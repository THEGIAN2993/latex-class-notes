\documentclass[11pt,twoside,openany]{memoir}
%\usepackage{mlmodern}
%\usepackage{tgpagella} % text only
%\usepackage{mathpazo}  % math & text
\usepackage[T1]{fontenc}
\usepackage[hidelinks]{hyperref}
\usepackage{amsmath}
\usepackage{amsthm}
\usepackage{amssymb}
%\usepackage{mathtools}
%\renewcommand*{\mathbf}[1]{\varmathbb{#1}}
%\usepackage{newpxtext}
%\usepackage{eulerpx}
%\usepackage{eucal}
\usepackage{datetime}
    \newdateformat{specialdate}{\THEYEAR\ \monthname\ \THEDAY}
\usepackage[margin=1in]{geometry}
\usepackage{fancyhdr}
    \fancyhf{}
    \pagestyle{fancy}
    \cfoot{\scriptsize }
    \fancyhead[R]{\scriptsize }
    \fancyhead[L]{\scriptsize }
    \renewcommand{\headrulewidth}{0pt}
    \renewcommand{\footrulewidth}{0pt} % if you also want to remove the footer rule
\usepackage{thmtools}
    \declaretheoremstyle[
        spaceabove=10pt,
        spacebelow=10pt,
        headfont=\normalfont\bfseries,
        notefont=\mdseries, notebraces={(}{)},
        bodyfont=\normalfont,
        postheadspace=0.5em
        %qed=\qedsymbol
        ]{defs}

    \declaretheoremstyle[ 
        spaceabove=10pt, % space above the theorem
        spacebelow=10pt,
        headfont=\normalfont\bfseries,
        bodyfont=\normalfont\itshape,
        postheadspace=0.5em
        ]{thmstyle}

    \declaretheorem[numberwithin=section,style=defs]{note}
    \declaretheorem[numbered=no,style=defs]{question}
    \declaretheorem[numbered=no,style=defs]{recall}
    \declaretheorem[numbered=no,style=remark]{answer}
    \declaretheorem[numbered=no,style=remark]{solution}
    \declaretheorem[numbered=no,style=defs]{definition}
    \declaretheorem[numbered=no,style=defs]{example}

    \declaretheorem[numbered=no,style=defs]{remark}
\usepackage{enumitem}
\usepackage{titlesec}
    \titleformat{\chapter}[display]
    {\bfseries\LARGE\raggedright}
    {Chapter {\thechapter}}
    {1ex minus .1ex}
    {\Huge}
    \titlespacing{\chapter}
    {3pc}{*3}{40pt}[3pc]

    \titleformat{\section}[block]
    {\normalfont\bfseries\Large}
    {\S\ \thesection.}{.5em}{}[]
    \titlespacing{\section}
    {0pt}{3ex plus .1ex minus .2ex}{3ex plus .1ex minus .2ex}
\usepackage[utf8x]{inputenc}
\usepackage{tikz}
\usepackage{tikz-cd}
\usepackage{wasysym}
\renewcommand{\int}{\varint}

\linespread{1}
%to make the correct symbol for Sha
%\newcommand\cyr{%
%\renewcommand\rmdefault{wncyr}%
%\renewcommand\sfdefault{wncyss}%
%\renewcommand\encodingdefault{OT2}%
%\normalfont \selectfont} \DeclareTextFontCommand{\textcyr}{\cyr}


\DeclareMathOperator{\ab}{ab}
\newcommand{\absgal}{\G_{\bbQ}}
\DeclareMathOperator{\ad}{ad}
\DeclareMathOperator{\adj}{adj}
\DeclareMathOperator{\alg}{alg}
\DeclareMathOperator{\Alt}{Alt}
\DeclareMathOperator{\Ann}{Ann}
\DeclareMathOperator{\arith}{arith}
\DeclareMathOperator{\Aut}{Aut}
\DeclareMathOperator{\Be}{B}
\DeclareMathOperator{\Bd}{Bd}
\DeclareMathOperator{\card}{card}
\DeclareMathOperator{\Char}{char}
\DeclareMathOperator{\csp}{csp}
\DeclareMathOperator{\codim}{codim}
\DeclareMathOperator{\coker}{coker}
\DeclareMathOperator{\coh}{H}
\DeclareMathOperator{\compl}{compl}
\DeclareMathOperator{\conj}{conj}
\DeclareMathOperator{\cont}{cont}
\DeclareMathOperator{\Cov}{Cov}
\DeclareMathOperator{\crys}{crys}
\DeclareMathOperator{\Crys}{Crys}
\DeclareMathOperator{\cusp}{cusp}
\DeclareMathOperator{\diag}{diag}
\DeclareMathOperator{\diam}{diam}
\DeclareMathOperator{\Dom}{Dom}
\DeclareMathOperator{\disc}{disc}
\DeclareMathOperator{\dist}{dist}
\DeclareMathOperator{\dR}{dR}
\DeclareMathOperator{\Eis}{Eis}
\DeclareMathOperator{\End}{End}
\DeclareMathOperator{\ev}{ev}
\DeclareMathOperator{\eval}{eval}
\DeclareMathOperator{\Eq}{Eq}
\DeclareMathOperator{\Ext}{Ext}
\DeclareMathOperator{\Fil}{Fil}
\DeclareMathOperator{\Fitt}{Fitt}
\DeclareMathOperator{\Frob}{Frob}
\DeclareMathOperator{\G}{G}
\DeclareMathOperator{\Gal}{Gal}
\DeclareMathOperator{\GL}{GL}
\DeclareMathOperator{\Gr}{Gr}
\DeclareMathOperator{\Graph}{Graph}
\DeclareMathOperator{\GSp}{GSp}
\DeclareMathOperator{\GUn}{GU}
\DeclareMathOperator{\Hom}{Hom}
\DeclareMathOperator{\id}{id}
\DeclareMathOperator{\Id}{Id}
\DeclareMathOperator{\Ik}{Ik}
\DeclareMathOperator{\IM}{Im}
\DeclareMathOperator{\Image}{im}
\DeclareMathOperator{\Ind}{Ind}
\DeclareMathOperator{\Inf}{inf}
\DeclareMathOperator{\Isom}{Isom}
\DeclareMathOperator{\J}{J}
\DeclareMathOperator{\Jac}{Jac}
\DeclareMathOperator{\lcm}{lcm}
\DeclareMathOperator{\length}{length}
\DeclareMathOperator*{\limit}{limit}
\DeclareMathOperator{\Log}{Log}
\DeclareMathOperator{\M}{M}
\DeclareMathOperator{\Mat}{Mat}
\DeclareMathOperator{\N}{N}
\DeclareMathOperator{\Nm}{Nm}
\DeclareMathOperator{\NIk}{N-Ik}
\DeclareMathOperator{\NSK}{N-SK}
\DeclareMathOperator{\new}{new}
\DeclareMathOperator{\obj}{obj}
\DeclareMathOperator{\old}{old}
\DeclareMathOperator{\ord}{ord}
\DeclareMathOperator{\Or}{O}
\DeclareMathOperator{\op}{op}
\DeclareMathOperator{\PGL}{PGL}
\DeclareMathOperator{\PGSp}{PGSp}
\DeclareMathOperator{\rank}{rank}
\DeclareMathOperator{\Ran}{Ran}
\DeclareMathOperator{\Rel}{Rel}
\DeclareMathOperator{\Real}{Re}
\DeclareMathOperator{\RES}{res}
\DeclareMathOperator{\Res}{Res}
%\DeclareMathOperator{\Sha}{\textcyr{Sh}}
\DeclareMathOperator{\Sel}{Sel}
\DeclareMathOperator{\semi}{ss}
\DeclareMathOperator{\sgn}{sign}
\DeclareMathOperator{\SK}{SK}
\DeclareMathOperator{\SL}{SL}
\DeclareMathOperator{\SO}{SO}
\DeclareMathOperator{\Sp}{Sp}
\DeclareMathOperator{\Span}{span}
\DeclareMathOperator{\Spec}{Spec}
\DeclareMathOperator{\spin}{spin}
\DeclareMathOperator{\st}{st}
\DeclareMathOperator{\St}{St}
\DeclareMathOperator{\SUn}{SU}
\DeclareMathOperator{\supp}{supp}
\DeclareMathOperator{\Sup}{sup}
\DeclareMathOperator{\Sym}{Sym}
\DeclareMathOperator{\Tam}{Tam}
\DeclareMathOperator{\tors}{tors}
\DeclareMathOperator{\tr}{tr}
\DeclareMathOperator{\Tr}{Tr}
\DeclareMathOperator{\un}{un}
\DeclareMathOperator{\Un}{U}
\DeclareMathOperator{\val}{val}
\DeclareMathOperator{\vol}{vol}

\DeclareMathOperator{\Sets}{S \mkern1.04mu e \mkern1.04mu t \mkern1.04mu s}
    \newcommand{\cSets}{\scalebox{1.02}{\contour{black}{$\Sets$}}}
    
\DeclareMathOperator{\Groups}{G \mkern1.04mu r \mkern1.04mu o \mkern1.04mu u \mkern1.04mu p \mkern1.04mu s}
    \newcommand{\cGroups}{\scalebox{1.02}{\contour{black}{$\Groups$}}}

\DeclareMathOperator{\TTop}{T \mkern1.04mu o \mkern1.04mu p}
    \newcommand{\cTop}{\scalebox{1.02}{\contour{black}{$\TTop$}}}

\DeclareMathOperator{\Htp}{H \mkern1.04mu t \mkern1.04mu p}
    \newcommand{\cHtp}{\scalebox{1.02}{\contour{black}{$\Htp$}}}

\DeclareMathOperator{\Mod}{M \mkern1.04mu o \mkern1.04mu d}
    \newcommand{\cMod}{\scalebox{1.02}{\contour{black}{$\Mod$}}}

\DeclareMathOperator{\Ab}{A \mkern1.04mu b}
    \newcommand{\cAb}{\scalebox{1.02}{\contour{black}{$\Ab$}}}

\DeclareMathOperator{\Rings}{R \mkern1.04mu i \mkern1.04mu n \mkern1.04mu g \mkern1.04mu s}
    \newcommand{\cRings}{\scalebox{1.02}{\contour{black}{$\Rings$}}}

\DeclareMathOperator{\ComRings}{C \mkern1.04mu o \mkern1.04mu m \mkern1.04mu R \mkern1.04mu i \mkern1.04mu n \mkern1.04mu g \mkern1.04mu s}
    \newcommand{\cComRings}{\scalebox{1.05}{\contour{black}{$\ComRings$}}}

\DeclareMathOperator{\hHom}{H \mkern1.04mu o \mkern1.04mu m}
    \newcommand{\cHom}{\scalebox{1.02}{\contour{black}{$\hHom$}}}

\renewcommand{\k}{\kappa}
\newcommand{\Ff}{F_{f}}
%\newcommand{\ts}{\,^{t}\!}


%Mathcal
\newcommand{\cA}{\mathcal{A}}
\newcommand{\cB}{\mathcal{B}}
\newcommand{\cC}{\mathcal{C}}
\newcommand{\cD}{\mathcal{D}}
\newcommand{\cE}{\mathcal{E}}
\newcommand{\cF}{\mathcal{F}}
\newcommand{\cG}{\mathcal{G}}
\newcommand{\cH}{\mathcal{H}}
\newcommand{\cI}{\mathcal{I}}
\newcommand{\cJ}{\mathcal{J}}
\newcommand{\cK}{\mathcal{K}}
\newcommand{\cL}{\mathcal{L}}
\newcommand{\cM}{\mathcal{M}}
\newcommand{\cN}{\mathcal{N}}
\newcommand{\cO}{\mathcal{O}}
\newcommand{\cP}{\mathcal{P}}
\newcommand{\cQ}{\mathcal{Q}}
\newcommand{\cR}{\mathcal{R}}
\newcommand{\cS}{\mathcal{S}}
\newcommand{\cT}{\mathcal{T}}
\newcommand{\cU}{\mathcal{U}}
\newcommand{\cV}{\mathcal{V}}
\newcommand{\cW}{\mathcal{W}}
\newcommand{\cX}{\mathcal{X}}
\newcommand{\cY}{\mathcal{Y}}
\newcommand{\cZ}{\mathcal{Z}}


%mathfrak (missing \fi)
\newcommand{\fa}{\mathfrak{a}}
\newcommand{\fA}{\mathfrak{A}}
\newcommand{\fb}{\mathfrak{b}}
\newcommand{\fB}{\mathfrak{B}}
\newcommand{\fc}{\mathfrak{c}}
\newcommand{\fC}{\mathfrak{C}}
\newcommand{\fd}{\mathfrak{d}}
\newcommand{\fD}{\mathfrak{D}}
\newcommand{\fe}{\mathfrak{e}}
\newcommand{\fE}{\mathfrak{E}}
\newcommand{\ff}{\mathfrak{f}}
\newcommand{\fF}{\mathfrak{F}}
\newcommand{\fg}{\mathfrak{g}}
\newcommand{\fG}{\mathfrak{G}}
\newcommand{\fh}{\mathfrak{h}}
\newcommand{\fH}{\mathfrak{H}}
\newcommand{\fI}{\mathfrak{I}}
\newcommand{\fj}{\mathfrak{j}}
\newcommand{\fJ}{\mathfrak{J}}
\newcommand{\fk}{\mathfrak{k}}
\newcommand{\fK}{\mathfrak{K}}
\newcommand{\fl}{\mathfrak{l}}
\newcommand{\fL}{\mathfrak{L}}
\newcommand{\fm}{\mathfrak{m}}
\newcommand{\fM}{\mathfrak{M}}
\newcommand{\fn}{\mathfrak{n}}
\newcommand{\fN}{\mathfrak{N}}
\newcommand{\fo}{\mathfrak{o}}
\newcommand{\fO}{\mathfrak{O}}
\newcommand{\fp}{\mathfrak{p}}
\newcommand{\fP}{\mathfrak{P}}
\newcommand{\fq}{\mathfrak{q}}
\newcommand{\fQ}{\mathfrak{Q}}
\newcommand{\fr}{\mathfrak{r}}
\newcommand{\fR}{\mathfrak{R}}
\newcommand{\fs}{\mathfrak{s}}
\newcommand{\fS}{\mathfrak{S}}
\newcommand{\ft}{\mathfrak{t}}
\newcommand{\fT}{\mathfrak{T}}
\newcommand{\fu}{\mathfrak{u}}
\newcommand{\fU}{\mathfrak{U}}
\newcommand{\fv}{\mathfrak{v}}
\newcommand{\fV}{\mathfrak{V}}
\newcommand{\fw}{\mathfrak{w}}
\newcommand{\fW}{\mathfrak{W}}
\newcommand{\fx}{\mathfrak{x}}
\newcommand{\fX}{\mathfrak{X}}
\newcommand{\fy}{\mathfrak{y}}
\newcommand{\fY}{\mathfrak{Y}}
\newcommand{\fz}{\mathfrak{z}}
\newcommand{\fZ}{\mathfrak{Z}}


%mathbf
\newcommand{\bfA}{\mathbf{A}}
\newcommand{\bfB}{\mathbf{B}}
\newcommand{\bfC}{\mathbf{C}}
\newcommand{\bfD}{\mathbf{D}}
\newcommand{\bfE}{\mathbf{E}}
\newcommand{\bfF}{\mathbf{F}}
\newcommand{\bfG}{\mathbf{G}}
\newcommand{\bfH}{\mathbf{H}}
\newcommand{\bfI}{\mathbf{I}}
\newcommand{\bfJ}{\mathbf{J}}
\newcommand{\bfK}{\mathbf{K}}
\newcommand{\bfL}{\mathbf{L}}
\newcommand{\bfM}{\mathbf{M}}
\newcommand{\bfN}{\mathbf{N}}
\newcommand{\bfO}{\mathbf{O}}
\newcommand{\bfP}{\mathbf{P}}
\newcommand{\bfQ}{\mathbf{Q}}
\newcommand{\bfR}{\mathbf{R}}
\newcommand{\bfS}{\mathbf{S}}
\newcommand{\bfT}{\mathbf{T}}
\newcommand{\bfU}{\mathbf{U}}
\newcommand{\bfV}{\mathbf{V}}
\newcommand{\bfW}{\mathbf{W}}
\newcommand{\bfX}{\mathbf{X}}
\newcommand{\bfY}{\mathbf{Y}}
\newcommand{\bfZ}{\mathbf{Z}}

\newcommand{\bfa}{\mathbf{a}}
\newcommand{\bfb}{\mathbf{b}}
\newcommand{\bfc}{\mathbf{c}}
\newcommand{\bfd}{\mathbf{d}}
\newcommand{\bfe}{\mathbf{e}}
\newcommand{\bff}{\mathbf{f}}
\newcommand{\bfg}{\mathbf{g}}
\newcommand{\bfh}{\mathbf{h}}
\newcommand{\bfi}{\mathbf{i}}
\newcommand{\bfj}{\mathbf{j}}
\newcommand{\bfk}{\mathbf{k}}
\newcommand{\bfl}{\mathbf{l}}
\newcommand{\bfm}{\mathbf{m}}
\newcommand{\bfn}{\mathbf{n}}
\newcommand{\bfo}{\mathbf{o}}
\newcommand{\bfp}{\mathbf{p}}
\newcommand{\bfq}{\mathbf{q}}
\newcommand{\bfr}{\mathbf{r}}
\newcommand{\bfs}{\mathbf{s}}
\newcommand{\bft}{\mathbf{t}}
\newcommand{\bfu}{\mathbf{u}}
\newcommand{\bfv}{\mathbf{v}}
\newcommand{\bfw}{\mathbf{w}}
\newcommand{\bfx}{\mathbf{x}}
\newcommand{\bfy}{\mathbf{y}}
\newcommand{\bfz}{\mathbf{z}}

%blackboard bold

\newcommand{\bbA}{\mathbb{A}}
\newcommand{\bbB}{\mathbb{B}}
\newcommand{\bbC}{\mathbb{C}}
\newcommand{\bbD}{\mathbb{D}}
\newcommand{\bbE}{\mathbb{E}}
\newcommand{\bbF}{\mathbb{F}}
\newcommand{\bbG}{\mathbb{G}}
\newcommand{\bbH}{\mathbb{H}}
\newcommand{\bbI}{\mathbb{I}}
\newcommand{\bbJ}{\mathbb{J}}
\newcommand{\bbK}{\mathbb{K}}
\newcommand{\bbL}{\mathbb{L}}
\newcommand{\bbM}{\mathbb{M}}
\newcommand{\bbN}{\mathbb{N}}
\newcommand{\bbO}{\mathbb{O}}
\newcommand{\bbP}{\mathbb{P}}
\newcommand{\bbQ}{\mathbb{Q}}
\newcommand{\bbR}{\mathbb{R}}
\newcommand{\bbS}{\mathbb{S}}
\newcommand{\bbT}{\mathbb{T}}
\newcommand{\bbU}{\mathbb{U}}
\newcommand{\bbV}{\mathbb{V}}
\newcommand{\bbW}{\mathbb{W}}
\newcommand{\bbX}{\mathbb{X}}
\newcommand{\bbY}{\mathbb{Y}}
\newcommand{\bbZ}{\mathbb{Z}}
\newcommand{\jota}{\jmath}

\newcommand{\bmat}{\left( \begin{matrix}}
\newcommand{\emat}{\end{matrix} \right)}

\newcommand{\bbmat}{\left[ \begin{matrix}}
\newcommand{\ebmat}{\end{matrix} \right]}

\newcommand{\pmat}{\left( \begin{smallmatrix}}
\newcommand{\epmat}{\end{smallmatrix} \right)}

\newcommand{\lat}{\mathscr{L}}
\newcommand{\mat}[4]{\begin{pmatrix}{#1}&{#2}\\{#3}&{#4}\end{pmatrix}}
\newcommand{\ov}[1]{\overline{#1}}
\newcommand{\res}[1]{\underset{#1}{\RES}\,}
\newcommand{\up}{\upsilon}

\newcommand{\tac}{\textasteriskcentered}

%mahesh macros
\newcommand{\tm}{\textrm}

%Comments
\newcommand{\com}[1]{\vspace{5 mm}\par \noindent
\marginpar{\textsc{Comment}} \framebox{\begin{minipage}[c]{0.95
\textwidth} \tt #1 \end{minipage}}\vspace{5 mm}\par}

\newcommand{\Bmu}{\mbox{$\raisebox{-0.59ex}
  {$l$}\hspace{-0.18em}\mu\hspace{-0.88em}\raisebox{-0.98ex}{\scalebox{2}
  {$\color{white}.$}}\hspace{-0.416em}\raisebox{+0.88ex}
  {$\color{white}.$}\hspace{0.46em}$}{}}  %need graphicx and xcolor. this produces blackboard bold mu 

\newcommand{\hooktwoheadrightarrow}{%
  \hookrightarrow\mathrel{\mspace{-15mu}}\rightarrow
}

\makeatletter
\newcommand{\xhooktwoheadrightarrow}[2][]{%
  \lhook\joinrel
  \ext@arrow 0359\rightarrowfill@ {#1}{#2}%
  \mathrel{\mspace{-15mu}}\rightarrow
}
\makeatother

\renewcommand{\geq}{\geqslant}
\renewcommand{\leq}{\leqslant}
\newcommand{\midd}{\hspace{4pt}\middle|\hspace{4pt}}
    
\newcommand{\bone}{\mathbf{1}}
\newcommand{\sign}{\mathrm{sign}}
\newcommand{\eps}{\varepsilon}
\newcommand{\textui}[1]{\uline{\textit{#1}}}

%\newcommand{\ov}{\overline}
%\newcommand{\un}{\underline}
\newcommand{\fin}{\mathrm{fin}}

\newcommand{\chnum}{\titleformat
{\chapter} % command
[display] % shape
{\centering} % format
{\Huge \color{black} \shadowbox{\thechapter}} % label
{-0.5em} % sep (space between the number and title)
{\LARGE \color{black} \underline} % before-code
}

\newcommand{\chunnum}{\titleformat
{\chapter} % command
[display] % shape
{} % format
{} % label
{0em} % sep
{ \begin{flushright} \begin{tabular}{r}  \Huge \color{black}
} % before-code
[
\end{tabular} \end{flushright} \normalsize
] % after-code
}

\newcommand{\nl}{\newline \mbox{}}

\newcommand{\h}[1]{\hspace{#1pt}}

\newcommand{\littletaller}{\mathchoice{\vphantom{\big|}}{}{}{}}
\newcommand\restr[2]{{% we make the whole thing an ordinary symbol
  \left.\kern-\nulldelimiterspace % automatically resize the bar with \right
  #1 % the function
  \littletaller % pretend it's a little taller at normal size
  \right|_{#2} % this is the delimiter
  }}

\newcommand{\mtext}[1]{\hspace{6pt}\text{#1}\hspace{6pt}}

\newcommand{\lnorm}{\left\lVert}
\newcommand{\rnorm}{\right\rVert}

\newcommand{\ds}{\displaystyle}
\newcommand{\ts}{\textstyle}


\newcommand{\sfrac}[2]{{}^{#1}\mskip -5mu/\mskip -3mu_{#2}}


\makeatletter
\newcommand*{\da@rightarrow}{\mathchar"0\hexnumber@\symAMSa 4B }
\newcommand*{\da@leftarrow}{\mathchar"0\hexnumber@\symAMSa 4C }
\newcommand*{\xdashrightarrow}[2][]{%
  \mathrel{%
    \mathpalette{\da@xarrow{#1}{#2}{}\da@rightarrow{\,}{}}{}%
  }%
}
\newcommand{\xdashleftarrow}[2][]{%
  \mathrel{%
    \mathpalette{\da@xarrow{#1}{#2}\da@leftarrow{}{}{\,}}{}%
  }%
}
\newcommand*{\da@xarrow}[7]{%
  % #1: below
  % #2: above
  % #3: arrow left
  % #4: arrow right
  % #5: space left 
  % #6: space right
  % #7: math style 
  \sbox0{$\ifx#7\scriptstyle\scriptscriptstyle\else\scriptstyle\fi#5#1#6\m@th$}%
  \sbox2{$\ifx#7\scriptstyle\scriptscriptstyle\else\scriptstyle\fi#5#2#6\m@th$}%
  \sbox4{$#7\dabar@\m@th$}%
  \dimen@=\wd0 %
  \ifdim\wd2 >\dimen@
    \dimen@=\wd2 %   
  \fi
  \count@=2 %
  \def\da@bars{\dabar@\dabar@}%
  \@whiledim\count@\wd4<\dimen@\do{%
    \advance\count@\@ne
    \expandafter\def\expandafter\da@bars\expandafter{%
      \da@bars
      \dabar@ 
    }%
  }%  
  \mathrel{#3}%
  \mathrel{%   
    \mathop{\da@bars}\limits
    \ifx\\#1\\%
    \else
      _{\copy0}%
    \fi
    \ifx\\#2\\%
    \else
      ^{\copy2}%
    \fi
  }%   
  \mathrel{#4}%
}
\makeatother

\renewcommand{\div}[2]{%
  \scalebox{0.92}{$#1$}%
  \hspace{1.2pt}%
  \scalebox{1.2}{$\mid$}%
  \hspace{0.75pt}%
  \scalebox{0.92}{$#2$}%
}


\begin{document}
\begin{center}
{\large Econ 272 \\[0.1in]Midterm II Study Guide \\[0.1in]}
\end{center}
\vspace{4pt}
%%%%%%%%%%%%%%%%%%%%%%%%%%%%%%%%%%%%%%%%%%%%%%%%%%%%%%%%%%%%%
    \section*{The Classical Assumptions}
        \begin{enumerate}[label = \Roman*,itemsep=1pt,topsep=3pt]
            \item The regression model is linear, is correctly specified, and has an additive error term.
                \begin{equation*}
                \begin{split}
                    Y_i = \beta_0 + \beta_1 X_{i,1} + \beta_2 X_{i,2} + ... + \beta_k X_{i,k} + \epsilon_i
                \end{split}
                \end{equation*}
            \item The error term has a zero population mean.
                \begin{equation*}
                \begin{split}
                    E[e_i] = 0 \h5\text{or}\h5 E[e_i \mid X_{i,j}] = 0.
                \end{split}
                \end{equation*}
            \item All explanatory variables are uncorrelated with the error term. (Strict Exogeneity)
                \begin{equation*}
                \begin{split}
                    \text{Cov}(X_{i,j},\epsilon_i) = 0 \h5\forall j.
                \end{split}
                \end{equation*}
            \item Observations of the error term are uncorrelated with each other (no serial correlation).
                \begin{equation*}
                \begin{split}
                    \text{Cov}(\epsilon_j,\epsilon_i) = 0 \h5\forall i\neq j.
                \end{split}
                \end{equation*}
            \item The error term has a constant variance. (Homoskedasticity).
                \begin{equation*}
                \begin{split}
                    \text{Var}(\epsilon_i) = \sigma ^2 \h5\forall i.
                \end{split}
                \end{equation*}
            \item No explanatory variable is a perfect linear function of any other explanatory variable(s).
        \end{enumerate}
%%%%%%%%%%%%%%%%%%%%%%%%%%%%%%%%%%%%%%%%%%%%%%%%%%%%%%%%%%%%%
    \section*{Omitted Variable Bias}
        \begin{definition}
            An \textit{omitted variable} is an important explanatory variable that has been left out of a regression equation.
        \end{definition}

        \begin{definition}
            An \textit{omitted variable bias} is the bias caused by leaving an omitted variable out of an OLS estimation.
        \end{definition}

        \begin{example}
            Suppose a true regression model is given by:
                \begin{equation*}
                \begin{split}
                    Y_i = \beta_0 = \beta_1 X_{i,1} + \beta_2 X_{i,2} + \epsilon_i.
                \end{split}
                \end{equation*}
            If we omit $X_2$ from the equation, we get:
                \begin{equation*}
                \begin{split}
                    Y_i = \beta_0^\ast + \beta_1^\ast X_{i,1} +  \epsilon^\ast_i.
                \end{split}
                \end{equation*}
            Recall that the \textit{stochastic error term} is a term which is added to a regression equation to introduce all the variation of $Y$ that cannot be explained by the included $X$'s. This means the stochastic error term includes the effects of any omitted variables, giving:
                \begin{equation*}
                \begin{split}
                    \epsilon^\ast_i = \epsilon_i + \beta_2 X_{i,2}.
                \end{split}
                \end{equation*}
            There is a reason our omitted variable equation includes $\beta_0^\ast$ and $\beta_1^\ast$. Note that $\beta_1$ is the impact of a one-unit increase in $X_1$ on $Y$ \textit{while holding $X_2$ constant}. But since $X_2$ isn't in our omitted variable equation, the OLS can't hold it constant, and as a result $\beta_1^\ast$ is the impact of a one-unit increase in $X_1$ on $Y$ \textit{not holding $X_2$ constant}.

            If we leave an important variable out of an equation, we violate Classical Assumption III. Most pairs of variables are correlated to some degree, so $X_1$ and $X_2$ from our true regression model are almost surely correlated. When $X_2$ is omitted from the equation, the impact of $X_2$ goes into $\epsilon^\ast$, so $\epsilon^\ast$ and $X_2$ are correlated, violating strict exogeneity.

            These paragraphs use words to describe why Classical Assumption III fails, which I think is dumb. From the above equations, note that:
                \begin{equation*}
                \begin{split}
                    E[\epsilon^\ast_i] 
                    & = E[\epsilon_i + \beta_2 X_{i,2}] \\
                    & = E[\epsilon_i] + E[\beta_2 X_{i,2}] \\
                    & = \beta_2 E[X_{i,2}].
                \end{split}
                \end{equation*}
            If $\beta_2$ and $E[X_{i,2}]$ are nonzero, then 
            
        \end{example}

        \begin{example}
            We want to quantify the amount and direction of bias. Suppose
                \begin{equation*}
                \begin{split}
                    Y_i = \beta_0 + \beta_1 X_{i,1} + \beta_2 X_{i,2} + \epsilon_i
                \end{split}
                \end{equation*}
            is our true regression equation. Since we know most pairs of variables are correlated, we can infer that:
                \begin{equation*}
                \begin{split}
                    X_{i,2} = \alpha_0 + \alpha_1 X_{i,1} + u_i.
                \end{split}
                \end{equation*}
            Subbing this into our true regression equation, we get:
                \begin{equation*}
                \begin{split}
                    Y_i 
                    & = \beta_0 + \beta_1 X_{i,1} + \beta_2 (\alpha_0 + \alpha_1 X_{i,1} + u_i) + \epsilon_i \\
                    & = (\beta_0 + \beta_2 \alpha_0) + (\beta_1 + \beta_2 \alpha_1) X_{i,1} + (\epsilon_i + \beta_2 u_i).
                \end{split}
                \end{equation*}
            The \textit{bias} is given by $\beta_2 \alpha_1$. Furthermore, note that:
                \begin{equation*}
                \begin{split}
                    E[\widehat{\beta_1}]
                    & = E[\beta_1 + \beta_2 \alpha_1] \\
                    & = \beta_1 + \beta_2 \alpha_1.
                \end{split}
                \end{equation*}
            In general, this bias exists unless:
                \begin{enumerate}[label = \arabic*. ,itemsep=1pt,topsep=3pt]
                    \item the true coefficient of our omitted variable equals zero;
                    \item the included and omitted variables are uncorrelated in the sample.
                \end{enumerate}
        \end{example}

        \begin{question}
            Consider the following population regression function:
                \begin{equation*}
                \begin{split}
                    \text{Unemp}_{cs} = \beta_0 + \beta_1 \text{MinWage}_{s} + \delta \mathbf{X}_{cs} + \epsilon_{cs}.
                \end{split}
                \end{equation*}
            Regression using OLS provides $\widehat{\beta_1} = -0.024$. An omitted variable is the history of labour movements in the state. Assume that states with a strong labour movement have higher minimum wages, or $\text{Cov}(\text{Unions}_s,\text{MinWage}_s)>0$. Is the estimated $\beta_1$ an over-estimate or an under-estimate of the true impact of minimum wage policies on unemployment? Show your steps and clearly state any assumptions you make.
        \end{question}
            \begin{solution}
                Note that our "true" regression function would be:
                    \begin{equation*}
                    \begin{split}
                        \text{Unemp}_{cs} = \beta_0 + \beta_1 \text{MinWage}_{s} + \beta_2 \text{Union}_s + \delta \mathbf{X}_{cs} + \epsilon_{cs}.
                    \end{split}
                    \end{equation*}
                Typical economic theory presumes higher labour movements results in higher unemployment, whence $\beta_2 > 0$. Since most pairs of variables are correlated to some degree, we can assume:
                    \begin{equation*}
                    \begin{split}
                        \text{MinWage}_s = \alpha_0 + \alpha_1 \text{Unions}_s + \epsilon_s.
                    \end{split}
                    \end{equation*}
                By the problem statement, we can assume $\alpha_1 > 0$. Thus our bias, which is denoted by $\alpha_1 \beta_2$, must be positive. Whence:
                    \begin{equation*}
                    \begin{split}
                        -0.024 = \beta_1 + \beta_2 \alpha_1 \\
                        \iff \\
                        -0.024 - \beta_2 \alpha_1 = \beta_1.
                    \end{split}
                    \end{equation*}
                Thus our estimator is []-biased.
            \end{solution}
%%%%%%%%%%%%%%%%%%%%%%%%%%%%%%%%%%%%%%%%%%%%%%%%%%%%%%%%%%%%%
    \section*{Interaction Terms}
        \begin{definition}
            An \textit{interaction term} is an independent variable in a regression equation that is a multiple of two or more other independent variables.
        \end{definition}

        For this section our interaction terms will strictly be the product of two independent variables. Interaction terms can involve two quantitative variables, or two dummy variables, but the most frequent application if interaction terms involves one quantitative variable and one dummy variable.

        \begin{example}
            We will often be asked to interpret the coefficients of an OLS estimation containing interaction terms. We start with a very general example. Let:
                \begin{equation*}
                \begin{split}
                    Y_i = \beta_0 + \beta_1 X_{i} + \beta_2 D_i + \beta_3 X_i D_i + \epsilon_i
                \end{split}
                \end{equation*}
            be a given linear model with $X_i$ a quantitative variable and $D_i$ a dummy variable. There are only two cases to consider. 
            
            Case 1: $D_i = 1$. Then:
                \begin{equation*}
                \begin{split}
                    Y_i = (\beta_0 + \beta_1) + (\beta_1 + \beta_3) X_i + \epsilon_i.
                \end{split}
                \end{equation*}
            From this, we can see that $\beta_3$ represents a "change of change"; it measures how much $Y_i$ changes per unit-change of $X_i$ \textit{assuming $D_i$} is true. Another interpretation would be it measures how much the effect of $X_i$ for $D_i$ differs from the effect of $X_i$ without $D_i.$ Similarly, our new constant term, $\beta_0 + \beta_3$, represent the mean value of the dependent variable when $D_i$ is true.

            Case 2: $D_i = 0$. Then:
                \begin{equation*}
                \begin{split}
                    Y_i = \beta_0 + \beta_1 X_{i} + \epsilon_i.
                \end{split}
                \end{equation*}
            Notice that $\beta_1$ isn't merely the change in $Y_i$ per unit change in $X_i$, rather it is the \textit{change in $Y_i$ per unit change in $X_i$ when $D_i$ is false}. The constant term $\beta_0$ is the mean value of our dependent variable assuming $D_i$ to be false.
        \end{example}

        \begin{example}
            
        \end{example}
%%%%%%%%%%%%%%%%%%%%%%%%%%%%%%%%%%%%%%%%%%%%%%%%%%%%%%%%%%%%%
    \section*{Linear Probability Models}
        \begin{definition}
            A \textit{linear probability model} is a OLS equation used to explain a dummy dependent variable:
                \begin{equation*}
                \begin{split}
                    D_i = \beta_0 + \beta_1 X_{i,1} + ... + \beta_k X_{i,k} + \epsilon_i.
                \end{split}
                \end{equation*}
        \end{definition}

        The term \textit{probability model} comes from the fact that taking the expected value of a dummy variable measures the probability that $D_i = 1$. This means:
            \begin{equation*}
            \begin{split}
                E[D_i] = \widehat{P(D_i = 1)}
                & = E[\beta_0 + \beta_1 X_{i,1} + ... + \beta_k X_{i,k} + \epsilon_i] \\
                & = \widehat{\beta_0} + \widehat{\beta_1} X_{i,1} + ... + \widehat{\beta_k} X_{i,k},
            \end{split}
            \end{equation*}
        where $P(D_i = 1)$ indicates the probability that $D_i = 1$ for the $i^\text{th}$ observation. So a unit change of $X_i$ results in a $\beta$ change in probability for $D_i = 1$ to occur.

        Using OLS to estimate the coefficients of an equation with a dummy dependent variable faces at least three problems:
            \begin{enumerate}[label = (\arabic*),itemsep=1pt,topsep=3pt]
                \item $\overline{R}^2$ is not an accurate measure of overall fit.
                \item $\widehat{D_i} = \widehat{P(D_i = 1)}$ is not bounded by 0 and 1. Any prediction that a probability equals something less than zero or greater than one is meaningless.
                \item The error term is neither homoskedastic nor normally distributed. In practice these problems on OLS estimation is minor, so it is typically ignored.
            \end{enumerate}
%%%%%%%%%%%%%%%%%%%%%%%%%%%%%%%%%%%%%%%%%%%%%%%%%%%%%%%%%%%%%
    \section*{Fixed Effects}
        \begin{definition}
            \textit{Panel data}, or \textit{longitudinal data} combines time-series and cross-sectional, by including observations on the same variables from the same cross-sectional sample from two or more different time periods.
        \end{definition}

        \begin{example}
            Suppose we surveyed 200 students when they graduated from college and then administered the same questionnaire to each student five years later. This would be a panel data set.
        \end{example}

        We are interested in estimating panel data equations.

        \begin{definition}
            The \textit{fixed effects model} estimates panel data equations by including enough dummy variables to allow each cross-sectional entity (like a state or country) and each time period to have a different intercept:
                \begin{equation*}
                \begin{split}
                    Y_{it} = \beta_0 + \beta_1 X_{it} + \alpha_2 E_2 + ... + \alpha_n E_n + \rho_2 T_2 + ... + \rho_m T_m + \epsilon_{it},
                \end{split}
                \end{equation*}
            where each $E_i$ is an entity fixed dummy variable (equal to 1 for the ith entity and 0 otherwise) and $T_i$ is a time fixed dummy variable (equal to 1 for the ith period and 0 otherwise).
        \end{definition}

        There's a reason this equation looks so complicated (even though it's just $n+m$ added dummy variables). If we estimated our model without accounting for the fact that our observations are from a panel data set. Then our equation looks like:
            \begin{equation*}
            \begin{split}
                Y_{it} = \beta_0 + \beta_1 X_{it} + V_{it},
            \end{split}
            \end{equation*}
        where $V_{it}$ represents the error term. To understand $V$, it's best to look at an example which is less general. Suppose our cross-sectional data is on the 50 U.S. states, and suppose our time-series data is from the years 2000-2010. Clearly no two states are alike \textemdash they have different cultures, histories, institutions, governments, etc. Likewise, a state's history and culture are relatively constant from year to year. Even if they are impossible to measure, we know that they don't change a lot, and in particular they distinguish each state from all the others. It is very likely that the unchanging and unmeasured differences between states are correlated with $X_{it}$, giving us omitted variable bias. Furthermore, it's also likely that the time-series data would add even more omitted variables (see example in text).

        The solution lies in examining what $V_{it}$ represents. We can break it into three components:
            \begin{equation*}
            \begin{split}
                V_{it} + \epsilon_{it} + a_i + z_t,
            \end{split}
            \end{equation*}
        where $\epsilon_{it}$ is the classical error term, $a_i$ refers to the entity characteristics omitted from the equation, and $z_t$ refers to the time characteristics omitted from the equation. If $a_i$ and $z_t$ are correlated with $X_{it}$, we will violate Classical Assumption III, and our estimate of $\beta_1$ will be biased. But simply including dummy variables for every entity (but one) and every time period (but one), we can control for the unchanging entity-effects and the time-fixed effects. Including each of the $n$ dummy variables for every entity and each of the $m$ dummy variables for each time-period results in the entity and time fixed effects not being omitted variables (because they are represented by dummy variables). We arrive at our original equation with the $m+n$ dummy variables.

        The major advantages of the fixed effects model is that it avoids bias due to omitted variables that don't change over time, or that change over time equally for all entities. The beauty lies in the fact we don't have to know exactly what things go into the entity and time fixed effects, the dummy variables include them all.

        There are drawbacks, however. No substantive explanatory variable that varies across entities, but not over time within each entity, can be used. They would create perfect multicollinearity.
%%%%%%%%%%%%%%%%%%%%%%%%%%%%%%%%%%%%%%%%%%%%%%%%%%%%%%%%%%%%%
\end{document}