\documentclass[10pt,twoside,openany]{memoir}
%\usepackage{mlmodern}
%\usepackage{tgpagella} % text only
%\usepackage{mathpazo}  % math & text
\usepackage[T1]{fontenc}
\usepackage[hidelinks]{hyperref}
\usepackage{amsmath}
\usepackage[fixamsmath]{mathtools}  % Extension to amsmath
\usepackage{amsthm}
\usepackage{amssymb}
\renewcommand*{\mathbf}[1]{\varmathbb{#1}}
\usepackage{newpxtext}
\usepackage{eulerpx}
\usepackage{eucal}
\usepackage{datetime}
    \newdateformat{specialdate}{\THEYEAR\ \monthname\ \THEDAY}
\usepackage[margin=1.5in]{geometry}
\usepackage{fancyhdr}
    \fancyhf{}
    \pagestyle{fancy}
    \cfoot{\scriptsize \thepage}
    \fancyhead[R]{\scalebox{0.7}{\rightmark}}
    \fancyhead[L]{\scalebox{0.7}{\leftmark}}
\usepackage{thmtools}
    \declaretheoremstyle[
        spaceabove=10pt,
        spacebelow=10pt,
        headfont=\normalfont\bfseries,
        notefont=\mdseries, notebraces={(}{)},
        bodyfont=\normalfont,
        postheadspace=0.5em
        %qed=\qedsymbol
        ]{defs}

    \declaretheoremstyle[ 
        spaceabove=10pt, % space above the theorem
        spacebelow=10pt,
        headfont=\normalfont\bfseries,
        bodyfont=\normalfont\itshape,
        postheadspace=0.5em
        ]{thmstyle}
    
    \declaretheorem[
        style=thmstyle,
        numberwithin=section
    ]{theorem}

    \declaretheorem[
        style=thmstyle,
        sibling=theorem,
    ]{proposition}

    \declaretheorem[
        style=thmstyle,
        sibling=theorem,
    ]{lemma}

    \declaretheorem[
        style=thmstyle,
        sibling=theorem,
    ]{corollary}

    \declaretheorem[
        numberwithin=section,
        style=defs,
    ]{example}

    \declaretheorem[
        numberwithin=section,
        style=defs,
    ]{definition}

    \declaretheorem[
        style=defs,
        sibling=theorem,
        numberwithin=section,
    ]{exercise}

    \declaretheorem[
        numbered=unless unique,
        shaded={rulecolor=black,
    rulewidth=1pt, bgcolor={rgb}{1,1,1}}
    ]{axiom}

    \declaretheorem[numberwithin=section,style=defs]{note}
    \declaretheorem[numbered=no,style=defs]{question}
    \declaretheorem[numbered=no,style=defs]{recall}
    \declaretheorem[numbered=no,style=remark]{answer}
    \declaretheorem[numbered=no,style=remark]{solution}
    \declaretheorem[numbered=no,style=defs]{remark}
\usepackage{enumitem}
\usepackage{titlesec}
    \titleformat{\chapter}[display]
    {\bfseries\Huge\raggedright}
    {Chapter {\thechapter}}
    {1ex minus .1ex}
    {\HUGE}
    \titlespacing{\chapter}
    {3pc}{*3}{40pt}[3pc]

    \titleformat{\section}[block]
    {\normalfont\bfseries\LARGE}
    {\S\ \thesection.}{.5em}{}[]
    \titlespacing{\section}
    {0pt}{3ex plus .1ex minus .2ex}{3ex plus .1ex minus .2ex}
\usepackage[utf8x]{inputenc}
\usepackage{tikz}
\usepackage{tikz-cd}
\usepackage{wasysym}
\linespread{1.00}
%%%%%%%%%%%%%%%%%%%%%%%%%%%%%%%%%%%%%%%%%%%%%%%%%%%%%%%%%%%%%
%%%%%%%%%%%%%%%%%%%%%%%%%%%%%%%%%%%%%%%%%%%%%%%%%%%%%%%%%%%%%
%to make the correct symbol for Sha
%\newcommand\cyr{%
%\renewcommand\rmdefault{wncyr}%
%\renewcommand\sfdefault{wncyss}%
%\renewcommand\encodingdefault{OT2}%
%\normalfont \selectfont} \DeclareTextFontCommand{\textcyr}{\cyr}


\DeclareMathOperator{\ab}{ab}
\newcommand{\absgal}{\G_{\bbQ}}
\DeclareMathOperator{\ad}{ad}
\DeclareMathOperator{\adj}{adj}
\DeclareMathOperator{\alg}{alg}
\DeclareMathOperator{\Alt}{Alt}
\DeclareMathOperator{\Ann}{Ann}
\DeclareMathOperator{\arith}{arith}
\DeclareMathOperator{\Aut}{Aut}
\DeclareMathOperator{\Be}{B}
\DeclareMathOperator{\Bd}{Bd}
\DeclareMathOperator{\card}{card}
\DeclareMathOperator{\Char}{char}
\DeclareMathOperator{\csp}{csp}
\DeclareMathOperator{\codim}{codim}
\DeclareMathOperator{\coker}{coker}
\DeclareMathOperator{\coh}{H}
\DeclareMathOperator{\compl}{compl}
\DeclareMathOperator{\conj}{conj}
\DeclareMathOperator{\cont}{cont}
\DeclareMathOperator{\Cov}{Cov}
\DeclareMathOperator{\crys}{crys}
\DeclareMathOperator{\Crys}{Crys}
\DeclareMathOperator{\cusp}{cusp}
\DeclareMathOperator{\diag}{diag}
\DeclareMathOperator{\diam}{diam}
\DeclareMathOperator{\Dom}{Dom}
\DeclareMathOperator{\disc}{disc}
\DeclareMathOperator{\dist}{dist}
\DeclareMathOperator{\dR}{dR}
\DeclareMathOperator{\Eis}{Eis}
\DeclareMathOperator{\End}{End}
\DeclareMathOperator{\ev}{ev}
\DeclareMathOperator{\eval}{eval}
\DeclareMathOperator{\Eq}{Eq}
\DeclareMathOperator{\Ext}{Ext}
\DeclareMathOperator{\Fil}{Fil}
\DeclareMathOperator{\Fitt}{Fitt}
\DeclareMathOperator{\Frob}{Frob}
\DeclareMathOperator{\G}{G}
\DeclareMathOperator{\Gal}{Gal}
\DeclareMathOperator{\GL}{GL}
\DeclareMathOperator{\Gr}{Gr}
\DeclareMathOperator{\Graph}{Graph}
\DeclareMathOperator{\GSp}{GSp}
\DeclareMathOperator{\GUn}{GU}
\DeclareMathOperator{\Hom}{Hom}
\DeclareMathOperator{\id}{id}
\DeclareMathOperator{\Id}{Id}
\DeclareMathOperator{\Ik}{Ik}
\DeclareMathOperator{\IM}{Im}
\DeclareMathOperator{\Image}{im}
\DeclareMathOperator{\Ind}{Ind}
\DeclareMathOperator{\Inf}{inf}
\DeclareMathOperator{\Isom}{Isom}
\DeclareMathOperator{\J}{J}
\DeclareMathOperator{\Jac}{Jac}
\DeclareMathOperator{\lcm}{lcm}
\DeclareMathOperator{\length}{length}
\DeclareMathOperator*{\limit}{limit}
\DeclareMathOperator{\Log}{Log}
\DeclareMathOperator{\M}{M}
\DeclareMathOperator{\Mat}{Mat}
\DeclareMathOperator{\N}{N}
\DeclareMathOperator{\Nm}{Nm}
\DeclareMathOperator{\NIk}{N-Ik}
\DeclareMathOperator{\NSK}{N-SK}
\DeclareMathOperator{\new}{new}
\DeclareMathOperator{\obj}{obj}
\DeclareMathOperator{\old}{old}
\DeclareMathOperator{\ord}{ord}
\DeclareMathOperator{\Or}{O}
\DeclareMathOperator{\op}{op}
\DeclareMathOperator{\PGL}{PGL}
\DeclareMathOperator{\PGSp}{PGSp}
\DeclareMathOperator{\rank}{rank}
\DeclareMathOperator{\Ran}{Ran}
\DeclareMathOperator{\Rel}{Rel}
\DeclareMathOperator{\Real}{Re}
\DeclareMathOperator{\RES}{res}
\DeclareMathOperator{\Res}{Res}
%\DeclareMathOperator{\Sha}{\textcyr{Sh}}
\DeclareMathOperator{\Sel}{Sel}
\DeclareMathOperator{\semi}{ss}
\DeclareMathOperator{\sgn}{sign}
\DeclareMathOperator{\SK}{SK}
\DeclareMathOperator{\SL}{SL}
\DeclareMathOperator{\SO}{SO}
\DeclareMathOperator{\Sp}{Sp}
\DeclareMathOperator{\Span}{span}
\DeclareMathOperator{\Spec}{Spec}
\DeclareMathOperator{\spin}{spin}
\DeclareMathOperator{\st}{st}
\DeclareMathOperator{\St}{St}
\DeclareMathOperator{\SUn}{SU}
\DeclareMathOperator{\supp}{supp}
\DeclareMathOperator{\Sup}{sup}
\DeclareMathOperator{\Sym}{Sym}
\DeclareMathOperator{\Tam}{Tam}
\DeclareMathOperator{\tors}{tors}
\DeclareMathOperator{\tr}{tr}
\DeclareMathOperator{\Tr}{Tr}
\DeclareMathOperator{\un}{un}
\DeclareMathOperator{\Un}{U}
\DeclareMathOperator{\val}{val}
\DeclareMathOperator{\vol}{vol}

\DeclareMathOperator{\Sets}{S \mkern1.04mu e \mkern1.04mu t \mkern1.04mu s}
    \newcommand{\cSets}{\scalebox{1.02}{\contour{black}{$\Sets$}}}
    
\DeclareMathOperator{\Groups}{G \mkern1.04mu r \mkern1.04mu o \mkern1.04mu u \mkern1.04mu p \mkern1.04mu s}
    \newcommand{\cGroups}{\scalebox{1.02}{\contour{black}{$\Groups$}}}

\DeclareMathOperator{\TTop}{T \mkern1.04mu o \mkern1.04mu p}
    \newcommand{\cTop}{\scalebox{1.02}{\contour{black}{$\TTop$}}}

\DeclareMathOperator{\Htp}{H \mkern1.04mu t \mkern1.04mu p}
    \newcommand{\cHtp}{\scalebox{1.02}{\contour{black}{$\Htp$}}}

\DeclareMathOperator{\Mod}{M \mkern1.04mu o \mkern1.04mu d}
    \newcommand{\cMod}{\scalebox{1.02}{\contour{black}{$\Mod$}}}

\DeclareMathOperator{\Ab}{A \mkern1.04mu b}
    \newcommand{\cAb}{\scalebox{1.02}{\contour{black}{$\Ab$}}}

\DeclareMathOperator{\Rings}{R \mkern1.04mu i \mkern1.04mu n \mkern1.04mu g \mkern1.04mu s}
    \newcommand{\cRings}{\scalebox{1.02}{\contour{black}{$\Rings$}}}

\DeclareMathOperator{\ComRings}{C \mkern1.04mu o \mkern1.04mu m \mkern1.04mu R \mkern1.04mu i \mkern1.04mu n \mkern1.04mu g \mkern1.04mu s}
    \newcommand{\cComRings}{\scalebox{1.05}{\contour{black}{$\ComRings$}}}

\DeclareMathOperator{\hHom}{H \mkern1.04mu o \mkern1.04mu m}
    \newcommand{\cHom}{\scalebox{1.02}{\contour{black}{$\hHom$}}}

\renewcommand{\k}{\kappa}
\newcommand{\Ff}{F_{f}}
%\newcommand{\ts}{\,^{t}\!}


%Mathcal
\newcommand{\cA}{\mathcal{A}}
\newcommand{\cB}{\mathcal{B}}
\newcommand{\cC}{\mathcal{C}}
\newcommand{\cD}{\mathcal{D}}
\newcommand{\cE}{\mathcal{E}}
\newcommand{\cF}{\mathcal{F}}
\newcommand{\cG}{\mathcal{G}}
\newcommand{\cH}{\mathcal{H}}
\newcommand{\cI}{\mathcal{I}}
\newcommand{\cJ}{\mathcal{J}}
\newcommand{\cK}{\mathcal{K}}
\newcommand{\cL}{\mathcal{L}}
\newcommand{\cM}{\mathcal{M}}
\newcommand{\cN}{\mathcal{N}}
\newcommand{\cO}{\mathcal{O}}
\newcommand{\cP}{\mathcal{P}}
\newcommand{\cQ}{\mathcal{Q}}
\newcommand{\cR}{\mathcal{R}}
\newcommand{\cS}{\mathcal{S}}
\newcommand{\cT}{\mathcal{T}}
\newcommand{\cU}{\mathcal{U}}
\newcommand{\cV}{\mathcal{V}}
\newcommand{\cW}{\mathcal{W}}
\newcommand{\cX}{\mathcal{X}}
\newcommand{\cY}{\mathcal{Y}}
\newcommand{\cZ}{\mathcal{Z}}


%mathfrak (missing \fi)
\newcommand{\fa}{\mathfrak{a}}
\newcommand{\fA}{\mathfrak{A}}
\newcommand{\fb}{\mathfrak{b}}
\newcommand{\fB}{\mathfrak{B}}
\newcommand{\fc}{\mathfrak{c}}
\newcommand{\fC}{\mathfrak{C}}
\newcommand{\fd}{\mathfrak{d}}
\newcommand{\fD}{\mathfrak{D}}
\newcommand{\fe}{\mathfrak{e}}
\newcommand{\fE}{\mathfrak{E}}
\newcommand{\ff}{\mathfrak{f}}
\newcommand{\fF}{\mathfrak{F}}
\newcommand{\fg}{\mathfrak{g}}
\newcommand{\fG}{\mathfrak{G}}
\newcommand{\fh}{\mathfrak{h}}
\newcommand{\fH}{\mathfrak{H}}
\newcommand{\fI}{\mathfrak{I}}
\newcommand{\fj}{\mathfrak{j}}
\newcommand{\fJ}{\mathfrak{J}}
\newcommand{\fk}{\mathfrak{k}}
\newcommand{\fK}{\mathfrak{K}}
\newcommand{\fl}{\mathfrak{l}}
\newcommand{\fL}{\mathfrak{L}}
\newcommand{\fm}{\mathfrak{m}}
\newcommand{\fM}{\mathfrak{M}}
\newcommand{\fn}{\mathfrak{n}}
\newcommand{\fN}{\mathfrak{N}}
\newcommand{\fo}{\mathfrak{o}}
\newcommand{\fO}{\mathfrak{O}}
\newcommand{\fp}{\mathfrak{p}}
\newcommand{\fP}{\mathfrak{P}}
\newcommand{\fq}{\mathfrak{q}}
\newcommand{\fQ}{\mathfrak{Q}}
\newcommand{\fr}{\mathfrak{r}}
\newcommand{\fR}{\mathfrak{R}}
\newcommand{\fs}{\mathfrak{s}}
\newcommand{\fS}{\mathfrak{S}}
\newcommand{\ft}{\mathfrak{t}}
\newcommand{\fT}{\mathfrak{T}}
\newcommand{\fu}{\mathfrak{u}}
\newcommand{\fU}{\mathfrak{U}}
\newcommand{\fv}{\mathfrak{v}}
\newcommand{\fV}{\mathfrak{V}}
\newcommand{\fw}{\mathfrak{w}}
\newcommand{\fW}{\mathfrak{W}}
\newcommand{\fx}{\mathfrak{x}}
\newcommand{\fX}{\mathfrak{X}}
\newcommand{\fy}{\mathfrak{y}}
\newcommand{\fY}{\mathfrak{Y}}
\newcommand{\fz}{\mathfrak{z}}
\newcommand{\fZ}{\mathfrak{Z}}


%mathbf
\newcommand{\bfA}{\mathbf{A}}
\newcommand{\bfB}{\mathbf{B}}
\newcommand{\bfC}{\mathbf{C}}
\newcommand{\bfD}{\mathbf{D}}
\newcommand{\bfE}{\mathbf{E}}
\newcommand{\bfF}{\mathbf{F}}
\newcommand{\bfG}{\mathbf{G}}
\newcommand{\bfH}{\mathbf{H}}
\newcommand{\bfI}{\mathbf{I}}
\newcommand{\bfJ}{\mathbf{J}}
\newcommand{\bfK}{\mathbf{K}}
\newcommand{\bfL}{\mathbf{L}}
\newcommand{\bfM}{\mathbf{M}}
\newcommand{\bfN}{\mathbf{N}}
\newcommand{\bfO}{\mathbf{O}}
\newcommand{\bfP}{\mathbf{P}}
\newcommand{\bfQ}{\mathbf{Q}}
\newcommand{\bfR}{\mathbf{R}}
\newcommand{\bfS}{\mathbf{S}}
\newcommand{\bfT}{\mathbf{T}}
\newcommand{\bfU}{\mathbf{U}}
\newcommand{\bfV}{\mathbf{V}}
\newcommand{\bfW}{\mathbf{W}}
\newcommand{\bfX}{\mathbf{X}}
\newcommand{\bfY}{\mathbf{Y}}
\newcommand{\bfZ}{\mathbf{Z}}

\newcommand{\bfa}{\mathbf{a}}
\newcommand{\bfb}{\mathbf{b}}
\newcommand{\bfc}{\mathbf{c}}
\newcommand{\bfd}{\mathbf{d}}
\newcommand{\bfe}{\mathbf{e}}
\newcommand{\bff}{\mathbf{f}}
\newcommand{\bfg}{\mathbf{g}}
\newcommand{\bfh}{\mathbf{h}}
\newcommand{\bfi}{\mathbf{i}}
\newcommand{\bfj}{\mathbf{j}}
\newcommand{\bfk}{\mathbf{k}}
\newcommand{\bfl}{\mathbf{l}}
\newcommand{\bfm}{\mathbf{m}}
\newcommand{\bfn}{\mathbf{n}}
\newcommand{\bfo}{\mathbf{o}}
\newcommand{\bfp}{\mathbf{p}}
\newcommand{\bfq}{\mathbf{q}}
\newcommand{\bfr}{\mathbf{r}}
\newcommand{\bfs}{\mathbf{s}}
\newcommand{\bft}{\mathbf{t}}
\newcommand{\bfu}{\mathbf{u}}
\newcommand{\bfv}{\mathbf{v}}
\newcommand{\bfw}{\mathbf{w}}
\newcommand{\bfx}{\mathbf{x}}
\newcommand{\bfy}{\mathbf{y}}
\newcommand{\bfz}{\mathbf{z}}

%blackboard bold

\newcommand{\bbA}{\mathbb{A}}
\newcommand{\bbB}{\mathbb{B}}
\newcommand{\bbC}{\mathbb{C}}
\newcommand{\bbD}{\mathbb{D}}
\newcommand{\bbE}{\mathbb{E}}
\newcommand{\bbF}{\mathbb{F}}
\newcommand{\bbG}{\mathbb{G}}
\newcommand{\bbH}{\mathbb{H}}
\newcommand{\bbI}{\mathbb{I}}
\newcommand{\bbJ}{\mathbb{J}}
\newcommand{\bbK}{\mathbb{K}}
\newcommand{\bbL}{\mathbb{L}}
\newcommand{\bbM}{\mathbb{M}}
\newcommand{\bbN}{\mathbb{N}}
\newcommand{\bbO}{\mathbb{O}}
\newcommand{\bbP}{\mathbb{P}}
\newcommand{\bbQ}{\mathbb{Q}}
\newcommand{\bbR}{\mathbb{R}}
\newcommand{\bbS}{\mathbb{S}}
\newcommand{\bbT}{\mathbb{T}}
\newcommand{\bbU}{\mathbb{U}}
\newcommand{\bbV}{\mathbb{V}}
\newcommand{\bbW}{\mathbb{W}}
\newcommand{\bbX}{\mathbb{X}}
\newcommand{\bbY}{\mathbb{Y}}
\newcommand{\bbZ}{\mathbb{Z}}
\newcommand{\jota}{\jmath}

\newcommand{\bmat}{\left( \begin{matrix}}
\newcommand{\emat}{\end{matrix} \right)}

\newcommand{\bbmat}{\left[ \begin{matrix}}
\newcommand{\ebmat}{\end{matrix} \right]}

\newcommand{\pmat}{\left( \begin{smallmatrix}}
\newcommand{\epmat}{\end{smallmatrix} \right)}

\newcommand{\lat}{\mathscr{L}}
\newcommand{\mat}[4]{\begin{pmatrix}{#1}&{#2}\\{#3}&{#4}\end{pmatrix}}
\newcommand{\ov}[1]{\overline{#1}}
\newcommand{\res}[1]{\underset{#1}{\RES}\,}
\newcommand{\up}{\upsilon}

\newcommand{\tac}{\textasteriskcentered}

%mahesh macros
\newcommand{\tm}{\textrm}

%Comments
\newcommand{\com}[1]{\vspace{5 mm}\par \noindent
\marginpar{\textsc{Comment}} \framebox{\begin{minipage}[c]{0.95
\textwidth} \tt #1 \end{minipage}}\vspace{5 mm}\par}

\newcommand{\Bmu}{\mbox{$\raisebox{-0.59ex}
  {$l$}\hspace{-0.18em}\mu\hspace{-0.88em}\raisebox{-0.98ex}{\scalebox{2}
  {$\color{white}.$}}\hspace{-0.416em}\raisebox{+0.88ex}
  {$\color{white}.$}\hspace{0.46em}$}{}}  %need graphicx and xcolor. this produces blackboard bold mu 

\newcommand{\hooktwoheadrightarrow}{%
  \hookrightarrow\mathrel{\mspace{-15mu}}\rightarrow
}

\makeatletter
\newcommand{\xhooktwoheadrightarrow}[2][]{%
  \lhook\joinrel
  \ext@arrow 0359\rightarrowfill@ {#1}{#2}%
  \mathrel{\mspace{-15mu}}\rightarrow
}
\makeatother

\renewcommand{\geq}{\geqslant}
\renewcommand{\leq}{\leqslant}
\newcommand{\midd}{\hspace{4pt}\middle|\hspace{4pt}}
    
\newcommand{\bone}{\mathbf{1}}
\newcommand{\sign}{\mathrm{sign}}
\newcommand{\eps}{\varepsilon}
\newcommand{\textui}[1]{\uline{\textit{#1}}}

%\newcommand{\ov}{\overline}
%\newcommand{\un}{\underline}
\newcommand{\fin}{\mathrm{fin}}

\newcommand{\chnum}{\titleformat
{\chapter} % command
[display] % shape
{\centering} % format
{\Huge \color{black} \shadowbox{\thechapter}} % label
{-0.5em} % sep (space between the number and title)
{\LARGE \color{black} \underline} % before-code
}

\newcommand{\chunnum}{\titleformat
{\chapter} % command
[display] % shape
{} % format
{} % label
{0em} % sep
{ \begin{flushright} \begin{tabular}{r}  \Huge \color{black}
} % before-code
[
\end{tabular} \end{flushright} \normalsize
] % after-code
}

\newcommand{\nl}{\newline \mbox{}}

\newcommand{\h}[1]{\hspace{#1pt}}

\newcommand{\littletaller}{\mathchoice{\vphantom{\big|}}{}{}{}}
\newcommand\restr[2]{{% we make the whole thing an ordinary symbol
  \left.\kern-\nulldelimiterspace % automatically resize the bar with \right
  #1 % the function
  \littletaller % pretend it's a little taller at normal size
  \right|_{#2} % this is the delimiter
  }}

\newcommand{\mtext}[1]{\hspace{6pt}\text{#1}\hspace{6pt}}

\newcommand{\lnorm}{\left\lVert}
\newcommand{\rnorm}{\right\rVert}

\newcommand{\ds}{\displaystyle}
\newcommand{\ts}{\textstyle}


\newcommand{\sfrac}[2]{{}^{#1}\mskip -5mu/\mskip -3mu_{#2}}


\makeatletter
\newcommand*{\da@rightarrow}{\mathchar"0\hexnumber@\symAMSa 4B }
\newcommand*{\da@leftarrow}{\mathchar"0\hexnumber@\symAMSa 4C }
\newcommand*{\xdashrightarrow}[2][]{%
  \mathrel{%
    \mathpalette{\da@xarrow{#1}{#2}{}\da@rightarrow{\,}{}}{}%
  }%
}
\newcommand{\xdashleftarrow}[2][]{%
  \mathrel{%
    \mathpalette{\da@xarrow{#1}{#2}\da@leftarrow{}{}{\,}}{}%
  }%
}
\newcommand*{\da@xarrow}[7]{%
  % #1: below
  % #2: above
  % #3: arrow left
  % #4: arrow right
  % #5: space left 
  % #6: space right
  % #7: math style 
  \sbox0{$\ifx#7\scriptstyle\scriptscriptstyle\else\scriptstyle\fi#5#1#6\m@th$}%
  \sbox2{$\ifx#7\scriptstyle\scriptscriptstyle\else\scriptstyle\fi#5#2#6\m@th$}%
  \sbox4{$#7\dabar@\m@th$}%
  \dimen@=\wd0 %
  \ifdim\wd2 >\dimen@
    \dimen@=\wd2 %   
  \fi
  \count@=2 %
  \def\da@bars{\dabar@\dabar@}%
  \@whiledim\count@\wd4<\dimen@\do{%
    \advance\count@\@ne
    \expandafter\def\expandafter\da@bars\expandafter{%
      \da@bars
      \dabar@ 
    }%
  }%  
  \mathrel{#3}%
  \mathrel{%   
    \mathop{\da@bars}\limits
    \ifx\\#1\\%
    \else
      _{\copy0}%
    \fi
    \ifx\\#2\\%
    \else
      ^{\copy2}%
    \fi
  }%   
  \mathrel{#4}%
}
\makeatother

\renewcommand{\div}[2]{%
  \scalebox{0.92}{$#1$}%
  \hspace{1.2pt}%
  \scalebox{1.2}{$\mid$}%
  \hspace{0.75pt}%
  \scalebox{0.92}{$#2$}%
}


\begin{document}
%This adds a "front cover" page.
%{\thispagestyle{empty}
%\vspace*{\fill}
%\begin{tabular}{l}
%\begin{tabular}{l}
%\includegraphics[scale=0.24]{oxy-logo.png}
%\end{tabular} \\
%\begin{tabular}{l}
%\Large \color{black} Module Theory, Linear Algebra, and Homological Algebra \\ \Large \color{black} Gianluca Crescenzo
%\end{tabular}
%\end{tabular}
%\newpage
    \pagenumbering{roman}
    \tableofcontents

    \chapter*{Preface}
    These are the notes I took for a second semester analysis course. \nl
    
    \noindent \textbf{Urgent Things That I Need To Fix}
    \begin{enumerate}[label = (\arabic*),itemsep=1pt,topsep=3pt]
        \item Any corollary, lemma, proposition, example, or theorem with three asterisks (***) means I don't understand the proof, or there is no proof altogether. There are 18 instances of these throughout the notes.
        \item Turn every "Exercise" into a proposition and do the proofs. Any proposition whose proof is an "exercise" needs to be filled in.
        \item The end of $\S$~\ref{sec:conv-seq}~\nameref{sec:conv-seq} is supposed to include one or two propositions related to the distance of an element $x \in X$ to a set $A \subseteq X$.
        \item $\S$~\ref{sec:cantor-set}~\nameref{sec:cantor-set} is incomplete. 
        \item The notes on applications of meager sets and the \nameref{thm:baires-theorem} need to be included at the end of $\S$~\ref{sec:completeness}~\nameref{sec:completeness}.
        \item $\S$~\ref{sec:top-of-metric-spaces}~\nameref{sec:top-of-metric-spaces}, $\S$~\ref{sec:conv-seq}~\nameref{sec:conv-seq}, and $\S$~\ref{sec:continuity}~\nameref{sec:continuity} are kind of messy. Can't put my finger on why but it doesn't feel as nice as some other parts of these notes.
    \end{enumerate}

    \noindent \textbf{Less Urgent Things That I'm Probably Gonna Do Instead}
    \begin{enumerate}[label = (\arabic*),itemsep=1pt,topsep=3pt]
        \item Clean up $\S$~\ref{sec:completeness}~\nameref{sec:completeness} and $\S$~\ref{sec:compactness}~\nameref{sec:compactness}. Finish $\S$~\ref{sec:connectedness}~\nameref{sec:connectedness}.
        \item Start Chapter 3. First section will be on Riemann Integration, will probably copy most of it from Bartle's \textit{An Introduction to Real Analysis}. Second section will be on measure theory. Third section will be on Lebesgue integration.
    \end{enumerate}

    \noindent \textbf{Things That I'm Really Happy With}
    \begin{enumerate}[label = (\arabic*),itemsep=1pt,topsep=3pt]
        \item Besides some stuff about inner-products, Chapter 1 is almost perfect.
    \end{enumerate}


    \vfill
    \specialdate
    Last update: \today

    \chapter{Vector Spaces, Algebras, and Normed Spaces}
\pagenumbering{arabic}

\noindent For the entirety of this chapter assume $F$ to be  $\bfR$ or $\bfC$.

\section{Vector Spaces}
    \begin{definition}
        A \textit{vector space} (or \textit{linear space}) over $F$ is a nonempty set $V$ equipped with two operations:
            \begin{equation*}
            \begin{split}
                V \times V \xrightarrow{+} V &\mtext{defined by} (v,w) \mapsto v+w \\
                F \times V \rightarrow V &\mtext{defined by} (\alpha,v) \mapsto \alpha v
            \end{split}
            \end{equation*}
        satisfying:
            \begin{enumerate}[label = (\arabic*),itemsep=1pt,topsep=3pt]
                \item $(V, +)$ is an abelian group:
                    \begin{enumerate}[label = (\roman*),itemsep=1pt,topsep=3pt]
                        \item $u + (v+w) = (u+v) + w$ for all $u,v,w \in V$;
                        \item there exists $0_V$ such that $v + 0_V = 0_V + v = v$ for all $v \in V$;
                        \item for all $v \in V$, there exists $w \in V$ satisfying $v+w = w+v = 0_V$;
                        \item $v + w = w+v$ for all $v,w \in V$;
                    \end{enumerate}

                \item $(\alpha + \beta)v = \alpha v + \beta v$ for all $\alpha,\beta \in F$, $v \in V$;
                \item $\alpha(\beta v) = (\alpha \beta)v$ for all $\alpha,\beta \in F$, $v \in V$;
                \item $\alpha(v+w) = \alpha v + \alpha w$ for all $\alpha \in F$, $v,w \in V$;
                \item $1_F v = v$ for all $v \in V$.
            \end{enumerate}
    \end{definition}

    It can be shown that the vector $0_V$ is unique, the additive inverse in (iii) is unique (which we denote as $-v$), that $0v = 0_V$, and $(-1)v = -v$.

    \begin{exercise}
        Show (iv) follows from the other axioms.
    \end{exercise}

    \begin{exercise}
        Show $nv = \underbrace{v + v + ... + v}_{n \hspace{-4pt}\mtext{times}}$ for $n \in \bfZ_{\geq 1}$.
    \end{exercise}

    It can be shown that a subspace is a vector space in its own right.

    \begin{example}
        Let $\{W_i\}_{i \in I}$ be a family of vector spaces. Then $\bigcap_{i \in I}W_i$ is also a vector space.
    \end{example}

    \begin{example}
        Planes and lines through the origin are subspaces of $\bfR^3$.
    \end{example}

    \begin{definition}
        Let $V$ be a vector space and $S \subseteq V$ a subset.
            \begin{enumerate}[label = (\arabic*),itemsep=1pt,topsep=3pt]
                \item A \textit{linear combination} from $S$ is a finite sum $\sum_{j = 1}^n \alpha_j v_j$ with $\alpha_j \in F$, $v_j \in S$.
                \item The \textit{linear span} of $S$ is:
                    \begin{equation*}
                    \begin{split}
                        \Span(S) := \left\{ \sum_{j = 1}^n \alpha_j v_j \hspace{4pt}\middle|\hspace{4pt} n \in \bfN, \alpha_j \in F, v_j \in S \right\}.
                    \end{split}
                    \end{equation*}
            \end{enumerate}
    \end{definition}

    \begin{exercise}
        Show that $\Span(S) \subseteq V$ is a subspace and:
            \begin{equation*}
            \begin{split}
                \Span(S) = \bigcap \hspace{2pt}\{W \mid S \subseteq W, W \hspace{-2pt}\mtext{is a subspace}\hspace{-4pt}\},
            \end{split}
            \end{equation*}
        that is, $\Span(S)$ is the smallest subspace of $V$ containing $S$.
    \end{exercise}

    \begin{definition}
        Let $V$ be a vector space and $S \subseteq V$ a subset.
            \begin{enumerate}[label = (\arabic*),itemsep=1pt,topsep=3pt]
                \item $S$ is \textit{spanning} for $V$ if $\Span(S) = V$.
                \item $S$ is \textit{independent} if, given $n \in \bfN$, $\alpha_1,...,\alpha_n \in F$, $v_1,...,v_n \in S$, then $\sum_{j = 1}^n \alpha_j v_j = 0$ implies $\alpha_j = 0$ for all $j$.
            \end{enumerate}
    \end{definition}

    \begin{center}
        \begin{tikzpicture}
            \draw[thick] (0.3,0) -- (2.3,0);
            \node at (2.39, 0) {$/\,$};
            \node at (2.56, 0) {$/\,$};
            \draw[thick] (2.6,0) -- (4.6,0);
        \end{tikzpicture}
    \end{center}

    Our goal is to show that every vector space admits a basis. As such, recall the following definitions from a standard course in Real Analysis.

    \begin{definition}
        An \textit{ordering} on a set $X$ is a relation $R \subseteq X \times X$ on $X$ that is reflexive, transitive, and antisymmetric. We write $xRy$ as $x \leq_R y$. The pair $(X,\leq_R)$ is called an \textit{ordered set}. An ordering $\leq$ on $X$ is called \textit{total} (or \textit{linear}) if for all $x,y \in X$, $x \leq y$ or $y \leq x$.
    \end{definition}

    Note that if $(X,\leq)$ is an ordered set and $Y \subseteq X$ is a subset, then $(Y,\leq)$ is an ordered set as well.

    \begin{definition}
        Let $(X, \leq)$ be an ordered set and $Y \subseteq X$. An \textit{upper bound} for $Y$ is an element $u \in X$ with $u \geq y$ for all $y \in Y$. An element $m \in X$ is called \textit{maximal} if $x \in X$, $x \geq m$ implies $x = m$.
    \end{definition}

    \begin{lemma}[Zorn's Lemma]
        Let $(X,\leq_X)$ be an ordered set. Suppose every subset $Y \subseteq X$ for which $(Y,\leq_X)$ is totally ordered has an upper bound in $X$. Then $X$ admits a maximal element.
    \end{lemma}

    The proof of Zorn's Lemma is outside the interest of this text. 

    \begin{theorem}
        Every vector space admits a basis. Moreover, every independent set is contained in a basis.
    \end{theorem}
        \begin{proof}
            Let $S \subseteq V$ be linearly independent. Define:
                \begin{equation*}
                \begin{split}
                    \fT(S) = \{T \subseteq V \mid S \subseteq T, T \hspace{-3pt}\mtext{linearly independent}\hspace{-5pt}\}.
                \end{split}
                \end{equation*}
            Let $\fC \subseteq \fT(S)$ be a totally ordered subset. Set $R = \bigcup_{T \in \fC} T$. Clearly $R \supseteq S$. Assume $\sum_{j = 1}^n \alpha_j v_j = 0$, where $\alpha_j \in F$ and $v_j \in R$. Since $\fC$ is totally ordered, there exists $T_0 \in \fC$ with $v_j \in T_0$ for all $j = 1,...,n$. Since $T_0$ is independent, $\alpha_j = 0$ for all $j = 1,...,n$. Thus $R$ is independent as well. Whence $R$ is an upper bound for $\fC$. By Zorn's Lemma, $\fT(S)$ admits a maximal element, call it $B$.

            Claim: $B$ is a basis for $V$. Suppose towards contradiction it's not, then there exists $v_0 \in V \setminus \Span(B)$. Consider $B \cup \{v_0\}$ and let $\alpha_0 v_0 + \sum_{j = 1}^n \alpha_j v_j = 0_V$. If $\alpha_0 \neq 0$, then $\sum_{j = 1}^n \alpha_j v_j = -\alpha_0 v_0 $, giving $v_0 \in \Span(B)$ which is a contradiction. If $\alpha_0 = 0$, then $\sum_{j = 1}^n \alpha_j v_j = 0_V$. Since $B$ is independent, $\alpha_j = 0$ for all $j =1,...,n$. Thus $B \cup \{v_0\}$ is independent, contradicting the maximality of $B$. Whence $B$ is a basis for $V$.
        \end{proof}

    \begin{theorem}
        If $B_1$ and $B_2$ are bases for $V$, then $\card(B_1) = \card(B_2)$.
    \end{theorem}

    \begin{definition}
        If $V$ is a vector space, its \textit{dimension} is the cardinality of any of its bases.
    \end{definition}

    \begin{corollary}
        If $B$ is a basis for $V$, then every $v \in V$ can be written $v = \sum_{j = 1}^n \alpha_k \beta_k$, $\alpha_k \in F$, $b_k \in B$ in a unique way.
    \end{corollary}

    \begin{theorem}
        Let $V$ be a linear space and $B \subseteq V$ a subset. The following are equivalent:
            \begin{enumerate}[label = (\arabic*),itemsep=1pt,topsep=3pt]
                \item $B$ is a basis for $V$;
                \item $B$ is a maximal element in $\fT = \{T \subseteq V \mid \hspace{-4pt}\mtext{$T$ independent}\hspace{-4pt}\}$;
                \item $B$ is a minimal element in $\fS = \{S \subseteq V \mid \hspace{-4pt}\mtext{$S$ spans $V$}\hspace{-4pt}\}$;
            \end{enumerate}
    \end{theorem}

    \begin{definition}
        Let $\{V_i\}_{i \in I}$ be a family of vector spaces over a field $F$.
            \begin{enumerate}[label = (\arabic*),itemsep=1pt,topsep=3pt]
                \item The \textit{product} of $\{V_i\}_{i \in I}$ is denoted:
                    \begin{equation*}
                    \begin{split}
                        \prod_{i \in I}V_i := \{(v_i)_{i \in I} \mid v_i \in V_i\}.
                    \end{split}
                    \end{equation*}
                \item The \textit{co-product} (or \textit{sum}) is denoted 
                    \begin{equation*}
                    \begin{split}
                        \bigoplus_{i \in I}V_i := \left\{(v_i)_{i \in I} \mid v_i \in V_i,\hspace{2pt} \supp\bigl((v_i)_{i \in I}\bigr) <\infty \right\}.
                    \end{split}
                    \end{equation*}
            \end{enumerate}
    \end{definition}
    
    \begin{exercise}
        \phantom{a}
        \begin{enumerate}[label = (\arabic*),itemsep=1pt,topsep=3pt]
            \item Show that $\prod_{i \in I}V_i$ equipped with pointwise operations:
                \begin{equation*}
                \begin{split}
                    (v_i)_{i \in I} + (w_i)_{i \in I} &= (v_i + w_i)_{i \in I} \\
                    \alpha(v_i)_{i \in I} &= (\alpha v_i)_{i \in I}
                \end{split}
                \end{equation*}
            is a linear space.

            \item Show that $\bigoplus_{i \in I}V_i$ is a subspace of $\prod_{i \in I}V_i$.
        \end{enumerate}
    \end{exercise}

    \begin{proposition}
        Let $V$ be a vector space over $F$ and $W \subseteq V$. The (additive, abelian) quotient group $V/W$ can be made into a vector space by defining multiplication by scalars as $\alpha(v + W) = \alpha v + W$ for all $\alpha \in F$, $v + W \in V/W$.
    \end{proposition}

    \begin{example}
        \phantom{a}
        \begin{enumerate}[label = (\arabic*),itemsep=1pt,topsep=3pt]
            \item The set $F^n = \{(x_1,...,x_n) \mid x_j \in F\}$ with component-wise operations is a vector space.
            \item The set $M_{n,m}(F) = \{(a_{ij}) \mid a_{ij} \in F\}$ with linear operations is a vector space.
            \item Let $\Omega$ be a nonempty set. Then $\cF(\Omega,F) = \{f \mid f:\Omega \rightarrow F\}$ with pointwise operations is a vector space.
            \item The set $\ell_\infty(\Omega,F) = \{f \in \cF(\Omega,F) \mid \lnorm f \rnorm _u < \infty \}$ with pointwise operations is a vector space.
                \begin{exercise}
                    Show $\ell_\infty(\Omega,F) \subseteq \cF(\Omega,F)$ is a subspace.
                \end{exercise}

            \item Let $f:[a,b] \rightarrow \bfR$ be any function. Let $\cP = \{a = x_0 < x_1 < ... < x_{n-1} < x_n = b\}$ be a partition of $[a,b]$. The \textit{variation of $f$ on $\cP$} is defined as:
            \begin{equation*}
            \begin{split}
                \text{Var}(f;\cP) = \sum_{k = 1}^n |f(x_k) - f(x_{k-1})|.
            \end{split}
            \end{equation*}
        We say $f$ is a \textit{bounded variation} if:
            \begin{equation*}
            \begin{split}
                \text{Var}(f) := \sup_{\cP}\text{Var}(f;\cP) < \infty.
            \end{split}
            \end{equation*}
        The set of all functions of bounded variation is defined:
            \begin{equation*}
            \begin{split}
                \text{BV}([a,b]) = \{f:[a,b] \rightarrow \bfR \mid \text{Var}(f) < \infty\}.
            \end{split}
            \end{equation*}
        This is a vector space by defining addition and scalar multiplication componentwise.
            \begin{exercise}\label{ex:bv-subspace}
                Show that $\text{BV([a,b])} \subseteq \ell_\infty([a,b],\bfR)$ is a subspace.
            \end{exercise}

            \item Let $K \subseteq V$ be a convex subset of a vector space $V$, that is, for all $v,w \in K$ and $t \in [0,1]$, then $(1-t)v + tw \in K$. A function $f:K \rightarrow F$ is said to be \textit{affine} if $x,y \in K$ and $t \in [0,1]$ implies $f((1-t)x + ty) = (1-t)f(x) + tf(y)$. The set \newline $\text{Aff}(K,F) = \{f \in \cF(\Omega,F) \mid f \hspace{3pt}\text{affine}\hspace{1pt}\}$ with pointwise operations is a vector space.
                \begin{exercise}
                    Show $\text{Aff}(\Omega,F) \subseteq \cF(\Omega,F)$ is a subspace.
                \end{exercise}
            \item The set $C([a,b],F) = \{f:[a,b] \rightarrow F \mid f \hspace{3pt}\text{continuous}\hspace{1pt}\}$ with pointwise operations is a vector space.
                \begin{exercise}
                    Explain why $C([a,b],F) \subseteq \ell_\infty([a,b],F)$ is a subspace.
                \end{exercise}
            \item Consider the following sequence spaces:
                \begin{itemize}
                    \item $s = \{(a_k)_k \mid a_k \in F\} = \cF(\bfN,F)$;
                    \item $\ell_\infty = \ell_\infty(\bfN,F) = \{(a_k)_k \mid \sup_{k \geq 1}|a_k| < \infty\}$;
                    \item $c = \{(a_k)_k \mid (a_k)_k \hspace{3pt}\text{converges}\hspace{2pt}\}$;
                    \item $c_0 = \{(a_k)_k \mid (a_k)_k \rightarrow 0\}$;
                    \item $c_{00} = \{(a_k)_k \mid \supp\bigl((a_k)_k\bigr) < \infty\}$;
                    \item $\ell_1 = \left\{ (a_k)_k \mid \sum_{k = 1}^\infty |a_k| < \infty \right\}$.
                \end{itemize}
            These are all vector spaces with pointwise operations. In fact, $c_{00} \subseteq c_0 \subseteq c \subseteq \ell_\infty \subseteq s$ are all subspaces.
                \begin{exercise}
                    Show that $\ell_1 \subseteq c_0$ is a subspace.
                \end{exercise}

            \item Consider the following continuous function spaces on $\bfR$:
                \begin{itemize}
                    \item $C(\bfR) = \{f:\bfR \rightarrow F \mid f \hspace{3pt}\text{continuous}\hspace{2pt}\}$;
                    \item $C_b(\bfR) = C(\bfR) \cap \ell_\infty(\bfR)$;
                    \item $C_0(\bfR) = \{f \in C(\bfR) \mid \limit_{x \rightarrow \pm\infty} f(x) = 0\}$;
                    \item Recall that a function is \textit{compactly supported} if for all $\epsilon > 0$, there exists $\alpha > 0$ such that $|x| \geq \alpha$ implies $f(x) = 0$. The set of compactly supported functions is denoted $C_c(\bfR) = \{f \in C(\bfR) \mid f \hspace{3pt}\text{compactly supported}\hspace{2pt}\}$.
                \end{itemize}
            These are all vector spaces with pointwise operations, and $C_c(\bfR) \subseteq C_0(\bfR) \subseteq C_b(\bfR) \subseteq C(\bfR)$ are all subspace inclusion.
        \end{enumerate}
    \end{example}

    \begin{definition}
        If $V$ and $W$ are linear spaces over a common field $F$, a map $T:V \rightarrow W$ is called \textit{linear} if $T(v_1 + \alpha v_2) = T(v_1) + \alpha T(v_2)$ for all $v_1,v_2 \in V$ and $\alpha \in F$.
    \end{definition}
    
    \begin{example}
        Let $A \in M_{m,n}(F)$. Then $T_A:F^n \rightarrow F^m$ defined by $T_A(v) = Av$ is linear. Let $\{e_1,...,e_n\}$ be a basis for $F^n$. If $T:F^n \rightarrow F^m$ is linear, set:
            \begin{equation*}
            \begin{split}
                [T] = \Bigl( T(e_1) \Bigm\vert T(e_2) \Bigm\vert \hspace{4.5pt}...\hspace{4.5pt}\Bigm\vert T(e_n) \Bigr).
            \end{split}
            \end{equation*}
        This gives $T(v) = [T]v$ for all $v \in F^n$. In fact, we also have $[T_A] = A$ and $T_{[T]} = T$.
    \end{example}

    \begin{example}
        The \textit{canonical projection} is linear:
            \begin{equation*}
            \begin{split}
                \pi_j: \prod_{i \in I}V_i \rightarrow V_j \mtext{defined by} \pi_j\bigl((v_i)_i\bigr) = v_i.
            \end{split}
            \end{equation*}
        We also have that the \textit{coordinate exclusions} are linear:
            \begin{equation*}
            \begin{split}
                \iota_j: V_j \hookrightarrow \bigoplus_{i \in I}V_i \mtext{defined by} \iota_j(v) = (v_i)_i \hspace{1.5pt}, \hspace{-2pt}\mtext{where} v_i = \begin{cases}
                    0_v, &i \neq j \\
                    v_j, &\text{otherwise}.
                \end{cases}
            \end{split}
            \end{equation*}
        The \textit{evaluation map} is linear as well. For $s \in S$, consider:
            \begin{equation*}
            \begin{split}
                e_s : \cF(S,F) \rightarrow F \mtext{defined by} e_s(f) = f(s).
            \end{split}
            \end{equation*}
    \end{example}

    \begin{proposition}
        Let $V$ be a vector space with basis $B$. Let $W$ be a vector space and suppose $\varphi:B \rightarrow W$ is a map. Then there exists a unique linear map $T_\varphi: V \rightarrow W$ with $T_\varphi(b) = \varphi(b)$ for all $b \in B$. We have the following diagram.
            \begin{center}
            \begin{tikzcd}
            B \arrow[rd, "\varphi"'] \arrow[r, "\iota", hook] & V \arrow[d, "T_\varphi", dotted] \\
                                                              & W                               
            \end{tikzcd}
            \end{center}
    \end{proposition}
        \begin{proof}
            Define $T_\varphi:V \rightarrow W$ by:
                \begin{equation*}
                \begin{split}
                    T_\varphi(v)
                    & = T_\varphi \left( \sum_{j = 1}^n \alpha_j b_j \right) \\
                    & = \sum_{j = 1}^n \alpha_j \varphi(b_j).
                \end{split}
                \end{equation*}
            Let $v_1,v_2 \in V$ and $c \in F$. We have that:
                \begin{equation*}
                \begin{split}
                    T_\varphi(v_1 + cv_2)
                    & = T_\varphi \left( \sum_{j = 1}^n \alpha_j b_j + c \sum_{j = 1}^n \beta_j b_j \right) \\
                    & = T_\varphi \left( \sum_{j = 1}^n (\alpha_j + c \beta_j)b_j \right) \\
                    & = \sum_{j  =1}^n (\alpha_j + c \beta_j)\varphi(b_j) \\
                    & = \sum_{j = 1}^n \alpha_j \varphi(b_j) + c \sum_{j =1}^n \beta_j \varphi(b_j) \\
                    & = T_\varphi(v_1) + c T_\varphi(v_2).
                \end{split}
                \end{equation*}
            Thus $T_\varphi$ is linear. Chasing the above diagram makes it clear that $T_\varphi(b) = \varphi(b)$. It remains to show that $T_\varphi$ is unique. Let $T$ be another linear transformation satisfying $T(b) = \varphi(b)$ for all $b \in B$. Then:
                \begin{equation*}
                \begin{split}
                    T(v)
                    & = T \left( \sum_{j = 1}^n \alpha_j b_j \right) \\
                    & = \sum_{j = 1}^n \alpha_j \varphi(b_j) \\
                    & = T_\varphi \left( \sum_{j = 1}^n \alpha_j b_j \right) \\
                    & = T_\varphi(v).
                \end{split}
                \end{equation*}
            Thus $T_\varphi$ is unique.
        \end{proof}

    \begin{proposition}
        Let $T:V \rightarrow W$ be linear.
            \begin{enumerate}[label = (\arabic*),itemsep=1pt,topsep=3pt]
                \item $\ker(T) = \{v \in V \mid T(v) = 0_W\}$ is a linear subspace of $V$.
                \item $\Image(T) = \{T(v) \mid v \in V\}$ is a linear subspace of $W$.
                \item $\ker(T) = \{0_V\}$ if and only if $T$ is injective.
                \item $\Image(T) = W$ if and only if $T$ is surjective.
            \end{enumerate}
    \end{proposition}
        \begin{proof}
            (1) Let $v_1,v_2 \in \ker(T)$ and $\alpha \in F$. Observe that:
                \begin{equation*}
                \begin{split}
                    T(v_1 + cv_2)
                    & = T(v_1) + cT(v_2) \\
                    & = 0.
                \end{split}
                \end{equation*}
            Thus $v_1 + cv_2 \in \ker(T)$, giving $\ker(T)$ as a linear subspace of $V$.

            (2) Let $w_1,w_2 \in \Image(T)$. Then there exists $v_1.v_2 \in V$ with $T(v_1) = w_1$ and $T(v_2) = w_2$. We have:
                \begin{equation*}
                \begin{split}
                    w_1 + cw_2 
                    & = T(v_1) + cT(v_2) \\
                    & = T(v_1 + cv_2).
                \end{split}
                \end{equation*}
            Whence $w_1 + cw_2 \in \Image(T)$, giving $\Image(T)$ as a linear subspace of $W$.

            (3) Let $\ker(T) = \{0\}$. Suppose $T(v_1) = T(v_2)$. Then $T(v_1) - T(v_2) = T(v_1 - v_2) = 0_W$. It must be that $v_1 -v_2 = 0_W$, giving $v_1 = v_2$. Thus $T$ is injective. Conversely, suppose $T$ is injective and let $v \in \ker(T)$. Then $T(v) = 0_W = T(0_V)$. Hence $v = 0_V$, establishing $\ker(T) = \{0\}$.

            (4) This is by definition of surjectivity.
        \end{proof}

    \begin{proposition}
        If $T:V \rightarrow W$ is linear and bijective, then the inverse map $T^{-1}:W \rightarrow V$ is linear.
    \end{proposition}
        \begin{proof}
            We have that:
                \begin{equation*}
                \begin{split}
                    T(T^{-1}(w_1) + \alpha T^{-1}(w_2)) = w_1 + \alpha w_2 = T \circ T^{-1}(w_1 + \alpha w_2).
                \end{split}
                \end{equation*}
            Applying $T^{-1}$ to both sides gives the desired result.
        \end{proof}

    \begin{proposition}[Vector Spaces are Injective]
        Let $U,V,W$ be vector spaces and $0 \rightarrow U \xrightarrow{j} V$ be exact (that is, $j$ is injective). Let $\varphi:U \rightarrow W$ be linear. There exists a linear map $\Psi:V \rightarrow W$ such that $\varphi = \Psi\circ j $; i.e., the following diagram commutes:
            \begin{center}
                \begin{tikzcd}
                    0 \arrow[r] & U \arrow[r, "j"] \arrow[d, "\varphi"'] & V \arrow[ld, "\Psi", dotted] \\
                                & W                                      &                             
                    \end{tikzcd}
            \end{center}
    \end{proposition}
        \begin{proof}
            Let $\{u_i\}_{i \in I}$ be a basis for $U$. We must first show that $\{j(u_i)\}_{i \in I}$ is linearly independent. Notice that:
                \begin{equation*}
                \begin{split}
                    0_V 
                    & = \sum_{i \in I}\alpha_i j(u_i) \\
                    & = j \left( \sum_{i \in I} \alpha_i u_i \right).
                \end{split}
                \end{equation*}
            By the injectivity of $j$, we have that $\sum_{i \in I}\alpha_i u_i = 0_U$. Thus $\alpha_i = 0$ for all $i \in I$, giving $\{j(u_i)\}_{i \in I}$ as linearly independent.

            Since $\{j(u_i)\}_{i \in I}$ is linearly independent in $V$, we can extend it to a basis \newline $B = \{v_i\}_{i \in J}$ where $I \subseteq J$ and $v_i = j(u_i)$ whenever $i \in I$. Now define $\psi:B\rightarrow W$ by:
                \begin{equation*}
                \begin{split}
                    \psi(v_i) = \begin{cases} \varphi(u_i), & i \in I \\ w, & i \in J\setminus I \h3, \end{cases}
                \end{split}
                \end{equation*}
            where $w \in W$ is arbitrary. Since this is a map of basis elements, there exists a unique linear map $\Psi:V \rightarrow W$ with $\Psi(v_i) = \psi(v_i)$ for all $v_i \in B$. We can finally see that:
                \begin{equation*}
                \begin{split}
                    \varphi(u_i)
                    & = \psi(v_i) \\
                    & = \Psi(v_i) \\
                    & = \Psi(j(u_i)).
                \end{split}
                \end{equation*}
            This establishes that $\varphi = \Psi \circ j$.
        \end{proof}

    \begin{proposition}[Vector Spaces are Projective]
        Let $U,V,W$ be vector spaces and $V \xrightarrow{\pi}U \rightarrow 0$ be exact (that is, $\pi$ is onto). Let $\varphi:W \rightarrow U$ be linear. There exists a linear map $\Psi:V \rightarrow W$ such that $\varphi = \pi \circ \Psi$; i.e., the following diagram commutes:
            \begin{center}
                \begin{tikzcd}
                    & W \arrow[d, "\varphi"] \arrow[ld, "\Psi"', dotted] &   \\
                    V \arrow[r, "\pi"] & U \arrow[r]                                        & 0
                \end{tikzcd}
            \end{center}
    \end{proposition}
        \begin{proof}
            Let $B = \{w_i\}_{i \in I}$ be a basis for $W$. Define $\psi:B \rightarrow V$ by $\psi(w_i) = \pi^{-1}(\varphi(w_i))$. Since this is a map of basis elements, it extends to a unique (dependent on $\pi^{-1}$) linear map $\Psi:W \rightarrow V$ with $\Psi(w_i) = \psi(w_i)$ for all $w_i \in B$. Moreover, we have that:
                \begin{equation*}
                \begin{split}
                    (\pi \circ \Psi)(w_i)
                    & = (\pi \circ \psi)(w_i) \\
                    & = (\pi \circ (\pi^{-1} \circ \varphi))(w_i) \\
                    & = \varphi(w_i). \qedhere
                \end{split}
                \end{equation*}
        \end{proof}

    \begin{definition}
        Let $V$ and $W$ be vector spaces over $F$. A \textit{linear isomorphism} between $V$ and $W$ is a bijective linear map $T:V \rightarrow W$. If such a $T$ exists, we say $V$ and $W$ are \textit{linearly isomorphic}, and write $V \cong W$.
    \end{definition}

    \begin{center}
        \begin{tikzpicture}
            \draw[thick] (0.3,0) -- (2.3,0);
            \node at (2.39, 0) {$/\,$};
            \node at (2.56, 0) {$/\,$};
            \draw[thick] (2.6,0) -- (4.6,0);
        \end{tikzpicture}
    \end{center}

    Finite dimensional vector spaces are boring. This is illustrated through the following theorem.

    \begin{theorem}
        Let $V$ and $W$ be finite-dimensional vector spaces over $F$. Then $V \cong W$ if and only if $\dim(V) = \dim(W)$.
    \end{theorem}
        \begin{proof}
            Suppose $V \cong W$. Then there is an isomorphism taking basis of $V$ to a basis of $W$. Therefore they have the same dimension.

            Conversely, if $\dim(V) = \dim(W) = n$, then they are each isomorphic to $F^n$, giving that they are isomorphic to eachother.
        \end{proof}

        \begin{center}
            \begin{tikzpicture}
                \draw[thick] (0.3,0) -- (2.3,0);
                \node at (2.39, 0) {$/\,$};
                \node at (2.56, 0) {$/\,$};
                \draw[thick] (2.6,0) -- (4.6,0);
            \end{tikzpicture}
        \end{center}

    \begin{example}
        Let $V$ be a vector space, $W \subseteq V$ a subspace. The \textit{natural projection}:
            \begin{equation*}
            \begin{split}
                \pi:V \rightarrow V/W \mtext{defined by}\pi(v) = v+W
            \end{split}
            \end{equation*}
        is a linear surjective map.
    \end{example}

    \begin{theorem}[First Isomorphism Theorem for Vector Spaces]
        Let $T:V \rightarrow V'$ be a linear map and $W \subseteq V$ a subspace.
        \begin{enumerate}[label = (\arabic*),itemsep=1pt,topsep=3pt]
            \item If $T$ "kills" $W$ (that is, $W \subseteq \ker(T)$), then there exists a linear map $\widetilde{T}:V/W \rightarrow V'$ with $\widetilde{T} \circ \pi = T$; i.e., the following diagram commutes.
                \begin{center}
                    \begin{tikzcd}
                        V \arrow[rr, "T"] \arrow[rd, "\pi"'] &                                & V' \\
                        & V/W \arrow[ru, "\widetilde{T}"', dotted] &   
                    \end{tikzcd}
                \end{center}

            \item If $\ker(T) = W$, then $\widetilde{T}$ is injective.
            \item If $\ker(T) = W$ and $\Image(T) = V'$, then $V/W \cong V'$.
        \end{enumerate}
    \end{theorem}
        \begin{proof}
            (1) As stipulated, define $\widetilde{T}(v+W) = T(v)$. We must show that $\widetilde{T}$ is well-defined: suppose $v_1 + W = v_2 + W$ for some $v_1,v_2 \in V$. Then $v_1 = v_2 + w$ for some $w \in W$. This gives:
                \begin{equation*}
                \begin{split}
                    \widetilde{T}(v_1 + W)
                    & = \widetilde{T}(v_2 + w + W) \\
                    & = \widetilde{T}(v_2 + W).
                \end{split}
                \end{equation*}
            Whence $\widetilde{T}$ is well-defined. Now given $v_1 + W, v_2 +W \in V/W$ and $\alpha \in F$, observe that:
                \begin{equation*}
                \begin{split}
                    \widetilde{T}\bigl((v_1 + W) + c (v_2 + W)\bigr)
                    & = \widetilde{T}\bigl( (v_1 + cv_2) + W\bigr) \\
                    & = T(v_1 + cv_2) \\
                    & = T(v_1) + cT(v_2) \\
                    & = \widetilde{T}(v_1 + W) + c \widetilde{T}(v_2 + W). 
                \end{split}
                \end{equation*}
            Thus $\widetilde{T}$ is linear.

            (2) If $\ker(T) = W$, then:
                \begin{equation*}
                \begin{split}
                    \ker(\widetilde{T})
                    & = \{v + W \mid \widetilde{T}(v+W) = 0_{V'}\} \\
                    & = \{v + W \mid T(v) = 0_{V'}\} \\
                    & = \{v + W \mid v \in \ker(T)\} \\
                    & = \{v + W \mid v \in W\} \\
                    & = \{0\}.
                \end{split}
                \end{equation*}
            Thus $\widetilde{T}$ is injective.

            (3) It remains to show that $\Image(T) = V'$ implies $\widetilde{T}$ is surjective. Observe that:
                \begin{equation*}
                \begin{split}
                    \Image(\widetilde{T})
                    & = \{\widetilde{T}(v+W) \mid v+W \in V/W \} \\
                    & = \{\widetilde{T}(\pi(v)) \mid v \in V \} \\
                    & = \{T(v) \mid v \in V\} \\
                    & = \Image(T) \\
                    & = V'.
                \end{split}
                \end{equation*}
            Thus $\widetilde{T}$ is surjective, which establishes it as a bijection. This gives $V/W \cong V'$.
        \end{proof}

    \begin{definition}
        Let $S$ be a nonempty set. The \textit{free vector space} of $S$ is:
            \begin{equation*}
            \begin{split}
                \bfF(S) = \{f:S \rightarrow F \mid \supp(f) <\infty \}.
            \end{split}
            \end{equation*}
    \end{definition}

    \begin{exercise}
        Show $\bfF(S) \subseteq \cF(S,F)$ is a subspace.
    \end{exercise}

    \begin{proposition}
        The set $\{\delta_s \mid s \in S\}$ is a basis for $\bfF(S)$, where $\delta_s :S \rightarrow F$ is defined by:
            \begin{equation*}
            \begin{split}
                \delta_s(t) = \begin{cases}1, & t =0 \\ 0, & \text{otherwise.} \end{cases}
            \end{split}
            \end{equation*}
    \end{proposition}
        \begin{proof}
            If $f \in \bfF(S)$ with $\supp(f) = \{s_1,...,s_n\}$, then $f = \sum_{k =1}^n f(s_k)\delta_{s_k}$. If \newline$\sum_{k = 1}^n \alpha_k \delta_{s_k} = 0$, then for $j = 1,...,n$ we have $0 = \left( \sum_{k  =1}^n \alpha_k \delta_{s_k} \right)(s_j) = \alpha_j$.
        \end{proof}

    \begin{theorem}
        Given any vector space $V$ and a map (of sets) $\varphi:S \rightarrow V$, there exists a unique linear map $T_\varphi : \bfF(S) \rightarrow V$ with $T_\varphi \circ \iota = \varphi$, where $\iota:S \rightarrow \bfF(S)$ is defined by $\iota(s) = \delta_s$ for all $s \in S$. The following diagram commutes:
            \begin{center}
                \begin{tikzcd}
                    S \arrow[r, "\iota"] \arrow[r] \arrow[rd, "\varphi"'] & \bfF(S) \arrow[d, "T_\varphi", dotted] \\
                                                                          & V                                     
                    \end{tikzcd}
            \end{center}
    \end{theorem}
        \begin{proof}
            By the previous proposition, we have that $B = \{\delta_s \mid s \in S\}$ is a basis for $\bfF(S)$. Define $T:B \rightarrow V$ by $T(\delta_s) = \varphi(s)$. Since this is a map of basis elements, there exists a unique linear map $T_\varphi:\bfF(S) \rightarrow V$ with $T_\varphi(\delta_s) = T(\delta_s)$ for all $\delta_s \in B$. The diagram commutes because:
                \begin{equation*}
                \begin{split}
                    \varphi(s)
                    & = T(\delta_s) \\
                    & = T_\varphi(\delta_s) \\
                    & = T_\varphi(\iota(s)).
                \end{split}
                \end{equation*}
            Moreover, if $T'$ satisfies $\varphi = T' \circ \iota$, then:
                \begin{equation*}
                \begin{split}
                    T'(\delta_s)
                    & = T'(\iota(s)) \\
                    & = \varphi(s) \\
                    & = T_\varphi(\iota(s)) \\
                    & = T_\varphi(\delta_s).
                \end{split}
                \end{equation*}
            Thus $T_\varphi$ is unique.
        \end{proof}

    \begin{definition}
        Let $V$ and $W$ be vector spaces. The set of linear transformations between $V$ and $W$ is $\cL(V,W) = \{T \mid T:V \rightarrow W \hspace{3pt}\text{linear}\hspace{2pt}\}$. The set of linear functionals is $V' := \cL(V,F)$.
    \end{definition}

    \begin{exercise}
        Show $\cL(V,W)$ is a vector space.
    \end{exercise}
    
    \begin{exercise}
        Show $M_{m,n}(F) \cong \cL(F^m,F^n)$ by $a \mapsto T_a:(v \mapsto av)$.
    \end{exercise}

\section{Algebras}
    \begin{definition}
        An \textit{algebra} over $F$ is a linear space $A$ over $F$ equipped with a multiplication operation:
            \begin{equation*}
            \begin{split}
                A \times A \rightarrow A \mtext{defined by} (a,b) \mapsto ab
            \end{split}
            \end{equation*}
        satisfying:
            \begin{enumerate}[label = (\arabic*),itemsep=1pt,topsep=3pt]
                \item $(ab)c = a(bc)$ for all $a,b,c \in A$;
                \item $(\alpha a)b = \alpha(ab) = a (\alpha b)$ for all $a,b \in A$, $\alpha \in F$;
                \item $a(b+c) = ab + ac$ for all $a,b,c \in A$;
                \item $(a+b)c = ac + bc$ for all $a,b,c \in A$.
            \end{enumerate}
        If $ab=ba$ for all $a,b \in A$ we say that $A$ is \textit{commutative}. If there exists $1_A \in A$ with $1_A a = a 1_A = a$ for all $a \in A$ we say $A$ is \textit{unital}.
    \end{definition}

    \begin{example}
        \phantom{a}
        \begin{enumerate}[label = (\arabic*),itemsep=1pt,topsep=3pt]
            \item $M_n(F)$ is a noncommutative unital algebra over $F$ under the usual matrix multiplication.
            \item If $V$ is a vector space over $F$, $\cL(V)$ is a unital algebra over $F$. It is noncommutative provided $\dim(V) > 1$.
            \item $\cF(S,F)$ is a unital commutative algebra over $F$.
        \end{enumerate}
    \end{example}

    \begin{definition}
        Let $B$ be a (unital) algebra over $F$.
        \begin{enumerate}[label = (\arabic*),itemsep=1pt,topsep=3pt]
            \item A (unital) \textit{subalgebra} of $B$ is a subspace $A \subseteq B$ ($1_B \in A$) satisfying the property that if $a,a' \in A$, then $aa' \in A$.
            \item An \textit{ideal} of $B$ is a subspace $I \subseteq B$ with $b \in B$, $a \in I$ implying $ba,ab \in I$.
        \end{enumerate} 
    \end{definition}

    \begin{example}
        \phantom{a}
        \begin{enumerate}[label = (\arabic*),itemsep=1pt,topsep=3pt]
            \item $\ell_\infty(\Omega,F) \subseteq \cF(\Omega,F)$ is a unital subalgebra.
            \item $c_{00} \subseteq c_0 \subseteq c \subseteq \ell_\infty \subseteq s$ are all subalgebras. In particular, $c_0 \subseteq \ell_\infty$ and $c_{00} \subseteq s$ are ideals.
            \item $C\bigl([a,b]\bigr) \subseteq \ell_\infty\bigl([a,b]\bigr)$ is a unital subalgebra.
            \item $C_c(\bfR) \subseteq C_0(\bfR) \subseteq C_b(\bfR) \subseteq \ell_\infty(\bfR)$ are all subalgebras. In fact, $C_b(\bfR) \subseteq C(\bfR)$ and $C_b(\bfR) \subseteq \ell_\infty(\bfR)$ are unital, whereas $C_0(\bfR) \subseteq C_b(\bfR)$ and $C_c(\bfR) \subseteq C(\bfR)$ are ideals.
            \item The set $T_n(F) = \{(a_{ij}) \in M_n(F) \mid a_{ij} = 0, i> j\}$ is a unital subalgebra of $M_n(F)$.
        \end{enumerate}
    \end{example}

    \begin{example}[Group Algebra]
        Let $\Gamma$ denote a group (not necessarily abelian). Take the free vector space $\bfF(\Gamma)$ and define multiplication as \textit{convolution}: given $f,g \in \bfF(\Gamma)$ let:
            \begin{equation*}
            \begin{split}
                (f \ast g)(r) = \sum_{
                    \mathclap{\substack{\bigl\{ (s,t) \hspace{2pt} \mid \\ s \in \supp(f), \\ t \in \supp(g), \\ st = r \bigr\}}}} f(s)g(t).
            \end{split}
            \end{equation*}
        Since $\supp(f)$ and $\supp(g)$ are finite, this is a finite sum. We often suppress this notation and write $(f \ast g)(r) = \sum_{st = r}f(s)g(t)$.

        We can also make substitutions:
            \begin{equation*}
            \begin{split}
                (f \ast g)(r)
                & = \sum_{st = r}f(s)g(t) \\
                & = \sum_{t \in \Gamma} f(rt^{-1})g(t) \\
                &  =\sum_{s \in \Gamma}f(s)g(s^{-1}r).
            \end{split}
            \end{equation*}
        It is clear that:
            \begin{equation*}
            \begin{split}
                (f+g)\ast h &= f \ast h + g \ast h \\\
                g \ast (g+h) &= f \ast g + f \ast h \\
                \alpha(f \ast g) &= (\alpha f)\ast g = f \ast (\alpha g)
            \end{split}
            \end{equation*}
        for $f,g,h \in \bfF(\Gamma)$, $\alpha \in F$. Associativity can be similarly shown using the above definition. Rather, we will prove associativity by first show that $\delta_s \ast \delta_t = \delta_{st}$. Given:
            \begin{equation*}
            \begin{split}
                (\delta_s \ast \delta_t)(r) = \sum_{q \in \Gamma} \delta_s(rq^{-1})\delta_t(q),
            \end{split}
            \end{equation*}
        notice that:
            \begin{equation*}
            \begin{split}
                \delta_s(rt^{-1}) = \begin{cases}
                    1, &s = rt^{-1} \\
                    0, &\text{otherwise}
                \end{cases}
                \hspace{5pt}=\hspace{5pt}
                \begin{cases}
                    1,& r = st \\
                    0 ,& \text{otherwise}
                \end{cases}
                \hspace{5pt} = \delta_{st}(r).
            \end{split}
            \end{equation*}
        Since $\{\delta_t \mid t \in \Gamma\}$ is a basis for $\bfF(\Gamma)$, every $f \in \bfF(\Gamma)$ looks like:
            \begin{equation*}
            \begin{split}
                f = \sum_{t \in J}\alpha_t \delta_t , \hspace{4pt}J \subseteq T \hspace{4pt}\text{finite}.
            \end{split}
            \end{equation*}
        Using distributivity we get:
            \begin{equation*}
            \begin{split}
                \delta_r \ast (\delta_s \ast \delta_t)
                & = \delta_r \ast \delta_{st} \\
                & = \delta_{rst} \\
                & = \delta_{rs} \ast \delta_t \\
                & = (\delta_r \ast \delta_s) \ast \delta_t.
            \end{split}
            \end{equation*}
        Whence convolution is associative.
    \end{example}

    \begin{exercise}
        Let $\{A_i\}_{i \in I}$ be a family of algebras over $F$.
        \begin{enumerate}[label = (\arabic*),itemsep=1pt,topsep=3pt]
            \item $\prod_{i \in I}A_i$ is an algebra under $(a_i)_i(b_i)_i = (a_ib_i)_i$.
            \item $\bigoplus_{i \in I}A_i \subseteq \prod_{i \in I}A_i$ is an ideal.
        \end{enumerate}
    \end{exercise}
    \begin{exercise}
        Let $A$ be an algebra over $F$ and $I \subseteq A$ an ideal. Then $A/I$ is an algebra under $(a+I)(b+I) = ab+I$.
    \end{exercise}

\section{Normed Vector Spaces}
    To each vector $v$ in a vector space $V$, we want to assign a "length", denoted $\lnorm v \rnorm$.

    \begin{definition}
        A \textit{norm} on a vector space $V$ is a map:
            \begin{equation*}
            \begin{split}
                \lnorm \cdot \rnorm :V \rightarrow [0,\infty), \h4 v \mapsto \lnorm v \rnorm
            \end{split}
            \end{equation*}
        satisfying:
            \begin{enumerate}[label = (\arabic*),itemsep=1pt,topsep=3pt]
                \item $\lnorm \alpha v \rnorm = |\alpha| \lnorm v \rnorm$ for all $\alpha \in F$, $v \in V$ (homogeneity);
                \item $\lnorm v + w \rnorm \leq \lnorm v \rnorm + \lnorm w \rnorm$ (triangle inequality);
                \item If $\lnorm v \rnorm = 0$, then $v = 0_V$ (positive-definite).
            \end{enumerate}
        If $\lnorm \cdot \rnorm$ satisfies (1) and (2), it is called a \textit{seminorm}. The pair $(V, \lnorm \cdot \rnorm)$ is called a \textit{normed space}.
    \end{definition}

    \begin{definition}
        Two norms $\lnorm \cdot \rnorm$ and $\lnorm \cdot \rnorm'$ on a vector space $V$ are called \textit{equivalent} if there exists $c_1 \geq 0$ and $c_2 \geq 0$ with $\lnorm v \rnorm \leq c_1 \lnorm v \rnorm'$ and $\lnorm v \rnorm' \leq c_2 \lnorm v \rnorm$ for all $v \in V$.
    \end{definition}

    \begin{exercise}
        If $p$ is a seminorm on $V$, show that $|p(v) - p(w)| \leq p(v-w)$.
    \end{exercise}

    \begin{definition}
        Let $(V, \lnorm v \rnorm)$ be a normed space.
            \begin{enumerate}[label = (\arabic*),itemsep=1pt,topsep=3pt]
                \item The \textit{closed unit ball} is denoted $B_V = \{v \in V \mid \lnorm v \rnorm \leq 1\}$.
                \item The \textit{open unit ball} is denoted $U_V = \{v \in V \mid \lnorm v \rnorm < 1\}$.
                \item The \textit{unit sphere} is denoted $S_V = \{v \in V \mid \lnorm v \rnorm = 1\}$. 
            \end{enumerate}
    \end{definition}

    \begin{example}
        Let $V = F^n$ and $x = (x_1,...,x_n)$. We define:
            \begin{equation*}
            \begin{split}
                \lnorm x \rnorm_1 &= \sum_{j  =1}^n |x_j|; \\
                \lnorm x \rnorm_\infty & = \max_{j=1}^n |x_j|; \\
                \lnorm x \rnorm_2 & = \left( \sum_{j = 1}^n |x_j|^2 \right)^{\frac{1}{2}}.
            \end{split}
            \end{equation*}
        For $p \geq 1$:
            \begin{equation*}
            \begin{split}
                \lnorm x \rnorm_p & = \left( \sum_{j = 1}^n |x_j|^p \right)^\frac{1}{p}.
            \end{split}
            \end{equation*}
    \end{example}

    \begin{exercise}
        Show that $\lnorm \cdot \rnorm_1$ and $\lnorm \cdot \rnorm_\infty$ are norms 
    \end{exercise}

    \begin{center}
        \begin{tikzpicture}
            \draw[thick] (0.3,0) -- (2.3,0);
            \node at (2.39, 0) {$/\,$};
            \node at (2.56, 0) {$/\,$};
            \draw[thick] (2.6,0) -- (4.6,0);
        \end{tikzpicture}
    \end{center}

    We aim to show that $\lnorm \cdot \rnorm_p$ is a norm for $p \in [0,\infty]$.

    \begin{lemma}\label{lemma:lemnorm1}
        Let $p,q \in [1,\infty]$ with $\frac{1}{p} + \frac{1}{q} = 1$. Let $f:[0,\infty) \rightarrow \bfR$ be defined by $f(t) = \frac{1}{p}t^p - t + \frac{1}{q}$. Then $f(t) \geq 0$ for $t \geq 0$.
    \end{lemma}
        \begin{proof}
            Note that $f'(t) = t^{p-1} - 1$. Since:
                \begin{equation*}
                \begin{split}
                    f'(1) &= 0 \\
                    f'(t) &> 0 \mtext{for} t>1 \\
                    f'(t) &<0 \mtext{for} 0\leq t < 1,
                \end{split}
                \end{equation*}
            we can see that $f(t) \geq 0$ for all $t \geq 0$.
        \end{proof}

    \begin{lemma}[Young's Inequality]
        Let $p,q \in [0,\infty]$ with $\frac{1}{p} + \frac{1}{q} = 1$. If $x,y \geq 0$, then $xy \leq \frac{1}{p}x^p + \frac{1}{q}y^q$. 
    \end{lemma}
        \begin{proof}
            By Lemma~\ref{lemma:lemnorm1}, $t \leq \frac{1}{p}t^p + \frac{1}{q}$. Multiplying both sides by $y^q$ gives:
                \begin{equation*}
                \begin{split}
                    ty^q \leq \frac{1}{p}t^p y^q + \frac{1}{q}y^q.
                \end{split}
                \end{equation*}
            Let $t = xy^{1-q}$. Then:
                \begin{equation*}
                \begin{split}
                    xy^{1-q}y^q \leq \frac{1}{p}x^p y^{p-pq}y^q + \frac{1}{q}y^q.
                \end{split}
                \end{equation*}
            Since $\frac{1}{p} + \frac{1}{q} = 1$, we have that $p-pq = -q$. Whence:
                \begin{equation*}
                \begin{split}
                    xy \leq \frac{1}{p}x^p + \frac{1}{q}y^q.
                \end{split}
                \end{equation*}
        \end{proof}

    \begin{lemma}[H\"olders Inequality]\label{lemma:holders}
        Let $p,q \in [0,\infty]$ with $\frac{1}{p} + \frac{1}{q} = 1$. Then for $x,y \in F^n$:
            \begin{equation*}
            \begin{split}
                \left|\sum_{j = 1}^n x_j y_j\right| \leq \lVert x \rVert _p \lVert y \rVert _q.
            \end{split}
            \end{equation*}
    \end{lemma}
        \begin{proof}
            We proceed by cases. \nl
            
            Case 1: $p=1$. Then:
                \begin{equation*}
                \begin{split}
                    \left|\sum_{j = 1}^n x_j y_j\right|
                    & \leq \sum_{j = 1}^n |x_j||y_j| \\
                    & \leq \sum_{j = 1}^n |x_j| \lVert y \rVert _\infty \\
                    & = \lVert x \rVert _1 \lVert y \rVert _\infty.
                \end{split}
                \end{equation*}

            Case 2: $p = \infty$. This follows similarly to Case 1. \nl
            
            Case 3: $1 < p < \infty$. Suppose $\lVert x \rVert _p = \lVert y \rVert _q =1 $. Then:
                \begin{equation*}
                \begin{split}
                    \left|\sum_{j = 1}^n x_j y_j\right|
                    & \leq \sum_{j = 1}^n |x_j||y_j| \\
                    & \leq \sum_{j =1}^n \left(\frac{1}{p}|x_j|^p + \frac{1}{q}|y_j|^q\right) \\
                    & = \frac{1}{p} \left(\sum_{j = 1}^n |x_j|^p\right) + \frac{1}{q} \left(\sum_{j = 1}^n|y_j|^q\right) \\
                    & = \frac{1}{p} + \frac{1}{q} \\
                    & = 1.
                \end{split}
                \end{equation*}
            Whence the inequality holds. Now suppose $\lVert x \rVert _p = 0$ or $\lVert y \rVert _q = 0$. Then $x = 0_{F^n}$ or $y = 0_{F^n}$, whence the inequality holds. Suppose $\lVert x \rVert _p \neq 0$ and $\lVert y \rVert _p \neq 0$. Set:
                \begin{equation*}
                \begin{split}
                    x' = \frac{x}{\lVert x \rVert _p} \\
                    y' = \frac{y}{ \lVert y \rVert _p}.
                \end{split}
                \end{equation*}
            Then $\lVert x' \rVert _p = 1 = \lVert y' \rVert _p$. Observe that:
                \begin{equation*}
                \begin{split}
                    1 &\geq \left|\sum_{j = 1}^n x_j' y_j'\right| \\
                    & = \left|\sum_{j = 1}^n \frac{x}{\lVert x \rVert _p}\frac{y}{ \lVert y \rVert _p}\right|.
                \end{split}
                \end{equation*}
            Multiplying both sides by $\lVert x \rVert _p \lVert y \rVert _q$ gives the desired result.
        \end{proof}

    \begin{lemma}[Minkowski's Inequality]\label{lemma:minkowski}
        Let $p,q \in [1,\infty]$ with $\frac{1}{p} + \frac{1}{q} = 1$. For $x,y \in F^n$:
            \begin{equation*}
            \begin{split}
                \lVert x + y \rVert _p \leq \lVert x \rVert _p + \lVert y \rVert _p.
            \end{split}
            \end{equation*}
    \end{lemma}
        \begin{proof}
            The only nontrivial case is for $1 < p < \infty$. Observe that:
                \begin{equation*}
                \begin{split}
                    \left(\lVert x + y \rVert _p\right)^p 
                    & = \sum_{j = 1}^n |x_j + y_j|^p \\
                    & = \sum_{j = 1}^n |x_j + y_j| |x_j + y_j|^{p-1} \\
                    & \leq \sum_{j = 1}^n |x_j||x_j + y_j|^{p-1} + 
                    |y_j||x_j + y_j|^{p-1} \\
                    & \leq \left(\sum_{j=1}^n |x_j|^p\right)^\frac{1}{p} \left(\sum_{j = 1}^n|x_j + y_j|^{(p-1)q}\right)^\frac{1}{q} + \left(\sum_{j = 1}^n |y_j|^p\right)^\frac{1}{p} \left(\sum_{j = 1}^n|x_j + y_j|^{(p-1)q}\right)^\frac{1}{q} \\
                    & = \left(\sum_{j=1}^n |x_j|^p\right)^\frac{1}{p} \left(\sum_{j = 1}^n|x_j + y_j|^{p-1 \left(\frac{p}{p-1}\right)}\right)^{1-\frac{1}{p}} + \left(\sum_{j = 1}^n |y_j|^p\right)^\frac{1}{p} \left(\sum_{j = 1}^n|x_j + y_j|^{p-1 \left(\frac{p}{p-1}\right)}\right)^{1-\frac{1}{p}} \\
                    & = \left(\sum_{j=1}^n |x_j|^p\right)^\frac{1}{p} \left(\sum_{j = 1}^n|x_j + y_j|^{p}\right)^{1-\frac{1}{p}} + \left(\sum_{j = 1}^n |y_j|^p\right)^\frac{1}{p} \left(\sum_{j = 1}^n|x_j + y_j|^{p}\right)^{1-\frac{1}{p}} \\
                    & = (\lVert x \rVert _p + \lVert y \rVert _p) \frac{\lVert x+y \rVert _p ^p}{\lVert x+y \rVert _p}.
                \end{split}
                \end{equation*}
            Multiplying boths sides by $\frac{\lVert x+y \rVert _p }{\lVert x+y \rVert _p ^p}$ gives the desired inequality.
        \end{proof}

    \begin{theorem}
        Let $V = F^n$. Then $(F^n, \lVert \cdot \rVert _p)$ is a normed space.
    \end{theorem}
        \begin{proof}
            Let $x = (x_1,...,x_n) \in F^n$ and $\alpha \in F$. Observe that:
                \begin{equation*}
                \begin{split}
                    \lVert  \alpha x \rVert _p
                    & = \left(\sum_{j = 1}^n |\alpha x_j|^p\right)^\frac{1}{p} \\
                    & = \left(\sum_{j = 1}^n |\alpha|^p |x_j|^p\right)^\frac{1}{p} \\
                    & = |\alpha| \lVert x \rVert _p.
                \end{split}
                \end{equation*}
            This satisfies homogeneity. Moreover, \nameref{lemma:minkowski} satisifes the triangle inequality. It remains to show that $\lVert \cdot \rVert _p$ is positive-definite. If $\lVert x \rVert _p = 0$, then $x_j = 0$ for all $1 \leq j \leq n$. Thus $x = 0_{F^n}$.
        \end{proof}

    \begin{corollary}
        Let $p \in [1,\infty]$. Then $\ell_p = \bigl\{(a_k)_k \mid \sum_{k = 1}^\infty |a_k|^p < \infty \bigr\}$ with norm $\lnorm (a_k)_k \rnorm_p = \left( \sum_{k = 1}^\infty |a_k|^p \right)^\frac{1}{p}$ is a normed space.
    \end{corollary}
        \begin{proof}
            Homogeneity and positive-definiteness are trivial to prove. Let $(x_k)_k,(y_k)_k \in \ell_p$. It is clear that:
            \begin{equation*}
                \begin{split}
                    \left(\sum_{k  =1}^n|x_k + y_k|^p\right)^{\frac{1}{p}}
                    & \leq \left(\sum_{k = 1}^n |x_k|^p\right)^\frac{1}{p} + \left(\sum_{k  =1}^n |y_k|^p\right)^\frac{1}{p} \\
                    & \leq \left(\sum_{k= 1}^\infty |x_k|^p\right)^\frac{1}{p} + \left(\sum_{k  =1}^\infty |y_k|^p\right)^\frac{1}{p} \\
                    & = \lnorm (x_k)_k \rnorm_p + \lnorm (y_k)_k \rnorm_p.
                \end{split}
                \end{equation*}
            We have that $\sum_{k = 1}^n |x_k + y_k|^p$ is increasing and bounded above by $(\lnorm (x_k)_k \rnorm_p + \lnorm (y_k)_k \rnorm_p)^p$. By the Monotone Convergence Theorem $\limit_{n \rightarrow \infty}\sum_{k = 1}^n |x_k + y_k|^p = \sum_{k = 1}^\infty |x_k + y_k|^p$ exists. Whence $\left( \sum_{k = 1}^\infty |x_k + y_k|^p \right)^\frac{1}{p}  = \lnorm (x_k)_k + (y_k)_k \rnorm_p \leq \lnorm (x_k)_k \rnorm_p + \lnorm (y_k)_k \rnorm_p$
        \end{proof}

    \begin{theorem}\label{thm:norms-equivalent}
        All $p$-norms on $\ell_p^n$ are equivalent for $1 \leq p \leq \infty$.
    \end{theorem}
        \begin{proof}
            Let $x \in \ell_p^n$. We have that:
                \begin{equation*}
                \begin{split}
                    &\lnorm x \rnorm_p = \left( \sum_{i = 1}^n |x_i|^p \right)^\frac{1}{p} 
                        \leq \left( \sum_{i = 1}^n \left( \max_{i = 1}^n |x_i| \right)^p \right)^\frac{1}{p} 
                        = \left( \sum_{i = 1}^n \lnorm x \rnorm_\infty ^p \right)^\frac{1}{p} 
                        = n^p \lnorm x \rnorm_\infty. \\
                    &\phantom{a} \\
                    &\lnorm x \rnorm_\infty = \left( \left( \max_{i = 1}^n |x_i| \right)^p \right)^\frac{1}{p} \leq \left( \sum_{i = 1}^n |x_i|^p \right)^\frac{1}{p} = \lnorm x \rnorm_p. \\
                    &\phantom{a} \\
                    &\lnorm x \rnorm_\infty = \max_{i = 1}^n |x_i| \leq \sum_{i = 1}^n |x_i| = \lnorm x_i \rnorm_1. \\
                    &\phantom{a} \\
                    &\lnorm x \rnorm_1 = \sum_{i = 1}^n |x_i| \leq \sum_{i = 1}^n \max_{i =1}^n |x_i| = n \max_{i = 1}^n |x_i| = n \lnorm x \rnorm_\infty.
                \end{split}
                \end{equation*}

                
            By transitivity, all norms on $\ell_p^n$ are equivalent.
        \end{proof}

    \begin{center}
        \begin{tikzpicture}
            \draw[thick] (0.3,0) -- (2.3,0);
            \node at (2.39, 0) {$/\,$};
            \node at (2.56, 0) {$/\,$};
            \draw[thick] (2.6,0) -- (4.6,0);
        \end{tikzpicture}
    \end{center}


    \begin{example}
        \phantom{a}
        \begin{enumerate}[label = (\arabic*),itemsep=1pt,topsep=3pt]
            \item $\bigl(\ell_\infty(\Omega,F), \lnorm \cdot \rnorm_u\bigr)$ where $\lnorm f \rnorm_u = \sup_{x \in \Omega}|f(x)|$ is a normed space. This includes its subspaces, such as $C \bigl([a,b],F\bigr) \subseteq \ell_\infty \bigl([a,b],F \bigr)$ and $C_c(\bfR) \subseteq C_0(\bfR) \subseteq \ell_\infty(\bfR)$, all with $\lnorm \cdot \rnorm_u$.
            \item Take $\Omega = \bfN$ in the previous example. Then $\bigl(\ell_\infty, \lnorm \cdot \rnorm_\infty\bigr)$ is a normed space. This includes its subspaces $c_{00} \subseteq c_0 \subseteq \ell_\infty$ with $\lnorm \cdot \rnorm_\infty$.
            \item $\bigl(\ell_1,\lnorm \cdot \rnorm_1\bigr)$ is a normed space.
            \item $\bigl(C\bigl([a,b]\bigr), \lnorm \cdot \rnorm_1\bigr)$ with $\lnorm f \rnorm_1 = \int_a^b |f(t)|dt$ is a normed space.
            \item $\bigl( \text{BV}\bigl([a,b]\bigr), \lnorm \cdot \rnorm_{\text{BV}}\bigr)$ where $\lnorm f \rnorm_{\text{BV}} = |f(a)| + \text{Var}(f)$ is a normed space.
            \item Let $(V,\lnorm \cdot \rnorm_V)$ and $(W,\lnorm \cdot \rnorm_W)$ be normed spaces. Then $\bigl(B(V,W),\lnorm \cdot \rnorm_\text{op}\bigr)$ is a normed space, where $B(V,W) = \{T \in \cL(V,W) \mid \lnorm T \rnorm_\text{op} < \infty\}$ is the set of bounded linear maps and $\lnorm T \rnorm_\text{op} = \sup_{v \in B_V}\lnorm T(v) \rnorm_W$. Intuitively, $\lnorm T \rnorm_\text{op}$ measures the radius of the smallest ball which contains $B_V$.
                \begin{exercise}
                    Show that $V^\ast := B(V,F)$ is a subspace of $V'$.
                \end{exercise}
            \item Let $S$ be a nonempty set. Both $\bigl( \bfF(S),\lnorm \cdot \rnorm_1\bigr)$ and $\bigl( \bfF(S),\lnorm \cdot \rnorm_p\bigr)$ are normed spaces, where $\lnorm f \rnorm_1 = \sum_{s \in S}|f(s)|$ and $\lnorm f \rnorm_p = \left( \sum_{s \in S}|f(s)|^p \right)^\frac{1}{p}$. Note that since $f(s) \neq 0$ for finitely many $s \in S$, both $\lnorm \cdot \rnorm_1$ and $\lnorm \cdot \rnorm_p$ are well-defined.
                \begin{exercise}
                    Show that $\lnorm f \rnorm_\infty := \sup_{s \in S}|f(s)|$ is a norm on $\bfF(S)$.
                \end{exercise}
        \end{enumerate}
    \end{example}
\newpage
\section{Inner Product Spaces}

    \begin{definition}
        Let $V$ be a vector space over $F$ and $\varphi:V \times V \rightarrow F$ a map.
            \begin{enumerate}[label = (\arabic*),itemsep=0pt,topsep=3pt]
                \item The map $\varphi$ is said to be a \textit{bilinear form} if is is linear in the first and second variable seperately; i.e., for all $v_1,v_2,v \in V$ and $c \in F$ we have:
                    \begin{enumerate}[label = (\roman*),itemsep=1pt,topsep=0pt]
                        \item $\varphi(cv_1 +v_2,v) = c\varphi(v_1,v) + \varphi(v_2,v)$
                        \item $\varphi(v,cv_1 + v_2) = c\varphi(v,v_1) + \varphi(v,v_2)$.
                    \end{enumerate}

                \item The map $\varphi$ is said to be a \textit{sesquilinear form} if it is linear in the first variable and conjugate linear in the second variable; i.e., for all $v_1,v_2,v \in V$ and $c \in F$ we have:
                    \begin{enumerate}[label = (\roman*),itemsep=1pt,topsep=0pt]
                        \item $\varphi(cv_1 +v_2,v) = c\varphi(v_1,v) + \varphi(v_2,v)$
                        \item $\varphi(v,cv_1 + v_2) = \overline{c}\varphi(v,v_1) + \varphi(v,v_2)$.
                    \end{enumerate}
            \end{enumerate}
        If we wish to keep track of a bilinear form on $V$ we write $(V,\varphi)$.
    \end{definition}

    \begin{definition}
        Let $V$ be a vector space over $F$.
            \begin{enumerate}[label = (\arabic*),itemsep=1pt,topsep=3pt]
                \item A bilinear form $\varphi$ on $V$ is said to be \textit{symmetric} if $\varphi(v,w) = \varphi(w,v)$ for all $v,w \in V$.
                \item A sesquilinear form $\varphi$ on $V$ is said to be \textit{Hermitian} if $\varphi(v,w) = \overline{\varphi(w,v)}$ for all $v,w \in V$.
            \end{enumerate}
    \end{definition}

    \begin{definition}
        Let $(V,\varphi)$ be a vector space over $F$ such that if $\varphi$ is symmetric, then $F = \bfR$ or if $\varphi$ is Hermitian, then $F = \bfC$. We say $\varphi$ is \textit{positive-definite} if for all nonzero $v \in V$ we have $\varphi(v,v) \neq 0$.
    \end{definition}

    \begin{definition}


        Let $(V,\varphi)$ be a vector space over $\bfR$ with $\varphi$ a positive-definite symmetric bilinear form or over $\bfC$ with $\varphi$ a positive-definite Hermitian sesquilinear form. Then we say $\varphi$ is an \textit{inner product} on $V$ and write $\varphi$ as $\langle \cdot,\cdot \rangle$. We say $(V,\langle \cdot,\cdot \rangle)$ is an \textit{inner product space}.
    \end{definition}

    \begin{center}
        \begin{tikzpicture}
            \draw[thick] (0.3,0) -- (2.3,0);
            \node at (2.39, 0) {$/\,$};
            \node at (2.56, 0) {$/\,$};
            \draw[thick] (2.6,0) -- (4.6,0);
        \end{tikzpicture}
    \end{center}
    
    \begin{definition}
        If $V$ is an inner product space we define $\lnorm v \rnorm_2 = \langle v,v \rangle^\frac{1}{2}$.
    \end{definition}

    \begin{definition}
        Let $V$ be an inner product space. Two vectors $v,w \in V$ are \textit{orthogonal} if $\langle u,v \rangle = 0$. We denote this as $u \bot v$.
    \end{definition}

    \begin{theorem}[Pythagorean Theorem]
        Let $v_1,...,v_n$ be mutually orthogonal. Then $\sum_{j = 1}^n \lnorm v_j \rnorm_2^2 = \lnorm \sum_{j = 1}^n v_j \rnorm_2^2$.
    \end{theorem}
        \begin{proof}
            Because $v_i \bot v_j$ for $1 \leq i,j \leq n$, we have $\langle v_i,v_j \rangle = 0$. Observe that:
                \begin{equation*}
                \begin{split}
                    \lnorm \sum_{j = 1}^n v_j \rnorm_2^2
                    & = \Biggl< \sum_{j = 1}^n v_j , \sum_{j = 1}^n v_j \Biggr> \\
                    & = \sum_{j = 1}^n \left( \sum_{i = 1}^n \langle v_j,v_i \rangle \right) \\
                    & = \sum_{j =1}^n \langle v_j,v_j \rangle \\
                    & = \sum_{j = 1}^n \lnorm v_j \rnorm_2^2.
                \end{split}
                \end{equation*}
        \end{proof}
    
    \begin{definition}
        Let $V$ be an inner product space and $w\in V$ nonzero. The \textit{projection} of a vector $v\in V$ onto $w$ is a map $P_W:V \rightarrow V$ defined by $P_W(v) = \frac{\langle v,w \rangle}{\langle w,w \rangle}w$.
        \begin{center}
            \begin{tikzpicture}[>=stealth,scale=0.8]
                % 1) Main horizontal vector w (length = 6)
                \draw[->, thick] (0,0) -- (6,0)
                     node[pos=1, above] {$\scriptstyle w$};
                % 2) Vector v: 20° angle, length = 4, black
                \draw[->, thick, black] (0,0) -- (20:4)
                     node[midway, above] {$\scriptstyle v$};
                % 3) Projection of v onto w, in blue (a bit thicker)
                %    Projected length = 4*cos(20°), purely along x-axis
                \draw[->, very thick, blue] (0,0) -- ({4*cos(20)}, 0)
                     node[midway, below] {$\scriptstyle P_W(v)$};
                % 4) Vector from the tip of P_W(v) to the tip of v (v - P_W(v)), in red
                \draw[->, thick, red] ({4*cos(20)}, 0) -- (20:4)
                     node[midway, right] {$\scriptstyle v - P_W(v)$};
            \end{tikzpicture}
        \end{center}
    \end{definition}

    \begin{proposition}***
        Let $V$ be an inner product space and $w \in V$ a nonzero vector. Then $P_w(v) \bot v - P_w(v)$.
    \end{proposition}
        \begin{proof}
            
        \end{proof}

    \begin{corollary}***
        Let $V$ be an inner product space and $w \in W$ a nonzero vector. Then $\lnorm v \rnorm_2^2 = \lnorm P_w(v) \rnorm_2^2 + \lnorm v - P_w(v) \rnorm_2^2$.
    \end{corollary}
        \begin{proof}
            
        \end{proof}

    \begin{lemma}[Cauchy-Schwartz Inequality]\label{lemma:c-s-ineq}
        Let $V$ be an inner product space and $v,w \in V$. Then $|\langle v,w \rangle| \leq \lnorm v \rnorm_2 \lnorm w \rnorm_2$.
    \end{lemma}
        \begin{proof}
            The previous corollary gives $\lnorm v \rnorm_2 \geq \lnorm P_w(v) \rnorm_2$. We have that:
                \begin{equation*}
                \begin{split}
                    \lnorm v \rnorm_2 
                    & \geq \lnorm \frac{\langle v,w \rangle}{\langle w,w \rangle}w \rnorm_2 \\
                    & = \frac{|\langle v,w \rangle|}{\lnorm w \rnorm_2^2}\lnorm w \rnorm_2 \\
                    & = \frac{\langle v,w \rangle}{\lnorm w \rnorm_2}.
                \end{split}
                \end{equation*}
            Multiplying both sides by $\lnorm w \rnorm_2$ gives the desired result.
        \end{proof}

    \begin{theorem}
        Let $V$ be an inner product space. Then $(V,\lnorm \cdot \rnorm_2)$ is a normed space.
    \end{theorem}
        \begin{proof}
            Let $v,w \in V$ and $\alpha \in F$. We have that:
                \begin{equation*}
                \begin{split}
                    \lnorm \alpha v \rnorm_2 
                    & = \langle \alpha v , \alpha v \rangle^\frac{1}{2} \\
                    & = \left( \alpha \overline{\alpha}\langle v,v \rangle \right)^\frac{1}{2} \\
                    & = \left( |\alpha|^2 \langle v,v \rangle\right)^\frac{1}{2}\\
                    & = |\alpha|\lnorm v \rnorm_2.
                \end{split}
                \end{equation*}
            Thus $\lnorm \cdot \rnorm_2$ satisfies homogeneity. It follows from the \nameref{lemma:c-s-ineq} that:
                \begin{equation*}
                \begin{split}
                    \lnorm v+w \rnorm_2^2
                    & = \langle v+w,v+w \rangle \\
                    & = \langle v,v \rangle + \langle v,w \rangle + \langle w,v \rangle + \langle w,w \rangle \\
                    & = \lnorm v \rnorm_2^2 + \langle v,w \rangle + \overline{\langle v,w \rangle} + \lnorm w \rnorm_2^2 \\
                    & = \lnorm v \rnorm_2^2 + 2\Re \left( \langle v,w \rangle \right) + \lnorm w \rnorm_2^2 \\
                    & \leq \lnorm v \rnorm_2^2 + 2|\langle v,w \rangle| + \lnorm w \rnorm_2^2 \\
                    & \leq \lnorm v \rnorm_2^2 + 2 \lnorm v \rnorm_2 \lnorm w \rnorm_2 + \lnorm w \rnorm_2^2 \\
                    & = \left( \lnorm v \rnorm_2 + \lnorm w \rnorm_2 \right)^2,
                \end{split}
                \end{equation*}
            where we used the fact that $2\Re (\langle v,w \rangle) = 2|\langle v,w \rangle|$. Squaring both sides proves that $\lnorm \cdot \rnorm_2$ satisfies the triangle inequality. It remains to show positive-definiteness. Suppose $\lnorm v \rnorm_2 = 0$. Then $\langle v,v \rangle = 0$, but since the inner-product is by definition positive-definite, we get that $v = 0_V$.
        \end{proof}

        \begin{center}
            \begin{tikzpicture}
                \draw[thick] (0.3,0) -- (2.3,0);
                \node at (2.39, 0) {$/\,$};
                \node at (2.56, 0) {$/\,$};
                \draw[thick] (2.6,0) -- (4.6,0);
            \end{tikzpicture}
        \end{center}

    \begin{example}
        \phantom{a}
        \begin{enumerate}[label = (\arabic*),itemsep=1pt,topsep=3pt]
            \item $\ell_2^n = F^n$ is an inner product space where $\langle (x_1,...,x_n),(y_1,...,y_n) \rangle := \sum_{j = 1}^n x_j \overline{y_j}$.
            \item $\ell_2$ is an inner product space where $\langle (a_k)_k,(b_k)_k \rangle := \sum_{k = 1}^\infty a_k \overline{b_k}$. Note that:
                \begin{equation*}
                \begin{split}
                    \sum_{k = 1}^n \left|a_k \overline{b_k}\right| 
                    & = \sum_{k = 1}^n |a_k||b_k| \\
                    & \leq \left( \sum_{k = 1}^n |a_k|^2 \right)^\frac{1}{2} \left( \sum_{k = 1}^n |b_k|^2 \right)^\frac{1}{2} \\
                    & \leq \left( \sum_{k = 1}^\infty |a_k|^2 \right)^\frac{1}{2} \left( \sum_{k = 1}^\infty |b_k|^2 \right)^\frac{1}{2} \\
                    & = \lnorm (a_k)_k \rnorm_2 \lnorm (b_k)_k \rnorm_2 \\
                    & < \infty \quad\quad\quad \text{\tiny (Because $(a_k)_k,(b_k)_k \in \ell_2)$.}
                \end{split}
                \end{equation*}
            Since $\sum_{k = 1}^n |a_k \overline{b_k}|$ is increasing and bounded above, the Monotone Convergence Theorem says $\sum_{k = 1}^\infty|a_k \overline{b_k}|$ exists and is finite. Whence $\langle (a_k)_k,(b_k)_k \rangle$ converges.

            \item Recall that $\Tr:M_n(\bfC) \rightarrow \bfC$ is defined by $\Tr(a_{ij}) = \sum_{i  =1}^n a_{ii}$.
            Then $M_n(\bfC)$ is an inner product space where $\langle a_{ij},b_{ij} \rangle := \Tr(b_{ij}^\ast a_{ij})$.

            \item $C([0,1])$ is an inner product space where $\langle f,g \rangle := \int_0^1 f(x) \overline{g(x)}dx$.
        \end{enumerate}
    \end{example}

    

\section{Normed Algebras}
    \begin{definition}
        A \textit{normed algebra} is an algebra $A$ equipped with a norm $\lnorm \cdot \rnorm_A$ such that $\lnorm ab \rnorm_A \leq \lnorm a \rnorm_A \lnorm b \rnorm_A$. If $A$ is unital, we require $\lnorm 1 \rnorm_A = 1$.
    \end{definition}

    \begin{example}
        \phantom{a}
        \begin{enumerate}[label = (\arabic*),itemsep=1pt,topsep=3pt]
            \item $\ell_\infty(\Omega)$ equipped with $\lnorm \cdot \rnorm_u$ is a normed algebra.
            \item $C_c(\bfR)$, $C_0(\bfR)$, and $C([0,1])$ are all normed algebras when equipped with $\lnorm \cdot \rnorm_u$.
            \item $M_n(F)$ equipped with $\lnorm \cdot \rnorm_\text{op}$ is a normed algebra.
            \item If $V$ is a normed space, then $B(V,V)$ with $\lnorm \cdot \rnorm_\text{op}$ is a normed algebra: for $T,S \in B(V,V)$ and $v \in B_V$, we have that
                \begin{equation*}
                \begin{split}
                    \lnorm (T \circ S)(v) \rnorm
                    & \leq \lnorm T \rnorm_\text{op}\lnorm S(v) \rnorm \\
                    & \leq \lnorm T \rnorm_\text{op}\lnorm S \rnorm_\text{op}.
                \end{split}
                \end{equation*}
            Taking the supremum over all $v \in B_V$ gives $\lnorm T \circ S \rnorm_\text{op} \leq \lnorm T \rnorm_\text{op}\lnorm S \rnorm_\text{op}$.

            \item Let $S$ be a group. Equip the algebra $\bfF(S)$ with $\lnorm \cdot \rnorm_1$. We get a normed algebra.
        \end{enumerate}
    \end{example}

    \begin{exercise}
        For $a,b \in \ell_1(\bfZ)$, define $a\ast b :\bfZ \rightarrow F$ by $(a \ast b)(n) = \sum_{k \in \bfZ}a(n-k)b(k)$. Show that $\ell_1(\bfZ)$ with this multiplication is a normed algbera.
    \end{exercise}
    

    




    
    

    \chapter{Metric Spaces}

\section{Basic Definitions and Examples}
    \begin{definition}
        A \textit{metric} on a nonempty set $X$ is a map
            \begin{equation*}
            \begin{split}
                d: X \times X \rightarrow [0,\infty)
            \end{split}
            \end{equation*}
        satisfying for all $x,y,z \in X$:
            \begin{enumerate}[label = (\arabic*),itemsep=1pt,topsep=3pt]
                \item $d(x,y) = d(y,x)$ (symmetry);
                \item $d(x,z) \leq d(x,y) + d(y,z)$ (triangle inequality);
                \item $d(x,x) = 0$;
                \item if $d(x,y) = 0$ then $x=y$ (positivity).
            \end{enumerate}
        If $d$ satisfies all but (iv), then $d$ is called a \textit{semi-metric}. The pair $(X,d)$ is called a \textit{metric space}.
    \end{definition}

    \begin{definition}
        Two metrics $d,\rho$ on $X$ are called \textit{equivalent} if there exists constants $c,c'$ with $d(x,y) \leq c \rho(x,y)$ and $\rho(x,y) \leq c'd(x,y)$.
    \end{definition}

    \begin{definition}
        A family of metrics $\{d_k\}_{k = 1}^\infty$ on $X$ is \textit{uniformly bounded} if, for all $x,y \in X$ and $k \in \bfN$, we have $d_k(x,y) \leq C$.
    \end{definition}

    \begin{example}
        \phantom{a}
        \begin{enumerate}[label = (\arabic*),itemsep=1pt,topsep=3pt]
            \item The \textit{discrete metric} on $X \neq \emptyset$ is:
                \begin{equation*}
                \begin{split}
                    d(x,y) = \begin{cases}1, & x \neq y \\ 0, & x =y  .\end{cases}
                \end{split}
                \end{equation*}

            \item The \textit{hamming distance} between two bit strings of equal length: given $X = \{0,1\}^n$, then $d_H:X \times X \rightarrow [0,\infty)$ is defined by $d_H\bigl( (x_j)_{n \geq 1}, (y_j)_{n \geq 1}\bigr) = \left| \left\{ j \mid x_j \neq y_j \right\} \right|$.
            
            \item If $(V,\lnorm \cdot \rnorm)$ is any normed space, then $d(v,w) = \lnorm v -w  \rnorm$ is a metric on $V$.
                \begin{exercise}
                    If $\lnorm \cdot \rnorm$ and $\lnorm \cdot \rnorm'$ are norms on a linear space $V$, show they are equivalent if and only if their induced metrics are equivalent.
                \end{exercise}

            \item If $(X,d)$ is a metric and $Y \subseteq X$ is a subset, then $(Y,d)$ is a metric space.
            
            \item Let $(X,\rho)$ be a metric space. Fix $p \in X$. Then:
                \begin{equation*}
                \begin{split}
                    d(x,y) := \begin{cases} 0, &x=y \\ \rho(x,p) + \rho(p,y), & x \neq y \end{cases}
                \end{split}
                \end{equation*}
            is a metric.

            \item It is often beneficial to work with metrics that are bounded. Let $\rho$ be a (semi)-metric of $X$. Set:
                \begin{equation*}
                \begin{split}
                    d(x,y) = \frac{\rho(x,y)}{1 + \rho(x,y)}.
                \end{split}
                \end{equation*}
            Defining $d(x,y)$ as above gives $0 \leq d(x,y) \leq 1$. Although $d$ and $\rho$ are not equivalent metrics, they are topologically equivalent. 
            
            Clearly $d$ is symmetric and $d(x,x) = 0$. Moreover, if $d(x,y) = 0$, then $\rho(x,y) = 0$, giving $x=y$ if $\rho$ is a metric. For the triangle inequality, consider the function $g:[0,\infty) \rightarrow [0,1)$ given by $g(t) = \frac{t}{1+t}$. We have that $g'(t) = \frac{(1+t) - t}{(1+t)^2} = \frac{1}{(1+t)^2} > 0$, whence $g$ is strictly increasing. Since we know $\rho(x,z) \leq \rho(x,y) + \rho(y,z)$, observe that:
                \begin{equation*}
                \begin{split}
                    d(x,z)
                    & = \frac{\rho(x,z)}{1 + \rho(x,z)} \\
                    & \leq \frac{\rho(x,y) + \rho(y,z)}{1 + \rho(x,y) + \rho(y,z)} \\
                    & = \frac{\rho(x,y)}{1 + \rho(x,y) + \rho(y,z)} + \frac{\rho(y,z)}{1 + \rho(x,y) + \rho(y,z)} \\
                    & \leq \frac{\rho(x,y)}{1 + \rho(x,y)} + \frac{\rho(y,z)}{1  + \rho(y,z)} \\
                    & = d(x,y) + d(y,z).
                \end{split}
                \end{equation*}

            \item If $d_1,...,d_n$ are metrics on $X$ and $c_1,...,c_n > 0$, then:
                \begin{equation*}
                \begin{split}
                    d(x,y) = \sum_{i = 1}^n c_i d_i(x,y)
                \end{split}
                \end{equation*}
            is a metric on $X$.

            \item Let $\{\rho_k\}_{k = 1}^\infty$ be a family of semi-metrics on $X$. Assume that the family is \textit{separating}: if $x,y \in X$ and $x \neq y$, then there exists $k$ such that $\rho_k(x,y) \neq 0$. Let $d_k(x,y) = \frac{\rho_k(x,y)}{1 + \rho_k(x,y)}$. Then:
                \begin{equation*}
                \begin{split}
                    d(x,y) = \sum_{k = 1}^\infty 2^{-k}d_k(x,y)
                \end{split}
                \end{equation*}
            is a metric on $X$. Since $0 \leq d_k(x,y) \leq 1$, by comparison $d(x,y)$ will converge.

            \item Let $(X_k, \rho_k)_{k \geq 1}$ be a sequence of metric spaces. For each $k$ let $d_k$ be as above. Let $X = \prod_{k = 1}^\infty X_k$. Then the map $D:X \times X \rightarrow [0,\infty)$ defined by
                \begin{equation*}
                \begin{split}
                    D(f,g) = \sum_{k = 1}^\infty 2^{-k}d_k(f(k),g(k))
                \end{split}
                \end{equation*}
            is a metric on $X$. Note that we need not make $d_k$ from $\rho_k$ if all the $d_k$ are uniformly bounded.

            \item Let $X = \{0,2\}$ with the discrete metric. The \textit{abstract Cantor set} is $\Delta := \prod_{k \in \bfN}X$. Then the map $D:\Delta \times \Delta \rightarrow [0,\infty)$ defined by 
                \begin{equation*}
                \begin{split}
                    D(f,g) = \sum_{k = 1}^\infty 3^{-k}|f(k) - g(k)|
                \end{split}
                \end{equation*}
            is a metric on $\Delta$.

            \item Let $\langle \cdot,\cdot \rangle$ be the standard inner product on $\bfR^3$ (or $\bfR^n$). The unit sphere $S^2 = \{x \in \bfR^3 \mid \lnorm x \rnorm_2 = 1\}$ paired with $d(x,y):=\arccos(\langle x,y \rangle)$ is a metric space.
                \begin{exercise}
                    Show that $(S^2, d)$ defined as above is indeed a metric space.
                \end{exercise}
        \end{enumerate}
    \end{example}

    \begin{definition}
        \phantom{a}
        \begin{enumerate}[label = (\arabic*),itemsep=1pt,topsep=3pt]
            \item Let $(X,d)$ be a metric space with $E \subseteq X$. The \textit{diameter} of $E$ is $\diam(E) = \sup_{x,y \in E}d(x,y)$. We say $E$ is \textit{bounded} (metrically) if $\diam(E) < \infty$.
            \item If $\Omega$ is any set and $(Y,d)$ is a metric space, $f:\Omega \rightarrow Y$ is \textit{bounded} if \newline $\diam(f(\Omega))<\infty$. The set of bounded functions is $\Bd(\Omega,Y):= \{f:\Omega \rightarrow Y \mid f \h3\text{bounded}\h2 \} $.
        \end{enumerate}
    \end{definition}

    \begin{exercise}
        If $(V,\lnorm \cdot \rnorm)$ is a normed space and $E \subseteq V$, the following are equivalent:
            \begin{enumerate}[label = (\arabic*),itemsep=1pt,topsep=3pt]
                \item $E$ is bounded;
                \item $\sup_{v \in E}\lnorm v \rnorm < \infty$;
                \item there exists $r > 0$ such that $E \subseteq B(0,r)$.
            \end{enumerate}
    \end{exercise}

    \begin{example}
        The set $\Bd(\Omega,Y)$ is a metric space with:
            \begin{equation*}
            \begin{split}
                D_u(f,g):= \sup_{x \in \Omega}d(f(x),g(x)).
            \end{split}
            \end{equation*}
        Clearly $D_u(f,g) = D_u(g,f)$ and $D_u(f,f) = 0$. If $D_u(f,g) = 0$, then $f(x) = g(x)$ for all $x \in \Omega$, giving $f = g$. Moreover, for every $x \in \Omega$:
            \begin{equation*}
            \begin{split}
                d(f(x),h(x)) &\leq d(f(x),g(x)) + d(g(x),h(x)) \\
                & \leq D_u(f,g) + D_u(g,h).
            \end{split}
            \end{equation*}
        Whence $D_u(f,h) \leq D_u(f,g) + D_u(g,h)$. Note that if we take the normed space $(\ell_\infty(\Omega),\lnorm \cdot \rnorm_u)$, the induced metric is:
            \begin{equation*}
            \begin{split}
                d(f,g)
                & = \lnorm f-g \rnorm_u \\
                & = \sup_{x \in \Omega}|(f-g)(x)| \\
                & = \sup_{x \in \Omega}|f(x) - g(x)| \\
                & = D_u(f,g).
            \end{split}
            \end{equation*}
        So as metric spaces, $\ell_\infty(\Omega) \cong \Bd(\Omega,F)$. Now consider the subset \newline $E = \{f \in \Bd(\Omega,F) \mid f(x) \in \{0,1\}\}$. We get:
            \begin{equation*}
            \begin{split}
                D_u(f,g)
                & = \sup_{x \in \Omega}|f(x) - g(x)| \\
                & = \begin{cases} 0, & f = g \\
                1, & f \neq g  \h2.\end{cases}
            \end{split}
            \end{equation*}
        So $(E,D_u)$ is discrete.
    \end{example}

\section{Topology of Metric Spaces}\label{sec:top-of-metric-spaces}
    Unless otherwise stated, let $(X,d)$ be a metric space.
    \begin{definition}
        Let $X$ be a set. A collection $T$ of subsets of $X$ is called a \textit{topology on $X$} if they satisfy:
            \begin{enumerate}[label = (\arabic*),itemsep=1pt,topsep=3pt]
                \item $\emptyset,X \in T$;
                \item arbitrary unions of elements in $T$ are in $T$;
                \item finite intersections of elements in $T$ are in $T$.
            \end{enumerate}
    \end{definition}

    \begin{definition}\label{def:2.2.2}
        Let $(X,d)$ be a metric space.
            \begin{enumerate}[label = (\arabic*),itemsep=1pt,topsep=3pt]
                \item Let $x_0 \in X$ and $\delta > 0$.
                    \begin{enumerate}[label = (\roman*),itemsep=1pt,topsep=3pt]
                        \item The \textit{open ball} centered at $x_0$ of radius $\delta$ is $U(x_0,\delta) = \{x \mid d(x,x_0) < \delta \}$.
                        \item The \textit{closed ball} centered at $x_0$ of radius $\delta$ is $B(x_0,\delta) = \{x \mid d(x,x_0) \leq \delta \}$.
                        \item The \textit{sphere} centered at $x_0$ of radius $\delta$ is $S(x_0,\delta) = \{x \mid d(x,x_0) = \delta \}$.
                    \end{enumerate}

                \item A subset $U \subseteq X$ is \textit{open in $X$} if:
                    \begin{equation*}
                    \begin{split}
                        (\forall x \in U)(\exists \delta > 0): U(x,\delta) \subseteq U.
                    \end{split}
                    \end{equation*}
                The collection of open sets is denoted $\tau_X :=\{U \subseteq X \mid U \h3\text{is open}\h1\}$.

                \item A subset $D \subseteq X$ is \textit{closed in $X$} if $D^c \subseteq X$ is open in $X$.
                
                \item If $x \in U \in \tau_X$, then $U$ is called an \textit{open neighborhood of $x$}. If $x \in U \in \tau_X$ and $U \subseteq N \subseteq X$, then $N$ is called a \textit{neighborhood of $x$}. The collection of neighborhoods of $x$ is denoted $\cN_x = \{N \mid N \h3 \text{is a neighborhood of $x$}\h1\}$.
                
                \item Let $A \subseteq X$.
                    \begin{enumerate}[label = (\roman*),itemsep=1pt,topsep=3pt]
                        \item The \textit{interior of $A$} is:
                            \begin{equation*}
                            \begin{split}
                                A^o := \bigcup \{V \in \tau_X \mid V \subseteq A\}.
                            \end{split}
                            \end{equation*}
                        \item The \textit{closure of $A$} is:
                            \begin{equation*}
                            \begin{split}
                                \overline{A}:= \bigcap \{C \mid C \supseteq A, \h1C \h3\text{closed}\h1\}.
                            \end{split}
                            \end{equation*}
                        \item The \textit{boundary of $A$} is $\partial A := \overline{A}\setminus A^o$.
                    \end{enumerate}
            \end{enumerate}
    \end{definition}

    \begin{exercise}
        Show that $\overline{A^c} = (A^o)^c$ and $\overline{A}^c = (A^c)^o$.
    \end{exercise}

    \begin{proposition}\label{prop:open-sets-topology}
        Let $(X,d)$ be a metric space. The open sets $\tau_X$ form a topology.
    \end{proposition}
        \begin{proof}
            Both $\emptyset$ and $X$ are open by assumption. Let $\{V_i\}_{i \in I}$ be a family of open sets of $X$. Let $x \in \bigcup_{i \in I} V_i$. Then $x \in V_i$ for some $i$. Since $V_i$ is open, there exists $\delta>0$ with $B(x,\delta) \subseteq V_i \subseteq \bigcup_{i \in I}V_i$. Whence the arbitrary union of open sets is open.
            
            Now let $\{V_k\}_{k = 1}^n$ be a family of open sets. Let $x \in \bigcap_{k = 1}^n V_k$. Then $x \in V_k$ for all $k$. Since $V_k$ is open, there exists $\delta_k > 0$ with $B(x,\delta_k) \subseteq V_k$. Pick $\delta = \min\{\delta_1,\delta_2,...,\delta_k\}$. Then $B(x,\delta) \subseteq \bigcap_{k = 1}^n V_k$. Whence $\bigcap_{k = 1}^n V_k$ is open.
        \end{proof}
    
    Note that an arbitrary intersection of open sets is not necessarily open. Consider the sequence of intervals $(I_n)_n = (-\frac{1}{n},\frac{1}{n})$. Then $\bigcap_{n = 1}^\infty I_n = \{0\}$, which is closed.

    \begin{exercise}
        Let $(X,d)$ be a metric space and consider a collection $\cC$ of subsets of $X$ with $\emptyset,X \in \cC$. Show that:
            \begin{enumerate}[label = (\arabic*),itemsep=1pt,topsep=3pt]
                \item if $\{C_i\}_{i \in I}$ is a family of closed sets, then $\bigcap_{i \in I}C_i$ is closed;
                \item if $\{C_i\}_{i = 1}^n$ is a family of closed sets then $\bigcup_{i = 1}^n C_i$ is closed.
            \end{enumerate}
    \end{exercise}

    \begin{proposition}
        Let $(X,d)$ be a metric space and $x \in X$.
            \begin{enumerate}[label = (\arabic*),itemsep=1pt,topsep=3pt]
                \item $N \in \cN_x$ if and only if there exists $\delta > 0$ such that $U(x,\delta) \subseteq N$.
                \item If $N \in \cN_x$ and $N \subseteq M$, then $M \in \cN_x$.
                \item If $N_1,N_2 \in \cN_x$, then $N_1 \cap N_2 \in \cN_x$.
            \end{enumerate}
    \end{proposition}
        \begin{proof}
            (1) Let $N \in \cN_x$. Then there is an open set $x \in U$ with $U \subseteq N \subseteq X$. Since $U$ is open, there exists $\delta > 0$ such that $U(x,\delta) \subseteq U \subseteq N$. Conversely, suppose there exists $\delta >0$ such that $U(x,\delta) \subseteq N$. Clearly $U(x,\delta) \subseteq N \subseteq X$, whence $N \in \cN_x$.

            (2) If $N \in \cN_x$, then there is an open set $U$ with $x \in U$ and $U \subseteq N \subseteq X$. So $U \subseteq N \subseteq M \subseteq X$. Whence $M \in \cN_x$.

            (3) If $N_1,N_2 \in \cN_x$, then there are open sets $U_1,U_2$ with $x \in U_1$, $x \in U_2$ and $U_1 \subseteq N_1 \subseteq X$, $U_2 \subseteq N_2 \subseteq X$. Whence $U_1 \cap U_2 \subseteq N_1 \cap N_2 \subseteq X$.
        \end{proof}

    \begin{proposition}\label{prop:partition-open}
        Let $U \subseteq \bfR$ be open. Then:
            \begin{equation*}
            \begin{split}
                U = \bigsqcup_{j \in J}I_j,
            \end{split}
            \end{equation*}
        where $J$ is countable and $I_j$ are open intervals.
    \end{proposition}
        \begin{proof}
            For each $x \in U$, define:
                \begin{equation*}
                \begin{split}
                    I_x := \bigcup \h2 \{I \mid x \in I \subseteq U, \h2I \h3\text{open interval}\h1\}.
                \end{split}
                \end{equation*}
            Clearly $x \in I_x \subseteq U$. If $s,t \in I_x$ with $s < t$, then there exists open intervals $I,I'$ with $x\in I \subseteq U$, $x \in I' \subseteq U$, and $s \in I$, $t \in I'$. Since $I \cap I' \neq \emptyset$, $I \cup I'$ is an open interval. Moreover, since $s,t \in I \cup I'$, we know $[s,t] \subseteq I \cup I' \subseteq I_x$. This shows $I_x$ is an interval \textemdash in particular, since $I_x$ is the union of open intervals, it must be open.

            Now suppose $x,y \in U$ and $I_x \cap I_y \neq \emptyset$. Then there exists $z \in I_x \cap I_y$, but $z \in I_x$ implies $I_x \subseteq I_z$ and $z \in I_y$ implies $I_y \subseteq I_z$. But  we also have $x \in I_x \subseteq I_z$ which gives $I_z \subseteq I_x$, and similarly $y \in I_y \subseteq I_z$ gives $I_z \subseteq I_y$. Together, we have $I_x = I_y$, which means for any $x,y \in U$, then $I_x \cap I_y = \emptyset$ or $I_x = I_y$. Thus there exists $J \subseteq U$ with $U = \bigsqcup_{j \in J}I_j$.

            It remains to show that $J$ is countable. Define $J \rightarrow \bfQ$ by $x \mapsto q_x$, where $q_x \in \bfQ \cap I_x$. This map is injective, establishing the proposition.
        \end{proof}
    
    \begin{proposition}\label{prop:interior-closure-boundary-property}
        Let $A \subseteq X$.
        \begin{enumerate}[label = (\arabic*),itemsep=1pt,topsep=3pt]
            \item $x \in A^o$ if and only if there exists $\delta > 0$ such that $U(x , \delta) \subseteq A$.
            \item $x \in \overline{A}$ if and only if for all $\delta > 0$, $U(x,\delta) \cap A \neq \emptyset$.
            \item $x \in \partial A$ if and only if for all $\delta > 0$, $U(x,\delta) \cap A \neq \emptyset$ and $U(x,\delta) \cap A^c \neq \emptyset$.
        \end{enumerate}
    \end{proposition}
        \begin{proof}
            
            (1) If $x \in A^o$, then by definition $x \in \bigcup \{V \in \tau_X \mid V \subseteq A\}.$ So there exists some $V \in \tau_X$ such that $x \in V \subseteq A$. Since $V$ is open, there exists $\delta > 0$ with $U(x,\delta) \subseteq V \subseteq A$. Conversely, if $U(x,\delta) \subseteq A$ for some $\delta > 0$, then clearly $x \in A^o$, as $U(x,\delta) \in \tau_X$.

            (2) We prove the converse of this statement. Note that $x \not\in \overline{A}$ if and only if $x \in (\overline{A})^c = (A^c)^o$. By (1) there exists $\delta > 0$ with $U(x,\delta) \subseteq (A^c)^o \subseteq A^c$. This is true if and only if $U(x,\delta) \cap A = \emptyset$, establishing the proposition.

            (3) Note that:
                \begin{equation*}
                \begin{split}
                    \partial A 
                    & = \overline{A} \setminus{A^o} \\
                    & = \overline{A} \cap (A^o)^c \\
                    & = \overline{A} \cap \overline{A^c}.
                \end{split}
                \end{equation*}
            So $x \in \partial A$ if and only if $x \in \overline{A} \cap \overline{A^c}$. Applying (2) to $x \in \overline{A}$ and $x \in \overline{A^c}$ establishes (3).
        \end{proof}

    \begin{exercise}
        Show that open balls are open, closed balls are closed, and spheres are closed.
    \end{exercise}

    \begin{proposition}***
        For any normed space:
            \begin{enumerate}[label = (\arabic*),itemsep=1pt,topsep=3pt]
                \item $\overline{U(x,\delta)} = B(x,\delta)$
                \item $B(x,\delta)^o = U(x,\delta)$ 
                \item $\partial U(x,\delta) = \partial B(x,\delta) = S(x,\delta)$.
            \end{enumerate}
    \end{proposition}
        \begin{proof}
        \end{proof}

    \begin{proposition}***
        Let $(X,d)$ be a metric space with $\{A_i\}_{i \in I}$ a family of subsets. Let $K \subseteq I$ be finite.
        \begin{enumerate}[label = (\arabic*),itemsep=5pt,topsep=8pt]
            \item $\ds \bigcup_{i \in I}A_i^o \subseteq \left( \bigcup_{i \in I}A_i \right)^o$ (Inclusion may be strict). 
            \item $\ds \overline{\bigcap_{i \in I} A_i} \subseteq \bigcap_{i \in I}\overline{A_i}$ (Inclusion may be strict). 
            \item $\ds \bigcap_{i \in K}A_i^o = \left( \bigcap_{i \in K}A_i \right)^o$
            \item $\ds \overline{\bigcup_{i \in K}A_i} = \bigcup_{i \in K}\overline{A_i}$.
        \end{enumerate}
    \end{proposition}
        \begin{proof}

        \end{proof}

    \begin{proposition}
        Let $S \subseteq X$.
        \begin{enumerate}[label = (\arabic*),itemsep=1pt,topsep=3pt]
            \item $\partial S = \partial S^c$.
            \item $\partial S$ is closed.
            \item $\overline{S} = S \cup \partial S$.
            \item $S \setminus \partial S = S^o$.
        \end{enumerate}
    \end{proposition}
        \begin{proof}
            (1) This follows from the characterization of $\partial S$. (2) We have $\partial S = \overline{S} \setminus S^o = \overline{S} \cap (S^o)^c$, which is closed. (3) Clearly $S \cup \partial S \subseteq \overline{S}$. Let $x \in \overline{S}$. If $x \in S$ we are done. Otherwise $x \in \overline{S} \setminus S \subseteq \overline{S} \setminus S^o = \partial S$. (4) Observe that:
                \begin{equation*}
                \begin{split}
                    S \setminus \partial S 
                    & = S \cap (\partial S) ^c \\
                    & = S \cap (\overline{S} \setminus S^o)^c \\
                    & = S \cap (\overline {S} \cap (S^o)^c)^c \\
                    & = S \cap (\overline{S}^c \cup S^o ) \\
                    & = S \cap \overline{S}^c \cup S \cap S^o \\
                    & = S^o.
                \end{split}
                \end{equation*}
        \end{proof}

    \begin{definition}
        Let $(X,d)$ be a metric space.
        \begin{enumerate}[label = (\arabic*),itemsep=1pt,topsep=3pt]
            \item A subset $A \subseteq X$ is \textit{$d$-dense} if $\overline{A} = X$.
            \item A subset $N \subseteq X$ is \textit{nowhere dense} if $(\overline{N})^o = \emptyset$.
            \item The space $(X,d)$ is \textit{separable} if there exists a countable dense subset $D \subseteq X$.
        \end{enumerate}
    \end{definition}

    \begin{exercise}
        If $N \subseteq X$ is closed, then $N$ is nowhere dense if and only if $N^c$ is dense.
    \end{exercise}

    \begin{proposition}\label{prop:dense-properties}***
        Let $A \subseteq X$. The following are equivalent:
            \begin{enumerate}[label = (\arabic*),itemsep=1pt,topsep=3pt]
                \item $A$ is dense;
                \item $(\forall U \in \tau_X),U \cap A \neq \emptyset$;
                \item $(\forall x \in X)(\forall \epsilon>0), U(x,\epsilon) \cap A \neq \emptyset$;
                \item $(\forall x \in X)(\forall \epsilon>0)(\exists a \in A):d(a,x) < \epsilon$.
            \end{enumerate}
    \end{proposition}
        \begin{proof}
            
        \end{proof}

    \begin{definition}
        Let $(X,d)$ be a metric space. 
        \begin{enumerate}[label = (\arabic*),itemsep=1pt,topsep=3pt]
            \item A \textit{base} for $\tau_X$ is a family of open subsets $\cB \subseteq \tau_X$ such that:
                \begin{equation*}
                \begin{split}
                    (\forall U \in \tau_X)(\forall x\in U)(\exists B \in \cB):x \in B \subseteq U.
                \end{split}
                \end{equation*}
            Equivalently, for all $U \in \tau_X$, we can write $U = \bigcup_{i \in I}B_i$, where $\{B_i\}_{i \in I} \subseteq \tau_X$.

            \item $X$ is \textit{second countable} if it has a countable base.
        \end{enumerate}
    \end{definition}

    Note that this definition can be generalized to any topological space. Clearly $\cB = \{U(x,\epsilon) \mid x \in X, \epsilon>0\}$ forms a base for any metric space.

    \begin{example}
        The set $\cB = \{U(q,\frac{1}{n}) \mid n \geq 1,q \in \bfQ^d\}$ is a base for $\bfR^d$.
    \end{example}

    \begin{proposition}
        Let $(X,d)$ be a metric space. $X$ is separable if and only if $X$ is second countable.
    \end{proposition}
        \begin{proof}
            Let $\cB = \{U_n\}_{n=1}^\infty$ be a countable base. Choose any $a_n \in U_n$. Then $\{a_n\}_{n = 1}^\infty$ is dense. Indeed, given any $x \in X$ and $\epsilon>0$, there exists $U_m$ with $x \in U_m \subseteq U(x,\epsilon)$ (since $U_m \in \cB$). Whence $d(a_m,x) < \epsilon$.

            Let $\{a_n\}_{n = 1}^\infty$ be dense. Consider:
                \begin{equation*}
                \begin{split}
                    \cB = \left\{U(a_n,{\ts \frac{1}{m}}) \mid n \geq 1, m \geq 1\right\}.
                \end{split}
                \end{equation*}
            Clearly $\cB$ is countable \textemdash it remains to show that it is a base for $X$. Given $x \in V \in \tau_X$, find $\epsilon > 0$ such that $U(x,\epsilon) \subseteq V$. Then there exists $m \geq 1$ with $\epsilon > \frac{1}{m}$. Since $\{a_n\}_{n = 1}^\infty$ is dense, there exists $a_j \in \{a_n\}_{n = 1}^\infty$ such that $d(a_j,x) < \frac{1}{2m}$. Let $y \in U(a_j,\frac{1}{2m})$. Observe that:
                \begin{equation*}
                \begin{split}
                    d(x,y)
                    & \leq d(x,a_j) + d(a_j,y) \\
                    & < \frac{1}{2m} + \frac{1}{2m} \\
                    & = \frac{1}{m} \\
                    & < \epsilon.
                \end{split}
                \end{equation*}
            So $y \in U(x,\epsilon)$. Thus $x \in U(a_j,\frac{1}{2m}) \subseteq U(x,\epsilon) \subseteq V$, establishing $\cB$ as a base.
        \end{proof}

    \begin{example}
        \phantom{a}
        \begin{enumerate}[label = (\arabic*),itemsep=1pt,topsep=3pt]
            \item The space $(\bfR^d,\lnorm \cdot \rnorm_p)$ is separable for any $1 \leq p \leq \infty$. Indeed, if $(r_1,...,r_d) \in \bfR^d$ and $\epsilon > 0$, find $q_j \in \bfQ$, $j = 1,...,d$ with:
                \begin{equation*}
                \begin{split}
                    |r_j - q_j| < \frac{\epsilon}{d}.
                \end{split}
                \end{equation*}
            Then:
                \begin{equation*}
                \begin{split}
                    \lnorm r-q \rnorm_1 = \sum_{j = 1}^d |r_j - q_j| < \epsilon.
                \end{split}
                \end{equation*}
            So $\bfQ^d$ is $\lnorm \cdot \rnorm_1$-dense in $\bfR^d$. For $1 \leq p \leq \infty$, let $C>0$ be such that \newline $\lnorm \cdot \rnorm_p \leq C \lnorm \cdot \rnorm_1$. So given $\epsilon > 0$, find $q \in \bfQ^d$ with $\lnorm r - q \rnorm_1 \leq \frac{\epsilon}{C}$. Then $\lnorm r - q \rnorm_p < \epsilon$.

            \item Similarly, $\bfC_\bfQ^d \subseteq \bfC^d$ is $\lnorm \cdot \rnorm_p$-dense, where:
                \begin{equation*}
                \begin{split}
                    \bfC_\bfQ = \{a+bi \mid a,b \in \bfQ\}.
                \end{split}
                \end{equation*}

            \item Recall that $c_{00} = \left\{(z_k)_k \mid \supp \bigl((z_k)_k\bigr) < \infty \right\}$. The space $(c_{00},\lnorm \cdot \rnorm_u)$ is separable.
            
            Note that $c_{00} = \bfC\text{\h1-}\Span\{e_k \mid k \in \bfN\}$. This space is not countable \textemdash clearly $\bfC\text{\h1-}\Span\{e_1\} = \{\alpha e_1 \mid \alpha \in \bfC\}$ is not countable, so it must be that $c_{00}$ is also not countable.

            Instead, consider:
                \begin{equation*}
                \begin{split}
                    \bfC_\bfQ\text{\h1-}\Span\{e_k \mid k \in \bfN \}
                    & = \left\{ \sum_{k = 1}^\infty t_k e_k \mid t_k \in \bfC_\bfQ,  \right\} \\
                    & = \bigcup_{k = 1}^\infty \{C_k \mid C_k = \bfC_\bfQ\text{\h1-}\Span\{e_1,...,e_k\}\}
                \end{split}
                \end{equation*}
            Note that $C_k$ is in bijection with $\bfQ^{2k}$, whence $\bfC_\bfQ\text{\h1-}\Span\{e_k \mid k \in \bfN \}$ is countable.

            Given $z \in c_{00}$, let $z = \sum_{k = 1}^N z_k e_k$ and $\epsilon > 0$. Find $t_k \in \bfC_\bfQ$ with $|z_k - t_k| < \epsilon$. Then:
                \begin{equation*}
                \begin{split}
                    \lnorm z - t \rnorm_u 
                    & = \lnorm \sum_{k = 1}^N z_k e_k - \sum_{k = 1}^K t_k e_k \rnorm_u \\
                    & = \lnorm \sum_{k = 1}^N (z_k - t_k)e_k \rnorm_u \\
                    & = \sup_{k = 1}^N |z_k - t_k| \\
                    & < \epsilon.
                \end{split}
                \end{equation*}
            Thus $\bfC_\bfQ\text{\h1-}\Span\{e_k \mid k \in \bfN \}$ is dense in $c_{00}$, whence the space $(c_{00},\lnorm \cdot \rnorm_u)$ is separable.
        \end{enumerate}
    \end{example}

    \begin{proposition}
        If $(X,d)$ is a separable metric space and $Y \subseteq X$, then $(Y,d)$ is separable.
    \end{proposition}
        \begin{proof}
            Let $A = \{a_k\}_{k = 1}^\infty$ be dense in $X$. Let:
                \begin{equation*}
                \begin{split}
                    N = \{(m,n) \mid U(a_m,{\ts \frac{1}{n}}) \cap Y \neq \emptyset \}.
                \end{split}
                \end{equation*}
            For each $(m,n) \in N$, choose $b_{(m,n)} \in Y \cap U(a_m,{\ts \frac{1}{n}})$. Claim: the set
                \begin{equation*}
                \begin{split}
                    \{b_{(m,n)} \mid (m,n) \in N\}
                \end{split}
                \end{equation*}
            is dense in $Y$. Let $y \in Y$ and $\epsilon > 0$. Then there exists $n \geq 1$ with $\frac{\epsilon}{2} > \frac{1}{n}$. Since $A$ is dense, $U(y,\frac{1}{n}) \cap A \neq \emptyset$ (this is because for all $U \in \tau_X$, we have $U \cap A \neq \emptyset$). So $d(a_m, y) < \frac{1}{n}$. Whence:
                \begin{equation*}
                \begin{split}
                    d(b_{(m,n)},y)
                    & \leq d(b_{(m,n)},a_m) +  d(a_m,y) \\
                    & < \frac{1}{n} + \frac{1}{n} \\
                    & < \epsilon.
                \end{split}
                \end{equation*}
        \end{proof}

    \begin{example}
        \phantom{a}
        \begin{enumerate}[label = (\arabic*),itemsep=1pt,topsep=3pt]
            \item The space $(\ell_\infty, \lnorm \cdot \rnorm_u)$ is not separable. If it were, consider:
                \begin{equation*}
                \begin{split}
                    E = \{(x_k)_k \mid x_k \in \{0,1\}\} \subseteq \ell_\infty.
                \end{split}
                \end{equation*}
            This set is uncountable, and by the previous proposition $(E, \lnorm \cdot \rnorm_u)$ is also separable. Let $a,b\in E$. We have:
                \begin{equation*}
                \begin{split}
                    \lnorm (a_k)_k - (b_k)_k \rnorm_u 
                    & = \lnorm (a_k - b_k)_k \rnorm_u \\
                    & = \sup_{k \geq 1}|a_k - b_k| \\
                    & = \begin{cases}
                        0, & (a_k)_k = (b_k)_k \\
                        1, & (a_k)_k \neq (b_k)_k \h2.
                    \end{cases}
                \end{split}
                \end{equation*}
            So $(E,\lnorm \cdot \rnorm_u)$ is discrete. Note that a metric space is discrete if and only if its singletons are open. Whence if $(E,\lnorm \cdot \rnorm_u)$ is separable, then $A = \overline{A}  = E$, which contradicts the countability of $A$. It must be that $(\ell_\infty, \lnorm \cdot \rnorm_u)$ is not separable.

            \item The space $(\ell_p, \lnorm \cdot \rnorm_p)$ is separable for $1 \leq p < \infty$. Let $a = (a_k)_k \in \ell_p$ and $\epsilon > 0$. Find $N$ large so that $\sum_{k > N}|a_k|^p < \frac{\epsilon^p}{2}$. Find $b_k \in \bfC_\bfQ$ with $|a_k - b_k| < \frac{\epsilon}{\text{\scalebox{.8}{$(2N)^\frac{1}{p}$}}}$. Let $b = (b_1,b_2,...,b_{N-1},b_N,0,0,...)$. We have:
                \begin{equation*}
                \begin{split}
                    \lnorm a-b \rnorm_p^p 
                    & = \sum_{k = 1}^\infty |a_k - b_k|^p \\
                    & = \sum_{k = 1}^N |a_k - b_k|^p + \sum_{k > N}|a_k|^p \\
                    & < N \cdot \frac{\epsilon^p}{2N} + \frac{\epsilon^p}{2} \\
                    & = \epsilon.
                \end{split}
                \end{equation*}
            Whence the set $\bfC_\bfQ\text{\h1-}\Span\{e_k \mid k \in \bfN\}$ is $\lnorm \cdot \rnorm_u$-dense, giving $(\ell_p,\lnorm \cdot \rnorm_p)$ as separable.

            \item We will eventually show that the set of polynomial functions:
                \begin{equation*}
                \begin{split}
                    P\bigl([0,1]\bigr) = \left\{ \sum_{k = 0}^n a_kx^k \mid a_k \in F,n \geq 0 \right\}
                \end{split}
                \end{equation*}
            is $\lnorm \cdot \rnorm_u$-dense in $C\bigl([0,1]\bigr)$ (note that this set is not countable). With this fact, we can show that $C\bigl([0,1]\bigr)$ is separable. Indeed, given $f \in C\bigl([0,1]\bigr)$ and $\epsilon > 0$, find $p \in P\bigl([0,1]\bigr)$ with $\lnorm f - p \rnorm_u < \frac{\epsilon}{2}$. Now let $p(x) = \sum_{k = 0}^n a_k x^k$. Find $b_k \in \bfC_\bfQ$ with $|a_k - b_k| < \frac{\epsilon}{2(n+1)}$ and define $q(x) = \sum_{k = 0}^n b_k x^k$. Observe that:
                \begin{equation*}
                \begin{split}
                    \lnorm f - q \rnorm_u
                    & = \lnorm f-p + p-q \rnorm_u \\
                    & \leq \lnorm f-p \rnorm_u + \lnorm p-q \rnorm_u \\
                    & = \lnorm f-p \rnorm_u + \sum_{k = 0}^n|a_k - b_k| \\
                    & < \frac{\epsilon}{2} + (n+1)\cdot \frac{\epsilon}{2(n+1)} \\
                    & = \epsilon.
                \end{split}
                \end{equation*}
            Thus the set $\bfC_\bfQ\text{\h1-}\Span\{x^k \mid k \in \bfN\}$ is $\lnorm \cdot \rnorm_u$-dense in $C\bigl([0,1]\bigr)$. In particular, since it is countable, $C\bigl([0,1]\bigr)$ is separable.
        \end{enumerate}
    \end{example}

    

    \begin{center}
        \begin{tikzpicture}
            \draw[thick] (0.3,0) -- (2.3,0);
            \node at (2.39, 0) {$/\,$};
            \node at (2.56, 0) {$/\,$};
            \draw[thick] (2.6,0) -- (4.6,0);
        \end{tikzpicture}
    \end{center}

    We have seen that subsets of metric spaces are metric spaces in their own right. Then what are their open sets?

    \begin{proposition}
        Let $(X,d)$ be a metric space and let $Y \subseteq X$ be any subset. Then $V \subseteq Y$ is open in $Y$ if and only if there exists an open set $U \subseteq X$ with $U \cap Y = V$. That is, $\tau_Y = \{U \cap Y \mid U \in \tau_X\}$.
    \end{proposition}   
        \begin{proof}
            Let $V \subseteq Y$ be open. Then for every $x \in V$, there exists $\delta_x > 0$ with $U_y(x,\delta_x) \subseteq V$, where:
                \begin{equation*}
                \begin{split}
                    U_y(x,\delta_x)
                    & = \{y \in Y \mid d(y,x) < \delta_x \} \\
                    & = U(x,\delta_x) \cap Y.
                \end{split}
                \end{equation*} 
            Set $U = \bigcup_{x \in V}U(x,\delta_x)$. Then $U$ is indeed open in $X$. Also:
                \begin{equation*}
                \begin{split}
                    U \cap Y 
                    & = \left(\bigcup_{x \in V} U(x,\delta_x) \right) \cap Y \\
                    & = \bigcup_{x \in V} \left(U(x,\delta_x) \cap Y \right) \\
                    & = \bigcup_{x \in V}U_y(x,\delta_x) \\
                    & = V.
                \end{split}
                \end{equation*}

            Conversely, suppose $V = U \cap Y$ for some open $U \in \tau_X$. Let $x \in V$. Since $x \in U$ and $U$ is open, there exists $\delta > 0$ such that $U(x,\delta) \subseteq U$. So $U_y(x,\delta) = U(x,\delta) \cap Y \subseteq U \cap Y = V$. Thus $V$ is open in $Y$.
        \end{proof}

        \begin{example}
            \phantom{a}
            \begin{enumerate}[label = (\arabic*),itemsep=1pt,topsep=3pt]
                \item $[0,\frac{1}{2})$ is not open in $\bfR$, but it is open in $[0,1]$.
                \item $\ell_\infty$ is not a discrete metric space, but $\{0,1\}^\bfN \subseteq \ell_\infty$ is.
            \end{enumerate}
        \end{example}

\section{The Cantor Set}\label{sec:cantor-set}
    
    Given the interval $[0,1]$, start by deleting the open middle third $\left(\frac{1}{3}, \frac{2}{3}\right)$, leaving two line segments $\left[0,\frac{1}{3}\right] \cup \left[\frac{2}{3},1\right]$.
    Next, the open middle third of each of these remaining segments is deleted, leaving four line segments $\left[0,\frac{1}{9}\right] \;\cup\; \left[\frac{2}{9},\frac{1}{3}\right] \;\cup\;\left[\frac{2}{3},\frac{7}{9}\right] \;\cup\; \left[\frac{8}{9},1\right]$. 

    \vspace{25pt}
    \begin{center}
    \begin{tikzpicture}[scale=0.8, thick]
      %-------------------------------------------------------------
      % Step 0 (C_0): [0,16], at y = 0
      %-------------------------------------------------------------
      \draw (0,0) -- (16,0)
        node[midway, above] {\scalebox{.6}{$[0,1]$}};
    
      %-------------------------------------------------------------
      % Step 1 (C_1): remove the open middle third (16/3, 32/3) ≈ (5.3333,10.6667)
      % => [0, 5.3333] & [10.6667, 16], at y = -0.6
      %-------------------------------------------------------------
      \draw (0,-0.6) -- (5.3333,-0.6)
        node[midway, above] {\scalebox{.6}{$[0,\frac{1}{3}]$}};
      \draw (10.6667,-0.6) -- (16,-0.6)
        node[midway, above] {\scalebox{.6}{$[\frac{2}{3},1]$}};
    
      %-------------------------------------------------------------
      % Step 2 (C_2): each segment length 5.3333
      % remove its middle third of length ~1.7778
      % => four segments, at y = -1.2
      %-------------------------------------------------------------
      \draw (0,-1.2) -- (1.7778,-1.2)
        node[midway, above] {\scalebox{.6}{$[0,\frac{1}{9}]$}};
      \draw (3.5556,-1.2) -- (5.3333,-1.2)
        node[midway, above] {\scalebox{.6}{$[\frac{2}{9},\frac{1}{3}]$}};
      \draw (10.6667,-1.2) -- (12.4444,-1.2)
        node[midway, above] {\scalebox{.6}{$[\frac{2}{3},\frac{7}{9}]$}};
      \draw (14.2222,-1.2) -- (16,-1.2)
        node[midway, above] {\scalebox{.6}{$[\frac{8}{9},1]$}};
    
      %-------------------------------------------------------------
      % Step 3 (C_3): each segment length ~1.7778
      % remove middle third ~0.5926
      % => 8 segments, at y = -1.5
      %-------------------------------------------------------------
      \draw (0,-1.5)        -- (0.5926,-1.5);
      \draw (1.1852,-1.5)   -- (1.7778,-1.5);
      \draw (3.5556,-1.5)   -- (4.1482,-1.5);
      \draw (4.7408,-1.5)   -- (5.3333,-1.5);
      \draw (10.6667,-1.5)  -- (11.2593,-1.5);
      \draw (11.8519,-1.5)  -- (12.4444,-1.5);
      \draw (14.2222,-1.5)  -- (14.8148,-1.5);
      \draw (15.4074,-1.5)  -- (16,-1.5);
    
      %=============================================================
      % Step 4 (unlabeled), at y = -1.8  (was -2.1 previously)
      % Each segment is length ~0.5926.
      % Remove the middle third (~0.1975).
      %=============================================================
      % From [0, 0.5926]
      \draw (0,-1.8)        -- (0.1975,-1.8);
      \draw (0.3951,-1.8)   -- (0.5926,-1.8);
    
      % From [1.1852,1.7778]
      \draw (1.1852,-1.8)   -- (1.3827,-1.8);
      \draw (1.5802,-1.8)   -- (1.7778,-1.8);
    
      % From [3.5556,4.1482]
      \draw (3.5556,-1.8)   -- (3.7531,-1.8);
      \draw (3.9506,-1.8)   -- (4.1482,-1.8);
    
      % From [4.7408,5.3333]
      \draw (4.7408,-1.8)   -- (4.9383,-1.8);
      \draw (5.1358,-1.8)   -- (5.3333,-1.8);
    
      % From [10.6667,11.2593]
      \draw (10.6667,-1.8)  -- (10.8642,-1.8);
      \draw (11.0617,-1.8)  -- (11.2593,-1.8);
    
      % From [11.8519,12.4444]
      \draw (11.8519,-1.8)  -- (12.0494,-1.8);
      \draw (12.2469,-1.8)  -- (12.4444,-1.8);
    
      % From [14.2222,14.8148]
      \draw (14.2222,-1.8)  -- (14.4197,-1.8);
      \draw (14.6172,-1.8)  -- (14.8148,-1.8);
    
      % From [15.4074,16]
      \draw (15.4074,-1.8)  -- (15.6049,-1.8);
      \draw (15.8024,-1.8)  -- (16,-1.8);
    
      %=============================================================
      % Step 5 (unlabeled), at y = -2.1  (was -3.0 previously)
      % Each segment now length ~0.1975; remove ~0.0658 from middle.
      % So each segment spawns 2 smaller segments.
      %=============================================================
      % From [0.0000, 0.1975]
      \draw (0.0000,-2.1) -- (0.0658,-2.1);
      \draw (0.1317,-2.1) -- (0.1975,-2.1);
    
      % From [0.3951, 0.5926]
      \draw (0.3951,-2.1) -- (0.4609,-2.1);
      \draw (0.5268,-2.1) -- (0.5926,-2.1);
    
      % From [1.1852, 1.3827]
      \draw (1.1852,-2.1) -- (1.2510,-2.1);
      \draw (1.3169,-2.1) -- (1.3827,-2.1);
    
      % From [1.5802,1.7778]
      \draw (1.5802,-2.1) -- (1.6460,-2.1);
      \draw (1.7119,-2.1) -- (1.7778,-2.1);
    
      % From [3.5556,3.7531]
      \draw (3.5556,-2.1) -- (3.6214,-2.1);
      \draw (3.6873,-2.1) -- (3.7531,-2.1);
    
      % From [3.9506,4.1482]
      \draw (3.9506,-2.1) -- (4.0164,-2.1);
      \draw (4.0823,-2.1) -- (4.1482,-2.1);
    
      % From [4.7408,4.9383]
      \draw (4.7408,-2.1) -- (4.8066,-2.1);
      \draw (4.8725,-2.1) -- (4.9383,-2.1);
    
      % From [5.1358,5.3333]
      \draw (5.1358,-2.1) -- (5.2016,-2.1);
      \draw (5.2675,-2.1) -- (5.3333,-2.1);
    
      % From [10.6667,10.8642]
      \draw (10.6667,-2.1) -- (10.7325,-2.1);
      \draw (10.7984,-2.1) -- (10.8642,-2.1);
    
      % From [11.0617,11.2593]
      \draw (11.0617,-2.1) -- (11.1275,-2.1);
      \draw (11.1934,-2.1) -- (11.2593,-2.1);
    
      % From [11.8519,12.0494]
      \draw (11.8519,-2.1) -- (11.9177,-2.1);
      \draw (11.9836,-2.1) -- (12.0494,-2.1);
    
      % From [12.2469,12.4444]
      \draw (12.2469,-2.1) -- (12.3127,-2.1);
      \draw (12.3786,-2.1) -- (12.4444,-2.1);
    
      % From [14.2222,14.4197]
      \draw (14.2222,-2.1) -- (14.2880,-2.1);
      \draw (14.3539,-2.1) -- (14.4197,-2.1);
    
      % From [14.6172,14.8148]
      \draw (14.6172,-2.1) -- (14.6830,-2.1);
      \draw (14.7489,-2.1) -- (14.8148,-2.1);
    
      % From [15.4074,15.6049]
      \draw (15.4074,-2.1) -- (15.4732,-2.1);
      \draw (15.5391,-2.1) -- (15.6049,-2.1);
    
      % From [15.8024,16.0000]
      \draw (15.8024,-2.1) -- (15.8682,-2.1);
      \draw (15.9341,-2.1) -- (16.0000,-2.1);
    
      %=============================================================
      % Step 6 (unlabeled), at y = -2.4
      % Each segment in Step 5 is ~0.0658 in length,
      % remove the middle third (~0.0219).
      % => 64 segments total
      %=============================================================
      % [a, b] => [a, a+0.0219], [a+0.0439, b]
      % 
      % 1) [0.0000, 0.0658]
      \draw (0.0000,-2.4) -- (0.0219,-2.4);
      \draw (0.0439,-2.4) -- (0.0658,-2.4);
    
      % 2) [0.1317, 0.1975]
      \draw (0.1317,-2.4) -- (0.1536,-2.4);
      \draw (0.1756,-2.4) -- (0.1975,-2.4);
    
      % 3) [0.3951, 0.4609]
      \draw (0.3951,-2.4) -- (0.4170,-2.4);
      \draw (0.4390,-2.4) -- (0.4609,-2.4);
    
      % 4) [0.5268, 0.5926]
      \draw (0.5268,-2.4) -- (0.5487,-2.4);
      \draw (0.5707,-2.4) -- (0.5926,-2.4);
    
      % 5) [1.1852, 1.2510]
      \draw (1.1852,-2.4) -- (1.2071,-2.4);
      \draw (1.2291,-2.4) -- (1.2510,-2.4);
    
      % 6) [1.3169, 1.3827]
      \draw (1.3169,-2.4) -- (1.3388,-2.4);
      \draw (1.3608,-2.4) -- (1.3827,-2.4);
    
      % 7) [1.5802, 1.6460]
      \draw (1.5802,-2.4) -- (1.6021,-2.4);
      \draw (1.6241,-2.4) -- (1.6460,-2.4);
    
      % 8) [1.7119, 1.7778]
      \draw (1.7119,-2.4) -- (1.7338,-2.4);
      \draw (1.7558,-2.4) -- (1.7778,-2.4);
    
      % 9) [3.5556, 3.6214]
      \draw (3.5556,-2.4) -- (3.5775,-2.4);
      \draw (3.5995,-2.4) -- (3.6214,-2.4);
    
      % 10) [3.6873, 3.7531]
      \draw (3.6873,-2.4) -- (3.7092,-2.4);
      \draw (3.7312,-2.4) -- (3.7531,-2.4);
    
      % 11) [3.9506, 4.0164]
      \draw (3.9506,-2.4) -- (3.9725,-2.4);
      \draw (3.9945,-2.4) -- (4.0164,-2.4);
    
      % 12) [4.0823, 4.1482]
      \draw (4.0823,-2.4) -- (4.1042,-2.4);
      \draw (4.1262,-2.4) -- (4.1482,-2.4);
    
      % 13) [4.7408, 4.8066]
      \draw (4.7408,-2.4) -- (4.7627,-2.4);
      \draw (4.7847,-2.4) -- (4.8066,-2.4);
    
      % 14) [4.8725, 4.9383]
      \draw (4.8725,-2.4) -- (4.8944,-2.4);
      \draw (4.9164,-2.4) -- (4.9383,-2.4);
    
      % 15) [5.1358, 5.2016]
      \draw (5.1358,-2.4) -- (5.1577,-2.4);
      \draw (5.1797,-2.4) -- (5.2016,-2.4);
    
      % 16) [5.2675, 5.3333]
      \draw (5.2675,-2.4) -- (5.2894,-2.4);
      \draw (5.3114,-2.4) -- (5.3333,-2.4);
    
      % 17) [10.6667, 10.7325]
      \draw (10.6667,-2.4) -- (10.6886,-2.4);
      \draw (10.7106,-2.4) -- (10.7325,-2.4);
    
      % 18) [10.7984, 10.8642]
      \draw (10.7984,-2.4) -- (10.8203,-2.4);
      \draw (10.8423,-2.4) -- (10.8642,-2.4);
    
      % 19) [11.0617, 11.1275]
      \draw (11.0617,-2.4) -- (11.0836,-2.4);
      \draw (11.1056,-2.4) -- (11.1275,-2.4);
    
      % 20) [11.1934, 11.2593]
      \draw (11.1934,-2.4) -- (11.2153,-2.4);
      \draw (11.2373,-2.4) -- (11.2593,-2.4);
    
      % 21) [11.8519, 11.9177]
      \draw (11.8519,-2.4) -- (11.8738,-2.4);
      \draw (11.8958,-2.4) -- (11.9177,-2.4);
    
      % 22) [11.9836, 12.0494]
      \draw (11.9836,-2.4) -- (12.0055,-2.4);
      \draw (12.0275,-2.4) -- (12.0494,-2.4);
    
      % 23) [12.2469, 12.3127]
      \draw (12.2469,-2.4) -- (12.2688,-2.4);
      \draw (12.2908,-2.4) -- (12.3127,-2.4);
    
      % 24) [12.3786, 12.4444]
      \draw (12.3786,-2.4) -- (12.4005,-2.4);
      \draw (12.4225,-2.4) -- (12.4444,-2.4);
    
      % 25) [14.2222, 14.2880]
      \draw (14.2222,-2.4) -- (14.2441,-2.4);
      \draw (14.2661,-2.4) -- (14.2880,-2.4);
    
      % 26) [14.3539, 14.4197]
      \draw (14.3539,-2.4) -- (14.3758,-2.4);
      \draw (14.3978,-2.4) -- (14.4197,-2.4);
    
      % 27) [14.6172, 14.6830]
      \draw (14.6172,-2.4) -- (14.6391,-2.4);
      \draw (14.6611,-2.4) -- (14.6830,-2.4);
    
      % 28) [14.7489, 14.8148]
      \draw (14.7489,-2.4) -- (14.7708,-2.4);
      \draw (14.7928,-2.4) -- (14.8148,-2.4);
    
      % 29) [15.4074, 15.4732]
      \draw (15.4074,-2.4) -- (15.4293,-2.4);
      \draw (15.4513,-2.4) -- (15.4732,-2.4);
    
      % 30) [15.5391, 15.6049]
      \draw (15.5391,-2.4) -- (15.5610,-2.4);
      \draw (15.5830,-2.4) -- (15.6049,-2.4);
    
      % 31) [15.8024, 15.8682]
      \draw (15.8024,-2.4) -- (15.8243,-2.4);
      \draw (15.8463,-2.4) -- (15.8682,-2.4);
    
      % 32) [15.9341, 16.0000]
      \draw (15.9341,-2.4) -- (15.9560,-2.4);
      \draw (15.9780,-2.4) -- (16.0000,-2.4);
    
    \end{tikzpicture}
    \end{center}
    \vspace{35pt}

    We are interested in studying the topological properties of the points which are not deleted at any step of this infinite process. The set of these points have a special name and are defined below.

    \begin{definition}
        Let $C_0 := [0,1]$ and $C_n = \frac{C_{n-1}}{3} \cup \left( \frac{2}{3} + \frac{C_{n-1}}{3}\right)$ for $n \geq 1$. The \textit{Cantor set} is $\fC = \bigcap_{n = 0}^\infty C_n$.
    \end{definition}

    \begin{proposition}
        The Cantor set is closed.
    \end{proposition}
        \begin{proof}
            Since the Cantor set is defined as the intersection of closed sets, it must be closed.
        \end{proof}

    \begin{proposition}
        The Cantor set is nowhere dense.
    \end{proposition}
        \begin{proof}
            Suppose towards contradiction its not, that is, $\overline{\fC}^o \neq \emptyset$. Then there is some $x \in \overline{\fC}^o$. We can find an $\epsilon > 0$ with $(x- \epsilon, x+\epsilon) \subseteq \fC$, in particular $(x- \epsilon, x+\epsilon) \subseteq C_n$ for all $n \geq 1$. Find $m$ large so that $\epsilon > \frac{1}{3^m}$ and consider $(x-\epsilon,x+\epsilon) \subseteq C_m$. We have that $C_m = \bigsqcup_{j = 1}^{2^m}C_{m,j}$ where $\text{length}(C_{m,j}) = \frac{1}{3^m}$. Since each $C_{m,j}$ is disjoint, it must be the case that $(x-\epsilon,x+\epsilon) \subseteq C_{m,j}$ for some $1 \leq j \leq 2^m$. But the length of $(x-\epsilon,x+\epsilon)$ is $2\epsilon$, which is impossible. It must be that $\fC$ is nowhere dense.
        \end{proof}
    
    \begin{proposition}
        The total length of the Cantor set is $0$.
    \end{proposition}
        \begin{proof}
            The total length of the removed intervals is:
                \begin{equation*}
                \begin{split}
                    \frac{1}{3} + \frac{2}{9} + \frac{4}{27} + ... 
                    & = \sum_{k=1}^\infty \frac{2^{k-1}}{3^k} \\
                    & = \frac{1}{2} \sum_{k=1}^\infty \left( \frac{2}{3} \right)^k \\
                    & = \frac{1}{2} \cdot \frac{\sfrac{2}{3}}{1 - \sfrac{2}{3}} \\
                    & = 1.
                \end{split}
                \end{equation*}
            Thus $\text{length}(\fC) = 0$.
        \end{proof}

    \begin{lemma}***
        
    \end{lemma}
        \begin{proof}
            
        \end{proof}

    \begin{lemma}***
    
    \end{lemma}
        \begin{proof}
            
        \end{proof}

    \begin{lemma}***

    \end{lemma}
        \begin{proof}
            
        \end{proof}

    \begin{proposition}***
        $\card(\fC) = \fc$.
    \end{proposition}
        \begin{proof}
            
        \end{proof}

\section{Convergent Sequences}\label{sec:conv-seq}
    \begin{definition}
        Let $(X,d)$ be a metric space.
        \begin{enumerate}[label = (\arabic*),itemsep=1pt,topsep=3pt]
            \item A \textit{sequence} in $X$ is a map $x_{\bullet}:\bfN \rightarrow X$ defined by $n \mapsto x_n$. We denote a sequence as $(x_n)_{n \geq 1}$, $(x_n)_{n = 1}^\infty$, or $(x_n)_n$.
            
            \item A \textit{natural sequence} is a sequence $(x_n)_n$ in $\bfN$ with $n_1 < n_2 < n_3 < ...$
            
            \item A \textit{subsequence} of a sequence $(x_n)_n$ is a sequence $(x_{n_k})_k$, where $(n_k)_k$ is a natural sequence. This is equivalent to the composition of maps
                    \begin{tikzcd}
                        \bfN \arrow[r, "k \mapsto n_k"] & \bfN \arrow[r, "n_k \mapsto x_{n_k}"] & X.
                    \end{tikzcd}
        \end{enumerate}
    \end{definition}

    \begin{definition}

        A sequence $(x_n)_n$ \textit{converges} to $x \in X$ if:
            \begin{equation*}
            \begin{split}
                (\forall \epsilon > 0)(\exists N \in \bfN):(\forall n \in \bfN)(n \geq N \implies d(x_n,x) < \epsilon).
            \end{split}
            \end{equation*}
        We write $(x_n)_n \xrightarrow{d} x$ or $\limit x_n = x$.
    \end{definition}

    \begin{exercise}
        Show that a sequence can have at most one limit.
    \end{exercise}

    \begin{proposition}\label{prop:sequence-equivalences}
        Let $(x_n)_n$ be a sequence in $X$ and $x \in X$. The following are equivalent:
        \begin{enumerate}[label = (\arabic*),itemsep=1pt,topsep=3pt]
            \item $(x_n)_n \rightarrow x$;
            \item $(d(x_n,x))_n \rightarrow 0$ in $\bfR$;
            \item $(\forall V \in \cN_x)(\exists N \in \bfN):(\forall n\in \bfN)(n \geq N \implies x_n \in V)$. 
        \end{enumerate}
    \end{proposition}
        \begin{proof}
            Exercise.
        \end{proof}

    \begin{exercise}
        Let $(X,d)$ be a metric space and $\rho(x,y) = \frac{d(x,y)}{1 + d(x,y)}$. Then $(x_n)_n \xrightarrow{d} x$ if and only if $(x_n)_n \xrightarrow{\rho} x$.
    \end{exercise}

    \begin{proposition}
        Convergent sequences are bounded.
    \end{proposition}
        \begin{proof}
            Suppose that $(x_n)_n \rightarrow x$ and let $\epsilon = 1$. Find $N$ large so that for $n \geq N$ we have $d(x_n,x) < 1$. Then for all $m,n \geq N$, we have $d(x_m,x_n) \leq d(x_m,x) + d(x,x_n) < 2$. Set $C = \max_{1 \leq n,m \leq N}d(x_m,x_n)$. Now if $n \geq N$ and $m \leq N$, we have:
                \begin{equation*}
                \begin{split}
                    d(x_n,x_m)
                    & \leq d(x_n,x_N) + d(x_N,x_m) \\
                    & \leq 1 + C.
                \end{split}
                \end{equation*}
            Let $K = \max\{2,1+C,C\}$. Then $\diam(\{x_n\}_{n \geq 1}) = \sup_{m,n \geq 1}d(x_n,x_m) \leq K$.
        \end{proof}

        \begin{definition}
            Let $(v_k)_k$ be a sequence in $(V,\lnorm \cdot \rnorm)$.
            \begin{enumerate}[label = (\arabic*),itemsep=1pt,topsep=3pt]
                \item A \textit{sequence of partial sums} $(s_n)_n$ is defined as $s_n = \sum_{k = 1}^n v_k$.
                \item If $(s_n)_n \rightarrow s$ in $V$ we say the \textit{series} $\sum_{k = 1}^\infty v_k$ converges and write $\sum_{k = 1}^\infty v_k = s$.
                \item The series $\sum v_k$ converges \textit{absolutely} if $\sum \lnorm v_k \rnorm$ converges.
            \end{enumerate}
        \end{definition}

    \begin{center}
        \begin{tikzpicture}
            \draw[thick] (0.3,0) -- (2.3,0);
            \node at (2.39, 0) {$/\,$};
            \node at (2.56, 0) {$/\,$};
            \draw[thick] (2.6,0) -- (4.6,0);
        \end{tikzpicture}
    \end{center}

    \begin{definition}
        Let $\Omega$ be a set, $(X,d)$ a metric space, and $(f_n)_n$ a sequence of functions in $X^\Omega$.
        \begin{enumerate}[label = (\arabic*),itemsep=-7pt,topsep=3pt]
            \item $(f_n)_n$ converges \textit{pointwise} to $f \in X^\Omega$ if:
                \begin{equation*}
                \begin{split}
                    (\forall x \in \Omega)(\forall \epsilon > 0)(\exists N_{x,\epsilon} \in \bfN) : (\forall n \in \bfN)( n \geq N \implies d(f_n(x), f(x)) < \epsilon).
                \end{split}
                \end{equation*}
            \item $(f_n)_n$ converges \textit{uniformly} to $f \in X^\Omega$ if:
                \begin{equation*}
                \begin{split}
                        (\forall \epsilon > 0)(\exists N_\epsilon \in \bfN) &\h4: (\forall n \in \bfN)(\forall x \in \Omega)\bigl(n \geq N \implies d(f_n(x), f(x)) < \epsilon\bigr) \\
                        &\equiv (\forall n \in \bfN)\bigl(n \geq N \implies {\ts \sup_{x \in \Omega}d(f_n(x),f(x))} < \epsilon\bigr).
                \end{split}
                \end{equation*}
        \end{enumerate}
    \end{definition}

    \begin{proposition}\label{prop:uniform-iff-bounded-norm}
        Let $\Omega$ be a set, $(X,d)$ a metric space, and $(f_n)_n$ a sequence in $X^\Omega$. The following are equivalent:
            \begin{enumerate}[label = (\arabic*),itemsep=1pt,topsep=3pt]
                \item The sequence $(f_n)_n$ converges uniformly to $f$ in $X^\Omega$;
                \item The sequence $(D_u(f_n,f))_n$ converges to 0 in $\bfR$.
            \end{enumerate}
    \end{proposition}
        \begin{proof}
            ($\Rightarrow$) Let $\epsilon > 0$. By assumption, find $N \in \bfN$ sufficiently large so that, for all $x \in X$, $n \geq N$ implies $d(f_n(x),f(x)) < \frac{\epsilon}{2}$. It follows then that:
                \begin{equation*}
                \begin{split}
                    |D_u(f_n,f)|
                    & = \left| \sup_{x \in X}d(f_n(x),f(x)) \right| \\
                    & \leq \frac{\epsilon}{2} \\
                    & < \epsilon.
                \end{split}
                \end{equation*}
            Thus $(D_u(f_n,f))_n \rightarrow 0$.

            ($\Leftarrow$) Let $\epsilon > 0$. Find $N \in \bfN$ large so that $n \geq N$ implies $D_u(f_n,f) < \epsilon$. If $n \geq N$ and $z \in X$, then:
                \begin{equation*}
                \begin{split}
                    d(f_n(z),f(z))
                    & \leq \sup_{x \in X}d(f_n(x),f(x)) \\
                    & < \epsilon.
                \end{split}
                \end{equation*}
            Since $z$ was arbitrary, $(f_n)_n \rightarrow f$ uniformly.
        \end{proof}

    \begin{example}
        Let $(X,d)$ be a metric space and $(f_n)_n$ a sequence of functions in $\bigl(\Bd(\Omega,X), D_u\bigr)$ which converges to $f \in \Bd(\Omega,X)$. Proposition~\ref{prop:sequence-equivalences} says this is equivalent to $(D_u(f_n,f))_n \rightarrow 0$ in $\bfR$, and by Proposition~\ref{prop:uniform-iff-bounded-norm} $(f_n)_n$ must converge uniformly to $f$. Thus convergence in $c_{00} \subseteq c_0 \subseteq c \subseteq \ell_\infty \subseteq \ell_{\infty}(\Omega) = \Bd(\Omega,F)$ is uniform.
    \end{example}

    \begin{proposition}
        Let $\{d_k\}_k$ be a separating family of semi-metrics which are uniformly bounded. Define:
            \begin{equation*}
            \begin{split}
                d(x,y) := \sum_{k = 1}^\infty 2^{-k}d_k(x,y).
            \end{split}
            \end{equation*}
        Then $(x_n)_n \xrightarrow{d} x$ if and only if $(d_k(x_n,x))_n \rightarrow 0$ for all $k$.
    \end{proposition}
        \begin{proof}
            ($\Rightarrow$) Let $k \geq 1$ be arbitrary. We have:
                \begin{equation*}
                \begin{split}
                    0 \leq 2^{-k}d_k(x_n,x) \leq d(x_n,x).
                \end{split}
                \end{equation*}
            Multiplying all sides of the above equation by $2^k$ gives $0 \leq d_k(x_n,x) \leq 2^k d(x_n,x)$. Since $\limit_{n \rightarrow \infty}2^kd(x_n,x) = 0$, by the Squeeze Theorem $(d_k(x_n,x))_n \rightarrow 0$.

            ($\Leftarrow$) Let $\epsilon > 0$. For all $x,y,k$, there exists $C > 0$ with $d_k(x,y) \leq C$. Since $\sum_{k = 1}^\infty 2^{-k}$ converges (in particular, it converges to 1), find $K$ large so:
                \begin{equation*}
                \begin{split}
                    \sum_{k > K}2^{-k} < \frac{\epsilon}{2C}.
                \end{split}
                \end{equation*}
            Note that $(d_k(x_n,x))_n \rightarrow 0$ for $k = 1,2,...,K$. So there exists $N_1,N_2,...,N_K \in \bfN$ with $n \geq N_j$ implying $d_j(x_n,x)<\frac{\epsilon}{2}$ for $1 \leq j \leq K$. Let $N = \max_{j = 1}^K N_j$. For $n \geq N$, we have:
                \begin{equation*}
                \begin{split}
                    d(x_n,x) 
                    & = \sum_{k = 1}^K 2^{-k}d_k(x_n,x) + \sum_{k > K}2^{-k}d_k(x_n,x) \\
                    & = \sum_{k = 1}^K 2^{-k}d_k(x_n,x) + \sum_{k > K}2^{-k} C \\
                    & < \sum_{k = 1}^K 2^{-k} \frac{\epsilon}{2} + C \cdot \frac{\epsilon}{2C} \\
                    & \leq \frac{\epsilon}{2} + \frac{\epsilon}{2} \\
                    & = \epsilon. \qedhere
                \end{split}
                \end{equation*}
            Thus $(x_n)_n \xrightarrow{d} x$.
            \end{proof}

    \begin{example}
        Consider the space $C(\bfR)$. How does one define a distance between two functions? Given $f,g \in C(\bfR)$, note that the uniform metric:
            \begin{equation*}
            \begin{split}
                d(f,g) := \sup_{x \in \Omega}|f(x) - g(x)|
            \end{split}
            \end{equation*}
        does not guarantee $d(f,g) < \infty$. We can fix this as follows: define $\rho_k(f,g) := \sup_{x \in [-k,k]}|f(x) - g(x)|$. Note that the family of metrics $\{\rho_k\}_k$ is separating, but not uniformly bounded. Defining $d_k(f,g):= \frac{\rho_k(f,g)}{1 + \rho_k(f,g)}$ gives $\{d_k\}_k$ as a family of uniformly bounded semi-metrics. We can now define what is called the \textit{Fr\'echet metric}:
            \begin{equation*}
            \begin{split}
                d_F(f,g) := \sum_{k = 1}^\infty 2^{-k}d_k(f,g).
            \end{split}
            \end{equation*}
        By the comparison test, $d_F(f,g) < \infty$. In $(C(\bfR),d_F)$, observe that:
            \begin{equation*}
            \begin{split}
                (f_n)_n \xrightarrow{d_F}f 
                &\iff \forall k, (d_k(f_n,f))_n \rightarrow 0 \\
                &\iff \forall k, (\rho_k(f_n,f))_n \rightarrow 0 \\
                &\iff \forall k, \Bigl(\ts \sup_{x \in [-k,k]}|f_n(x) - f(x)|\Bigr)_n \rightarrow 0 \\
                &\iff \forall k, (f_n)_n \rightarrow f \h5\text{uniformly on $[-k,k]$}.
            \end{split}
            \end{equation*}
        We've obtained a new type of convergence called \textit{compact convergence}.
    \end{example}


    \iffalse
    \begin{center}
        \begin{tikzpicture}
            \draw[thick] (0.3,0) -- (2.3,0);
            \node at (2.39, 0) {$/\,$};
            \node at (2.56, 0) {$/\,$};
            \draw[thick] (2.6,0) -- (4.6,0);
        \end{tikzpicture}
    \end{center}

    \begin{example}
        Let $(X,d)$, $(Y,\rho)$ be metric spaces. There are different metrics we may put on $X \times Y$, for example:
            \begin{itemize}
                \item $D_1((x,y),(x',y')) = d(x,x') + \rho(y,y')$ 
                \item $D_2((x,y),(x',y')) = \sqrt{d(x,x')^2 + \rho(y,y')^2}$
                \item $D_\infty((x,y),(x',y')) = \max\{d(x,x') + \rho(y,y')\}$.
            \end{itemize}
        These metrics are all equivalent. A sequence of points $(x_n,y_n)_n$ converges to $(x,y)$ in $X \times Y$ with respect to any of these metrics if and only if $(x_n)_n \xrightarrow{d} x$ and $(y_n)_n \xrightarrow{\rho} y$.
    \end{example}
    \fi

    \iffalse
    \begin{example}
        Let $\{(X_k,d_k)_k\}_k$ be a family of metric spaces where the $d_k$ are uniformly bounded. We looked at the product:
            \begin{equation*}
            \begin{split}
                X = \prod_{k = 1}^\infty X_k
            \end{split}
            \end{equation*}
        with:
            \begin{equation*}
            \begin{split}
                d(f,g) = \sum_{k = 1}^\infty 2^{-k}d_k(f(k),g(k)).
            \end{split}
            \end{equation*}
        We have that $(f_n)_n \xrightarrow{d} f$ if and only if $(f_n(k))_n \rightarrow f(k)$ for all $k$ (pointwise convergence).
    \end{example}

    \begin{example}
        Let $(V,\lnorm \cdot \rnorm)$ be a normed space. Then $(v_n)_n \rightarrow v$ if and only if $(\lnorm v_n - v \rnorm)_n \rightarrow 0$.
    \end{example}
    \fi

    \begin{center}
        \begin{tikzpicture}
            \draw[thick] (0.3,0) -- (2.3,0);
            \node at (2.39, 0) {$/\,$};
            \node at (2.56, 0) {$/\,$};
            \draw[thick] (2.6,0) -- (4.6,0);
        \end{tikzpicture}
    \end{center}

    \begin{proposition}\label{prop:1}
        Let $(X,d)$ be a metric space and $A \subseteq X$. We have $x \in \overline{A}$ if and only if there exists a sequence $(a_n)_n$ in $A$ with $(a_n)_n \rightarrow x$.
    \end{proposition}
        \begin{proof}
            ($\Rightarrow$) If $x \in \overline{A}$, then for each $n \geq 1$ we have $U(x,\frac{1}{n}) \cap A \neq \emptyset$. For each $n$ choose $a_n \in U(x,\frac{1}{n}) \cap A$. Then $d(x,a_n)<\frac{1}{n}$, so $(a_n)_n \rightarrow x$. ($\Leftarrow$) Given $\epsilon > 0$, find $N$ large so $d(x,a_N) < \epsilon$. Then $a_N \in U(x,\epsilon) \cap A$. Since $U(x,\epsilon) \cap A \neq \emptyset$, we have $x \in \overline{A}$.
        \end{proof}

    \begin{proposition}\label{prop:closed-iff-seq}
        Let $(X,d)$ be a metric space and $A \subseteq X$. The following are equivalent:
            \begin{enumerate}[label = (\arabic*),itemsep=1pt,topsep=3pt]
                \item $A$ is closed;
                \item If $(a_n)_n$ is a sequence in $A$ which converges to $x \in X$, then $x \in A$.
            \end{enumerate}
    \end{proposition}
        \begin{proof}
            Let $(a_n)_n$ be a sequence in $A$ which converges to $x \in X$. By Proposition~\ref{prop:closed-iff-seq} $x \in \overline{A}$. Since $A$ is closed, $A = \overline{A}$. Thus $x \in A$.

            We will show $A$ is closed by proving $A = \overline{A}$ Clearly $A \subseteq \overline{A}$. Let $x \in \overline{A}$. Then there exists a sequence $(a_n)_n$ in $A$ with $(a_n)_n \rightarrow x$. Thus $x \in A$, giving $\overline{A} \subseteq A$.
        \end{proof}

    \begin{exercise}
        Show $x \in \overline{A}$ if and only if $x \in A$ or there exists a sequence $(a_n)_n$ in $A \setminus\{a\}$ with $(a_n)_n \rightarrow x$.
    \end{exercise}
    
    \newpage

    \begin{proposition}***
        \phantom{a}
        \begin{enumerate}[label = (\arabic*),itemsep=1pt,topsep=3pt]
            \item The space $c_0 \subseteq \ell_\infty$ is closed.
            \item $\overline{c_{00}} = c_0$.
        \end{enumerate}
    \end{proposition}
        \begin{proof}
            (1) Let $(z_n)_n$ be a sequence in $c_0$ converging to $f \in \ell_\infty$. Let $\epsilon > 0$. Find $N$ large so that:
                \begin{equation*}
                \begin{split}
                    \lnorm z_N - f \rnorm_u < \frac{\epsilon}{2}.
                \end{split}
                \end{equation*}
            Since $z_N \in c_0$, we know $\limit_{k \rightarrow \infty} z_N(k) = 0$. Find $K$ large so that for $k \geq K$:
                \begin{equation*}
                \begin{split}
                    |z_N(k)| < \frac{\epsilon}{2}.
                \end{split}
                \end{equation*}
            Together, for $k \geq K$ we have:
                \begin{equation*}
                \begin{split}
                    |f(k)|
                    & = |f(k) - z_N(k) + z_N(k)| \\
                    & \leq |f(k) - z_N(k)| + |z_N(k)| \\
                    & < \frac{\epsilon}{2} + \frac{\epsilon}{2} \\
                    & = \epsilon.
                \end{split}
                \end{equation*}
            Thus $\limit_{k \rightarrow \infty}f(k) = 0$; i.e., $f \in c_0$. By Proposition~\ref{prop:closed-iff-seq}, $c_0$ is closed in $\ell_\infty$.
        \end{proof}

    \newpage

    \begin{definition}
        Let $(X,d)$ be a metric space and $A \subseteq X$. The \textit{distance} of an element $x \in X$ to $A$ is defined as the map $\dist_A:X \rightarrow [0,\infty)$ given by $\dist_A(x) = \inf_{a \in A}d(x,a)$.
    \end{definition}

    \begin{proposition}
        Let $(X,d)$ be a metric space and $A \subseteq X$.
        \begin{enumerate}[label = (\arabic*),itemsep=1pt,topsep=3pt]
            \item $\overline{A} = \{x \mid \dist_A(x) = 0 \}$.
            \item $\dist_A = \dist_{\overline{A}}$.
            \item $\left| \dist_A(x) - \dist_A(y) \right| \leq d(x,y).$
        \end{enumerate}
    \end{proposition}
        \begin{proof}
            (1) Let $D:=\{x \mid \dist_A(x) = 0\}$. By Proposition~\ref{prop:1}, $x \in \overline{A}$ if and only if there exists a sequence $(a_n)_n$ in $A$ which converges to $x$. Equivalently, the seqence $(d(x,a_n))_n$ converges to $0$ in $\bfR$. Since $\inf_{a \in A}d(x,a) \leq d(x,a_n)$, it must be the case that $\dist_A(x) = 0$. Thus $\overline{A} \subseteq D$. Now suppose $x \in D$. Then $\inf_{a \in A}d(x,a) = 0$ implies the existance of a sequence $(a_n)_n$ in $A$ with $(d(x,a_n))_n \rightarrow 0$. Thus $(a_n)_n \rightarrow x$, establishing that $x \in \overline{A}$ by Proposition~\ref{prop:1}. Thus $D = \overline{A}$.

            (2) Let $x \in X$. Since $A \subseteq \overline{A}$, it is clear that:
                \begin{equation*}
                \begin{split}
                    \dist_{\overline{A}}(x) 
                    & = \inf_{a \in \overline{A}} d(x,a) \\
                    & \leq \inf_{a \in A}d(x,a) \\
                    & = \dist_A(x).
                \end{split}
                \end{equation*}
            Let $\epsilon > 0$ and $y \in \overline{A}$. From part (1), we have that $\dist_A(y) = 0$. In particular, $\dist_A(y) < \epsilon$. So there exists $a \in A$ such that $d(a,y) < \epsilon$. Observe that:
                \begin{equation*}
                \begin{split}
                    \dist_A(x) 
                    & = \inf_{a \in A}d(x,a) \\
                    & \leq d(x,a) \\
                    & \leq d(x,y) + d(y,a) \\
                    & < d(x,y) + \epsilon.
                \end{split}
                \end{equation*}
            It follows that $\dist_A(x) \leq d(x,y)$ for all $y \in \overline{A}$. Finally:
                \begin{equation*}
                \begin{split}
                    \dist_A(x) 
                    & \leq \inf_{y \in \overline{A}}d(x,y) \\
                    & = \dist_{\overline{A}}(x).
                \end{split}
                \end{equation*}
            Thus $\dist_A = \dist_{\overline{A}}$.

            (3)
        \end{proof}

\newpage
\section{Continuity}\label{sec:continuity}
    \begin{definition}\label{def:2.5}
        Let $(X,d)$ and $(Y,\rho)$ be metric spaces. A map $f:X \rightarrow Y$ is \textit{continuous at $x_0 \in X$} if any of the equivalent definitions are satisfied:
            \begin{enumerate}[label = (\arabic*),itemsep=1pt,topsep=3pt]
                \item $(\forall \epsilon > 0)(\exists \delta > 0):(\forall x \in X)\Bigl(d(x,x_0) < \delta \implies \rho\bigl(f(x),f(x_0)\bigr) < \epsilon\Bigr)$
                \item $(\forall \epsilon > 0)(\exists \delta > 0):(\forall x \in X)\Bigl(x \in U_X(x_0, \delta) \implies f(x) \in U_Y\bigl(f(x_0),\epsilon\bigr)\Bigr)$
                \item $(\forall \epsilon > 0)(\exists \delta > 0):f\bigl(U_X(x_0,\delta)\bigr) \subseteq U_Y\bigl(f(x_0),\epsilon\bigr)$
            \end{enumerate}
    \end{definition}

    \begin{proposition}
        Let $f:(X,d) \rightarrow (Y,\rho)$ be a map between metric spaces and $x_0 \in X$. The following are equivalent:
            \begin{enumerate}[label = (\arabic*),itemsep=1pt,topsep=3pt]
                \item $f$ is continuous at $x_0$;
                \item $(\forall V \in \cN_{f(x_0)})(\exists U \in \cN_{x_0}):f(U) \subseteq V$;
                \item $(\forall (x_n)_n \in X^\bfN)\bigl((x_n)_n \rightarrow x_0 \implies (f(x_n))_n \rightarrow f(x_0)\bigr)$
            \end{enumerate}
    \end{proposition}
        \begin{proof}
            (1)$\Rightarrow$(2) follows from Definition~\ref{def:2.5}. 
            
            (1)$\Rightarrow$(3) Let $(x_n)_n \rightarrow x_0$. Let $\epsilon > 0$. Since $f$ is continuous, find $\epsilon > 0$ so that $x \in U(x,\delta)$ implies $f(x) \in U(f(x_0),\epsilon)$. Whence $d(f(x_n),f(x_0)) < \epsilon$, establishing $(f(x_n))_n \rightarrow f(x_0)$.

            (3)$\Rightarrow$(1) We prove the contrapositive of this statement. If $f$ is not continuous, choose $\epsilon_0>0$ so that $d(x_n,x_0) < \frac{1}{n}$ and $d(f(x_n),f(x_0)) \geq \epsilon_0$. Whence $(x_n)_n \rightarrow x_0$ and $(f(x_n))_n \not\rightarrow f(x_0)$.
        \end{proof}

    \begin{proposition}
        Let $f:(X,d) \rightarrow (Y,\rho)$ be a map of metric spaces. The following are equivalent:
            \begin{enumerate}[label = (\arabic*),itemsep=1pt,topsep=3pt]
                \item $f$ is continuous;
                \item $(\forall V \in \tau_Y),f^{-1}(V) \in \tau_X$;
            \end{enumerate}
    \end{proposition}
        \begin{proof}
            Let $V \subseteq Y$ be open. If $f^{-1}(V) = \emptyset$, we're done. If not, let $x \in f^{-1}(V)$. Then $f(x) \in V$. Since $V$ is open, find $\epsilon > 0$ so that $U(f(x),\epsilon) \subseteq V$. Since $f$ is continuous, find $\delta > 0$ so that $f(U(x,\delta)) \subseteq U(f(x),\epsilon) \subseteq V$. Whence $U(x,\delta) \subseteq f^{-1}(V)$.

            Let $x \in X$ and $\epsilon > 0$. Since $U(f(x),\epsilon) \in \tau_Y$, we have $f^{-1}(U(f(x)),\epsilon) \in \tau_X$. Note that $x \in f^{-1}(U(f(x)),\epsilon)$. Since this set is open, find $\delta > 0$ so that $U(x,\delta) \subseteq f^{-1}(U(f(x)),\epsilon)$. Thus $f(U(x,\delta)) \subseteq U(f(x),\epsilon)$; i.e., $f$ is continuous.
        \end{proof}

    \begin{proposition}***
        Let $(X,d) \xrightarrow{f} (Y,\rho) \xrightarrow{g} (Z,\gamma)$ be maps of metric spaces. If $f$ is continuous at $x \in X$ and $g$ is continuous at $y = f(x)$, then $g \circ f$ is continuous at $x$.
    \end{proposition}
        \begin{proof}
            Exercise.
        \end{proof}

    \begin{proposition}
        Let $(X,d)$ be a metric space with $A \subseteq X$ dense. Let $f:X \rightarrow F$ be a continuous and bounded function. Then $\sup_{x \in A}f(x) = \sup_{x \in X}f(x)$.
    \end{proposition}
        \begin{proof}
            Since $A \subseteq X$ we have $\sup_{x \in A}f(x) \leq \sup_{x \in X}f(x)$. Conversely, let $\epsilon > 0$. Find $x' \in X$ so that $\sup_{x \in X}f(x) - \frac{\epsilon}{2} < f(x')$. Since $f$ is continuous, find $\delta > 0$ so that for all $x \in X$, $d(x,x') < \delta$ implies $|f(x)-f(x')| < \frac{\epsilon}{2}$. Since $A$ is dense, find $a \in A$ so that $d(x',a) < \delta$. This implies $|f(x')-f(a)| < \frac{\epsilon}{2}$, or equivalently $f(x')- \frac{\epsilon}{2} < f(a)$. This gives:
                \begin{equation*}
                \begin{split}
                    \sup_{x \in X}f(x) - \frac{\epsilon}{2}
                    & < f(x') \\
                    & <f(a) + \frac{\epsilon}{2}.
                \end{split}
                \end{equation*}
            Upon simplifying, we have:
                \begin{equation*}
                \begin{split}
                    \sup_{x \in X}f(x) - \epsilon 
                    & < f(a).
                \end{split}
                \end{equation*}
            Hence for $\epsilon > 0$ we have:
                \begin{equation*}
                \begin{split}
                    \sup_{x \in X}f(x)< \sup_{x \in A}f(x) + \epsilon.
                \end{split}
                \end{equation*}
            Taking $\epsilon \rightarrow 0$ gives $\sup_{x \in X}f(x) \leq \sup_{x \in A}f(x)$.
        \end{proof}

    \begin{definition}
        Let $f:(X,d) \rightarrow (Y,\rho)$ be a map of metric spaces.
        \begin{enumerate}[label = (\arabic*),itemsep=1pt,topsep=3pt]
            \item We say $f$ is \textit{uniformly continuous} if:
                \begin{equation*}
                \begin{split}
                    (\forall \epsilon > 0)(\exists \delta > 0): (\forall x,y \in X)\bigl(d(x,y) < \delta \implies \rho(f(x),f(y)) < \epsilon\bigr).
                \end{split}
                \end{equation*}
            \item We say $f$ is \textit{Lipschitz} if:
                \begin{equation*}
                \begin{split}
                    (\exists C > 0):(\forall x,y \in X)\bigl(\rho(f(x),f(y)) \leq C d(x,y)\bigr).
                \end{split}
                \end{equation*}
            If $C < 1$, we say $f$ is \textit{contractive}.

            \item We say $f$ is an \textit{isometry} if:
                \begin{equation*}
                \begin{split}
                    (\forall x,x' \in X)\bigl(\rho(f(x),f(x')) = d(x,x')\bigr)
                \end{split}
                \end{equation*}
        \end{enumerate}
    \end{definition}

    \begin{exercise}***
        Show that Lipschitz implies uniform continuity. Show that uniform continuity implies continuity. Show that the converse direction fails in general.
    \end{exercise}

    \begin{example}
        Let $(V, \lnorm \cdot \rnorm)$ be a normed space. Then $V \xrightarrow{\lnorm \cdot \rnorm} [0,\infty)$ is continuous. Indeed, we have $\left| \lnorm v \rnorm - \lnorm w \rnorm \right| \leq \lnorm v - w \rnorm$, so $\lnorm \cdot \rnorm$ is Lipschitz.
    \end{example}

    \begin{example}
        Let $(X,d)$ be a metric space and equip $X \times X$ with the product metric $D_1$. Claim: $d:X \times X \rightarrow [0,\infty)$ is continuous. Indeed, given $(x,y),(x',y') \in X \times X$, then we have:
            \begin{equation*}
            \begin{split}
                d(x,y) \leq d(x,x') + d(x',y') + d(y',y).
            \end{split}
            \end{equation*}
        So we have $d(x,y) - d(x',y') \leq d(x,x') + d(y,y')$. But this is equivalent to $|d(x,y) - d(x',y')| \leq D_1((x,y),(x',y'))$. So $d$ is Lipschitz.
    \end{example}

    \begin{example}
        If $(X,d)$ is a metric space and $A \subseteq X$, then $\dist_A : X \rightarrow [0,\infty)$ is continuous. We've shown that $|\dist_A(x) - \dist_A(y)| \leq d(x,y)$, so $\dist_A$ is Lipschitz.
    \end{example}

    \begin{definition}
        Let $X$ be a topological space. We say $X$ is \textit{normal} (or T4) if, given any disjoint closed sets $E,F \subseteq X$, there are neighbourhoods $U$
    \end{definition}

    \begin{definition}
        Let $X$ be a topological space. We say $X$ is \textit{normal} (or T4) if, for any $A,B \subseteq X$ closed satisfying $A \cap B = \emptyset$, then there exists $U,V \in \tau_X$ with $A \subseteq U$, $B \subseteq V$ satisfying $U \cap V = \emptyset$.
    \end{definition}

    \begin{proposition}
        Metric spaces are normal.
    \end{proposition}
        \begin{proof}
            Let $A,B \subseteq (X,d)$ with $A \cap B = \emptyset$. Define $f:(X,d) \rightarrow \bfR$ by:
                \begin{equation*}
                \begin{split}
                    f(x) = \frac{\dist_A(x)}{\dist_A(x) + \dist_B(x)}.
                \end{split}
                \end{equation*}
            Then $f$ is continuous. Moreover, define:
                \begin{equation*}
                \begin{split}
                    U &:= f^{-1}\left(\left(-\sfrac{1}{2},\sfrac{1}{2}\right)\right) \\
                    V &:= f^{-1}\left(\left(-\sfrac{1}{2},\sfrac{3}{2}\right)\right)
                \end{split}
                \end{equation*}
            Then $U$ and $V$ open with $U \cap V = \emptyset$.
        \end{proof}

    \begin{proposition}\label{prop:linear-implies-continuous}
        Let $V$ and $W$ be normed spaces and $T:V \rightarrow W$ linear. The following are equivalent:
            \begin{enumerate}[label = (\arabic*),itemsep=1pt,topsep=3pt]
                \item $T$ is continuous at $0_V$;
                \item $T$ is continuous;
                \item $T$ is uniformly continuous;
                \item $T$ is Lipschitz;
                \item there exists $C \geq 0$ such that $\lnorm Tv \rnorm \leq C \lnorm v \rnorm$ for all $v \in V$;
                \item $T$ is bounded.
            \end{enumerate}
    \end{proposition}
        \begin{proof}
            We will show (6)$\Leftrightarrow$(5)$\Rightarrow$(4)$\Rightarrow$(3)$\Rightarrow$(2)$\Rightarrow$(1)$\Rightarrow$(5). Let $T$ be bounded. Given $v \in V$, $v \neq 0$, we have:
                \begin{equation*}
                \begin{split}
                    \lnorm T \rnorm_{\op} 
                    & = \sup_{v \in B_V}\lnorm Tv \rnorm \\
                    & \geq \lnorm T  \frac{v}{\lnorm v \rnorm} \rnorm \h5\text{for all $v \in V$}\\
                    & = \frac{1}{\lnorm v \rnorm} \lnorm Tv \rnorm \h5\text{for all $v \in V$}.
                \end{split}
                \end{equation*}
            Thus $\lnorm Tv \rnorm \leq \lnorm T \rnorm_{\op} \lnorm v \rnorm$ for all $v \in V$. The converse is clear by inspection.

            Suppose there exists $C \geq 0$ satisfying (5). We have that $\lnorm Tv - Tv' \rnorm \leq C \lnorm v - v' \rnorm$. Thus $T$ is Lipschitz.

            Let $T$ be Lipschitz. Let $\epsilon > 0$ and find $\delta = \frac{\epsilon}{c}$. Then $\lnorm v - v' \rnorm < \delta$ implies:
                \begin{equation*}
                \begin{split}
                    \lnorm Tv - Tv' \rnorm
                    & \leq C \lnorm v - v' \rnorm \\
                    & < c \frac{\epsilon}{c} \\
                    & = \epsilon.
                \end{split}
                \end{equation*}
            Thus $T$ is uniformly continuous.

            Suppose that $T$ be uniformly continuous. Fix $x \in V$. Given $\epsilon > 0$, we can find $\delta > 0$ so that $\lnorm v - x \rnorm < \delta$ implies $\lnorm Tv - Tx \rnorm < \epsilon$. Thus $T$ is continuous at $x \in V$. Since $x$ was arbitrary, $T$ is continuous. Moreover, $T$ will be continuous at $0_V$, establishing (1).
            
            We will now show (1) implies (5). Let $\epsilon = 1$. We can find a $\delta > 0$ such that $T\bigl(U(0,\delta)\bigr) \subseteq U(0,1)$. If $v \in V$, $v \neq 0$, then $\frac{\delta v}{2 \lnorm v \rnorm} \in U(0,\delta)$. Since $T$ is continuous at $0$, we have $T\frac{\delta v}{2 \lnorm v \rnorm} \in U(0,1)$\footnote{Recall that $T(0) = 0$.}. This gives $\lnorm T \frac{\delta v}{2 \lnorm v \rnorm} \rnorm < 1$, which is equivalent to $\lnorm Tv \rnorm < \frac{2}{\delta}\lnorm v \rnorm$. This establishes (5).
        \end{proof}

    \begin{corollary}\label{cor:finite-lin-implies-continuous}
        Let $V$ be a normed space with $\dim(V) = n$. If $T: \ell_p^n \rightarrow V$ is linear, then $T$ is continuous.
    \end{corollary}
        \begin{proof}
            Let $\cB = \{e_1,...,e_n\}$ be a basis for $\ell_p^n$. Let $v \in V$. Then $v = \sum_{j = 1}^n \alpha_j e_j$ where each $\alpha_j \in \ell_p$. Then:
                \begin{equation*}
                \begin{split}
                    \lnorm Tv \rnorm
                    & = \lnorm T \left( \sum_{j = 1}^n \alpha_j e_J \right) \rnorm \\
                    & = \lnorm \sum_{j = 1}^n \alpha_j Te_j \rnorm \\
                    & \leq \sum_{j = 1}^n |\alpha_j| \lnorm T e_j \rnorm.
                \end{split}
                \end{equation*}
            Let $c = \max_{j = 1}^n \lnorm T e_j \rnorm$. We have:
                \begin{equation*}
                \begin{split}
                    \sum_{j = 1}^n |\alpha_j| \lnorm T e_j \rnorm 
                    & \leq c\sum_{j = 1}^n |\alpha_j| \\
                    & = c \lnorm \sum_{j = 1}^n \alpha_j e_j \rnorm_1.
                \end{split}
                \end{equation*}
            We showed that all norms are equivalent in Theorem~\ref{thm:norms-equivalent}. So there exists $c' > 0$ such that $\lnorm \cdot \rnorm_1 \leq \lnorm \cdot \rnorm_p$. Thus:
                \begin{equation*}
                \begin{split}
                    \lnorm Tv \rnorm
                    & \leq c\cdot c' \lnorm \sum_{j = 1}^n \alpha_j e_j \rnorm_p \\
                    & = c \cdot c' \lnorm v \rnorm_p.
                \end{split}
                \end{equation*}
            By Proposition~\ref{prop:linear-implies-continuous}, we have that $T$ is continuous.
        \end{proof}

    \begin{proposition}
        Let $(X,d)$ be a metric space with $A \subseteq X$ dense. If $f,g :X \rightarrow (Y,\rho)$ are continuous with $f(a) = g(a)$ for all $a \in A$, then $f = g$.
    \end{proposition}
        \begin{proof}
            If $x \in X$, we can find a sequence $(a_n)_n$ in $A$ with $(a_n)_n \xrightarrow{d} x$. Then:
                \begin{equation*}
                \begin{split}
                    &(f(a_n))_n \rightarrow f(x) \\
                    & \h9\h3\shortparallel \\
                    & (g(a_n))_n \rightarrow g(x).
                \end{split}
                \end{equation*}
            Thus $f(x) = g(x)$.
        \end{proof}

    \begin{definition}
        Let $(X,d)$ and $(Y,\rho)$ be metric spaces and $f:X \rightarrow Y$.
        \begin{enumerate}[label = (\arabic*),itemsep=1pt,topsep=3pt]
            \item $f$ is a \textit{homeomorphism} if $f$ is bijective with $f$ and $f^{-1}$ continuous. If such an $f$ exists, we say $X \cong Y$ are \textit{homeomorphic}.
            \item $f$ is a \textit{uniformism} if $f$ is bijective with $f$ and $f^{-1}$ uniformly continuous. If such an $f$ exists, we say $X \cong Y$ are \textit{uniformly isomorphic}.
            \item $f$ is an \textit{metric isomorphism} if $f$ is bijective with $f$ and $f^{-1}$ Lipschitz. If such an $f$ exists, we say $X \cong Y$ are \textit{metrically isomorphic}.
            \item $f$ is an \textit{isometric isomorphism} if $f$ and $f^{-1}$ are isometries. We say $X \cong Y$ are \textit{isometrically isomorphic}.
        \end{enumerate}
    \end{definition}

    \begin{example}
        $(0,1) \cong \bfR$ are homeomorphic, but not uniformly isomorphic.
    \end{example}

    \begin{example}***
        Let $a = (a_k)_k \in \ell_1$. Define $\varphi_a:c_0 \rightarrow F$ by $\varphi_a(z) = \sum_{k \geq 1}a_k z_k$. This series converges since:
    \end{example}

    \begin{definition}
        Let $X$ be a set with two metrics $d_1$ and $d_2$.
        \begin{enumerate}[label = (\arabic*),itemsep=1pt,topsep=3pt]
            \item $d_1$ and $d_2$ are \textit{metrically equivalent} if $\id:(X,d_1) \rightarrow (X,d_2)$ and $\id^{-1}:(X,d_2) \rightarrow (X,d_1)$ are Lipschitz.
            \item $d_1$ and $d_2$ are \textit{uniformly equivalent} if $\id:(X,d_1) \rightarrow (X,d_2)$ and $\id^{-1}:(X,d_2) \rightarrow (X,d_1)$ are uniformisms.
            \item $d_1$ and $d_2$ are \textit{topologically equivalent} if $\id:(X,d_1) \rightarrow (X,d_2)$ and $\id^{-1}:(X,d_2) \rightarrow (X,d_1)$ are homeomorphisms.
        \end{enumerate}
    \end{definition}

    \begin{example}***
        
    \end{example}

\section{Completeness}\label{sec:completeness}
    \begin{definition}
        A sequence $(x_n)_n$ in a metric space $(X,d)$ is $d$-\textit{Cauchy} if:
            \begin{equation*}
            \begin{split}
                (\forall \epsilon > 0)(\exists N \in \bfN): (\forall p,q \in \bfN)(p,q \geq N \implies d(x_p,x_q) < \epsilon).
            \end{split}
            \end{equation*}
    \end{definition}

    \begin{proposition}
        Let $(x_n)$ be a sequence in $(X,d)$.
        \begin{enumerate}[label = (\arabic*),itemsep=1pt,topsep=3pt]
            \item If $(x_n)_n$ converges, then $(x_n)_n$ is Cauchy.
            \item If $(x_n)_n$ is Cauchy, then $(x_n)_n$ is bounded.
        \end{enumerate}
    \end{proposition}
        \begin{proof}
            (1) Let $x \in X$ and suppose $(x_n)_n \rightarrow x$. Let $\epsilon > 0$. Find $N$ large so that for $p \geq N$ we have $d(x_p,x) < \frac{\epsilon}{2}$. Then $p,q \geq N$ implies:
                \begin{equation*}
                \begin{split}
                    d(x_p,x_q) 
                    & \leq d(x_p,x) + d(x,x_q) \\
                    & < \frac{\epsilon}{2} + \frac{\epsilon}{2} \\
                    & = \epsilon.
                \end{split}
                \end{equation*}

            (2) Let $\epsilon = 1$. Find $N$ large so $p,q \geq N$ implies $d(x_p,x_q) < 1$. Let $C = \max_{1 \leq p,q \leq N}d(x_p,x_q)$. Without loss of generality, if $p \geq N $ and $q \leq N$, then $d(x_p,x_q) \leq d(x_p,x_N) + d(x_N,x_q) < 1 + C$. Set $K = \max\{1,1+C\}$. Then $\diam(\{x_n\}_{n \geq 1}) = \sup_{p,q \geq 1}d(x_p,x_q) < K$.
        \end{proof}

    \begin{proposition}
        Let $(x_n)_n$ be a Cauchy sequence in $X$ and suppose there exists a subsequence $(x_{n_k})_k$ converging to $x \in X$. Then $(x_n)_n$ converges to $x$.
    \end{proposition}
        \begin{proof}
            Let $\epsilon > 0$. Since $(x_n)_n$ is Cauchy, there exists $N$ large so $n,n_k \geq N$ implies $d(x_n,x_{n_k}) < \frac{\epsilon}{2}$. This gives:
                \begin{equation*}
                \begin{split}
                    d(x_n,x) 
                    & = d(x_n,\limit_{k \rightarrow \infty}x_{n_k}) \\
                    & = \limit_{k \rightarrow \infty} d(x_n,x_{n_k}) \\
                    & \leq \frac{\epsilon}{2} \\
                    & < \epsilon. \qedhere
                \end{split}
                \end{equation*}
        \end{proof}

    \begin{definition}
        A metric space is said to be \textit{complete} if every Cauchy sequence converges. A complete normed space is called a \textit{Banach space}. A complete inner product space is called a \textit{Hilbert space}.
    \end{definition}

    \begin{lemma}
        Let $f:(X,d) \rightarrow (Y,\rho)$ be uniformly continuous. If $(x_n)_n$ is $d$-Cauchy then $(f(x_n))_n$ is $\rho$-Cauchy.
    \end{lemma}
        \begin{proof}
            Let $\epsilon > 0$. Find $\delta > 0$ so that $d(x,x') < \delta$ implies $\rho(f(x),f(x')) < \epsilon$. Pick $N$ sufficiently large so that $p,q \geq N$ implies $d(x_p,x_q) < \delta$. This gives $\rho(f(x_p),f(x_q)) < \epsilon$, whence $(f(x_n))_n$ is $\rho$-Cauchy.
        \end{proof}

    \begin{corollary}
        If $f:(X,d) \rightarrow (Y,\rho)$ is a uniformism, then $(X,d)$ is complete if and only if $(Y,\rho)$ is complete.
    \end{corollary}
        \begin{proof}
            Let $(X,d)$ be complete. If $(y_n)_n$ is $\rho$-Cauchy, then $(f^{-1}(y_n))_n$ is $d$-Cauchy in $X$. So we can find some $x \in X$ such that $(f^{-1}(y_n))_n \rightarrow x$. Then $(f(f^{-1}(y_n)))_n = (y_n)_n \rightarrow f(x)$. The converse follows similarly.
        \end{proof}

    \begin{corollary}
        If $d_1$ and $d_2$ are uniformly equivalent metrics on a set $X$, then $(X,d_1)$ is complete if and only if $(X,d_2)$ is complete.
    \end{corollary}
        \begin{proof}
            Since the map $\id:(X,d_1) \rightarrow (X,d_2)$ is a uniformism, the previous corollary gives that $(X,d_1)$ is complete if and only if $(X,d_2)$ is complete.
        \end{proof}
    

    \begin{proposition} 
        $\ell_p^d$ is a Banach space for $1 \leq p \leq \infty$.
    \end{proposition}
        \begin{proof}
            We only need to show this for $\ell_\infty^d$ since all the $p$-norms are equivalent. Let $(x_n)_n$ be $\lnorm \cdot \rnorm_\infty$-Cauchy in $\ell_\infty^d$. Let $\epsilon > 0$. Find $N$ large so for $n,m \geq N$ we have $\lnorm x_n - x_m \rnorm_\infty < \epsilon$. Observe that:
                \begin{equation*}
                \begin{split}
                    |x_n(k) - x_m(k)|
                    & \leq \max_{1 \leq k \leq d}|x_n(k) - x_m(k)| \\
                    & = \lnorm x_n - x_m \rnorm_\infty \\
                    & < \epsilon,
                \end{split}
                \end{equation*}
            where $x_n(k)$ is the $k^\text{th}$ entry of the $d$-tuple $x_n$. So for each $k = 1,...,d$, we know $(x_n(k))_{n}$ is Cauchy in $F$. Set $\limit_{n \rightarrow \infty}x_n(k) = x(k)$ for $k=1,..,d$. This gives:
                \begin{equation*}
                \begin{split}
                    \lnorm x - x_n \rnorm_\infty 
                    & = \max_{1 \leq k \leq d}|x(k) - x_n(k)| \xrightarrow{n \rightarrow \infty} 0.
                \end{split}
                \end{equation*}
            Whence $(x_n)_n \rightarrow x$ in $\ell_\infty^d$. Now set $y = (y_1,...,y_n)$. We have:
                \begin{equation*}
                \begin{split}
                    \lnorm x_n - y \rnorm_p 
                    & \leq c' \lnorm x_n - y \rnorm_1 \\
                    & = c' \sum_{j = 1}^d |x_n(j) - y_j | \\
                    & \leq c' \max_{1 \leq j \leq d}\left| x_n(j) - y_j \right| \xrightarrow{n\rightarrow \infty} 0.
                \end{split}
                \end{equation*}
            Thus $\ell_p^d$ is complete.
        \end{proof}

    \begin{proposition}
        $\ell_p$ is a Banach space for $1 \leq p \leq \infty$.
    \end{proposition}
        \begin{proof}
            Suppose that $(f_n)_n$ is $\lnorm \cdot \rnorm_{\ell_p}$-Cauchy. Observe that
                \begin{equation*}
                \begin{split}
                    |f_n(k) - f_m(k)|^p
                    & \leq \sum_{j = 1}^\infty |f_n(j) - f_m(j)|^p \\
                    & = \lnorm f_n - f_m \rnorm_{\ell_p}^p
                \end{split}
                \end{equation*}
            So $(f_n(k))_n$ is Cauchy in $F$. Since this space is complete, define $\limit_{n \rightarrow \infty}f_n(k) := f(k)$.

            Our goal is to find some function $f:\bfN \rightarrow F$ satisfying $f \in \ell_p$ and $\lnorm f_n -f \rnorm_{\ell_p} \rightarrow 0$. The $f(k)$ we've just obtained will lead us to the most suitable candidate.

            Since $(f_n)_n$ is $\lnorm \cdot \rnorm_{\ell_p}$-Cauchy, it is bounded by some constant. Fix $K \geq 1$ and observe that:
                \begin{equation*}
                \begin{split}
                    \sum_{k = 1}^K |f(k)|^p 
                    & = \sum_{k = 1}^K\left|\limit_{n \rightarrow \infty}f_n(k)\right|^p \\
                    & = \limit_{n \rightarrow \infty}\sum_{k = 1}^K |f_n(k)|^p \\
                    & \leq \sup_{n \geq 1}\lnorm f_n \rnorm_{\ell_p}^p \\
                    & := C.
                \end{split}
                \end{equation*}
            Since the sequence $\left( \sum_{k = 1}^K |f(k)|^p \right)_{K = 1}^\infty$ is increasing and bounded above $C$, the Monotone Convergence Theorem says that it's limit exists. We obtain:
                \begin{equation*}
                \begin{split}
                    \limit_{K \rightarrow \infty}\sum_{k = 1}^K |f(k)|^p 
                    & = \sum_{k = 1}^\infty |f(k)|^p \\
                    & = \lnorm f \rnorm_{\ell_p}^p \\
                    & < \infty.
                \end{split}
                \end{equation*}
            Thus $f \in \ell_p$.

            It remains to show that $(f_n)_n$ converges to $f$. Given $\epsilon > 0$, find $N$ large so that $n,m \geq N$ implies $\lnorm f_n - f_m \rnorm_p < \epsilon$. For every $n,m \geq N$ we have:
                \begin{equation*}
                \begin{split}
                    \sum_{k = 1}^K |f_m(k) - f_n(k)|^p 
                    & \leq \lnorm f_m - f_n \rnorm_{\ell_p}^p \\
                    & < \epsilon^p.
                \end{split}
                \end{equation*}
            Taking the limit as $m \rightarrow \infty$ and considering all $n \geq N$ gives:
                \begin{equation*}
                \begin{split}
                    \sum_{k = 1}^K |f(k) - f_n(k)|^p < \epsilon^p.
                \end{split}
                \end{equation*}
            Finally, taking the limit as $K \rightarrow \infty$ and simplifying gives $\lnorm f - f_n \rnorm_{\ell_p} < \epsilon$. Thus $\ell_p$ is complete.
        \end{proof}

    \begin{proposition}
        Let $(Y,d)$ be a complete metric space. The set of bounded functions $\Bd(\Omega,Y)$ with $\lnorm \cdot \rnorm_u$ is complete.
    \end{proposition}
        \begin{proof}
            Let $(f_n)_n$ be $D_u$-Cauchy. Fix $x,x' \in \Omega$ and let $\epsilon > 0$. Find $N$ large so that $n,m \geq N$ implies:
                \begin{equation*}
                \begin{split}
                    d(f_n(x),f_m(x)) \leq D_u(f_n,f_m) < \epsilon.
                \end{split}
                \end{equation*}
            Thus $(f_n(x))_n$ is Cauchy in $Y$. Since $Y$ is complete, define $\limit_{n \rightarrow \infty}f_n(x) := f(x)$. Now find $N_{1}$ large so $n \geq N_{1}$ implies $d(f(x),f_n(x)) < \epsilon$. Find $N_{2}$ large so $n \geq N_{2}$ implies $d(f(x'),f_n(x')) < \epsilon$. Observe that:
                \begin{equation*}
                \begin{split}
                    d(f(x),f(x'))
                    & \leq d(f(x),f_{N_1}(x)) + d(f_{N_1}(x),f_{N_2}(x')) + d(f_{N_2}(x'),f(x')) \\
                    & < 2\epsilon + d(f_{N_1}(x),f_{N_2}(x')) \\
                    & \leq 2\epsilon + d(f_{N_1}(x),f_{N_2}(x)) + d(f_{N_2}(x),f_{N_2}(x')) \\
                    & \leq 2\epsilon + \sup_{n,m \geq 1}d(f_n(x),f_m(x)) + \sup_{x,x' \in \Omega}d(f_{N_2}(x),f_{N_2}(x')) \\
                    & = 2\epsilon +\diam(\{f_n(x)\}_{n \geq 1}) + \diam(f_{N_2}(\Omega)).
                \end{split}
                \end{equation*}
            Note that the sequence $(f_n(x))_n$ is bounded because it is Cauchy \textemdash so there exists $C_1 \geq 0$ such that $\diam(\{f_n(x)\}_{n \geq 1}) < C_1$. Moreover, since $f_{N_2} \in \Bd(\Omega,Y)$, there exists $C_2 \geq 0$ such that $\diam(f_{N_2}(\Omega)) < C_2$. This gives:
                \begin{equation*}
                \begin{split}
                    d(f(x),f(x')) < 2\epsilon + C_1 + C_2.
                \end{split}
                \end{equation*}
            Since this inequality is independent of any $n$, and since $x,x' \in \Omega$ was arbitrary, we have that $\diam(f(\Omega)) < \infty$; i.e., $f \in \Bd(\Omega,Y)$. With the same $\epsilon$ as above, find $N_3$ large so that for all $n,m \geq N_3$ then $D_u(f_n,f_m) <\frac{\epsilon}{2}$. We know:
                \begin{equation*}
                \begin{split}
                    d(f_n(x),f_m(x)) \leq D_u(f_n,f_m) < \frac{\epsilon}{2}.
                \end{split}
                \end{equation*}
            Taking $m \rightarrow \infty$ gives $d(f_n(x),f(x)) \leq \frac{\epsilon}{2}$ for all $n \geq N$. But note that $N$ does not depend on our fixed $x \in \Omega$. It follows that $D_u(f_n,f) \leq \frac{\epsilon}{2} < \epsilon$. Thus $(f_n)_n \rightarrow f$, establishing $\Bd(\Omega,Y)$ as complete.
        \end{proof}

    \begin{corollary}
        $\ell_\infty(\Omega)$ is complete.
    \end{corollary}

    \begin{proposition}\label{prop:complete-iff-closed}
        Let $(X,d)$ be a complete metric space and $Y \subseteq X$. $Y$ is complete if and only if $Y$ is closed.
    \end{proposition}
        \begin{proof}
            Let $(y_n)_n$ be a sequence in $Y$ converging to $x \in X$. Then $(y_n)_n$ is sequence in $X$. Since $X$ is complete, $(y_n)_n$ is Cauchy. Since $Y$ is complete, $(y_n)_n$ must converge to some $y \in Y$. Since sequences can have at most one limit, it must be that $y = x$. Thus $x \in Y$; i.e., $Y$ is closed.

            Conversely, if $(y_n)_n$ is Cauchy in $Y$, then it is Cauchy in $X$. Since $X$ is complete, there exists $x \in X$ with $(y_n)_n \rightarrow x$. Since $Y$ is closed, $x \in Y$. Thus $Y$ is complete.
        \end{proof}

    Proposiiton~\ref{prop:complete-iff-closed} and Proposition~\ref{prop:closed-iff-seq} are extremely useful tools for showing a space is complete.

    \begin{corollary}
        Let $(X,d)$ and $(Y,\rho)$ be metric spaces.
        \begin{enumerate}[label = (\arabic*),itemsep=1pt,topsep=3pt]
            \item $C_b(X,Y) := C(X,Y) \cap \Bd(X,Y)$ is $D_u$-complete.
            \item $C_b(X)$ is a $\lnorm \cdot \rnorm_u$-Banach space.
            \item $C_0(\bfR)$ is a $\lnorm \cdot \rnorm_u$-Banach space.
        \end{enumerate}
    \end{corollary}
        \begin{proof}
            (1) Let $(f_n)_n$ be a sequence in $C_b(X,Y)$ converging to $f \in \Bd(X,Y)$. Let $x\in X$ and $\epsilon > 0$. Find $N$ large so that $D_u(f_N,f) < \frac{\epsilon}{3}$. Find $\delta > 0$ so that for all $x' \in X$, $d(x,x') < \delta$ implies $\rho(f_N(x),f_N(x')) < \frac{\epsilon}{3}$. For $d(x,x') < \delta$:
                \begin{equation*}
                \begin{split}
                    \rho(f(x),f(x'))
                    & \leq \rho(f(x),f_N(x)) + \rho(f_N(x),f_N(x')) + \rho(f_N(x'),f(x')) \\
                    & \leq 2D_u(f_n,f) + \rho(f_N(x),f_N(x')) \\
                    & < \frac{2\epsilon}{3} + \frac{\epsilon}{3} \\
                    & = \epsilon.
                \end{split}
                \end{equation*}
            Thus $f \in C_b(X,Y)$ because it is bounded and continuous. Since $C_b(X,Y) \subseteq \Bd(X,Y)$ is closed, it is complete.

            (2) As we've just shown, the space $C_b(X)$ is complete. It only remains to show it is a vector space.

            (3) Let $(f_n)_n$ be a sequence in $C_0(\bfR)$ converging to $f \in C_b(\bfR)$. Let $\epsilon > 0$ and find $N$ large so that $\lnorm f - f_N \rnorm_u < \frac{\epsilon}{2}$. Since $f_N \in C_0(\bfR)$, we know that $\limit_{x \rightarrow \pm \infty}f_N(x) = 0$. So there exists $M > 0$ with $|x| \geq M$ implying $|f_N(x)| < \frac{\epsilon}{2}$. For $|x| \geq M$ observe that:
                \begin{equation*}
                \begin{split}
                    |f(x)|
                    & \leq |f(x) - f_N(x)| + |f_N(x)| \\
                    & \leq \lnorm f - f_N \rnorm_u + |f_N(x)| \\
                    & < \frac{\epsilon}{2} + \frac{\epsilon}{2} \\
                    & = \epsilon.
                \end{split}
                \end{equation*}
            Thus $\limit_{x \rightarrow \infty}f(x) = 0$; i.e., $f \in C_0(\bfR)$. Since $C_0(\bfR) \subseteq C_b(\bfR)$ is closed, it is complete.
        \end{proof}

    \begin{proposition}\label{prop:bounded-ops-is-complete}
        Let $V$ be a normed space and $W$ a Banach space. $B(V,W)$ with $\lnorm \cdot \rnorm_{\op}$ is a Banach space.
    \end{proposition}
        \begin{proof}
            Let $(T_n)_n$ be $\lnorm \cdot \rnorm_{\op}$-Cauchy. Let $v \in V$ and $\epsilon > 0$. Find $N_1$ large so that $n,m \geq N_1$ implies $\lnorm T_n - T_m \rnorm_{\op} < \frac{\epsilon}{\lnorm v \rnorm}$. We can see:
                \begin{equation*}
                \begin{split}
                    \lnorm T_nv - T_mv \rnorm_W
                    & = \lnorm (T_n - T_m)v \rnorm_W \\
                    & \leq \lnorm T_n - T_m \rnorm_{\op}\lnorm v \rnorm \\
                    & < \frac{\epsilon}{\lnorm v \rnorm} \cdot \lnorm v \rnorm \\
                    & = \epsilon.
                \end{split}
                \end{equation*}
            So $(T_nv)_n$ is $\lnorm \cdot \rnorm_W$-Cauchy. Since $W$ is a Banach space, define $\limit_{n \rightarrow \infty}T_n v = Tv$. We must show that $T$ is linear, bounded, and $\lnorm T_n - T \rnorm_{\op} \rightarrow 0$. Given $v_1,v_2 \in V$ and $c \in F$ we can see:
                \begin{equation*}
                \begin{split}
                    T(v_1 + cv_2) 
                    & = \limit_{n \rightarrow \infty} T_n(v_1 + c v_2) \\
                    & = \limit_{n \rightarrow \infty}T_nv_1 + c T_n v_2 \\
                    & = \limit_{n \rightarrow \infty}Tv_1 + c\limit_{n \rightarrow \infty}T v_2 \\
                    & = Tv_1 + cT v_2.
                \end{split}
                \end{equation*}
            Thus $T$ is linear. Now since $(T_n)_n$ is $\lnorm \cdot \rnorm_{\op}$-Cauchy, it is bounded, so there exists $C > 0$ with $\lnorm T_n \rnorm_{\op} \leq C$ for all $n \geq 1$. Using the fact norms are continuous, we have:
                \begin{equation*}
                \begin{split}
                    \lnorm Tv \rnorm_{W}
                    & = \lnorm \limit_{n\rightarrow \infty}T_n v \rnorm_{W} \\
                    & = \limit_{n \rightarrow \infty} \lnorm T_n v \rnorm_W \\
                    & \leq \limsup_{n \rightarrow \infty} \lnorm T_n \rnorm_{\op}\lnorm v \rnorm \\
                    & \leq C \lnorm v \rnorm.
                \end{split}
                \end{equation*}
            Thus $T \in B(V,W)$. With the same epsilon as before, find $N_2$ so that $n,m \geq N_2$ implies $\lnorm T_n - T_m \rnorm_{\op} < \frac{\epsilon}{2}$. We can show:
                \begin{equation*}
                \begin{split}
                    \lnorm T_nv - T_mv \rnorm_{W}
                    & \leq \lnorm T_n - T_m \rnorm_{\op} < \frac{\epsilon}{2}.
                \end{split}
                \end{equation*}
            Taking $m \rightarrow \infty$ gives:
                \begin{equation*}
                \begin{split}
                    \lnorm T_nv - Tv \rnorm_{W} \leq \frac{\epsilon}{2}.
                \end{split}
                \end{equation*}
            Taking the supremum over all $v \in B_V$ gives:
                \begin{equation*}
                \begin{split}
                    \lnorm T_n - T \rnorm_{\op} \leq \frac{\epsilon}{2} < \epsilon.
                \end{split}
                \end{equation*}
            Thus $B(V,W)$ is complete.
        \end{proof}

    \begin{proposition}
        Let $(V,\lnorm \cdot \rnorm)$ be a normed space. The following are equivalent:
            \begin{enumerate}[label = (\arabic*),itemsep=1pt,topsep=3pt]
                \item $V$ is a Banach space;
                \item If $(v_k)_k$ is a sequence in $V$ with $\sum_{k = 1}^\infty \lnorm v_k \rnorm$ convergent, then $\sum_{ k =1}^\infty v_k$ converges.
            \end{enumerate}
    \end{proposition}
        \begin{proof}
            Suppose $V$ is a Banach space. Let $s_n = \sum_{k = 1}^n v_k$ and $t_n = \sum_{k = 1}^n \lnorm v_k \rnorm$. For $p > q > 1$:
                \begin{equation*}
                \begin{split}
                    \lnorm s_p - s_q \rnorm
                    & = \lnorm \sum_{k = q +1}^p v_k \rnorm \\
                    & \leq \sum_{k = q + 1}^p \lnorm v_k \rnorm \\
                    & = |t_p - t_k|.
                \end{split}
                \end{equation*}
            Since $(t_n)_n$ is convergent, it is Cauchy. So $(s_n)_n$ is Cauchy, implying it is convergent. Thus $\sum_{k = 1}^\infty v_k$ converges.

            Now let $(v_n)_n$ be Cauchy. Find $n_1 \in \bfN$ such that $p,q \geq n_1$ implies $\lnorm v_p - v_q \rnorm < 2^{-1}$. Find $n_2 > n_1$ such that $p,q \geq n_2$ implies $\lnorm v_p - v_q \rnorm < 2^{-2}$. Inductively, find $n_k > n_{k-1}$ such that $p,q \geq n_k$ implies $\lnorm v_p - v_q \rnorm < 2^{-k}$. Consider the sequence $(v_{n_{k+1}} - v_{n_k})_k$. Then:
                \begin{equation*}
                \begin{split}
                    \sum_{k = 1}^\infty \lnorm v_{n_{k+1}} - v_{n_k} \rnorm \leq \sum_{k = 1}^\infty 2^{-k} = 1.
                \end{split}
                \end{equation*}
            By our hypothesis, $\sum_{k = 1}^\infty (v_{n_{k+1}} - v_{n_k})$ converges. So the sequence of partial sums:
                \begin{equation*}
                \begin{split}
                    w_m 
                    & = \sum_{k = 1}^m v_{n_{k+1}} - v_{n_k} \\
                    & = v_{n_m} - v_{n_1}
                \end{split}
                \end{equation*}
            also converges to some $w \in V$ as $m \rightarrow \infty$. However, notice:
                \begin{equation*}
                \begin{split}
                    (v_{n_m})_m 
                    & = (v_{n_m} - v_{n_1})_m + v_{n_1} \\
                    & \xrightarrow{m \rightarrow \infty} w + v_{n_1}.
                \end{split}
                \end{equation*}
            Since $(v_n)_n$ is a Cauchy sequence which admits a convergent subsequence, $(v_n)_n$ converges. Thus $V$ is a Banach space.
        \end{proof}

    \begin{example}***
        Let $\cH$ be a Hilbert space. Suppose $(e_n)_n$ is an orthonormal sequence in $\cH$ and $(t_k)_k \in \ell_2$. We will show $\sum_{k = 1}^\infty t_k e_k$ converges in $\cH$, but not absolutely in general.

        Let $s_n = \sum_{k = 1}^n t_k e_k$. For $n>m > 1$:
            \begin{equation*}
            \begin{split}
                \lnorm s_n - s_m \rnorm^2 
                & = \lnorm \sum_{k = m+1}^n t_k e_k \rnorm^2 \\
                & = \sum_{k = m+1}^n |t_k|^2.
            \end{split}
            \end{equation*}
        Since $ \left( \sum_{k = 1}^n |t_k|^2 \right)_n$ is Cauchy, then $(s_n)_n$ is Cauchy, whence $\sum_{k = 1}^\infty t_k e_k$ converges. But notice $\lnorm s_n \rnorm^2 = \sum_{k = 1}^n |t_k|^2$. As $n \rightarrow \infty$, we see $\lnorm \sum_{k = 1}^\infty t_k e_k \rnorm^2 = \sum_{k = 1}^\infty |t_k|^2$.
    \end{example}

    \begin{center}
        \begin{tikzpicture}
            \draw[thick] (0.3,0) -- (2.3,0);
            \node at (2.39, 0) {$/\,$};
            \node at (2.56, 0) {$/\,$};
            \draw[thick] (2.6,0) -- (4.6,0);
        \end{tikzpicture}
    \end{center}

    Recall that if $f:(X,d) \rightarrow (Y,\rho)$ is a uniformly continuous map between metric spaces and $(x_n)_n$ is Cauchy, then $(f(x_n))_n$ is Cauchy. Complete spaces have the unique property that, given a map from a dense subset $A \rightarrow Y$, we can define an extension of such function.

    \begin{theorem}\label{thm:unif-cont-extension}
        Let $(X,d)$ and $(Y,\rho)$ be metric spaces with $Y$ complete. Suppose $A \subseteq X$ is dense and $f:A \rightarrow Y$ is uniformly continuous. There exists a unique uniformly continuous map $\widetilde{f}:X \rightarrow Y$ with $\widetilde{f}(x) = f(x)$ for all $x \in A$.
    \end{theorem}
        \begin{proof}
            Let $x \in X$. We know there exists a sequence $(a_n)_n$ in $A$ with $(a_n)_n \rightarrow x$. Since $(a_n)_n$ is convergent, it is Cauchy, so $(f(a_n))_n$ is also Cauchy. By the completeness of $Y$, $(f(a_n))_n$ converges. Define $\widetilde{f}(x) := \limit_{n \rightarrow \infty}f(a_n)$.

            We must show this extension is well-defined. Suppose $(b_n)_n$ is another sequence in $A$ with $(b_n)_n \rightarrow x$. Then the mixed sequence $(a_1,b_1,a_2,b_2,...)$ will converge to $x$. The same reasoning as above tells us $(f(a_1),f(b_1),f(a_2),f(b_2),...)$ converges in $Y$. The two subsequences $(f(a_n))_n$ and $(f(b_n))_n$ must then converge to the same limit.

            We will now show that $\widetilde{f}$ is uniformly continuous. Let $\epsilon > 0$. Find $\delta > 0$ so that for all $a,b \in A$, $d(a,b) < \delta$ implies $\rho(f(a),f(b)) < \frac{\epsilon}{2}$. Now let $x,x' \in X$ with $d(x,x') < \frac{\delta}{4}$. Find sequences $(a_n)_n$ and $(b_n)_n$ in $A$ with $(a_n)_n \rightarrow x$ and $(b_n)_n \rightarrow x'$. Find $N$ large so that $n \geq N$ implies $d(a_n,x) < \frac{\delta}{4}$ and $d(b_n,x) < \frac{\delta}{4}$. The triangle inequality gives $d(a_n,b_n) < \frac{3\delta}{4} < \delta$. So $\rho(f(a_n),f(b_n)) < \frac{\epsilon}{2}$ for all $n \geq N$. Observe that:
                \begin{equation*}
                \begin{split}
                    \rho(\widetilde{f}(x),\widetilde{f}(x'))
                    & = \limit_{n \rightarrow \infty} \rho(f(a_n),f(b_n)) \\
                    & \leq \frac{\epsilon}{2} \\
                    & < \epsilon.
                \end{split}
                \end{equation*}
            Thus $\widetilde{f}$ is uniformly continuous. It remains to show that $\widetilde{f}$ is unique. Suppose $g:X\rightarrow Y$ is also a continuous extension of $f$. Then $g(x) = f(x) = \widetilde{f}(x)$ for all $x \in A$. Since $g$ and $\widetilde{f}$ agree on elements of a dense set, we must have $g = \widetilde{f}$.
        \end{proof}

    \begin{proposition}
        Let $(X,d)$ and $(Y,\rho)$ be metric spaces with $A \subseteq X$ dense, $Y$ complete, and $f:A \rightarrow Y$ an isometry. Then the continuous extension $\widetilde{f}:X \rightarrow Y$ is an isometry.
    \end{proposition}
        \begin{proof}
            Let $x,x' \in X$. Let $(a_n)_n$ and $(b_n)_n$ be sequences in $A$ with $(a_n)_n \rightarrow x$ and $(b_n)_n \rightarrow x'$. We have:
                \begin{equation*}
                \begin{split}
                    \rho(\widetilde{f}(x),\widetilde{f}(x'))
                    & = \limit_{n \rightarrow \infty}\rho(f(a_n),f(b_n)) \\
                    & = \limit_{n \rightarrow \infty} d(a_n,b_n) \\
                    & = d(x,x'). \qedhere
                \end{split}
                \end{equation*}
        \end{proof}

    \begin{corollary}
        Let $V$ be a normed space, $W$ a Banach space, and $U \subseteq V$ a dense linear subspace. Let $T_0 \in B(U,W)$. There exists a unique $T \in B(V,W)$ with $\lnorm T \rnorm_{\op} = \lnorm T_0 \rnorm_{\op}$. Moreover, if $T_0$ is isometric, then so if $T$.
    \end{corollary}
        \begin{proof}
            We only need to show $T$ is linear and $\lnorm T \rnorm_{\op} = \lnorm T_0 \rnorm_{\op}$. Let $v,v' \in V$ and $\alpha \in F$. Let $(x_n)_n$ and $(y_n)_n$ be sequences in $U$ with $(x_n)_n \rightarrow v$ and $(y_n)_n \rightarrow v'$. Observe that:
                \begin{equation*}
                \begin{split}
                    T(v + \alpha v')
                    & = \limit_{n \rightarrow \infty} T_0(x_n + \alpha y_n) \\
                    & = \limit_{n \rightarrow \infty} T_0(x_n) + \alpha \limit_{n \rightarrow \infty} T_0(y_n) \\
                    & = T(v) + \alpha T(v').
                \end{split}
                \end{equation*}
            
            Note that the composition $V \xrightarrow{\h6T\h6} W \xrightarrow{\lnorm \cdot \rnorm_W} F$ will be continuous and bounded, so by Proposition~\ref{} we have:
                \begin{equation*}
                \begin{split}
                   \lnorm T \rnorm_{\op} 
                   & = \sup_{v \in B_V}\lnorm T(v) \rnorm \\
                   & = \sup_{v \in B_U}\lnorm T(v) \rnorm \\
                   & = \sup_{v \in B_U}\lnorm T_0(v) \rnorm \\
                   & = \lnorm T_0 \rnorm_{\op}.
                \end{split}
                \end{equation*}
            Thus $T \in B(V,W)$.
        \end{proof}

    \begin{example}***
        (Something about Hilbert spaces...)
    \end{example}

    \begin{proposition}
        Let $T:V \rightarrow W$ be a continuous linear map between normed spaces which is \textit{bounded below}, that is, there is a $C>0$ with $\lnorm Tv \rnorm \geq C \lnorm v \rnorm$ for all $v \in V$. If $V$ is complete, then $\Image(T) \subseteq W$ is a closed subspace, and $V \cong \Image(T)$ are uniformly isomorphic.
    \end{proposition}
        \begin{proof}
            Let $(T(v_n))_n$ be a sequence in $\Image(T)$ converging to $w \in W$. Given $\epsilon$, find $N$ large so that $n \geq M$ implies $\lnorm T(v_n)-w \rnorm < \frac{C\epsilon}{2}$. For $n,m \geq N$, observe that:
                \begin{equation*}
                \begin{split}
                    \lnorm v_n - v_m \rnorm
                    & \leq \frac{1}{C}\lnorm T(v_n - v_m) \rnorm \\
                    & = \frac{1}{C}\lnorm T(v_n) - T(v_m) \rnorm \\
                    & \leq \frac{1}{C}\lnorm T(v_n) - w \rnorm + \frac{1}{C}\lnorm w - T(v_m) \rnorm \\
                    & < \frac{\epsilon}{2} + \frac{\epsilon}{2} \\
                    & = \epsilon.
                \end{split}
                \end{equation*}
            Thus $(v_n)_n$ is Cauchy. Since $V$ is complete, let $v_0 := \limit_{n \rightarrow \infty}v_n$. Since $T$ is continuous, we can see $(T(v_n))_n \rightarrow T(v_0)$. It must be the case that $T(v_0) = w$; i.e., $w \in \Image(T)$. Thus $\Image(T)$ is a closed subspace.

            Since $T$ is continuous, there exists some $\alpha > 0$ such that $\lnorm Tv \rnorm \leq \alpha \lnorm  v \rnorm$. Clearly if $v = 0$, then $Tv = 0$, implying that $T$ is injective. Whence $V \cong \Image(T)$ as vector spaces. Since $T$ is continuous, it is uniformly continuous, so it remains to show that $T^{-1}:\Image(T) \rightarrow V$ (which exists) is also continuous. Let $w \in \Image(T)$, then there exists $v \in V$ with $T(v) = w$. Observe that:
                \begin{equation*}
                \begin{split}
                    \lnorm T^{-1} w \rnorm
                    & = \lnorm T^{-1}(T(v)) \rnorm \\
                    & = \lnorm v \rnorm \\
                    & \leq \frac{1}{C}\lnorm Tv \rnorm \\
                    & = \frac{1}{C}\lnorm w \rnorm.
                \end{split}
                \end{equation*}
            Thus $T$ is a uniformism.
        \end{proof}

    \begin{center}
        \begin{tikzpicture}
            \draw[thick] (0.3,0) -- (2.3,0);
            \node at (2.39, 0) {$/\,$};
            \node at (2.56, 0) {$/\,$};
            \draw[thick] (2.6,0) -- (4.6,0);
        \end{tikzpicture}
    \end{center}

    \begin{definition}
        Let $(X,d)$ be a metric space. A \textit{completion} of $(X,d)$ is a pair $\bigl((Z,\rho),\iota\bigr)$ where:
            \begin{enumerate}[label = (\arabic*),itemsep=1pt,topsep=3pt]
                \item $(Z,\rho)$ is a complete metric space;
                \item $\iota:X \hookrightarrow Z$ is an isometry;
                \item $\overline{\iota(X)}^\rho = Z$.
            \end{enumerate}
    \end{definition}

    \begin{example}
        $\bigl(([0,1],\lnorm \cdot \rnorm), \iota(t) = t\bigr)$ is a completion of $(0,1)$.
    \end{example}

    \begin{lemma}\label{lemma:isometry-closed}
        Let $f:(X,d) \rightarrow (Y,\rho)$ be an isometry between metric spaces. If $X$ is complete, then $f(X) \subseteq Y$ is closed. In particular, if $f(X)$ is dense, then $f$ is onto. 
    \end{lemma}
        \begin{proof}
            If $(f(x_n))_n \rightarrow y$ in $Y$, then $(f(x_n))_n$ is $\rho$-Cauchy. Since $f$ is an isometry, $d(x_n,x_m) = \rho(f(x_n),f(x_m))$, so $(x_n)_n$ is $d$-Cauchy. Let $x \in X$ such that $(x_n)_n \rightarrow x$. Since $f$ is continuous, we have $(f(x_n))_n \rightarrow f(x)$. It must be that $y = f(x) \in f(X)$.

            Since we've just shown that $f(X)$ is closed, if it were also dense then $f(X) = \overline{f(X)}^\rho = Y$, whence $f$ is surjective.
        \end{proof}

    \begin{theorem}\label{thm:unique-completion}
        Let $(X,d)$ be a metric space. If $\bigl((Z,\rho),\iota\bigr)$ and $\bigl((Z',\rho'),\jota\bigr)$ are completions of $X$, there exists a unique isometric isomorphism $\varphi:Z \rightarrow Z'$ with $\varphi \circ \iota = \jota$; that is, the following diagram commutes:
        \begin{center}
            \begin{tikzcd}
                X \arrow[rd, "\jota"'] \arrow[r, "\iota"] & Z \arrow[d, "\varphi", dotted] \\
                                                    & Z'                            
                \end{tikzcd}
        \end{center}
            \begin{proof}
                Let $z \in Z$. Since $\iota(X)$ is dense in $Z$, there exists a sequence $(\iota(x_n))_n$ in $\iota(X)$ such that $(\iota(x_n))_n \rightarrow z$. Then $(\iota(x_n))_n$ is $\rho$-Cauchy since it is convergent, and furthermore we have:
                    \begin{equation*}
                    \begin{split}
                        \rho'(\jota(x_n),\jota(x_m))
                        & = d(x_n,x_m) \\
                        & = \rho(\iota(x_n),\iota(x_m)).
                    \end{split}
                    \end{equation*}
                So if $(\iota(x_n))_n$ is $\rho$-Cauchy in $Z$, then $(\jota(x_n))_n$ is $\rho'$-Cauchy in $Z'$. Since $\jota(X)$ is dense, the limit of this sequence is in $Z'$. Define $\varphi(z) := \limit_{n \rightarrow \infty}\jota(x_n)$.

                We will show that $\varphi:Z \rightarrow Z'$ is well-defined, that is, it does not depend on our particular choice of sequence. Let $(y_n)$ be another sequence in $X$ with $(y_n)_n \rightarrow z$. We can see that:
                    \begin{equation*}
                    \begin{split}
                        d(x_n,y_n) 
                        & = \rho(\iota(x_n),\iota(y_n)) \\
                        & \leq \rho(\iota(x_n), z) + \rho(z, \iota(y_n)) \\
                        & \rightarrow 0,
                    \end{split}
                    \end{equation*}
                which gives:
                    \begin{equation*}
                    \begin{split}
                        \rho'(\jota(y_n), \varphi(z))
                        & \leq \rho'(\jota(y_n),\jota(x_n)) + \rho'(\jota(x_n),\varphi(z)) \\
                        & = d(y_n,x_n) + \rho'(\jota(x_n),\varphi(z)) \\
                        & \rightarrow 0.
                    \end{split}
                    \end{equation*}
                Thus $(\jota(y_n))_n \rightarrow \varphi(z)$. Since $\limit_{n \rightarrow \infty}\jota(x_n) = \limit_{n \rightarrow \infty}\jota(y_n)$, $\varphi$ is well-defined.

                We will now show that $\varphi$ is an isometric isomorphism. Given $z_1,x_2 \in Z$, let $(\iota(x_n))_n \rightarrow z_1$ and $(\iota(y_n))_n \rightarrow z_2$, corresponding to $(\jota(x_n))_n \rightarrow \varphi(z_1)$ and $(\jota(y_n))_n \rightarrow \varphi(z_2)$. Observe that:
                    \begin{equation*}
                    \begin{split}
                        \rho'(\varphi(z_1),\varphi(z_2))
                        & = \limit_{n \rightarrow \infty} \rho'(\jota(x_n),\jota(y_n)) \\
                        & = \limit_{n \rightarrow \infty}d(x_n,y_n) \\
                        & = \limit_{n \rightarrow \infty}\rho(\iota(x_n),\iota(y_n)) \\
                        & = \rho(z_1,z_2).
                    \end{split}
                    \end{equation*}
                Since $\varphi$ is an isometry, it is injective. Chasing the above diagram will yield $\varphi \circ \iota  = \jota$, so we have $\jota(X) = \varphi(\iota(X)) \subseteq \varphi(Z)$. Then $Z' = \overline{\jota(X)}^{\rho'} \subseteq \overline{\varphi(Z)}^{\rho'}$. Since $Z$ is complete, by Lemma~\ref{lemma:isometry-closed} the set $\varphi(Z)$ is closed, whence $Z' \subseteq \overline{\varphi(Z)}^{\rho'} = \varphi(Z)$. Thus $\varphi$ is onto.

                It remains to show that $\varphi$ is unique. Let $\psi:Z \rightarrow Z'$ be another isometric isomorphism satisfying $\psi \circ \iota = \jota$. Given $z \in Z$, we can find a sequence $(\iota(x_n))_n$ in $\iota(X)$ such that $(\iota(x_n))_n \rightarrow z$. Observe that:
                    \begin{equation*}
                    \begin{split}
                        \varphi(z)
                        & = \limit_{n \rightarrow \infty} \varphi(\iota(x_n)) \\
                        & = \limit_{n \rightarrow \infty} \jota(x_n) \\
                        & = \limit_{n \rightarrow \infty} \psi(\iota(x_n)) \\
                        & = \psi(z).
                    \end{split}
                    \end{equation*}
                Since $z \in Z$ was arbitrary, this proves $\varphi = \psi$. 
            \end{proof}
    \end{theorem}

    \begin{lemma}
        If $(X,d)$ is a metric space and $i:(X,d) \rightarrow (Y,\rho)$ is an isometry into a complete metric space $(Y,\rho)$, then $\Bigl( \bigl(\overline{i(X)}^\rho , \rho\bigr), i\Bigr)$ is a completion of $X$.
    \end{lemma}
        \begin{proof}
            The space $\bigl(\overline{i(X)}^\rho,\rho\bigr)$ is complete by Proposition~\ref{prop:complete-iff-closed}. Clearly $i:X \rightarrow \overline{i(X)}^\rho$ is an isometry because $\overline{i(X)}^\rho \subseteq Y$.
        \end{proof}

    \begin{theorem}
        Every metric space admits a unique completion up to isometric isomorphism.
    \end{theorem}
        \begin{proof}
            Uniqueness was shown in Theorem~\ref{thm:unique-completion}. Given $(X,d)$, consider the Banach space $(C_b(X),\lnorm \cdot \rnorm_u)$. By the previous lemma we only need to construct an isometry $X \xhookrightarrow{i} C_b(X)$.

            Fix any $x_0 \in X$. Define $f_x :X \rightarrow F$ by $f_x(t) = d(t,x) -d(t,x_0)$. Clearly $f_x$ is continuous, and it is bounded because $|f_x(t)| = |d(t,x) - d(t,x_0)| \leq d(x,x_0)$. So $f_x \in C_b(X)$. Now define $i:X \rightarrow C_b(X)$ by $x \mapsto f_x$. Observe that:
                \begin{equation*}
                \begin{split}
                    \lnorm f_x - f_y \rnorm_u
                    & = \sup_{t \in X}|f_x(t) - f_y(t)| \\
                    & = \sup_{t \in X}|d(t,x) - d(t,x_0) - d(t,y) + d(t,x_0)| \\
                    & = \sup_{t \in X} |d(t,x) - d(t,y)| \\
                    & = d(x,y).
                \end{split}
                \end{equation*}
            Thus $i$ is an isometry, making $\Bigl( \bigl(\overline{i(X)}^{\lnorm \cdot \rnorm_u} , \lnorm \cdot \rnorm_u\bigr), i\Bigr)$ a completion of $X$.
        \end{proof}

    \begin{theorem}
        Let $(X,d)$ be a metric space with completion $\bigl((Z,\rho), i\bigr)$. If $f:X \rightarrow Y$ is a uniformly continuous function into a complete metric space $Y$, then there exists a unique uniformly continuous function $\widetilde{f}:Z \rightarrow Y$ such that $\widetilde{f} \circ i = f$; i.e., the following diagram commutes:
            \begin{center}
                \begin{tikzcd}
                    X \arrow[rd, "f"'] \arrow[r, "i"] & Z \arrow[d, "\widetilde{f}", dotted] \\
                                                      & Y                                   
                    \end{tikzcd}
            \end{center}
    \end{theorem}
        \begin{proof}
            Define $g:i(X) \rightarrow Z$ by $g = f \circ i^{-1}$. Since $g$ is uniformly continuous and $i(X) \subseteq Z$ is dense, Theorem~\ref{thm:unif-cont-extension} says there exists a unique uniformly continuous map $\widetilde{f}:Z \rightarrow Y$. Clearly $\widetilde{f} \circ i = f$.
        \end{proof}

    \begin{theorem}***\label{thm:completion-of-norm}
        Let $(V,\lnorm \cdot \rnorm)$ be a normed space. The completion of $V$ is a Banach space.
    \end{theorem}
        \begin{proof}
            Let $\bigl((W,\rho),i\bigr)$ be the completion of $V$. We must first verify that $W$ is a vector space. In doing so, we have to define what the vector space operations of $W$ are. Let $w \in W$ and $\alpha \in F$. We know there exists a sequence $(i(v_n))_n$ in $i(V)$ converging to $w$. Since this sequence is convergent, it is $\rho$-Cauchy. From the following:
                \begin{equation*}
                \begin{split}
                    \rho(i(\alpha v_n), i(\alpha v_m))
                    & = \lnorm \alpha v_n - \alpha v_m \rnorm \\
                    & = |\alpha| \lnorm v_n - v_m \rnorm \\
                    & = |\alpha| \rho(i(v_n),i(v_m)),
                \end{split}
                \end{equation*}
            we can see $(i(\alpha v))_n$ is also $\rho$-Cauchy, whence it is convergent. Define $s:F \times W \rightarrow W$ by $s(\alpha,w) = \limit_{n \rightarrow \infty} i(\alpha v_n) := \alpha w$. We will first show that this action is well-defined. Let $(i(u_n))_n$ be another sequence in $i(V)$ converging to $w$. As above, we have:
                \begin{equation*}
                \begin{split}
                    \rho(i(\alpha v_n), i(\alpha u_n))
                    & = |\alpha| \rho(i(v_n), i( u_n)) \\
                    & \leq \alpha \bigl( \rho(i(v_n),w) + \rho(w, i(u_n))\bigr) \\
                    & \rightarrow 0.
                \end{split}
                \end{equation*}
            This gives:
                \begin{equation*}
                \begin{split}
                    \rho(i(\alpha u_n), \alpha w) 
                    & \leq \rho(i(\alpha u_n), i(\alpha v_n)) + \rho(i(\alpha v_n),\alpha w) \\
                    & \rightarrow 0.
                \end{split}
                \end{equation*}
            Thus $(i(\alpha u_n))_n \rightarrow \alpha w$, meaning scalar multiplication is well-defined. Now let $w_1,w_2 \in W$. Let $(i(v_n))_n$ and $(i(u_n))_n$ be sequences in $i(V)$ converging respectively to $w_1$ and $w_2$. Since these sequences are convergent, they are $\rho$-Cauchy. From the fact that:
                \begin{equation*}
                \begin{split}
                    \rho(i(v_n + u_n), i(v_m + u_m))
                    & = \lnorm v_n + u_n - v_m - u_m \rnorm \\
                    & \leq \lnorm v_n - v_m \rnorm + \lnorm u_n - u_m \rnorm \\
                    & = \rho(i(v_n),i(v_m)) + \rho(i(u_n),i(u_m)),
                \end{split}
                \end{equation*}
            we can see $(i(v_n + u_n))_n$ is also $\rho$-Cauchy, hence it is convergent. Define $a: W \times W \rightarrow W$ by $a(w_1,w_2) = \limit_{n \rightarrow \infty} i(v_n + u_n) := w_1 + w_2$. We will show this binary operation is well-defined. Let $(i(x_n))_n$ and $(i(y_n))_n$ be sequences in $i(V)$ also converging to $w_1$ and $w_2$ respectively. Note that:
                \begin{equation*}
                \begin{split}
                    \rho(i(v_n + u_n), i(x_n + y_n))
                    & = \lnorm v_n + u_n - x_n - y_n \rnorm \\
                    & \leq \lnorm v_n - x_n \rnorm + \lnorm u_n - y_n \rnorm \\
                    & = \rho(i(v_n),i(x_n)) + \rho(i(u_n),i(y_n)) \\
                    & \leq \rho(i(v_n),w_1) + \rho(w_1,(x_n)) + \rho(i(u_n),w_2) + \rho(w_2,i(y_n)) \\
                    & \rightarrow 0.
                \end{split}
                \end{equation*}
            This gives:
                \begin{equation*}
                \begin{split}
                    \rho(i(x_n + y_n), w_1 + w_2)
                    & \leq \rho(i(x_n + y_n), i(v_n+u_n)) + \rho(i(v_n + u_n),w_1 + w_2) \\
                    & \rightarrow 0.
                \end{split}
                \end{equation*}
            Thus $(i(x_n + y_n))_n \rightarrow w_1 + w_2$, meaning vector addition is well-defined.

            Before showing $W$ paired with the above operations is a vector space, we need to verify that the isometry $i:V \rightarrow W$ is linear. If $v,v' \in V$ and $\alpha \in F$, by taking $v_n = v$ and $u_n = v'$ for all $n\geq 1$, we can see that:
                \begin{equation*}
                \begin{split}
                    i(v + \alpha v')
                    & = \limit_{n \rightarrow \infty}i(v + \alpha v') \\
                    & = i(v) + \alpha i(v').
                \end{split}
                \end{equation*}
            With this fact, we can show that $W$ satsifies all of the vector-space axioms \textemdash this is left as an exercise. Moreover, we have that $i(V) \subseteq W$ is a $\rho$-dense linear subspace\footnote{If $T:V \rightarrow W$ is injective, then $V \cong T(V)$}.

            It remains to show that $W$ is a normed space. Exactly as before, if $w \in W$, we can find a sequence $(i(v_n))_n$ in $i(V)$ converging to $w$. Since this sequence is Cauchy, observe that:
                \begin{equation*}
                \begin{split}
                    \Bigl| \lnorm v_n \rnorm - \lnorm v_m \rnorm \Bigr|
                    & \leq \lnorm v_n - v_m \rnorm \\
                    & = \rho(i(v_n),i(v_m)).
                \end{split}
                \end{equation*}
            So $\bigl(\lnorm v_n \rnorm\bigr)_n$ is Cauchy, hence it is convergent. Define $\lnorm w \rnorm_W := \limit_{n \rightarrow \infty} \lnorm v_n \rnorm$. Showing this definition is well-defined, and that is satisfies the properties of a norm are left as an exercise.
        \end{proof}

        \begin{center}
            \begin{tikzpicture}
                \draw[thick] (0.3,0) -- (2.3,0);
                \node at (2.39, 0) {$/\,$};
                \node at (2.56, 0) {$/\,$};
                \draw[thick] (2.6,0) -- (4.6,0);
            \end{tikzpicture}
        \end{center}

    There is a softer proof of Theorem~\ref{thm:completion-of-norm}, but it requires heavier machinery. If $V$ is a vector space over $F$, recall that
        \begin{equation*}
        \begin{split}
            V' = \{\varphi \mid \varphi:V \rightarrow F \h3\text{linear}\h1\}
        \end{split}
        \end{equation*}
    is the linear space of all linear functionals on $V$. By Zorn's Lemma, $V' \neq \emptyset$. If $V$ is a normed space, then 
        \begin{equation*}
        \begin{split}
            V^\ast = \{\varphi \in V' \mid \varphi \h3\text{continuous}\h1\}
        \end{split}
        \end{equation*}
    is called the \textit{continuous dual space}. This is in fact a Banach space with norm $\lnorm \varphi \rnorm_{\op} = \sup_{v \in B_V}|\varphi(v)|$. However, is $V^\ast \neq \emptyset$?

    \begin{theorem}[Hahn-Banach]
        Let $V$ be a normed space. For any nonzero $v_0 \in V$, there is a $\varphi_{v_0} \in V^\ast$ with $\varphi_{v_0}(v_0) = \lnorm v_0 \rnorm$. Such a $\varphi_{v_0}$ is called a \textit{norming functional}.
    \end{theorem}

    \begin{corollary}
        Let $V$ be a normed space and $v \in V$. Then $\lnorm v \rnorm = \sup_{\varphi \in B_{V^\ast}}|\varphi(v)|$.
    \end{corollary}
        \begin{proof}
            If $\varphi \in B_{V^\ast}$, then $|\varphi(v)| \leq \lnorm \varphi \rnorm_{\op} \lnorm v \rnorm \leq \lnorm v \rnorm$. Choose $\varphi_v \in V^\ast$. Since $\varphi_v(v) = \lnorm v \rnorm$, we have $\sup_{v \in B_{V^\ast}}|\varphi(v)| \geq |\varphi_v(V)| = \lnorm v \rnorm$.
        \end{proof}

    \begin{corollary}
        There is a linear isometry $V \hookrightarrow \left( V^\ast \right)^\ast$.
    \end{corollary}
        \begin{proof}
            Let $v \in V$. Define $\widehat{v}:V^\ast \rightarrow F$ by $\varphi \mapsto \varphi(v)$. We can easily verify that $\widehat{v} \in (V^\ast)'$. By the above corollary:
                \begin{equation*}
                \begin{split}
                    \lnorm \widehat{v} \rnorm
                    & = \sup_{\varphi \in B_{V^\ast}}|\widehat{v}(\varphi)| \\
                    & = \sup_{\varphi \in B_{V^\ast}}|\varphi(v)| \\
                    & = \lnorm v \rnorm.
                \end{split}
                \end{equation*} 
            Whence $\widehat{v} \in \left( V^\ast \right)^\ast$. Define $i_V:V \rightarrow \left( V^\ast \right)^\ast$ by $v \mapsto \widehat{v}$. Note that this is an isometry since $\lnorm i_V(v) \rnorm = \lnorm \widehat{v} \rnorm = \lnorm v \rnorm$. Given $v_1,v_2 \in V$, $c \in F$, and $\varphi \in V^\ast$ we can see:
            \begin{equation*}
            \begin{split}
                i_V(v_1 + c v_2)(\varphi)
                & = \widehat{v_1 + cv_2}(\varphi) \\
                & = \varphi(v_1 + cv_2) \\
                & = \varphi(v_1) + c \varphi(v_2) \\
                & = \widehat{v_1}(\varphi) + c \widehat{v_2}(\varphi) \\
                & = i_V(v_1)(\varphi) + c i_V(v_2)(\varphi).
            \end{split}
            \end{equation*}
        Thus $i_V$ is linear.
        \end{proof}

    Using the tools above, we can now demonstrate in a cleaner way that the completion of any normed space forms a Banach space.

    \begin{proof}[\protect{Alternative proof of Theorem~\ref{thm:completion-of-norm}}]
        If $V$ is a normed space, then $\left( V^\ast \right)^\ast = B(V^\ast,F)$ is complete by Proposition~\ref{prop:bounded-ops-is-complete}. Then $\overline{i_V(V)}^{\lnorm \cdot \rnorm_{\op}} \subseteq \left( V^\ast \right)^\ast$ is complete by Proposition~\ref{prop:complete-iff-closed}. Clearly $i_V:V \rightarrow \overline{i_V(V)}^{\lnorm \cdot \rnorm_{\op}}$ is an isometry because $\overline{i_V(V)}^{\lnorm \cdot \rnorm_{\op}} \subseteq \left( V^\ast \right)^\ast$. Thus the completion of $V$ is  $\Bigl(\bigl(\overline{i_V(V)}^{\lnorm \cdot \rnorm_{\op}},\lnorm \cdot \rnorm_{\op}\bigr), i_V\Bigr)$
    \end{proof}

    \begin{lemma}
        Let $T:V \rightarrow W$ be continuous and linear. There is an induced continous and linear map $T^\ast:W^\ast \rightarrow V^\ast$ with $T^\ast (\psi) = \psi \circ T$; i.e., the following diagram commutes:
            \begin{center}
                \begin{tikzcd}
                    V \arrow[rd, "T^\ast(\psi)"'] \arrow[r, "T"] & W \arrow[d, "\psi"] \\
                                                                 & F                  
                    \end{tikzcd}
            \end{center}
    \end{lemma}
        \begin{proof}
            Let $\psi_1\,\psi_2 \in W^\ast$ and $\alpha \in F$. We have:
                \begin{equation*}
                \begin{split}
                    T^\ast(\psi_1 + \alpha \psi_2)
                    & = (\psi_1 + \alpha \psi_2) \circ T \\
                    & = \psi_1 \circ T + \alpha (\psi_2 \circ T) \\
                    & = T^\ast(\psi_1) + \alpha T^\ast(\psi_2).
                \end{split}
                \end{equation*}
            Thus $T^\ast$ is linear. Moreover:
                \begin{equation*}
                \begin{split}
                    \lnorm T^\ast \rnorm_{\op}
                    & = \sup_{\lnorm \psi \rnorm_{\op} \leq 1}\lnorm T^\ast(\psi) \rnorm_{\op} \\
                    & = \sup_{\lnorm \psi \rnorm_{\op} \leq 1}\lnorm \psi \circ T \rnorm_{\op} \\
                    & \leq \sup_{\lnorm \psi \rnorm_{\op} \leq 1}\left( \lnorm \psi \rnorm_{\op} \lnorm T \rnorm_{\op} \right)\\
                    & = \lnorm T \rnorm_{\op} \cdot \sup_{\lnorm \psi \rnorm_{\op} \leq 1}\lnorm \psi \rnorm_{\op} \\
                    & \leq \lnorm T \rnorm_{\op}.
                \end{split}
                \end{equation*}
            Thus $T^\ast$ is continuous.
        \end{proof}

    \begin{theorem}
        Let $V$ and $W$ be normed spaces with completions $\widetilde{V} = \overline{i_V(V)}$ and $\widetilde{W} = \overline{i_W(W)}$. If $T:V \rightarrow W$ is a continuous and linear map, there exists a unique continuous and linear map $\widetilde{T}:\widetilde{V} \rightarrow \widetilde{W}$ such that $\widetilde{T} \circ i_V = i_W \circ T$; i.e., the following diagram commutes:
            \begin{center}
                \begin{tikzcd}
                    \widetilde{V} \arrow[r, "\widetilde{T}"] & \widetilde{W}       \\
                    V \arrow[r, "T"'] \arrow[u, "i_V"]       & W \arrow[u, "i_W"']
                    \end{tikzcd}
            \end{center}
    \end{theorem}
        \begin{proof}
            Since $T$ is continuous, we can induce the map $T^\ast:W^\ast \rightarrow V^\ast$ where $T^\ast(\psi) = \psi \circ T$. Since $T^\ast$ is continuous, we can induce the map $T^{\ast \ast}:V^{\ast \ast} \rightarrow W^{\ast \ast}$ where $T^{\ast \ast}(\xi) = \xi \circ T^\ast$. If $v \in V$ and $\psi \in V^\ast$, note that:
                \begin{equation*}
                \begin{split}
                    (\widehat{v}\circ T^\ast)(\psi) 
                    & = (T^\ast \circ \psi)(v) \\
                    & = (\psi \circ T)(v) \\
                    & = \widehat{T(v)}(\psi).
                \end{split}
                \end{equation*}
            This allows us to show:
                \begin{equation*}
                \begin{split}
                    T^{\ast \ast} \circ i_V(v)
                    & = \widehat{v} \circ T^\ast \\
                    & = \widehat{T(v)} \\
                    & = i_W(T(v)).
                \end{split}
                \end{equation*}
            Whence $T^{\ast \ast}(i_V(V)) \subseteq i_W(W)$. Since $T^{\ast \ast}$ is continuous:
                \begin{equation*}
                \begin{split}
                    T^{\ast \ast}\left(\widetilde{V}\right)
                    & = T^{\ast \ast}\left(\overline{i_V(V)}\right) \\
                    & \subseteq \overline{T^{\ast \ast}(i_V(V))} \\
                    & \subseteq \overline{i_W(W)} \\
                    & = \widetilde{W}.
                \end{split}
                \end{equation*}
            Thus $\restr{T^{\ast \ast}}{\widetilde{V}}: \widetilde{V} \rightarrow \widetilde{W}$ is the desired linear extension. We obtain the following diagram:
                \begin{center}
                    \begin{tikzcd}
                        V^{\ast \ast} \arrow[r, "T^{\ast \ast}"]                                                        & W^{\ast \ast}                                   \\
                        \widetilde{V} \arrow[u, phantom, sloped, "\subseteq"] \arrow[r, "\restr{T^{\ast\ast}}{\widetilde{V}}"] & \widetilde{W} \arrow[u, phantom, sloped, "\subseteq"] \\
                        i_V(V) \arrow[u, phantom, sloped, "\subseteq"] \arrow[r, "\restr{T^{\ast}}{i_V(V)}"]                   & i_W(W) \arrow[u, phantom, sloped, "\subseteq"]        \\
                        V \arrow[u, "i_V", hook] \arrow[r, "T"']                                                        & W \arrow[u, "i_W"', hook]                      
                        \end{tikzcd}
                \end{center}
            
        \end{proof}

    \begin{exercise}
        Show that $\lnorm \widetilde{T} \rnorm_{\op} = \lnorm T \rnorm_{\op}$.
    \end{exercise}

    \begin{definition}
        If $V$ is a normed space and $i_V:V \rightarrow \left( V^\ast \right)^\ast$ is surjective, then $V$ is called \textit{reflexive}.
    \end{definition}

    \begin{example}
        Hilbert spaces are reflexive by the Riesz representation theorem.
    \end{example}

    \begin{center}
        \begin{tikzpicture}
            \draw[thick] (0.3,0) -- (2.3,0);
            \node at (2.39, 0) {$/\,$};
            \node at (2.56, 0) {$/\,$};
            \draw[thick] (2.6,0) -- (4.6,0);
        \end{tikzpicture}
    \end{center}

    Recall that $A \subseteq X$ is \textit{nowhere dense} if $\overline{A}^o = \emptyset$. For example, if $f:\bfR \rightarrow \bfR$ is any map, the set $\{(x,y) \in \bfR^2 \mid y = f(x)\}$ is nowhere dense.

    \begin{proposition}\label{prop:nowhere-dense-properties}
        For a metric space $(X,d)$ and $A \subseteq X$, the following are equivalent:
            \begin{enumerate}[label = (\arabic*),itemsep=1pt,topsep=3pt]
                \item $A$ is nowhere dense;
                \item There exists a closed subset $F \subseteq X$ such that $F^o = \emptyset$ and $F \supseteq A$;
                \item There exists an open and dense subset $U \subseteq X$ such that $U \subseteq A^c$.
            \end{enumerate}
    \end{proposition}
        \begin{proof}
            $(1)\Rightarrow(2)$ Take $F = \overline{A}$.

            $(2)\Rightarrow(1)$ Let $F$ be such a set. Then $\overline{A} \subseteq \overline{F}$. So $\overline{A}^o \subseteq \overline{F}^o = \emptyset$.

            $(2)\Rightarrow(3)$ Let $F$ be such a set. Let $U = F^c$. Then $\overline{U} = \overline{F^c} = (F^o)^c = \emptyset^c = X$. We also have $U = F^c \subseteq A^c$. 

            $(3)\Rightarrow(2)$ Let $U$ be such a set. Take $F = U^c$. Then $F^o = (U^c)^o = (\overline{U})^c = X^c = \emptyset$. Also $F = U^c \supseteq (A^c)^c = A$.
        \end{proof}

    \begin{definition}
        A point $x \in X$ is \textit{isolated} if there exists an $\epsilon > 0$ such that $U(x,\epsilon) = \{x\}$.
    \end{definition}

    \begin{proposition}
        Let $(X,d)$ be a metric space.
        \begin{enumerate}[label = (\arabic*),itemsep=1pt,topsep=3pt]
            \item If $A \subseteq X$ is nowhere dense and $B \subseteq A$, then $B$ is nowhere dense.
            \item If $A \subseteq X$ is nowhere dense, then $\overline{A}$ is nowhere dense.
            \item If $A_1,A_2,...,A_n$ are nowhere dense, then $\bigcup_{k = 1}^n A_k$ is nowhere dense.
            \item If $X$ has no isolated points, then every finite subset is nowhere dense.
        \end{enumerate}
    \end{proposition}
        \begin{proof}
            (1) If $B \subseteq A$ then $\overline{B} \subseteq \overline{A}$, so $(\overline{B})^o \subseteq (\overline{A})^o = \emptyset$.
            
            (2) Note that $\overline{A} = \overline{\overline{A}}$. So $(\overline{\overline{A}})^o = (\overline{A})^o = \emptyset$.

            (3) We show this for $n=2$. Let $A_1$ and $A_2$ be nowhere dense. By Proposition~\ref{prop:nowhere-dense-properties}, $A_2^c \supseteq U_1$, where $U_1$ is open and dense. Similarly, $A_2^c \supseteq U_2$, where $U_2$ is open and dense. Then:
                \begin{equation*}
                \begin{split}
                    (A_1 \cup A_2)^c 
                    & = A_1^c \cap A_2^c \\
                    & \supseteq U_1 \cap U_2.
                \end{split}
                \end{equation*}
            Clearly $U_1 \cap U_2$ is open by Proposition~\ref{prop:open-sets-topology}. Claim: $U_1 \cap U_2$ is dense. Let $x \in X$ and $\epsilon > 0$. We'd like to show, by Proposition~\ref{prop:dense-properties}, that $(U_1 \cap U_2) \cap U(x,\epsilon) \neq \emptyset$. Since $U_1$ is dense, we know $U_1 \cap U(x,\epsilon) \neq \emptyset$. Let $z \in U_1 \cap U(x,\epsilon)$. Since $U_1 \cap U(x,\epsilon)$ is open, by Definition~\ref{def:2.2.2} there exists a $\delta > 0$ such that $U(z,\delta) \subseteq U_1 \cap U(x,\epsilon)$. Now since $U_2$ is dense, $U(z,\delta) \cap U_2 \neq \emptyset$. Therefore $U(x,\epsilon) \cap (U_1 \cap U_2) \neq \emptyset$. Thus $U_1 \cap U_2$ is dense.

            (4) Since $X$ has no isolated points, $\{x\}$ is closed but not open. Then $(\overline{\{x\}})^o = \{x\}^o = \emptyset$. So $\{x\}$ is nowhere dense. By (3), any finite set $\{x_1,x_2,...,x_n\} = \{x_1\} \cup \{x_2\} \cup ... \cup \{x_n\}$ is nowhere dense.
        \end{proof}

    Note that $\bfQ$ is not nowhere dense, but it is the \textit{countable} union of nowhere dense sets.

    \begin{definition}
        Let $(X,d)$ be a metric space.
            \begin{enumerate}[label = (\arabic*),itemsep=1pt,topsep=3pt]
                \item A set $A \subseteq X$ is \textit{meager} if $A = \bigcup_{k = 1}^\infty A_k$, where each $A_k$ is nowhere dense.
                \item A set $B \subseteq X$ is \textit{residual} if $B^c$ is meager.
            \end{enumerate}
    \end{definition}

    \begin{example}
        A meager set is "topologically small", whereas a residual set is "topologically big". Consider $\bfQ \subseteq \bfR$ meager. Then $\bfR\setminus \bfQ \subseteq \bfR$ is residual.

        Note that this definition only applies to \textit{subsets} of a specified metric space. For instance, $\bfZ \subseteq \bfR$ is meager, but $\bfZ \subseteq \bfZ$ is not meager. Indeed, $\bfZ$ with the discrete topology means that every subset of $\bfZ$ is open, and every subset of $\bfZ$ is closed. So given $A \subseteq \bfZ$ nowhere dense, we have $A = (\overline{A})^c = \emptyset$. So if $A \subseteq \bfZ$ is meager, then $A \neq \emptyset$.
    \end{example}

    \begin{proposition}
        Let $(X,d)$ be a metric space.
        \begin{enumerate}[label = (\arabic*),itemsep=1pt,topsep=3pt]
            \item If $A \subseteq X$ is meager and $B \subseteq A$, then $B$ is meager.
            \item If $A_1,A_2,...,A_k$ are meager, then $\bigcup_{k = 1}^\infty A_k$ is meager.
            \item If $X$ has no isolated points, every countable set is meager.
        \end{enumerate}
    \end{proposition}
        \begin{proof}
            (1) If $A$ is meager, then $A = \bigcup_{k = 1}^\infty A_k$, where each $A_k$ is nowhere dense. Then:
                \begin{equation*}
                \begin{split}
                    B 
                    & = B \cap A \\
                    & = B \cap \bigcup_{k = 1}^\infty A_k \\
                    & = \bigcup_{k = 1}^\infty (B \cap A_k)
                \end{split}
                \end{equation*}
            Since $B \cap A_k$ is nowhere dense for all $k$, $B$ is meager.

            (2) If $A_k$ is meager, then $A_k = \bigcup_{j = 1}^\infty {A_k}_j$, where ${A_k}_j$ is nowhere dense for each $j$. We have:
                \begin{equation*}
                \begin{split}
                    \bigcup_{k = 1}^\infty A_k 
                    & = \bigcup_{k = 1}^\infty \bigcup_{j = 1}^\infty {A_k}_j.
                \end{split}
                \end{equation*}
            Since the countable union of countable sets is countable, we can see $\bigcup_{k = 1}^\infty A_k$ is meager.

            (3) We can write $\{x_k \mid k \geq 1\}$ as $\bigcup_{k = 1}^\infty \{x_k\}$. Since singletons are nowhere dense, every countable set is meager.
        \end{proof}

    \begin{proposition}[Cantor's Intersection Theorem]\label{prop:cantor-intersection-thm}
        Let $(X,d)$ be a complete metric space and $F_1 \supseteq F_2 \supseteq F_3 \supseteq ...$ be a nested sequence of closed nonempty sets with $(\diam(F_n))_n \rightarrow 0$. Then $\bigcap_{n \geq 1}F_n = \{x\}$ for some $x \in X$.
    \end{proposition}
        \begin{proof}
            Let $x_n \in F_n$ for all $n \geq 1$. Note that $(x_n)_n$ is Cauchy \textemdash given $\epsilon > 0$, let $N$ be large so that $n \geq N$ implies $\diam(F_n) < \epsilon$. For $m,n \geq N$, we have $x_m,x_n \in F_N$ (because $F_N \supseteq F_{N+1} \supseteq ...$), which gives:
                \begin{equation*}
                \begin{split}
                    d(x_n,x_m)
                    & \leq \sup_{x_n,x_m \in F_N} d(x_n,x_m) \\
                    & = \diam(F_N) \\
                    & < \epsilon.
                \end{split}
                \end{equation*}
            Since $X$ is complete, $(x_n)_n \rightarrow x_0$ for some $x_0 \in X$. 

            Claim: $\bigcap_{n \geq 1}F_n = \{x_0\}$. To see this, fix $m \in \bfN$ and consider the sequence $(x_{m+k})_k$ in $F_m$. Since $(x_n)_n$ converges to $x_0$, we know the tail $(x_{m+k})_k \xrightarrow{k \rightarrow \infty} x_0$. Since $F_m$ is closed, we know $x_0 \in F_m$. Since $m$ was arbitrary, $x_0 \in \bigcap_{n \geq 1}F_n$. Now suppose $x_0,x \in \bigcap_{n \geq 1}F_n$ with $d(x_0,x) > 0$. We can find $N$ large so that $\diam(F_n) < d(x_0,x)$ But $x_0,x \in \bigcap_{n \geq 1}F_n$ implies $x_0,x \in F_n$, giving $d(x_0,x) < d(x_0,x)$, which is a contradiction. Thus $\bigcap_{n \geq 1}F_n = \{x_0\}$.
        \end{proof}

    \begin{example}
        Note that $\bigcap_{n \geq 1}(0,\frac{1}{n}) = \emptyset$ becaue $(0,\frac{1}{n})$ is not closed, and $\bigcap_{n \geq 1}[n,\infty) = \emptyset$ because $(\diam([n,\infty)))_n \nrightarrow 0$. 
    \end{example}

    \begin{theorem}[Baire Category Theorem]\label{thm:baires-theorem}
        Let $(X,d)$ be a complete metric space.
        \begin{enumerate}[label = (\arabic*),itemsep=1pt,topsep=3pt]
            \item If $\{V_k\}_{k = 1}^\infty$ is a countable family of open and dense subsets, then $\bigcap_{k = 1}^\infty V_k$ is also dense (but not necessarily open).
            \item $X$ is not meager.
        \end{enumerate}
    \end{theorem}
        \begin{proof}
            (1) Let $U_0$ be any open ball. Since $V_1$ is open and dense, $U_0 \cap V_1$ is open and nonempty. So there exists an open ball $U(x,\delta) \subseteq U_0 \cap V_1$. Notice we also have $U_1:= U(x,\frac{\delta}{2}) \subseteq U_0 \cap V_1$. Define $B_1 = \overline{U_1} \subseteq U_0 \cap V_1$. Since $\delta$ was arbitrary, we arbitrarily shrink out open ball to ensure $\diam(B_1) < 1$. Now $U_1 \cap V_2$ is open and nonempty. So there exists some open ball contained in $U_1 \cap V_2$. Exactly as we just did, find $B_2 := \overline{U_2} \subseteq U_1 \cap V_2$ such that $\diam(B_2) < \frac{1}{2}$. Inductively, having constructed $U_1,B_1,U_2,B_2,...$, we note that $U_{n-1} \cap V_n$ is open and nonempty, so it contains an open ball. Find $B_n := \overline{U_n} \subseteq U_{n-1} \cap V_n$ such that $\diam(B_n) < \frac{1}{n}$. Observe that:
                \begin{equation*}
                \begin{split}
                    B_1 \supseteq U_1 \supseteq B_2 \supseteq U_2 \supseteq B_3 \supseteq U_3 \supseteq ...
                \end{split}
                \end{equation*}
            So $\{B_n\}_{n \geq 1}$ is a nested sequence of closed sets with $(\diam(B_n))_n \rightarrow 0$. By Cantor's Intersection Theorem, $\cap_{n \geq 1}B_n = \{x\}$. Claim: $x \in U_0 \cap \bigl( \bigcap_{k=1}^\infty V_k \bigr)$. Note that $B_n \subseteq U_{n-1} \cap V_n \subseteq V_n$. So $x \in \bigcap_{n \geq 1}B_n \subseteq \bigcap_{n \geq 1}V_n$. Also, $x \in B_1 = \overline{U_1} \subseteq U_0 \cap V_1 \subseteq U_0$. This means $x \in U_0 \cap \bigl( \bigcap_{k=1}^\infty V_k \bigr)$. Since $U_0$ was an arbitrary open set, and since it's intersection with $\bigcap_{k=1}^\infty V_k$ is nonempty, we can conclude that $\bigcap_{k=1}^\infty V_k$ is dense.

            (2) Suppose towards contadiction that $X$ is meager, that is, $X = \bigcup_{k = 1}^\infty A_k$, where each $A_k$ is nowhere dense. Fixing $k$, by Proposition~\ref{prop:nowhere-dense-properties} we can find an open and dense set $V_k$ with $V_k \subseteq A_k^c$. But then:
                \begin{equation*}
                \begin{split}
                    \emptyset
                    & = X^c \\
                    & = \left( \bigcup_{k = 1}^\infty A_k \right)^c \\
                    & = \bigcap_{k = 1}^\infty A_k^c \\
                    & \supseteq \bigcap_{k = 1}^\infty V_k.
                \end{split}
                \end{equation*}
            This is a contradiction. Thus, if $X$ is complete, then $X$ is not meager.
        \end{proof}

    \begin{exercise}
        Show that $\bfR\setminus \bfQ$ is not meager.
    \end{exercise}

    \vspace{5pt}
    \begin{center}
        {Skipped all of the "application" stuff}
    \end{center}
    \vspace{5pt}

    \begin{example}
        Let $X = \bigl(C([0,1]),\lnorm \cdot \rnorm_u\bigr)$. Recall that $f(x) = |x|$ is continuous everywhere, but not differentiable at $x = 0$. Does there exist a function $f:X \rightarrow \bfR$ which is continuous \textit{everywhere}, but differentiable \textit{nowhere}. The answer is yes, and in fact, the set:
            \begin{equation*}
            \begin{split}
                \left\{ f \in X \mid f \h3\text{differentiable nowhere}\h1 \right\}
            \end{split}
            \end{equation*}
        is \textit{residual}. This is surprising, as the complement of this set\textemdash the set containing every continuous function which is differentiable at one or more points, is "topologically small."
    \end{example}

\section{Compactness}\label{sec:compactness}
    Compactness is a generalization of "closed and bounded in $\bfR^n$ for metric spaces. It is the algebraic analogue of "finite-dimensional".

    \begin{definition}
        Let $(X,d)$ be a metric space and $K \subseteq X$ a subset.
        \begin{enumerate}[label = (\arabic*),itemsep=1pt,topsep=3pt]
            \item A \textit{cover} for $K$ is a family $\cU = \{U_i\}_{i \in I}$ with $U_i \subseteq X$ for all $i$ and $K \subseteq K \subseteq \bigcup_{i \in I}U_i$.
            \item The family $\cU$ is an \textit{open cover} for $K$ if each $U_i$ is open.
            \item The family $\cU$ is a \textit{finite cover} for $K$ if $I$ is finite.
            \item A \textit{subcover} of $\cU$ is a subfamily $\{U_i\}_{i \in J}$, where $J \subseteq I$ and $K \subseteq \bigcup_{i \in J}U_i$.
            \item $K$ is called \textit{compact} if every open cover $\cU$ of $K$ admits a finite subcover. That is, if $K \subseteq \bigcup_{i \in I}U_i$ with each $U_i \subseteq X$ open, there exists a finite subset $F \subseteq I$ with $K \subseteq \bigcup_{i \in F}U_i$.
        \end{enumerate}
    \end{definition}

    Since $K$ is not necessarily proper, a metric space $X$ is compact if it can be written as the union of finitely many open sets.

    \begin{example}
        The set $(0,1]$ is not compact in $\bfR$. The family $\left\{\left(\frac{1}{n},2\right)\right\}_{n \geq 1}$ is an open cover of $(0,1]$, but admits no finite subcover.
    \end{example}

    \begin{exercise}
        Let $(X,d)$ be a metric space with $Y \subseteq X$ and $K \subseteq Y$. $K \subseteq X$ is compact if and only if $K \subseteq Y$ is compact.
    \end{exercise}

    \begin{proposition}\label{prop:compact-subsets}
        Let $(X,d)$ be a metric space and $K \subseteq X$.
            \begin{enumerate}[label = (\arabic*),itemsep=1pt,topsep=3pt]
                \item If $K$ is compact, then $K$ is closed and bounded,
                \item If $X$ is compact and $K$ is closed, then $K$ is compact.
            \end{enumerate}
    \end{proposition}
        \begin{proof}
            (1) We will first show that $K$ is bounded. Pick $x_0 \in K$, Note that $\{U(x_0,n)\}_{n = 1}^\infty$ is a cover for $K$, whence $K \subseteq \bigcup_{n \geq 1}U(x_0,n)$. By the compactness of $K$, $K \subseteq \bigcup_{n = 1}^N U(x_0,n)$. For $x,x' \in K$:
                \begin{equation*}
                \begin{split}
                    d(x,x')
                    & \leq d(x,x_0) + d(x_0,x') \\
                    & \leq 2N.
                \end{split}
                \end{equation*}
            Thus $\diam(K) \leq 2N$, so $K$ is bounded. We will now show that $K^c$ is open. Let $x_0 \in K^c$ be arbitrary. For each $x \in K$, we can find a $\delta_x > 0$ such that $U(x_0,\delta_x) \cap U(x,\delta_x) = \emptyset$. Note that $K \subseteq \bigcup_{x \in K}U(x,\delta_x)$. Since $K$ is compact, there exists $x_1,...,x_n \in K$ with $K \subseteq \bigcup_{j = 1}^n U(x_j,\delta_{x_j})$. Let $\delta = \min_{j = 1}^n \delta_{x_j}$. Claim: $U(x_0,\delta) \subseteq K^c$. If not, we can find some $z \in U(x_0,\delta) \cap K$, which implies $z \in U(x_0,\delta) \cap U(x_j,\delta_{x_j})$ for some $j$. But this means $\emptyset \neq U(x_0,\delta) \cap U(x_j,\delta_{x_j}) \subseteq U(x_0,\delta_{x_j}) \cap U(x_j,\delta_{x_j}) = \emptyset$. Thus $K$ is closed.

            (2) Let $K \subseteq \bigcup_{i \in I}U_i$, where $U_i \subseteq X$ is open. Then $X = K^c \cup \left( \bigcup_{i \in I}U_i \right)$. Since $K^c \cup \left( \bigcup_{i \in I}U_i \right)$ is an open cover of $X$, there exists a finite subset $F \subseteq I$ with $X = K^c \cup \left( \bigcup_{i \in F}U_i \right)$. It follows that $K \subseteq \bigcup_{i \in F}U_i$.
        \end{proof}
    
    \begin{definition}
        Let $(X,d)$ be a metric space. A non-empty family of subsets $\{A_i\}_{i \in I}$ is said to have the \textit{finite intersection property} if for any finite subset $F \subseteq I$, we have $\bigcap_{i \in F}A_i \neq \emptyset$.
    \end{definition}

    \begin{proposition}\label{prop:closed-fip}
        Let $(X,d)$ be a metric space. The following are equivalent:
            \begin{enumerate}[label = (\arabic*),itemsep=1pt,topsep=3pt]
                \item $X$ is compact;
                \item For any family of closed subsets $\{C_i\}_{i \in I}$ satisfying the finite intersection property, we have $\bigcap_{i \in I}C_i \neq \emptyset$.
            \end{enumerate}
    \end{proposition}
        \begin{proof}
            Suppose that $X$ is compact and let $\{C_i\}_{i \in I}$ be a family of closed sets satisfying the finite intersection property. Assume towards contradiction that $\bigcap_{i \in I}C_i = \emptyset$. Then $\bigcup_{i \in I}C_i^c = X$. Since $X$ is compact, there exists some finite subset $F \subseteq I$ with $\bigcup_{i \in F}C_i^c = X$. But this implies $\bigcap_{i \in F}C_i = \emptyset$, contradicting our family of closed subsets satisfying the finite intersection property.

            We will assume the contrapositive of (2). Let $X = \bigcup_{i \in I}V_i$, where each $V_i$ is open. Then $\emptyset = \bigcap_{i \in I}V_i^c$. Since $\{V_i^c\}_{i \in I}$ is a family of closed subsets, it must be the case that $\bigcap_{i \in F}V_i^c = \emptyset$ for some finite subset $F \subseteq I$. Whence $\bigcup_{i \in F}V_i = X$.
        \end{proof}

    \begin{proposition}***
        Every compact metric space is separable.
    \end{proposition}
        \begin{proof}
            Fix $n \geq 1$. We have an open cover $X = \bigcup_{x \in X}U(x,\frac{1}{n})$, which admits a subcover $X = \bigcup_{j = 1}^{J_n} U\left(x_{n,j}, \frac{1}{n}\right)$. Let $S = \{x_{n,j} \mid n \geq 1, 1 \leq j \leq J_n\}$. 

            Let $x \in X$ and $\epsilon > 0$. Let $m$ be such that $\epsilon > \frac{1}{m}$. Find $j \in \{1,...,J_m\}$ with $x \in U(x_{m,j},\frac{1}{m})$. Thus $d(x,x_{m,j})< \frac{1}{m}$. Whence $S \cap U(x,\epsilon) \neq \emptyset$. So $S$ is dense and certainly countable.
        \end{proof}

    \begin{definition}
        Let $(X,d)$ be a metric space and let $K \subseteq X$, $K$ is called \textit{sequentially compact} if every sequence $(x_n)_n$ in $K$ admits a convergent subsequence in $K$.
    \end{definition}

    \begin{proposition}\label{prop:compact-implies-seq-comp}
        Let $(X,d)$ be a metric space. If $K \subseteq X$ is compact, then $K$ is sequentially compact.
    \end{proposition}
        \begin{proof}
            Let $(x_k)_k$ be a sequence in $K$. Consider $C_n = \overline{\{x_k \mid k > n\}}$. Note that:
                \begin{equation*}
                    C_1 \supseteq C_2 \supseteq C_3 \supseteq...
                \end{equation*}
            In particular, $\bigcap_{j = 1}^n C_j \neq \emptyset$ for any $n > 1$. So $\{C_n\}_{n \geq 1}$ is a family of closed sets which satisfies the finite intersection property. By Proposition~\ref{prop:closed-fip} we have $\bigcap_{n \geq 1}C_n \neq \emptyset$. Let $x \in \bigcap_{n \geq 1}C_n$. 
            
            Clearly $x \in C_1$. This means $U(x,1)\cap \{x_k \mid k >1 \} \neq \emptyset$. Let $x_{k_1} \in U(x,1) \cap C_1$. Since $x_{k_1} \in C_1$, we know that $k_1 > 1$. Moreover, $d(x,x_{k_1}) < 1$. 
            
            Similarly, $x \in C_{k_1}$. This means $U(x,\frac{1}{2}) \cap \{x_k \mid k > k_1\} \neq \emptyset$. Let $x_{k_2} \in U(x,\frac{1}{2}) \cap \{x_k \mid k > k_1\}$. Since $x_{k_2} \in\{x_k \mid k > k_1\}$, we know $k_2 > k_1$. Moreover, $d(x,x_{k_2}) < \frac{1}{2}$. 
            
            Inductively, $x \in C_{k_{j-1}}$. This means $U(x,\frac{1}{j}) \cap \{x_k \mid k > k_{j - 1}\} \neq \emptyset$. Let $x_{k_j} \in U(x,\frac{1}{j}) \cap \{x_k \mid k > k_{j - 1}\}$. Since $x_{k_j} \in \{x_k \mid k > k_{j - 1}\}$, we know $k_j > k_{j-1}$. Moreover, $d(x,x_{k_j}) < \frac{1}{j}$. We've obtained a sequence $(x_{k_j})_j$ which converges to $x$. Thus $(X,d)$ is sequentially compact.
        \end{proof}

    \begin{proposition}\label{prop:seq-implies-complete}
        If $(X,d)$ is sequentially compact, then $(X,d)$ is complete.
    \end{proposition}
        \begin{proof}
            Let $(x_n)_n$ be Cauchy. We know there exists a subsequence $(x_{n_k})_k$ which converges to some $x \in X$. Let $\epsilon > 0$. Find $N$ large so $p,q \geq N$ implies $d(x_p,x_q) < \frac{\epsilon}{2}$. Find $K$ large so $k \geq K$ implies $d(x_{n_k},x) < \frac{\epsilon}{2}$. For $n,k \geq \max\{N,K\}$:
                \begin{equation*}
                \begin{split}
                    d(x_n,x)
                    & \leq d(x_n,x_{n_k}) + d(x_{n_k}, x) \\
                    & < \frac{\epsilon}{2} + \frac{\epsilon}{2} \\
                    & = \epsilon.
                \end{split}
                \end{equation*}
            Thus $(x_n)_n \rightarrow x$. Whence $(X,d)$ is complete.
        \end{proof}

    \begin{definition}
        Let $(X,d)$ be a metric space. We say $Y \subseteq X$ is \textit{totally bounded} if for all $\epsilon > 0$, there exists a finite subset $F\subseteq Y$ such that $Y \subseteq \bigcup_{z \in F}U(z,\epsilon)$.
    \end{definition}

    \begin{proposition}\label{prop:sequential-implies-tot-bounded}
        Let $(X,d)$ be a metric space. If $K \subseteq X$ is sequentially compact, then $K$ is totally bounded.
    \end{proposition}
        \begin{proof}
            Suppose towards contradiction that $K$ is not totally bounded. Then there exists some $\epsilon_0 > 0$ such that, for any finite subset $F \subseteq K$, we have $K \not\subseteq \bigcup_{z \in F}U(z,\epsilon_0)$. Let $x_1 \in K$ be arbitrary. Then $K \not\subseteq U(x_1,\epsilon_0)$, since $K$ cannot be covered by finitely many open balls. Find $x_2 \in K \setminus U(x_1,\epsilon_0)$. Then $K \not\subseteq U(x_1,\epsilon_0) \h1\cup\h1 U(x_2,\epsilon_0)$. Find $x_3 \in K \setminus \left( U(x_1,\epsilon_0) \cup U(x_2,\epsilon_0) \right)$. Then $K \not\subseteq U(x_1,\epsilon_0) \cup U(x_2,\epsilon_0) \cup U(x_3,\epsilon_0)$. Inductively, we obtain a sequence $(x_n)_n$, where $x_n \in K \setminus \bigcup_{j = 1}^{n-1}U(x_j,\epsilon_0)$. Since $x_n \not\in \bigcup_{j = 1}^{n-1}U(x_j,\epsilon_0)$, we have that $d(x_n,x_j) \geq \epsilon_0$ for all $j \neq n$. Now since $X$ is sequentially compact, $(x_n)_n$ admits a convergent subsequence, call it $(x_{n_k})_k$. Since $(x_{n_k})_k$ converges, it is Cauchy. But this is a contradiction, because $d(x_{n_l},x_{n_k}) \geq \epsilon_0$ for all $l \neq k$. It must be the case that $K$ is totally bounded.
        \end{proof}

    \begin{proposition}
        If $(X,d)$ is a metric space and $K \subseteq X$ is totally bounded, then $K$ is bounded.
    \end{proposition}
        \begin{proof}
            Let $\epsilon = 1$. We have that $K \subseteq \bigcup_{j = 1}^n U(x_j,1)$. Define $C = \max_{1 \leq i,j \leq n}d(x_i,x_j)$. If $x,y \in K$, then $x \in U(x_i,1)$ and $y \in U(x_j,1)$ for some $i,j$. This gives:
                \begin{equation*}
                \begin{split}
                    d(x,y)
                    & \leq d(x,x_i) + d(x_i,x_j) + d(x_j,y) \\
                    & \leq 2 + C.
                \end{split}
                \end{equation*}
            Thus $\diam(K) \leq 2+C$.
        \end{proof}

    \begin{corollary}\label{cor:compact-sup}
        If $K \subseteq \bfR$ is compact and nonempty, then $\sup(K) \in K$.
    \end{corollary}
        \begin{proof}
            Define $u:= \sup(K)$. Let $(x_n)_n$ be a sequence in $K$ converging to $u$. Note that $u \in \bfR$ since $K$ is bounded. Since $K$ is closed, $u \in K$.
        \end{proof}

    \begin{theorem}\label{thm:gen-heine-borel}
        Let $(X,d)$ be a metric space. The following are equivalent:
            \begin{enumerate}[label = (\arabic*),itemsep=1pt,topsep=3pt]
                \item $X$ is compact;
                \item $X$ is sequentially compact;
                \item $X$ is complete and totally bounded.
            \end{enumerate}
    \end{theorem}
        \begin{proof}
            Propositions \ref{prop:compact-implies-seq-comp}, \ref{prop:seq-implies-complete}, and \ref{prop:sequential-implies-tot-bounded} established (1)$\Rightarrow$(2) and (2)$\Rightarrow$(3). It remains to prove (3)$\Rightarrow$(1).
            Suppose towards contradiction $\cV$ is an open cover for $X$ which fails to admit a finite subcover. Let $\epsilon = 1$. By total boundedness:
                \begin{equation*}
                \begin{split}
                    X = U(x_1,1) \cup U(x_2,1) \cup ... \cup U(x_{m_1},1).
                \end{split}
                \end{equation*}
            There must be at least one $U(x_j,1)$ that can't be covered by finitely many members of $\cV$. Label one of these balls at $B_{1}(x_1)$. Let $\epsilon = \frac{1}{2}$. By total boundedness,
                \begin{equation*}
                \begin{split}
                    X = U(x_1,\sfrac{1}{2}) \cup U(x_2,\sfrac{1}{2}) \cup ... \cup U(x_{m_2},\sfrac{1}{2}).
                \end{split}
                \end{equation*}
            Note that:
                \begin{equation*}
                \begin{split}
                    \bigcup_{j = 1}^{m_2}\Bigl( B_{1}(x_1) \cap U(x_j,\sfrac{1}{2}) \Bigr) 
                    & = B_{1}(x_1) \cap \left( \bigcup_{j = 1}^{m_2} U(x_j,\sfrac{1}{2})\right) \\
                    & = B_{1}(x_1) \cap X \\
                    & = B_{1}(x_1).
                \end{split}
                \end{equation*}
            There must be some $j$ such that $B_{x_1}(1) \cap U(x_j,\sfrac{1}{2})$ cannot be covered by finitely many mebers of $\cV$. Label $U(x_j,\sfrac{1}{2})$ as $B_{\sfrac{1}{2}}(x_2)$. We continue this process inductively, along which we obtain a sequence $(x_n)_n$ such that
                \begin{equation*}
                \begin{split}
                    F_n := B_{1}(x_1) \cap B_{\sfrac{1}{2}}(x_2) \cap B_{\sfrac{1}{3}}(x_3) \cap ... \cap B_{\sfrac{1}{n}}(x_n),
                \end{split}
                \end{equation*}
            cannot be covered by finitely many members of $\cV$. Define $C_n = \overline{F_n}$. Note that $C_1 \supseteq C_2 \supseteq C_3 \supseteq...$ and $\diam(C_n) \leq \frac{2}{n}$. Since $X$ is complete, by \nameref{prop:cantor-intersection-thm} $\bigcap_{n \geq 1}C_n = \{x\}$. Locate this $x$ in $V \in \cV$. Since $V$ is open, we can find some $\epsilon > 0$ with $U(x,\epsilon) \subseteq V$. Find $N$ large so that $\frac{2}{N} < \epsilon$. We can see that $C_N \subseteq U(x,\epsilon)$ since $d(z,x) \leq \frac{2}{N} < \epsilon$ for all $z \in C_N$. But this means $F_N \subseteq C_N \subseteq U(x,\epsilon) \subseteq V$. This is a contradiction, as we've asserted $F_n$ cannot be covered by finitely many members of $\cV$. Thus $X$ is compact.
        \end{proof}

    \begin{center}
        \begin{tikzpicture}
            \draw[thick] (0.3,0) -- (2.3,0);
            \node at (2.39, 0) {$/\,$};
            \node at (2.56, 0) {$/\,$};
            \draw[thick] (2.6,0) -- (4.6,0);
        \end{tikzpicture}
    \end{center}

    We're interested in studying compact subsets of $\bfR^d$.

    (Heine-Borel stuff here...)

    \begin{center}
        \begin{tikzpicture}
            \draw[thick] (0.3,0) -- (2.3,0);
            \node at (2.39, 0) {$/\,$};
            \node at (2.56, 0) {$/\,$};
            \draw[thick] (2.6,0) -- (4.6,0);
        \end{tikzpicture}
    \end{center}

    Compactness and Continuity 

    \begin{proposition}\label{prop:compact-cont}
        Let $f:(X,d) \rightarrow (Y,d)$ be continuous and $K \subseteq X$ compact. Then $f(K) \subseteq Y$ is compact. 
    \end{proposition}
        \begin{proof}
            Let $f(K) \subseteq \bigcup_{i \in I}V_i$, where each $V_i \subseteq Y$ is open. Then:
                \begin{equation*}
                \begin{split}
                    K 
                    & \subseteq f^{-1} (f(K)) \\
                    & \subseteq f^{-1} \left( \bigcup_{i \in I}V_i \right) \\
                    & = \bigcup_{i \in I}f^{-1}(V_i).
                \end{split}
                \end{equation*}
            Since $f$ is continuous, $f^{-1}(V_i) \subseteq X$ is open. So there exists a finite subset $F \subseteq I$ with $K \subseteq \bigcup_{i \in F}f^{-1}(V_i)$. Whence:
                \begin{equation*}
                \begin{split}
                    f(K)
                    & \subseteq f \left( \bigcup_{i \in F}f^{-1}(V_i) \right) \\
                    & = \bigcup_{i \in F}f(f^{-1}(V_i)) \\
                    & = \bigcup_{i \in F}V_i. \qedhere
                \end{split}
                \end{equation*}
        \end{proof}

    \begin{corollary}
        If $d_1$ and $d_2$ are topologically equivalent metrics on $X$, then $K \subseteq X$ is $d_1$-compact if and only if $K \subseteq X$ is $d_2$-compact.
    \end{corollary}
        \begin{proof}
            By the definition of topological equivalence $id:(X,d_1) \rightarrow (X,d_2)$ is continuous. Let $K \subseteq X$ be $d_1$-compact. Then $K = \id(K) \subseteq (X,d_2)$. The previous proposition says $K$ is $d_2$-compact. The converse direction is identical.
        \end{proof}

    \begin{theorem}[Extreme Value Theorem]
        Let $f:(X,d) \rightarrow \bfR$ be continuous and $K \subseteq X$ compact. Then there exists $x_m,x_M \in K$ with $f(x_m) = \inf_{x \in K}f(x)$ and $f(x_M) = \sup_{x \in K}f(x)$.
    \end{theorem}
        \begin{proof}
            If $f$ is continuous, then $f(K) \subseteq \bfR$ is compact. Then Proposition~\ref{cor:compact-sup} gives $\inf(f(K)),\sup(f(K)) \in f(K)$. Hence there exists $x_m,x_M $ with $f(x_m) = \inf(K)$ and $f(x_M) = \sup(K)$.
        \end{proof}

    \begin{proposition}
        Let $V$ be a finite dimensional vector space over $F$. All norms on $V$ are equivalent.
    \end{proposition}
        \begin{proof}
            Let $\{v_1,...,v_n\}$ be a basis for $V$. Define $\lnorm \cdot \rnorm_1 : V \rightarrow [0,\infty)$ by:
                \begin{equation*}
                \begin{split}
                    \lnorm \sum_{i = 1}^n t_j v_j \rnorm_1 = \sum_{i = 1}^n |t_j|.
                \end{split}
                \end{equation*}
            Then $\varphi:\ell_1^n \rightarrow (V,\lnorm \cdot \rnorm_1)$ given by $(t_1,...,t_n) \mapsto \sum_{i = 1}^n t_i v_i$ is a linear isomorphism and isometry. Indeed:
                \begin{equation*}
                \begin{split}
                    \lnorm \varphi((t_1,...,t_n)) - \varphi((t_1',...,t_n')) \rnorm_1
                    & = \lnorm \sum_{i = 1}^n t_i v_i - \sum_{i = 1}^n t_i' v_i \rnorm_1 \\
                    & = \lnorm \sum_{i = 1}^n(t_i - t_i')v_i \rnorm_1 \\
                    & = \sum_{i = 1}^n |t_i - t_i'| \\
                    & = \lnorm (t_1,...,t_n) - (t_1',...,t_n') \rnorm_{\ell_2^n}.
                \end{split}
                \end{equation*}
            By the Heine-Borel Theorem $B_{\ell_1^n} = \{t \in \ell_1^n \mid \lnorm t \rnorm_{\ell_1^n} \leq 1\}$ is compact as it is a closed and bounded subset of $\ell_1^n$. Moreover, $B_1 = \{v \in V \mid \lnorm V \rnorm \leq 1\}$ is compact by Proposition~\ref{prop:compact-cont} since $B_1 = \varphi(B_{\ell_1^n})$. Now $S_1 = \{v \in V \mid \lnorm v \rnorm_1 =1 \} \subseteq B_1$ is closed, hence compact. Let $\lnorm \cdot \rnorm$ be any norm on $V$. We can see:
                \begin{equation*}
                \begin{split}
                    \lnorm \sum_{i = 1}^n t_i v_i \rnorm
                    & = \sum_{i = 1}^n |t_i|\lnorm v_i \rnorm \\
                    & \leq c \sum_{i = 1}^n |t_i| \\
                    & = c \lnorm \sum_{i = 1} ^n t_i v_i\rnorm_1,
                \end{split}
                \end{equation*}
            where $c = \max_{i = 1}^n \lnorm v_j \rnorm$. 
            
            Now consider $g:(V,\lnorm \cdot \rnorm_1) \rightarrow \bfR$ given by $g(v) = \lnorm v \rnorm$. We have:
                \begin{equation*}
                \begin{split}
                    |g(v) - g(v')| 
                    & = | \lnorm v \rnorm - \lnorm v' \rnorm| \\
                    & \leq \lnorm v - v' \rnorm \\
                    & \leq c \lnorm v-v' \rnorm_1.
                \end{split}
                \end{equation*}
            Thus $g$ is continuous. Since $S_1 \subseteq (V,\lnorm \cdot \rnorm_1)$ is compact, the Extreme Value Theorem says there exists $w \in S_1$ with $g(w) = \inf_{v \in S_1}g(v)$. If $g(w) = \lnorm w \rnorm = 0$, then $w = 0$. This is a contradiction since $w \in S_1$, so it must be the case that $g(w) > 0$. Define $c' := g(w)$, then $c' \leq g(v) = \lnorm v \rnorm$ for every $v \in V$. Using the fact that $\frac{v}{\lnorm v \rnorm_1} \in S_1$ for any nonzero $v \in V$, we can see $c' \leq \lnorm \frac{v}{\lnorm v \rnorm_1} \rnorm $ implies $\lnorm v \rnorm_1 \leq \frac{1}{c'}\lnorm v \rnorm$. By transitivity, all norms on $V$ are equivalent.
        \end{proof}

    \begin{corollary}
        Let $V$ be a normed space and $W \subseteq V$ a finite dimensional subspace. Then $W \subseteq V$ is closed. 
    \end{corollary}
        \begin{proof}
            We know that all norms on $V$ are equivalent, so there exists a uniformism and linear isomorphism $\varphi:W \rightarrow \ell_1^n$. Let $(w_n)_n$ be a sequence in $W$ converging to some $v \in V$. Since $(w_n)_n$ is convergent, it is $\lnorm \cdot \rnorm$-Cauchy. Since $\varphi$ is uniformly continuous, $(\varphi(w_n))_n$ is $\lnorm \cdot \rnorm_{\ell_1^n}$-Cauchy, whence it converges to some $x \in \ell_1^n$. Since $\varphi^{-1}$ is uniformly continuous, we have that $(w_n)_n \rightarrow \varphi^{-1}(x) \in W$. Thus $v = \varphi^{-1}(x) \in W$; i.e., $W$ is closed.
        \end{proof}

    \begin{lemma}[Riesz]***
        Let $V$ be a normed space and $W$ a closed, proper subspace of $V$. For each $t \in (0,1)$, there is a $v_t \in S_V$ with $\dist_W(v_t) \geq t$.
    \end{lemma}
        \begin{proof}
            Recall that $\dist_W:V \rightarrow [0,\infty)$ is given by $v \mapsto \inf_{w \in W}\lnorm v - w \rnorm$. Let $v_0 \in V\setminus W$ and define $\delta:= \dist_W(v_0)$. Clearly $\delta > 0$ \textemdash if not the closedness of $W$ implies $v_0 \in W$, which is a contradiction. Since $t \in (0,1)$, we have $\delta < \frac{\delta}{t}$. Choose $w_t \in W$ with $\delta < \lnorm v_0 - w_t \rnorm < \frac{\delta}{t}$. Define $v_t := \frac{w_t - v_0}{\lnorm w_t - v_0 \rnorm}$.
        \end{proof}

    \begin{theorem}\label{thm:finite-iff-compact}
        Let $V$ be a normed space and $B_V = \{v \in V \mid \lnorm v \rnorm \leq 1\}$. The following are equivalent:
            \begin{enumerate}[label = (\arabic*),itemsep=1pt,topsep=3pt]
                \item $B_V$ is compact;
                \item $\dim(V) < \infty$.
            \end{enumerate}
    \end{theorem}
        \begin{proof}
            Let $B_V$ be compact and suppose towards contradiction $\dim(V) = \infty$. Choose $v_1 \in S_V$. Since $V$ is infinite dimensional, $\Span\{v_1\} \neq V$. By Riesz' Lemma, there exists a $v_2 \in S_V$ so that $\dist_{\Span\{v_1\}}(v_2) \geq \frac{1}{2}$. In particular, $\lnorm v_2 - v_1 \rnorm \geq \frac{1}{2}$. Again, since $V$ is infinite dimensional, $\Span\{v_1,v_2\} \neq V$. By Riesz' Lemma, there exists a $v_3 \in S_V$ so that $\dist_{\Span\{v_1,v_2\}}(v_3) \geq \frac{1}{2}$. Then $\lnorm v_3 - v_2 \rnorm \geq \frac{1}{2}$ and $\lnorm v_3 - v_1 \rnorm \geq \frac{1}{2}$. Inductively, we obtain a sequence $(v_n)_n$ in $S_V$ with $\lnorm v_n - v_j \rnorm \geq \frac{1}{2}$ for all $1 \leq j \leq n-1$. Since $(v_n)_n$ is not Cauchy, by Proposition~\ref{prop:seq-implies-complete} $(v_n)_n$ does not admit a convergent subsequence. So $S_V$ is not compact by Theorem~\ref{thm:gen-heine-borel}. But this contradicts Proposition~\ref{prop:compact-subsets}, as we've assumed $B_V \supseteq S_V$ is compact.

            Suppose $V$ is finite dimensional. By Proposition~\ref{cor:finite-lin-implies-continuous} there exists a continuous linear isomorphism $f:\bfC^n \rightarrow V$. Note that $B_V = \{v \in V \mid \lnorm v \rnorm \leq 1\}$ is closed and bounded. Since $f$ is continuous, by definition $f^{-1}(B_V)$ is closed. Since $f^{-1}$ is continuous, Proposition~\ref{prop:linear-implies-continuous} says $f^{-1}$ is Lipschitz. So $f^{-1}(B_V)$ is bounded . By the Heine-Borel Theorem, $f^{-1}(B_V)$ is compact. Thus $f(f^{-1}(B_V)) = B_V$ is also compact by Proposition~\ref{prop:compact-cont}.
        \end{proof}

    \begin{proposition}\label{prop:compact-cont-implies-unif}
        Let $(X,d)$ be compact and $f:(X,d) \rightarrow (Y,\rho)$ continuous. Then $f$ is uniformly continuous
    \end{proposition}
        \begin{proof}
            Let $\epsilon > 0$. Since $f$ is continuous, for every $x \in X$ there exists a $\delta_x > 0$ such that, for any $z \in X$, we have $d(x,z) < \delta_x$ implies $\rho(f(x),f(z))<\frac{\epsilon}{2}$. Note that $X = \bigcup_{x \in X}U(x,\frac{\delta_x}{2})$. Since $X$ is compact, there exists $x_1,...,x_n \in X$ with $X = \bigcup_{i = 1}^n U\bigl(x_i, \frac{\delta_{x_i}}{2}\bigr)$. Let $x,x' \in X$ be arbitrary. Define $\delta = \min_{i = 1}^n \delta_{x_i}$. Locate $U\bigl(x_i, \frac{\delta_{x_i}}{2}\bigr) \ni x$. If $d(x,x') < \delta$, then
                \begin{equation*}
                \begin{split}
                    d(x',x_i)
                    & \leq d(x',x) + d(x,x_i) \\
                    & < \delta + \frac{\delta_{x_i}}{2} \\
                    & \leq \delta_{x_i},
                \end{split}
                \end{equation*}
            which gives:
                \begin{equation*}
                \begin{split}
                    \rho(f(x),f(x'))
                    & \leq \rho(f(x),f(x_i)) + \rho(f(x_i),f(x')) \\
                    & < \frac{\epsilon}{2} + \frac{\epsilon}{2} \\
                    & = \epsilon.
                \end{split}
                \end{equation*}
        \end{proof}
        
    \begin{proof}[\protect{Alternative proof of Proposition~\ref{prop:compact-cont-implies-unif}}]
        Suppose towards contradiction $f$ is not uniformly continuous. Then there exists an $\epsilon_0 > 0$ and sequences $(u_n)_n$, $(v_n)_n$ in $X$ such that $d(u_n,v_n) < \frac{1}{n}$ and $\rho(f(u_n),f(v_n)) \geq \epsilon_0$. Since $X$ is compact there exists a subsequence $(u_{n_k})_k$ convering to $u \in X$. Then:
            \begin{equation*}
            \begin{split}
                d(v_{n_k}, u) 
                &\leq d(v_{n_k},u_{n_k}) + d(u_{n_k},u)\\
                & < \frac{1}{n_k} + d(u_{n_k},u).
            \end{split}
            \end{equation*}
        Taking the limit as $k \rightarrow \infty$ gives $(d(v_{n_k},u))_k \rightarrow 0$; i.e., $(v_{n_k})_k$ converges to $u$. Since $f$ is continuous, both $(f(u_{n_k}))_k$ and $(f(v_{n_k}))_k$ converge to $f(u)$. But this is a contradiction, as $\rho(f(u_n),f(v_n)) \rightarrow 0 \not\geq \epsilon_0$. 
    \end{proof}

    \begin{lemma}\label{lemma:dini}
        Let $(X,d)$ be compact. Suppose $(f_n)_n$ is a monotonically decreasing seqence of continuous real-valued functions on $X$ which converges pointwise to $0$. Then $(f_n)_n$ converges uniformly to $0$.
    \end{lemma}
        \begin{proof}
            Let $\epsilon > 0$. Since $(f_n)_n$ converges pointwise to $0$, for each $x \in X$ there exists $N_x \in \bfN$ such that $n \geq N_x$ implies $f_n(x) < \frac{\epsilon}{2}$. Because $f_{N_x}$ is continuous at $x$, there exists $\delta_x > 0$ such that, for every $z \in X$, $d(x,z) < \delta_x$ implies $|f_{N_x}(x) - f_{N_x}(z)| < \frac{\epsilon}{2}$. The collection $\{U(x,\delta_x)\}_{x \in X}$ covers $X$, so by compactness there is a finite set $F \subseteq X$ with $X = \bigcup_{x \in F} U(x,\delta_x)$. Set $N = \max_{x \in F} N_x$. Let $z \in X$ be arbitrary and locate $x \in F$ such that $z \in U(x,\delta_x)$. Notice that our choice of $N$ does not depend on $z$. For $n \geq N$:
                \begin{equation*}
                \begin{split}
                    f_n(z) 
                    & \leq f_{N_x}(z) \\
                    & = f_{N_x}(z) - f_{N_x}(x) + f_{N_x}(x) \\
                    & \leq |f_{N_x}(z) - f_{N_x}(x)| + f_{N_x}(x) \\
                    & < \frac{\epsilon}{2} + \frac{\epsilon}{2} \\
                    & = \epsilon.
                \end{split}
                \end{equation*}
            Thus for $n \geq N$, we have $\lnorm f_n \rnorm_u \leq \epsilon$.
        \end{proof}

        \begin{proof}[\protect{Alternative proof of Lemma~\ref{lemma:dini}}]
            Let $\epsilon > 0$. For $n \geq 1$, define:
                \begin{equation*}
                \begin{split}
                    U_n 
                    & := \{x \mid f_n(x) < \frac{\epsilon}{2}\} \\
                    &= f_n^{-1}\bigl((-\infty,\frac{\epsilon}{2})\bigr)
                \end{split}
                \end{equation*}
            Note $U_1 \subseteq U_2 \subseteq U_3 \subseteq ...$, and in particular $\bigcup_{n \geq 1}U_n = X$. Since $X$ is compact, $X = U_N$ for some $N$. That is, $f_N < \frac{\epsilon}{2}$ for all $x$, so $\lnorm f_n \rnorm_u < \epsilon$.
        \end{proof}

    \begin{theorem}[Dini's Theorem]
        Let $(X,d)$ be compact. Suppose $(f_n)_n$ is a monotone sequence of continuous real-valued functions on $X$ which converges pointwise to $f$. Then $(f_n)_n$ converges uniformly to $f$.
    \end{theorem}
        \begin{proof}
            If $(f_n)_n$ is decreasing, apply Lemma~\ref{lemma:dini} to $f_n - f$. If $(f_n)_n$ is increasing, apply Lemma~\ref{lemma:dini} to $f - f_n$.
        \end{proof}

    \begin{center}
        \begin{tikzpicture}
            \draw[thick] (0.3,0) -- (2.3,0);
            \node at (2.39, 0) {$/\,$};
            \node at (2.56, 0) {$/\,$};
            \draw[thick] (2.6,0) -- (4.6,0);
        \end{tikzpicture}
    \end{center}

    The goal of this subsection is to develop the structure of $C(X)$. If $(X,d)$ is compact, by the Extreme Value Theorem $C(X) = C_b(X)$; i.e., $C(X)$ is a Banach algebra. We know that compact subsets of $C(X)$ must be closed and bounded. Conversely however, we'd like to know which closed and bounded subsets of $C(X)$ are compact. This leads us to the following definition.

    \begin{definition}
        Let $(X,d)$ be a metric space and $\cF \subseteq C(X)$.
        \begin{enumerate}[label = (\arabic*),itemsep=1pt,topsep=3pt]
            \item $\cF$ is \textit{pointwise equicontinuous} if:
                \begin{equation*}
                \begin{split}
                    \text{{\scalebox{.85}{$(\forall x_0 \in X)(\forall \epsilon > 0)(\exists \delta > 0):(\forall x \in X)( \forall f \in \cF)(d(x,x_0) < \delta \implies |f(x) - f(x_0)| < \epsilon)$}}} 
                \end{split}
                \end{equation*}
            \item $\cF$ is \textit{uniformly equicontinuous} if:
                \begin{equation*}
                \begin{split}
                    \text{{\scalebox{.85}{$(\forall \epsilon > 0)(\exists \delta > 0):(\forall x,x' \in X)(\forall f \in \cF)(d(x,x') < \delta \implies |f(x) - f(x')| < \epsilon)$}}} 
                \end{split}
                \end{equation*}
        \end{enumerate}
    \end{definition}

    \begin{proposition}
        Let $\cF \subseteq C(X)$ be finite. Then $\cF$ is pointwise equicontinuous.
    \end{proposition}
        \begin{proof}
            Suppose $\cF = \{f_1,f_2,...,f_n\}$. Let $x_0 \in X$ and $\epsilon > 0$. For each $k \in \{1,2,...,n\}$, there exists $\delta_k > 0$ such that, for any $x \in X$, $d(x,x_0)< \delta_k$ implies $|f_k(x) - f_k(x_0)| < \epsilon$. Set $\delta = \min_{k = 1}^n \delta_k$. If $x \in X$ and $d(x,x_0)$, then by our choice of $\delta$ we have $|f(x) - f(x_0)| < \epsilon$ for any $f \in \cF$. Hence $\cF$ is pointwise equicontinuous.
        \end{proof}

    \begin{theorem}[Arzela-Ascoli]
        Let $(X,d)$ be compact and $\cF \subseteq C(X)$. The following are equivalent:
            \begin{enumerate}[label = (\arabic*),itemsep=1pt,topsep=3pt]
                \item $\cF$ is compact;
                \item $\cF$ is closed, bounded, and uniformly equicontinuous.
            \end{enumerate}
    \end{theorem}
        \begin{proof}
            Suppose $\cF$ is compact. Then $\cF$ is closed and (totally) bounded. We only need to show that $\cF$ is uniformly equicontinuous. Let $\epsilon > 0$. Since $\cF$ is totally bounded, there exists $f_1,...,f_n \in \cF$ with $\cF \subseteq \bigcup_{j = 1}^n U(f_j, \frac{\epsilon}{3})$. Now since $X$ is compact, each $f_1,...,f_n$ are uniformly continuous.  So there exists $\delta_j > 0$ such that, for all $x,y \in X$, $d(x,y) < \delta_j$ implies $|f_j(x) - f_j(y)| < \frac{\epsilon}{3}$. Let $\delta = \min_{j = 1}^n \delta_j$. Given $f \in F$, locate $f_j$ such that $f \in U(f_j, \frac{\epsilon}{3})$. Then given any $x,y \in X$ satsifying $d(x,y) < \delta$, we have:
                \begin{equation*}
                \begin{split}
                    |f(x) - f(y)|
                    & \leq |f(x) - f_j(x)| + |f_j(x) - f_j(y)| + |f_j(y) + f(y)| \\
                    & \leq 2 \lnorm f - f_j \rnorm_u + \frac{\epsilon}{3} \\
                    & < \frac{2\epsilon}{3} + \frac{\epsilon}{3} \\
                    & = \epsilon.
                \end{split}
                \end{equation*}
            Thus $\cF$ is uniformly equicontinuous.

            Now suppose $\cF$ is closed, bounded, and uniformly equicontinuous. Since $X$ is compact, $C(X)$ is complete. Since $\cF$ is closed, it is complete. We only need to show $\cF$ is totally bounded. Let $\epsilon > 0$. By the uniform equicontinuity of $\cF$, there exists $\delta > 0$ such that, for any $f \in \cF$ and $x,y \in X$ satisfying $d(x,y) < \delta$, we have $|f(x) - f(y)| < \frac{\epsilon}{4}$. Note that since $X$ is compact, it is totally bounded. So we can find $x_1,...,x_n \in X$ with $X = \bigcup_{ i =1}^n U(x_i,\delta)$. Note that any $f \in \cF$ is uniformly continuous, hence bounded. So it must be the case that $C_{\cF} = \bigl\{\bigl(f(x_1),...,f(x_n)\bigr) \mid f \in \cF  \bigr\} \subseteq \bfR^n$ is bounded, hence totally bounded. As a result, there exists $f_1,...,f_m \in \cF$ such that $C_{\cF} \subseteq \bigcup_{i = 1}^m U \bigl( (f_i(x_1),...,f_i(x_n)), \frac{\epsilon}{4}\bigr)$. This means, given any $f \in \cF$, there exists $i=1,...,m$ such that $(f(x_1),...,f(x_n)) \in \bigcup_{i = 1}^m U \bigl( (f_i(x_1),...,f_i(x_n)), \frac{\epsilon}{4}\bigr)$. Claim: $\cF \subseteq \bigcup_{i = 1}^m U(f_i,\epsilon)$. Let $f \in \cF$ and $x \in X$ be arbitrary. Locate $x_j$ so that $x \in U(x_j,\delta)$. Locate $i$ so that $(f(x_1),...,f(x_n)) \in U \bigl( (f_i(x_1),...,f_i(x_n)), \frac{\epsilon}{4}\bigr)$. Then $\sum_{j = 1}^n |f(x_j) - f_i(x_j)| < \frac{\epsilon}{4}$. We finally have:
                \begin{equation*}
                \begin{split}
                    |f(x) - f_i(x)| 
                    & \leq |f(x) - f(x_j)| + |f(x_j) - f_i(x_j)| + |f_i(x_j) - f_i(x)| \\
                    & < \frac{3\epsilon}{4}.
                \end{split}
                \end{equation*}
            Thus $\lnorm f - f_i \rnorm < \epsilon$, so our claim holds. Since $\cF$ is both complete and totally bounded, it is compact.
        \end{proof}

    Just as we were interested in compact subsets of $C(X)$, we'd like to know which subalgebras of $C(X)$ are dense. It turns out that the crucial property that a subalgebra must satisfy is that it separates points: a set $A$ of functions defined on $X$ is said to \textit{separate} points if, for every two different points $x$ and $y$ in $X$, there exists a function $f$ in $A$ with $f(x) \neq f(y)$. We restate this in one of the most vital results in mathematical analysis.

    \begin{theorem}[Stone-Weierstrass]
        Let $(X,d)$ be a compact metric space. Suppose $A \subseteq C(X)$ is a separating unital subalgebra. Then $A$ is $\lnorm \cdot \rnorm_u$-dense in $C(X)$.
    \end{theorem}

    The Stone-Weierstrass Theorem for continuous complex-valued functions requires $A \subseteq C(X,\bfC)$ to be separating, unital, \textit{and} self-adjoint.

    \begin{example}
        \phantom{a}
        \begin{enumerate}[label = (\arabic*),itemsep=1pt,topsep=3pt]
            \item The polynomials $\left\{ \sum_{k = 0}^n a_k x^k \mid a_k \in F,\h2 n\geq 0 \right\}$ are dense in $C([a,b])$.
            \item By Theorem~\ref{thm:finite-iff-compact} and Proposition~\ref{prop:compact-subsets}, the set $S_\bfC = \{z \in \bfC \mid |z| = 1\}$ is compact. Consider the set of all trigonometric polynomials $\cT = \{z \mapsto \sum_{k=-n}^n a_k z^k \mid a_k \in \bfC\}$. Note that $\overline{z} = z^{-1}$ for $z \in S_\bfC$. Then $\cT \subseteq C(S_\bfC)$ is dense.
        \end{enumerate}
    \end{example}

\section{Connectedness}\label{sec:connectedness}
    \begin{definition}
        Let $(X,d)$ be a metric space and $Y \subseteq X$.
            \begin{enumerate}[label = (\arabic*),itemsep=1pt,topsep=3pt]
                \item A \textit{splitting for $Y$ in $X$} is a pair of open subsets $U,V \subseteq X$ with $Y \subseteq U \cup V$ and $U \cap V \cap Y = \emptyset$.
                \item Such a splitting for $Y$ is \textit{trivial} if $U \cap Y = \emptyset$ or $V \cap Y = \emptyset$.
                \item $Y$ is \textit{connected in $X$} if every splitting is trivial. Otherwise, $Y$ is said to be \textit{disconnected}.
            \end{enumerate}
    \end{definition}

    \begin{lemma}
        
    \end{lemma}

    \newpage
    \begin{proposition}
        $[a,b] \subseteq \bfR$ is connected.
    \end{proposition}
        \begin{proof}
            Let $[a,b] \subseteq U \cup V$ be a splitting. Suppose towards contradiction that $V \cap [a,b] \neq \emptyset$; i.e., our splitting is not trivial. Set $c = \inf (V \cap [a,b])$. Since $U$ is open, there exists $\epsilon > 0$ such that $[a,a+\epsilon) \subseteq U$.

            Claim: $V \cap [a,b] \subseteq [a+\epsilon,b]$. If not, then there exists so $x \in V \cap [a,b]$ with $x \not\in [a+\epsilon,b]$. It must be the case that $x \in [a,a+\epsilon) \subseteq U$. But this gives $x \in U \cap V \cap [a,b]$, which contradicts $U,V$ being a splitting. From this, it follows that $c \geq a+\epsilon > a$.

            Claim: $[a,c) \subseteq U$. If not, there must be some $x \in [a,c)$ with $x \not\in U$. Since $U,V$ splits $[a,b]$, it must be that $x \in V$. But $x < c$, and $x \in V \cap [a,b]$, contradicting our definition of $c$.

            Claim: $c \in V$. Suppose towards contradiction $c \in U$. We proceed by cases. Case 1: $c < b$. Since $U$ is open, there exists some $\delta > 0$ with $(c-\delta,c+\delta) \subseteq U$. Then $[a,c+\delta) \subseteq U \cap [a,b]$. Consider $x \in (c,c+\delta)$. If $x \in V$, then $x \in U \cap V \cap [a,b]$, contradicting the fact that $U,V$ is a splitting. If $x \in U$, then this contradicts $c$ being the least upper bound of $V \cap [a,b]$. Case 2: $c = b$. Then $[a,b) \subseteq U$ and $\{b\} \subseteq U$. But this means $[a,b] \subseteq U$. In particular, $U \cap [a,b] = [a,b]$. Thus $U \cap V \cap [a,b] \neq \emptyset$, contradicting again that $U,V$ is a splitting.

            Since $V$ is open, there exists $\gamma > 0$ such that $(c-\gamma,c+\gamma) \subseteq V$. This clearly contradicts $c$ being the least upper bound, as we can find an element less than $c$ still contained in $V \cap [a,b]$.
        \end{proof}



    

    

    
    





    
    
    \chapter{Measure and Integration}
    %\appendix
    %\chapter{Sequences of Functions and Series}

Only Real Valued Functions

\section{Sequences of Functions}
    \begin{definition}
        Let $\Omega$ be a set, $(X,d)$ a metric space, and $(f_n)_n$ a sequence of functions in $X^\Omega$.
        \begin{enumerate}[label = (\arabic*),itemsep=1pt,topsep=3pt]
            \item $(f_n)_n$ converges \textit{pointwise} to $f \in X^\Omega$ if:
                \begin{equation*}
                \begin{split}
                    (\forall x \in \Omega)(\forall \epsilon > 0)(\exists N_{x,\epsilon} \in \bfN) : (\forall n \in \bfN)( n \geq N \implies d(f_n(x), f(x)) < \epsilon).
                \end{split}
                \end{equation*}

            \item $(f_n)_n$ converges \textit{uniformly} to $f \in X^\Omega$ if:
                \begin{equation*}
                \begin{split}
                        (\forall \epsilon > 0)(\exists N_\epsilon \in \bfN) : (\forall n \in \bfN)(\forall x \in \Omega)(n \geq N &\implies d(f_n(x), f(x)) < \epsilon).
                \end{split}
                \end{equation*}
        \end{enumerate}
    \end{definition}

    \begin{theorem}
        Let $(X,d)$ and $(Y,\rho)$ be metric spaces. If $(f_n)_n$ is a sequence of functions in $C(X,Y)$ which converges uniformly to $f:X \rightarrow Y$, then $f$ is continuous.
    \end{theorem}
        \begin{proof}
            Let $\epsilon > 0$. Since $(f_n)_n$ converges uniformly to $f$, pick $N$ large so that $n \geq N$ implies $\rho(f_n(x),f(x)) < \frac{\epsilon}{3}$ for all $x \in X$. Let $c \in X$ be arbitrary. Since $f_N \in C(X,Y)$, there exists $\delta > 0$ such that $d(x,c) < \delta$ implies $\rho(f(x),f(c)) < \frac{\epsilon}{3}$. If $d(x,c) < \delta$, then:
                \begin{equation*}
                \begin{split}
                    \rho(f(x),f(c)) 
                    & \leq \rho(f(x),f_N(x)) + \rho(f_N(x),f_N(c)) + \rho(f_N(c),f(c)) \\
                    & < \frac{\epsilon}{3} + \frac{\epsilon}{3} + \frac{\epsilon}{3} \\
                    & = \epsilon.
                \end{split}
                \end{equation*}
            Thus $f \in C(X,Y)$.
        \end{proof}



\section{Series of Functions}
    For the remainder of this section assume $\Omega \subseteq \bfR$.
    \begin{definition}
        \phantom{a}
        \begin{enumerate}[label = (\arabic*),itemsep=1pt,topsep=3pt]
            \item If $(f_n:\Omega \rightarrow \bfR)_n$ is a sequence of functions, the \textit{partial sums} $(s_n)_n$ of the infinite series $\sum f_n$ is defined for $x \in \Omega$ by:
                \begin{equation*}
                \begin{split}
                    s_1(x) &:= f_1(x), \\
                    s_2(x) &:= s_1(x) + f_2(x), \\
                    &\vdots \\
                    s_{n+1}(x) &:= s_n(x) + f_{n+1}(x).
                \end{split}
                \end{equation*}
            If the sequence $(s_n)_n$ of functions converges to a function $f:\Omega \rightarrow \bfR$, we say that the infinite series of functions $\sum f_n$ \textit{converges} to $f$.

            \item If the series $\sum \left| f_n(x) \right|$ converges for each $x \in \Omega$, we say that $\sum f_n$ is \textit{absolutely convergent} on $\Omega$.
            \item If the sequence $(s_n)_n$ of partial sums is uniformly convergent on $\Omega$ to $f$, we say that $\sum f_n$ is \textit{uniformly convergent} on $\Omega$.
        \end{enumerate}
    \end{definition}

    \begin{theorem}
        If $f_n:\Omega \rightarrow \bfR$ is continuous for each $n \in \bfN$ and if $\sum f_n$ converges to $f$ uniformly on $\Omega$, then $f$ is continuous on $\Omega$.
    \end{theorem}
\end{document}