\documentclass[11pt,twoside,openany]{memoir}
%\usepackage{mlmodern}
%\usepackage{tgpagella} % text only
%\usepackage{mathpazo}  % math & text
\usepackage[T1]{fontenc}
\usepackage[hidelinks]{hyperref}
\usepackage{amsmath}
\usepackage[fixamsmath]{mathtools}  % Extension to amsmath
\usepackage{amsthm}
\usepackage{amssymb}
%\renewcommand*{\mathbf}[1]{\varmathbb{#1}}
\usepackage[cal=cm, scr=rsfs, frak=euler, bb=mma]{mathalpha}
\usepackage{newpxtext}
\usepackage{eulerpx}
\usepackage{eucal}
\usepackage{datetime}
    \newdateformat{specialdate}{\THEYEAR\ \monthname\ \THEDAY}
\usepackage[margin=1in]{geometry}
\usepackage{fancyhdr}
    \fancyhf{}
    \pagestyle{fancy}
\usepackage{thmtools}
    \declaretheoremstyle[
        spaceabove=10pt,
        spacebelow=10pt,
        headfont=\normalfont\bfseries,
        notefont=\mdseries, notebraces={(}{)},
        bodyfont=\normalfont,
        postheadspace=0.5em
        %qed=\qedsymbol
        ]{defs}

    \declaretheoremstyle[ 
        spaceabove=10pt, % space above the theorem
        spacebelow=10pt,
        headfont=\normalfont\bfseries,
        bodyfont=\normalfont\itshape,
        postheadspace=0.5em
        ]{thmstyle}
    
    \declaretheorem[
        style=thmstyle,
    ]{theorem}

    \declaretheorem[
        style=thmstyle,
        sibling=theorem,
    ]{proposition}

    \declaretheorem[
        style=thmstyle,
        sibling=theorem,
    ]{lemma}

    \declaretheorem[
        style=thmstyle,
        sibling=theorem,
    ]{corollary}

    \declaretheorem[
        style=defs,
    ]{example}

    \declaretheorem[
        style=defs,
    ]{definition}

    \declaretheorem[
        style=defs,
        sibling=theorem,
    ]{exercise}

    \declaretheorem[
        numbered=unless unique,
        shaded={rulecolor=black,
    rulewidth=1pt, bgcolor={rgb}{1,1,1}}
    ]{axiom}

    \declaretheorem[numberwithin=section,style=defs]{note}
    \declaretheorem[numbered=no,style=defs]{question}
    \declaretheorem[numbered=no,style=defs]{recall}
    \declaretheorem[numbered=no,style=remark]{answer}
    \declaretheorem[numbered=no,style=remark]{solution}
    \declaretheorem[numbered=no,style=defs]{remark}
    \declaretheorem[numbered=unless unique,style=defs]{problem}
    \declaretheorem[numbered=no,style=remark]{proof.}
\usepackage{enumitem}
\usepackage{titlesec}
    \titleformat{\chapter}[display]
    {\bfseries\Huge\raggedright}
    {Chapter {\thechapter}}
    {1ex minus .1ex}
    {\HUGE}
    \titlespacing{\chapter}
    {3pc}{*3}{40pt}[3pc]

    \titleformat{\section}[block]
    {\normalfont\bfseries\Large}
    {\S\ \thesection.}{.5em}{}[]
    \titlespacing{\section}
    {0pt}{3ex plus .1ex minus .2ex}{3ex plus .1ex minus .2ex}
\usepackage[utf8x]{inputenc}
\usepackage{tikz}
\usepackage{tikz-cd}
\usepackage{wasysym}
\linespread{1.00}
\setcounter{secnumdepth}{0}
%%%%%%%%%%%%%%%%%%%%%%%%%%%%%%%%%%%%%%%%%%%%%%%%%%%%%%%%%%%%%
%%%%%%%%%%%%%%%%%%%%%%%%%%%%%%%%%%%%%%%%%%%%%%%%%%%%%%%%%%%%%
%to make the correct symbol for Sha
%\newcommand\cyr{%
%\renewcommand\rmdefault{wncyr}%
%\renewcommand\sfdefault{wncyss}%
%\renewcommand\encodingdefault{OT2}%
%\normalfont \selectfont} \DeclareTextFontCommand{\textcyr}{\cyr}


\DeclareMathOperator{\ab}{ab}
\newcommand{\absgal}{\G_{\bbQ}}
\DeclareMathOperator{\ad}{ad}
\DeclareMathOperator{\adj}{adj}
\DeclareMathOperator{\alg}{alg}
\DeclareMathOperator{\Alt}{Alt}
\DeclareMathOperator{\Ann}{Ann}
\DeclareMathOperator{\arith}{arith}
\DeclareMathOperator{\Aut}{Aut}
\DeclareMathOperator{\Be}{B}
\DeclareMathOperator{\Bd}{Bd}
\DeclareMathOperator{\card}{card}
\DeclareMathOperator{\Char}{char}
\DeclareMathOperator{\csp}{csp}
\DeclareMathOperator{\codim}{codim}
\DeclareMathOperator{\coker}{coker}
\DeclareMathOperator{\coh}{H}
\DeclareMathOperator{\compl}{compl}
\DeclareMathOperator{\conj}{conj}
\DeclareMathOperator{\cont}{cont}
\DeclareMathOperator{\Cov}{Cov}
\DeclareMathOperator{\crys}{crys}
\DeclareMathOperator{\Crys}{Crys}
\DeclareMathOperator{\cusp}{cusp}
\DeclareMathOperator{\diag}{diag}
\DeclareMathOperator{\diam}{diam}
\DeclareMathOperator{\Dom}{Dom}
\DeclareMathOperator{\disc}{disc}
\DeclareMathOperator{\dist}{dist}
\DeclareMathOperator{\dR}{dR}
\DeclareMathOperator{\Eis}{Eis}
\DeclareMathOperator{\End}{End}
\DeclareMathOperator{\ev}{ev}
\DeclareMathOperator{\eval}{eval}
\DeclareMathOperator{\Eq}{Eq}
\DeclareMathOperator{\Ext}{Ext}
\DeclareMathOperator{\Fil}{Fil}
\DeclareMathOperator{\Fitt}{Fitt}
\DeclareMathOperator{\Frob}{Frob}
\DeclareMathOperator{\G}{G}
\DeclareMathOperator{\Gal}{Gal}
\DeclareMathOperator{\GL}{GL}
\DeclareMathOperator{\Gr}{Gr}
\DeclareMathOperator{\Graph}{Graph}
\DeclareMathOperator{\GSp}{GSp}
\DeclareMathOperator{\GUn}{GU}
\DeclareMathOperator{\Hom}{Hom}
\DeclareMathOperator{\id}{id}
\DeclareMathOperator{\Id}{Id}
\DeclareMathOperator{\Ik}{Ik}
\DeclareMathOperator{\IM}{Im}
\DeclareMathOperator{\Image}{im}
\DeclareMathOperator{\Ind}{Ind}
\DeclareMathOperator{\Inf}{inf}
\DeclareMathOperator{\Isom}{Isom}
\DeclareMathOperator{\J}{J}
\DeclareMathOperator{\Jac}{Jac}
\DeclareMathOperator{\lcm}{lcm}
\DeclareMathOperator{\length}{length}
\DeclareMathOperator*{\limit}{limit}
\DeclareMathOperator{\Log}{Log}
\DeclareMathOperator{\M}{M}
\DeclareMathOperator{\Mat}{Mat}
\DeclareMathOperator{\N}{N}
\DeclareMathOperator{\Nm}{Nm}
\DeclareMathOperator{\NIk}{N-Ik}
\DeclareMathOperator{\NSK}{N-SK}
\DeclareMathOperator{\new}{new}
\DeclareMathOperator{\obj}{obj}
\DeclareMathOperator{\old}{old}
\DeclareMathOperator{\ord}{ord}
\DeclareMathOperator{\Or}{O}
\DeclareMathOperator{\op}{op}
\DeclareMathOperator{\PGL}{PGL}
\DeclareMathOperator{\PGSp}{PGSp}
\DeclareMathOperator{\rank}{rank}
\DeclareMathOperator{\Ran}{Ran}
\DeclareMathOperator{\Rel}{Rel}
\DeclareMathOperator{\Real}{Re}
\DeclareMathOperator{\RES}{res}
\DeclareMathOperator{\Res}{Res}
%\DeclareMathOperator{\Sha}{\textcyr{Sh}}
\DeclareMathOperator{\Sel}{Sel}
\DeclareMathOperator{\semi}{ss}
\DeclareMathOperator{\sgn}{sign}
\DeclareMathOperator{\SK}{SK}
\DeclareMathOperator{\SL}{SL}
\DeclareMathOperator{\SO}{SO}
\DeclareMathOperator{\Sp}{Sp}
\DeclareMathOperator{\Span}{span}
\DeclareMathOperator{\Spec}{Spec}
\DeclareMathOperator{\spin}{spin}
\DeclareMathOperator{\st}{st}
\DeclareMathOperator{\St}{St}
\DeclareMathOperator{\SUn}{SU}
\DeclareMathOperator{\supp}{supp}
\DeclareMathOperator{\Sup}{sup}
\DeclareMathOperator{\Sym}{Sym}
\DeclareMathOperator{\Tam}{Tam}
\DeclareMathOperator{\tors}{tors}
\DeclareMathOperator{\tr}{tr}
\DeclareMathOperator{\Tr}{Tr}
\DeclareMathOperator{\un}{un}
\DeclareMathOperator{\Un}{U}
\DeclareMathOperator{\val}{val}
\DeclareMathOperator{\vol}{vol}

\DeclareMathOperator{\Sets}{S \mkern1.04mu e \mkern1.04mu t \mkern1.04mu s}
    \newcommand{\cSets}{\scalebox{1.02}{\contour{black}{$\Sets$}}}
    
\DeclareMathOperator{\Groups}{G \mkern1.04mu r \mkern1.04mu o \mkern1.04mu u \mkern1.04mu p \mkern1.04mu s}
    \newcommand{\cGroups}{\scalebox{1.02}{\contour{black}{$\Groups$}}}

\DeclareMathOperator{\TTop}{T \mkern1.04mu o \mkern1.04mu p}
    \newcommand{\cTop}{\scalebox{1.02}{\contour{black}{$\TTop$}}}

\DeclareMathOperator{\Htp}{H \mkern1.04mu t \mkern1.04mu p}
    \newcommand{\cHtp}{\scalebox{1.02}{\contour{black}{$\Htp$}}}

\DeclareMathOperator{\Mod}{M \mkern1.04mu o \mkern1.04mu d}
    \newcommand{\cMod}{\scalebox{1.02}{\contour{black}{$\Mod$}}}

\DeclareMathOperator{\Ab}{A \mkern1.04mu b}
    \newcommand{\cAb}{\scalebox{1.02}{\contour{black}{$\Ab$}}}

\DeclareMathOperator{\Rings}{R \mkern1.04mu i \mkern1.04mu n \mkern1.04mu g \mkern1.04mu s}
    \newcommand{\cRings}{\scalebox{1.02}{\contour{black}{$\Rings$}}}

\DeclareMathOperator{\ComRings}{C \mkern1.04mu o \mkern1.04mu m \mkern1.04mu R \mkern1.04mu i \mkern1.04mu n \mkern1.04mu g \mkern1.04mu s}
    \newcommand{\cComRings}{\scalebox{1.05}{\contour{black}{$\ComRings$}}}

\DeclareMathOperator{\hHom}{H \mkern1.04mu o \mkern1.04mu m}
    \newcommand{\cHom}{\scalebox{1.02}{\contour{black}{$\hHom$}}}

\renewcommand{\k}{\kappa}
\newcommand{\Ff}{F_{f}}
%\newcommand{\ts}{\,^{t}\!}


%Mathcal
\newcommand{\cA}{\mathcal{A}}
\newcommand{\cB}{\mathcal{B}}
\newcommand{\cC}{\mathcal{C}}
\newcommand{\cD}{\mathcal{D}}
\newcommand{\cE}{\mathcal{E}}
\newcommand{\cF}{\mathcal{F}}
\newcommand{\cG}{\mathcal{G}}
\newcommand{\cH}{\mathcal{H}}
\newcommand{\cI}{\mathcal{I}}
\newcommand{\cJ}{\mathcal{J}}
\newcommand{\cK}{\mathcal{K}}
\newcommand{\cL}{\mathcal{L}}
\newcommand{\cM}{\mathcal{M}}
\newcommand{\cN}{\mathcal{N}}
\newcommand{\cO}{\mathcal{O}}
\newcommand{\cP}{\mathcal{P}}
\newcommand{\cQ}{\mathcal{Q}}
\newcommand{\cR}{\mathcal{R}}
\newcommand{\cS}{\mathcal{S}}
\newcommand{\cT}{\mathcal{T}}
\newcommand{\cU}{\mathcal{U}}
\newcommand{\cV}{\mathcal{V}}
\newcommand{\cW}{\mathcal{W}}
\newcommand{\cX}{\mathcal{X}}
\newcommand{\cY}{\mathcal{Y}}
\newcommand{\cZ}{\mathcal{Z}}


%mathfrak (missing \fi)
\newcommand{\fa}{\mathfrak{a}}
\newcommand{\fA}{\mathfrak{A}}
\newcommand{\fb}{\mathfrak{b}}
\newcommand{\fB}{\mathfrak{B}}
\newcommand{\fc}{\mathfrak{c}}
\newcommand{\fC}{\mathfrak{C}}
\newcommand{\fd}{\mathfrak{d}}
\newcommand{\fD}{\mathfrak{D}}
\newcommand{\fe}{\mathfrak{e}}
\newcommand{\fE}{\mathfrak{E}}
\newcommand{\ff}{\mathfrak{f}}
\newcommand{\fF}{\mathfrak{F}}
\newcommand{\fg}{\mathfrak{g}}
\newcommand{\fG}{\mathfrak{G}}
\newcommand{\fh}{\mathfrak{h}}
\newcommand{\fH}{\mathfrak{H}}
\newcommand{\fI}{\mathfrak{I}}
\newcommand{\fj}{\mathfrak{j}}
\newcommand{\fJ}{\mathfrak{J}}
\newcommand{\fk}{\mathfrak{k}}
\newcommand{\fK}{\mathfrak{K}}
\newcommand{\fl}{\mathfrak{l}}
\newcommand{\fL}{\mathfrak{L}}
\newcommand{\fm}{\mathfrak{m}}
\newcommand{\fM}{\mathfrak{M}}
\newcommand{\fn}{\mathfrak{n}}
\newcommand{\fN}{\mathfrak{N}}
\newcommand{\fo}{\mathfrak{o}}
\newcommand{\fO}{\mathfrak{O}}
\newcommand{\fp}{\mathfrak{p}}
\newcommand{\fP}{\mathfrak{P}}
\newcommand{\fq}{\mathfrak{q}}
\newcommand{\fQ}{\mathfrak{Q}}
\newcommand{\fr}{\mathfrak{r}}
\newcommand{\fR}{\mathfrak{R}}
\newcommand{\fs}{\mathfrak{s}}
\newcommand{\fS}{\mathfrak{S}}
\newcommand{\ft}{\mathfrak{t}}
\newcommand{\fT}{\mathfrak{T}}
\newcommand{\fu}{\mathfrak{u}}
\newcommand{\fU}{\mathfrak{U}}
\newcommand{\fv}{\mathfrak{v}}
\newcommand{\fV}{\mathfrak{V}}
\newcommand{\fw}{\mathfrak{w}}
\newcommand{\fW}{\mathfrak{W}}
\newcommand{\fx}{\mathfrak{x}}
\newcommand{\fX}{\mathfrak{X}}
\newcommand{\fy}{\mathfrak{y}}
\newcommand{\fY}{\mathfrak{Y}}
\newcommand{\fz}{\mathfrak{z}}
\newcommand{\fZ}{\mathfrak{Z}}


%mathbf
\newcommand{\bfA}{\mathbf{A}}
\newcommand{\bfB}{\mathbf{B}}
\newcommand{\bfC}{\mathbf{C}}
\newcommand{\bfD}{\mathbf{D}}
\newcommand{\bfE}{\mathbf{E}}
\newcommand{\bfF}{\mathbf{F}}
\newcommand{\bfG}{\mathbf{G}}
\newcommand{\bfH}{\mathbf{H}}
\newcommand{\bfI}{\mathbf{I}}
\newcommand{\bfJ}{\mathbf{J}}
\newcommand{\bfK}{\mathbf{K}}
\newcommand{\bfL}{\mathbf{L}}
\newcommand{\bfM}{\mathbf{M}}
\newcommand{\bfN}{\mathbf{N}}
\newcommand{\bfO}{\mathbf{O}}
\newcommand{\bfP}{\mathbf{P}}
\newcommand{\bfQ}{\mathbf{Q}}
\newcommand{\bfR}{\mathbf{R}}
\newcommand{\bfS}{\mathbf{S}}
\newcommand{\bfT}{\mathbf{T}}
\newcommand{\bfU}{\mathbf{U}}
\newcommand{\bfV}{\mathbf{V}}
\newcommand{\bfW}{\mathbf{W}}
\newcommand{\bfX}{\mathbf{X}}
\newcommand{\bfY}{\mathbf{Y}}
\newcommand{\bfZ}{\mathbf{Z}}

\newcommand{\bfa}{\mathbf{a}}
\newcommand{\bfb}{\mathbf{b}}
\newcommand{\bfc}{\mathbf{c}}
\newcommand{\bfd}{\mathbf{d}}
\newcommand{\bfe}{\mathbf{e}}
\newcommand{\bff}{\mathbf{f}}
\newcommand{\bfg}{\mathbf{g}}
\newcommand{\bfh}{\mathbf{h}}
\newcommand{\bfi}{\mathbf{i}}
\newcommand{\bfj}{\mathbf{j}}
\newcommand{\bfk}{\mathbf{k}}
\newcommand{\bfl}{\mathbf{l}}
\newcommand{\bfm}{\mathbf{m}}
\newcommand{\bfn}{\mathbf{n}}
\newcommand{\bfo}{\mathbf{o}}
\newcommand{\bfp}{\mathbf{p}}
\newcommand{\bfq}{\mathbf{q}}
\newcommand{\bfr}{\mathbf{r}}
\newcommand{\bfs}{\mathbf{s}}
\newcommand{\bft}{\mathbf{t}}
\newcommand{\bfu}{\mathbf{u}}
\newcommand{\bfv}{\mathbf{v}}
\newcommand{\bfw}{\mathbf{w}}
\newcommand{\bfx}{\mathbf{x}}
\newcommand{\bfy}{\mathbf{y}}
\newcommand{\bfz}{\mathbf{z}}

%blackboard bold

\newcommand{\bbA}{\mathbb{A}}
\newcommand{\bbB}{\mathbb{B}}
\newcommand{\bbC}{\mathbb{C}}
\newcommand{\bbD}{\mathbb{D}}
\newcommand{\bbE}{\mathbb{E}}
\newcommand{\bbF}{\mathbb{F}}
\newcommand{\bbG}{\mathbb{G}}
\newcommand{\bbH}{\mathbb{H}}
\newcommand{\bbI}{\mathbb{I}}
\newcommand{\bbJ}{\mathbb{J}}
\newcommand{\bbK}{\mathbb{K}}
\newcommand{\bbL}{\mathbb{L}}
\newcommand{\bbM}{\mathbb{M}}
\newcommand{\bbN}{\mathbb{N}}
\newcommand{\bbO}{\mathbb{O}}
\newcommand{\bbP}{\mathbb{P}}
\newcommand{\bbQ}{\mathbb{Q}}
\newcommand{\bbR}{\mathbb{R}}
\newcommand{\bbS}{\mathbb{S}}
\newcommand{\bbT}{\mathbb{T}}
\newcommand{\bbU}{\mathbb{U}}
\newcommand{\bbV}{\mathbb{V}}
\newcommand{\bbW}{\mathbb{W}}
\newcommand{\bbX}{\mathbb{X}}
\newcommand{\bbY}{\mathbb{Y}}
\newcommand{\bbZ}{\mathbb{Z}}
\newcommand{\jota}{\jmath}

\newcommand{\bmat}{\left( \begin{matrix}}
\newcommand{\emat}{\end{matrix} \right)}

\newcommand{\bbmat}{\left[ \begin{matrix}}
\newcommand{\ebmat}{\end{matrix} \right]}

\newcommand{\pmat}{\left( \begin{smallmatrix}}
\newcommand{\epmat}{\end{smallmatrix} \right)}

\newcommand{\lat}{\mathscr{L}}
\newcommand{\mat}[4]{\begin{pmatrix}{#1}&{#2}\\{#3}&{#4}\end{pmatrix}}
\newcommand{\ov}[1]{\overline{#1}}
\newcommand{\res}[1]{\underset{#1}{\RES}\,}
\newcommand{\up}{\upsilon}

\newcommand{\tac}{\textasteriskcentered}

%mahesh macros
\newcommand{\tm}{\textrm}

%Comments
\newcommand{\com}[1]{\vspace{5 mm}\par \noindent
\marginpar{\textsc{Comment}} \framebox{\begin{minipage}[c]{0.95
\textwidth} \tt #1 \end{minipage}}\vspace{5 mm}\par}

\newcommand{\Bmu}{\mbox{$\raisebox{-0.59ex}
  {$l$}\hspace{-0.18em}\mu\hspace{-0.88em}\raisebox{-0.98ex}{\scalebox{2}
  {$\color{white}.$}}\hspace{-0.416em}\raisebox{+0.88ex}
  {$\color{white}.$}\hspace{0.46em}$}{}}  %need graphicx and xcolor. this produces blackboard bold mu 

\newcommand{\hooktwoheadrightarrow}{%
  \hookrightarrow\mathrel{\mspace{-15mu}}\rightarrow
}

\makeatletter
\newcommand{\xhooktwoheadrightarrow}[2][]{%
  \lhook\joinrel
  \ext@arrow 0359\rightarrowfill@ {#1}{#2}%
  \mathrel{\mspace{-15mu}}\rightarrow
}
\makeatother

\renewcommand{\geq}{\geqslant}
\renewcommand{\leq}{\leqslant}
\newcommand{\midd}{\hspace{4pt}\middle|\hspace{4pt}}
    
\newcommand{\bone}{\mathbf{1}}
\newcommand{\sign}{\mathrm{sign}}
\newcommand{\eps}{\varepsilon}
\newcommand{\textui}[1]{\uline{\textit{#1}}}

%\newcommand{\ov}{\overline}
%\newcommand{\un}{\underline}
\newcommand{\fin}{\mathrm{fin}}

\newcommand{\chnum}{\titleformat
{\chapter} % command
[display] % shape
{\centering} % format
{\Huge \color{black} \shadowbox{\thechapter}} % label
{-0.5em} % sep (space between the number and title)
{\LARGE \color{black} \underline} % before-code
}

\newcommand{\chunnum}{\titleformat
{\chapter} % command
[display] % shape
{} % format
{} % label
{0em} % sep
{ \begin{flushright} \begin{tabular}{r}  \Huge \color{black}
} % before-code
[
\end{tabular} \end{flushright} \normalsize
] % after-code
}

\newcommand{\nl}{\newline \mbox{}}

\newcommand{\h}[1]{\hspace{#1pt}}

\newcommand{\littletaller}{\mathchoice{\vphantom{\big|}}{}{}{}}
\newcommand\restr[2]{{% we make the whole thing an ordinary symbol
  \left.\kern-\nulldelimiterspace % automatically resize the bar with \right
  #1 % the function
  \littletaller % pretend it's a little taller at normal size
  \right|_{#2} % this is the delimiter
  }}

\newcommand{\mtext}[1]{\hspace{6pt}\text{#1}\hspace{6pt}}

\newcommand{\lnorm}{\left\lVert}
\newcommand{\rnorm}{\right\rVert}

\newcommand{\ds}{\displaystyle}
\newcommand{\ts}{\textstyle}


\newcommand{\sfrac}[2]{{}^{#1}\mskip -5mu/\mskip -3mu_{#2}}


\makeatletter
\newcommand*{\da@rightarrow}{\mathchar"0\hexnumber@\symAMSa 4B }
\newcommand*{\da@leftarrow}{\mathchar"0\hexnumber@\symAMSa 4C }
\newcommand*{\xdashrightarrow}[2][]{%
  \mathrel{%
    \mathpalette{\da@xarrow{#1}{#2}{}\da@rightarrow{\,}{}}{}%
  }%
}
\newcommand{\xdashleftarrow}[2][]{%
  \mathrel{%
    \mathpalette{\da@xarrow{#1}{#2}\da@leftarrow{}{}{\,}}{}%
  }%
}
\newcommand*{\da@xarrow}[7]{%
  % #1: below
  % #2: above
  % #3: arrow left
  % #4: arrow right
  % #5: space left 
  % #6: space right
  % #7: math style 
  \sbox0{$\ifx#7\scriptstyle\scriptscriptstyle\else\scriptstyle\fi#5#1#6\m@th$}%
  \sbox2{$\ifx#7\scriptstyle\scriptscriptstyle\else\scriptstyle\fi#5#2#6\m@th$}%
  \sbox4{$#7\dabar@\m@th$}%
  \dimen@=\wd0 %
  \ifdim\wd2 >\dimen@
    \dimen@=\wd2 %   
  \fi
  \count@=2 %
  \def\da@bars{\dabar@\dabar@}%
  \@whiledim\count@\wd4<\dimen@\do{%
    \advance\count@\@ne
    \expandafter\def\expandafter\da@bars\expandafter{%
      \da@bars
      \dabar@ 
    }%
  }%  
  \mathrel{#3}%
  \mathrel{%   
    \mathop{\da@bars}\limits
    \ifx\\#1\\%
    \else
      _{\copy0}%
    \fi
    \ifx\\#2\\%
    \else
      ^{\copy2}%
    \fi
  }%   
  \mathrel{#4}%
}
\makeatother

\renewcommand{\div}[2]{%
  \scalebox{0.92}{$#1$}%
  \hspace{1.2pt}%
  \scalebox{1.2}{$\mid$}%
  \hspace{0.75pt}%
  \scalebox{0.92}{$#2$}%
}


\begin{document}\renewcommand{\qedsymbol}{}
%%%%%%%%%%%%%%%%%%%%%%%%%%%%%%%%%%%%%%%%%%%%%%%%%%%%%%%%%%%%%
%-----------------------------------------------------------%
%-----------------------Problem 1---------------------------%
%-----------------------------------------------------------%
%%%%%%%%%%%%%%%%%%%%%%%%%%%%%%%%%%%%%%%%%%%%%%%%%%%%%%%%%%%%%
\newpage 
\fancyhead[L]{\scalebox{0.9}{Sequences}}
\fancyhead[R]{\scalebox{0.9}{Appeared on: S14}}
\begin{problem}
    \phantom{a}
    \begin{enumerate}[label = (\arabic*),itemsep=1pt,topsep=3pt]
        \item Assume that $\sum_{n = 1}^\infty a_n$ is convergent and $(b_n)_n$ is a bounded sequence. Does $\sum_{n = 1}^\infty a_n b_n$ converge? Prove or provide a counterexample.
        \item Assume that $\sum_{n = 1}^\infty \left| a_n \right|$ is convergent and $(b_n)_n$ is a bounded sequence. Does $\sum_{n = 1}^\infty a_n b_n$ converge? Prove or provide a counter example.
    \end{enumerate}
\end{problem}
\begin{proof}
\end{proof}

%%%%%%%%%%%%%%%%%%%%%%%%%%%%%%%%%%%%%%%%%%%%%%%%%%%%%%%%%%%%%
%-----------------------------------------------------------%
%-----------------------Problem 2---------------------------%
%-----------------------------------------------------------%
%%%%%%%%%%%%%%%%%%%%%%%%%%%%%%%%%%%%%%%%%%%%%%%%%%%%%%%%%%%%%
\newpage 
\fancyhead[L]{\scalebox{0.9}{Sequences}}
\fancyhead[R]{\scalebox{0.9}{Appeared on: S15}}
\begin{problem}
    Show that the least upper bound property of the real numbers implies the Cauchy completeness property, that is, show that the property that every bounded set of real numbers has a least upper bound implies that every Cauchy sequence of real numbers converges in $\bfR$.
\end{problem}
\begin{proof}
\end{proof}

%%%%%%%%%%%%%%%%%%%%%%%%%%%%%%%%%%%%%%%%%%%%%%%%%%%%%%%%%%%%%
%-----------------------------------------------------------%
%-----------------------Problem 3---------------------------%
%-----------------------------------------------------------%
%%%%%%%%%%%%%%%%%%%%%%%%%%%%%%%%%%%%%%%%%%%%%%%%%%%%%%%%%%%%%
\newpage 
\fancyhead[L]{\scalebox{0.9}{Sequences}}
\fancyhead[R]{\scalebox{0.9}{Appeared on: F14}}
\begin{problem}
    Let $(x_n)_n$ be a sequence in $\bfR$ with $|x_n - x_{n+1}| < \frac{1}{n}$ for all $n \in \bfN$
    \phantom{a}
    \begin{enumerate}[label = (\arabic*)]
        \item If $(x_n)_n$ is bounded, must $(x_n)_n$ converge?
        \item If the subsequence $(x_{2n})_n$ converges, must $(x_n)$ converge?
    \end{enumerate}
\end{problem}
\begin{proof}
\end{proof}

%%%%%%%%%%%%%%%%%%%%%%%%%%%%%%%%%%%%%%%%%%%%%%%%%%%%%%%%%%%%%
%-----------------------------------------------------------%
%-----------------------Problem 4---------------------------%
%-----------------------------------------------------------%
%%%%%%%%%%%%%%%%%%%%%%%%%%%%%%%%%%%%%%%%%%%%%%%%%%%%%%%%%%%%%
\newpage
\fancyhead[L]{\scalebox{0.9}{Sequences}}
\fancyhead[R]{\scalebox{0.9}{Appeared on: W21}}
\begin{problem}
    Suppose that $(a_n)_n$ and $(b_n)_n$ are Cauchy sequences in $\bfR$. Prove, using the definition of Cauchy sequences, that $\left( \left| a_n - b_n \right| \right)_n$ converges in $\bfR$.
\end{problem}
\begin{proof}
\end{proof}

%%%%%%%%%%%%%%%%%%%%%%%%%%%%%%%%%%%%%%%%%%%%%%%%%%%%%%%%%%%%%
%-----------------------------------------------------------%
%-----------------------Problem 5---------------------------%
%-----------------------------------------------------------%
%%%%%%%%%%%%%%%%%%%%%%%%%%%%%%%%%%%%%%%%%%%%%%%%%%%%%%%%%%%%%
\newpage
\fancyhead[L]{\scalebox{0.9}{Sequences}}
\fancyhead[R]{\scalebox{0.9}{Appeared on: F20}}
\begin{problem}
    Let $s_1 = \sqrt{2}$ and $s_{n+1} = \sqrt{2 + s_n}$ for $n \in \bfN$.
        \phantom{a}
        \begin{enumerate}[label = (\arabic*)]
            \item Show that $s_n \leq 2$ for all $n$.
            \item Show that $(s_n)_n$ converges and then compute the limit of the sequence.
        \end{enumerate}
\end{problem}
\begin{proof}
\end{proof}

%%%%%%%%%%%%%%%%%%%%%%%%%%%%%%%%%%%%%%%%%%%%%%%%%%%%%%%%%%%%%
%-----------------------------------------------------------%
%-----------------------Problem 6---------------------------%
%-----------------------------------------------------------%
%%%%%%%%%%%%%%%%%%%%%%%%%%%%%%%%%%%%%%%%%%%%%%%%%%%%%%%%%%%%%
\newpage
\fancyhead[L]{\scalebox{0.9}{Sequences}}
\fancyhead[R]{\scalebox{0.9}{Appeared on: W20}}
\begin{problem}
    Consider the sequence $(a_n)_n$ given by:
        \begin{equation*}
        \begin{split}
            a_n = \frac{1}{n+1} + \frac{1}{n+2} + ... + \frac{1}{2n}.
        \end{split}
        \end{equation*}
    \begin{enumerate}[label = (\arabic*)]
        \item Prove that $(a_n)_n$ is increasing.
        \item Prove that $(a_n)_n$ converges.
    \end{enumerate}
\end{problem}
\begin{proof}
\end{proof}

%%%%%%%%%%%%%%%%%%%%%%%%%%%%%%%%%%%%%%%%%%%%%%%%%%%%%%%%%%%%%
%-----------------------------------------------------------%
%-----------------------Problem 7---------------------------%
%-----------------------------------------------------------%
%%%%%%%%%%%%%%%%%%%%%%%%%%%%%%%%%%%%%%%%%%%%%%%%%%%%%%%%%%%%%
\newpage
\fancyhead[L]{\scalebox{0.9}{Sequences}}
\fancyhead[R]{\scalebox{0.9}{Appeared on: F17}}
\begin{problem}
    Prove that the sequence $(a_n)_n$, where:
        \begin{equation*}
        \begin{split}
            a_n = \sum_{k = 1}^n \frac{1}{\sqrt{n^2 + k}},
        \end{split}
        \end{equation*}
    converges and compute its limit.
\end{problem}
\begin{proof}
\end{proof}

%%%%%%%%%%%%%%%%%%%%%%%%%%%%%%%%%%%%%%%%%%%%%%%%%%%%%%%%%%%%%
%-----------------------------------------------------------%
%-----------------------Problem 8---------------------------%
%-----------------------------------------------------------%
%%%%%%%%%%%%%%%%%%%%%%%%%%%%%%%%%%%%%%%%%%%%%%%%%%%%%%%%%%%%%
\newpage
\fancyhead[L]{\scalebox{0.9}{Sequences}}
\fancyhead[R]{\scalebox{0.9}{Appeared on: F16}}
\begin{problem}
    \phantom{a}
    \begin{enumerate}[label = (\arabic*)]
        \item Prove that the sequence defined by $x_1 = 3$ and $x_{n+1} = \frac{1}{4 - x_n}$ converges.
        \item Explicitly compute $\limit_{n \rightarrow \infty} x_n$.
    \end{enumerate}
\end{problem}
\begin{proof}
\end{proof}

%%%%%%%%%%%%%%%%%%%%%%%%%%%%%%%%%%%%%%%%%%%%%%%%%%%%%%%%%%%%%
%-----------------------------------------------------------%
%-----------------------Problem 9---------------------------%
%-----------------------------------------------------------%
%%%%%%%%%%%%%%%%%%%%%%%%%%%%%%%%%%%%%%%%%%%%%%%%%%%%%%%%%%%%%
\newpage
\fancyhead[L]{\scalebox{0.9}{Sequences}}
\fancyhead[R]{\scalebox{0.9}{Appeared on: S16}}
\begin{problem}
    \phantom{a}
    \begin{enumerate}[label = (\arabic*)]
        \item Argue from the definition of Cauchy sequences that if $(a_n)_n$ and $(b_n)_n$ are Cauchy sequences, then so is $(a_n b_n)_n$.
        \item Give an example of a sequence $(a_n)$ with $\limit_{n \rightarrow \infty} |a_{n+1} - a_n| = 0$ but which is \textit{not} Cauchy.
    \end{enumerate}
\end{problem}
\begin{proof}
\end{proof}

%%%%%%%%%%%%%%%%%%%%%%%%%%%%%%%%%%%%%%%%%%%%%%%%%%%%%%%%%%%%%
%-----------------------------------------------------------%
%-----------------------Problem 10--------------------------%
%-----------------------------------------------------------%
%%%%%%%%%%%%%%%%%%%%%%%%%%%%%%%%%%%%%%%%%%%%%%%%%%%%%%%%%%%%%
\newpage
\fancyhead[L]{\scalebox{0.9}{Sequences}}
\fancyhead[R]{\scalebox{0.9}{Appeared on: F15}}
\begin{problem}
    Let $(x_n)_n$ be a sequence of real numbers satisfying:
        \begin{equation*}
        \begin{split}
            \left| x_{n+1} - x_n \right| \leq C \left| x_n - x_{n-1} \right|,
        \end{split}
        \end{equation*}
    for all $n \geq 1$, where $0 < C < 1$ is a constant. Prove that $(x_n)_n$ converges.
\end{problem}
\begin{proof}
\end{proof}

%%%%%%%%%%%%%%%%%%%%%%%%%%%%%%%%%%%%%%%%%%%%%%%%%%%%%%%%%%%%%
%-----------------------------------------------------------%
%-----------------------Problem 11---------------------------%
%-----------------------------------------------------------%
%%%%%%%%%%%%%%%%%%%%%%%%%%%%%%%%%%%%%%%%%%%%%%%%%%%%%%%%%%%%%
\newpage
\fancyhead[L]{\scalebox{0.9}{Sequences}}
\fancyhead[R]{\scalebox{0.9}{Appeared on: F24}}
\begin{problem}
    \phantom{a}
    \begin{enumerate}[label = (\arabic*)]
        \item Exhibit, with proof, a sequence of real numbers which has $[0,1]$ as its set of limit points.
        \item Does there exist a sequence with $(0,1)$ as its set of limit points? Give an example with proof or prove that no such sequence exists.
    \end{enumerate}
\end{problem}
\begin{proof}
\end{proof}

%%%%%%%%%%%%%%%%%%%%%%%%%%%%%%%%%%%%%%%%%%%%%%%%%%%%%%%%%%%%%
%-----------------------------------------------------------%
%-----------------------Problem 12---------------------------%
%-----------------------------------------------------------%
%%%%%%%%%%%%%%%%%%%%%%%%%%%%%%%%%%%%%%%%%%%%%%%%%%%%%%%%%%%%%
\newpage
\fancyhead[L]{\scalebox{0.9}{Sequences}}
\fancyhead[R]{\scalebox{0.9}{Appeared on: W23}}
\begin{problem}
    Show that the sequence $(x_n)_n$ is Cauchy, where:
        \begin{equation*}
        \begin{split}
            x_n = \int_1^n \frac{\cos t}{t^2}dt.
        \end{split}
        \end{equation*}
\end{problem}
\begin{proof}
\end{proof}

%%%%%%%%%%%%%%%%%%%%%%%%%%%%%%%%%%%%%%%%%%%%%%%%%%%%%%%%%%%%%
%-----------------------------------------------------------%
%-----------------------Problem 13---------------------------%
%-----------------------------------------------------------%
%%%%%%%%%%%%%%%%%%%%%%%%%%%%%%%%%%%%%%%%%%%%%%%%%%%%%%%%%%%%%
\newpage
\fancyhead[L]{\scalebox{0.9}{Sequences}}
\fancyhead[R]{\scalebox{0.9}{Appeared on: S22}}
\begin{problem}
    Prove that every convergent sequence of real numbers has a maximum or minimum value.
\end{problem}
\begin{proof}
\end{proof}

%%%%%%%%%%%%%%%%%%%%%%%%%%%%%%%%%%%%%%%%%%%%%%%%%%%%%%%%%%%%%
%-----------------------------------------------------------%
%-----------------------Problem 14---------------------------%
%-----------------------------------------------------------%
%%%%%%%%%%%%%%%%%%%%%%%%%%%%%%%%%%%%%%%%%%%%%%%%%%%%%%%%%%%%%
\newpage
\fancyhead[L]{\scalebox{0.9}{Sequences}}
\fancyhead[R]{\scalebox{0.9}{Appeared on: S22}}
\begin{problem}
    Suppose that for a function $f:\bfR \rightarrow \bfR$, there is a number $k \in (0,1)$ such that for all $x,y \in \bfR$:
        \begin{equation*}
        \begin{split}
            |f(x) - f(y)| \leq k|x-y|.
        \end{split}
        \end{equation*}
    Fix a number $x_0$, and define a sequence by:
        \begin{equation*}
        \begin{split}
            x_n = f(x_{n-1})
        \end{split}
        \end{equation*}
    for each $n \geq 1$. Prove that $(x_n)_n$ is a Cauchy sequence.
\end{problem}
\begin{proof}
\end{proof}

%%%%%%%%%%%%%%%%%%%%%%%%%%%%%%%%%%%%%%%%%%%%%%%%%%%%%%%%%%%%%
%-----------------------------------------------------------%
%-----------------------Problem 15---------------------------%
%-----------------------------------------------------------%
%%%%%%%%%%%%%%%%%%%%%%%%%%%%%%%%%%%%%%%%%%%%%%%%%%%%%%%%%%%%%
\newpage
\fancyhead[L]{\scalebox{0.9}{Sequences}}
\fancyhead[R]{\scalebox{0.9}{Appeared on: W22}}
\begin{problem}
    Let $(x_n)_n$ be a sequence such that $(x_{2n})_n$, $(x_{2n+1})_n$, and $(x_{3n})_n$ are convergent. Show that $(x_n)_n$ is convergent.
\end{problem}
\begin{proof}
\end{proof}

%%%%%%%%%%%%%%%%%%%%%%%%%%%%%%%%%%%%%%%%%%%%%%%%%%%%%%%%%%%%%
%-----------------------------------------------------------%
%-----------------------Problem 16---------------------------%
%-----------------------------------------------------------%
%%%%%%%%%%%%%%%%%%%%%%%%%%%%%%%%%%%%%%%%%%%%%%%%%%%%%%%%%%%%%
\newpage
\fancyhead[L]{\scalebox{0.9}{Series}}
\fancyhead[R]{\scalebox{0.9}{Appeared on: S18}}
\begin{problem}
    Prove that the series $\sum_{n = 1}^\infty \frac{n^2}{3^n}$ converges by showing that the sequence of partial sums is Cauchy.
\end{problem}
\begin{proof}
\end{proof}

%%%%%%%%%%%%%%%%%%%%%%%%%%%%%%%%%%%%%%%%%%%%%%%%%%%%%%%%%%%%%
%-----------------------------------------------------------%
%-----------------------Problem 17---------------------------%
%-----------------------------------------------------------%
%%%%%%%%%%%%%%%%%%%%%%%%%%%%%%%%%%%%%%%%%%%%%%%%%%%%%%%%%%%%%
\newpage
\fancyhead[L]{\scalebox{0.9}{Series}}
\fancyhead[R]{\scalebox{0.9}{Appeared on: W22}}
\begin{problem}
    Suppose that $\sum_{n = 1}^\infty x_n$ is convergent series of positive terms. Show that $\sum_{n = 1}^\infty x_n^2$ and $\sum_{n = 1}^\infty \sqrt{x_n x_{n+1}}$ are also convergent.
\end{problem}
\begin{proof}
\end{proof}

%%%%%%%%%%%%%%%%%%%%%%%%%%%%%%%%%%%%%%%%%%%%%%%%%%%%%%%%%%%%%
%-----------------------------------------------------------%
%-----------------------Problem 18---------------------------%
%-----------------------------------------------------------%
%%%%%%%%%%%%%%%%%%%%%%%%%%%%%%%%%%%%%%%%%%%%%%%%%%%%%%%%%%%%%
\newpage
\fancyhead[L]{\scalebox{0.9}{Series}}
\fancyhead[R]{\scalebox{0.9}{Appeared on: --}}
\begin{problem}
    \phantom{a}
    \begin{enumerate}[label = (\arabic*)]
        \item Let $(f_n)_n$ and $(g_n)_n$ be sequences of bounded functions on a subset $A$ of $\bfR$. Suppose that $(f_n)_n$ converges uniformly to a bounded function $f$ and $(g_n)_n$ converges uniformly to a bounded function $g$. Show that $(f_n g_n)_n \rightarrow fg$ uniformly on $A$.
        \item Show that (a) may be false if $g$ is unbounded. \textit{Hint:} Consider $f_n(x) = \frac{1}{n}$ and $g_n(x) = x + \frac{1}{n}$. Prove that the convergence $(f_n g_n)_n \rightarrow fg$ in this case is not uniform on $\bfR$.
    \end{enumerate}
\end{problem}
\begin{proof}
\end{proof}

%%%%%%%%%%%%%%%%%%%%%%%%%%%%%%%%%%%%%%%%%%%%%%%%%%%%%%%%%%%%%
%-----------------------------------------------------------%
%-----------------------Problem 19---------------------------%
%-----------------------------------------------------------%
%%%%%%%%%%%%%%%%%%%%%%%%%%%%%%%%%%%%%%%%%%%%%%%%%%%%%%%%%%%%%
\newpage
\fancyhead[L]{\scalebox{0.9}{Continuity}}
\fancyhead[R]{\scalebox{0.9}{Appeared on: W25}}
\begin{problem}
    Suppose that $f:[0,\infty) \rightarrow \bfR$ is a continuous, increasing, and bounded function. Prove that $f$ is uniformly continuous on $[0,\infty)$.
\end{problem}
\begin{proof}
\end{proof}

%%%%%%%%%%%%%%%%%%%%%%%%%%%%%%%%%%%%%%%%%%%%%%%%%%%%%%%%%%%%%
%-----------------------------------------------------------%
%-----------------------Problem 20---------------------------%
%-----------------------------------------------------------%
%%%%%%%%%%%%%%%%%%%%%%%%%%%%%%%%%%%%%%%%%%%%%%%%%%%%%%%%%%%%%
\newpage
\fancyhead[L]{\scalebox{0.9}{Continuity}}
\fancyhead[R]{\scalebox{0.9}{Appeared on: --}}
\begin{problem}
    Let $(f_n)_n$ be a sequence of functions defined on $A \subseteq \bfR$.
    \begin{enumerate}[label = (\arabic*)]
        \item Prove if each $f_n$ is uniformly continuous on $A$ and $(f_n)_n$ converges uniformly on $A$ to a function $f$, then $f$ is uniformly continuous on $A$.
        \item Give a counter example to show that $(a)$ is false if we assume pointwise convergence instead of uniform convergence.
    \end{enumerate}
\end{problem}
\begin{proof}
\end{proof}

%%%%%%%%%%%%%%%%%%%%%%%%%%%%%%%%%%%%%%%%%%%%%%%%%%%%%%%%%%%%%
%-----------------------------------------------------------%
%-----------------------Problem 21---------------------------%
%-----------------------------------------------------------%
%%%%%%%%%%%%%%%%%%%%%%%%%%%%%%%%%%%%%%%%%%%%%%%%%%%%%%%%%%%%%
\newpage
\fancyhead[L]{\scalebox{0.9}{Continuity}}
\fancyhead[R]{\scalebox{0.9}{Appeared on: S14}}
\begin{problem}
    \phantom{a}
    \begin{enumerate}[label = (\arabic*)]
        \item Let $f:\bfR \rightarrow \bfR$ be uniformly continuous. Show that if $(x_n)_n$ is a Cauchy sequence of real numbers, then $(f(x_n))_n$ is a Cauchy sequence.
        \item Suppose that $f_n$ is a sequence of continuous functions that converge uniformly on a subset $A \subseteq \bfR$ to a function $f$. Show that $f$ is continuous on $A$. 
    \end{enumerate}
\end{problem}
\begin{proof}
\end{proof}

%%%%%%%%%%%%%%%%%%%%%%%%%%%%%%%%%%%%%%%%%%%%%%%%%%%%%%%%%%%%%
%-----------------------------------------------------------%
%-----------------------Problem 22---------------------------%
%-----------------------------------------------------------%
%%%%%%%%%%%%%%%%%%%%%%%%%%%%%%%%%%%%%%%%%%%%%%%%%%%%%%%%%%%%%
\newpage
\fancyhead[L]{\scalebox{0.9}{Continuity}}
\fancyhead[R]{\scalebox{0.9}{Appeared on: W21}}
\begin{problem}
    Consider the function:
        \begin{equation*}
        \begin{split}
            g(x)
            & = \begin{cases}
                e^x,& x \in \bfQ,\\
                1,&  x \not\in \bfQ.
            \end{cases}
        \end{split}
        \end{equation*}
    Find, with proof, the set $C = \{x \in \bfR \mid g \h3\text{is continuous at}\h3 x \}$.
\end{problem}
\begin{proof}
\end{proof}

%%%%%%%%%%%%%%%%%%%%%%%%%%%%%%%%%%%%%%%%%%%%%%%%%%%%%%%%%%%%%
%-----------------------------------------------------------%
%-----------------------Problem 23---------------------------%
%-----------------------------------------------------------%
%%%%%%%%%%%%%%%%%%%%%%%%%%%%%%%%%%%%%%%%%%%%%%%%%%%%%%%%%%%%%
\newpage
\fancyhead[L]{\scalebox{0.9}{Continuity}}
\fancyhead[R]{\scalebox{0.9}{Appeared on: F20}}
\begin{problem}
    Define $f:(-1,0) \cup (0,1) \rightarrow \bfR$ by:
    \begin{equation*}
        \begin{split}
            f(x)
            & = \begin{cases}
                4,& x \in (-1,0),\\
                5,&  x \in (0,1).
            \end{cases}
        \end{split}
        \end{equation*}
    \begin{enumerate}[label = (\arabic*)]
        \item Show that $f$ is continuous on $(-1,0) \cup (0,1)$.
        \item Show that $f$ is not uniformly continuous on $(-1,0) \cup (0,1)$.
    \end{enumerate}
\end{problem}
\begin{proof}
\end{proof}

%%%%%%%%%%%%%%%%%%%%%%%%%%%%%%%%%%%%%%%%%%%%%%%%%%%%%%%%%%%%%
%-----------------------------------------------------------%
%-----------------------Problem 24---------------------------%
%-----------------------------------------------------------%
%%%%%%%%%%%%%%%%%%%%%%%%%%%%%%%%%%%%%%%%%%%%%%%%%%%%%%%%%%%%%
\newpage
\fancyhead[L]{\scalebox{0.9}{Continuity}}
\fancyhead[R]{\scalebox{0.9}{Appeared on: S20}}
\begin{problem}
    Let $(f_n)_n$ be a sequence of functions $f_n:\bfR \rightarrow \bfR$ and let $f:\bfR \rightarrow \bfR$ be a function. Suppose $f_n$ is bounded for each $n \in \bfN$.
    \begin{enumerate}[label = (\arabic*)]
        \item Prove that if $(f_n)_n \rightarrow f$ uniformly on $\bfR$, then $f$ is bounded.
        \item If each $f_n$ is continuous and $(f_n)_n \rightarrow f$ pointwise on $\bfR$, does $f$ have to be bounded? Give a proof or a counterexample.
    \end{enumerate}
\end{problem}
\begin{proof}
\end{proof}

%%%%%%%%%%%%%%%%%%%%%%%%%%%%%%%%%%%%%%%%%%%%%%%%%%%%%%%%%%%%%
%-----------------------------------------------------------%
%-----------------------Problem 25---------------------------%
%-----------------------------------------------------------%
%%%%%%%%%%%%%%%%%%%%%%%%%%%%%%%%%%%%%%%%%%%%%%%%%%%%%%%%%%%%%
\newpage
\fancyhead[L]{\scalebox{0.9}{Continuity}}
\fancyhead[R]{\scalebox{0.9}{Appeared on: F19}}
\begin{problem}
    Show that the sequence of functions:
        \begin{equation*}
        \begin{split}
            f_n(x) = \frac{n^2 x}{1 + n^4 x^2}
        \end{split}
        \end{equation*}
    converges pointwise to $f(x) = 0$ on $[0,1]$, but does not converge uniformly.
\end{problem}
\begin{proof}
\end{proof}

%%%%%%%%%%%%%%%%%%%%%%%%%%%%%%%%%%%%%%%%%%%%%%%%%%%%%%%%%%%%%
%-----------------------------------------------------------%
%-----------------------Problem 26---------------------------%
%-----------------------------------------------------------%
%%%%%%%%%%%%%%%%%%%%%%%%%%%%%%%%%%%%%%%%%%%%%%%%%%%%%%%%%%%%%
\newpage
\fancyhead[L]{\scalebox{0.9}{Continuity}}
\fancyhead[R]{\scalebox{0.9}{Appeared on: F18}}
\begin{problem}
    \phantom{a}
    \begin{enumerate}[label = (\arabic*)]
        \item Let $A \subseteq \bfR$. Define what it means for $f:A \rightarrow \bfR$ to be uniformly continuous.
        \item Use the definition to show that $f(x) = \frac{1}{x}$ is uniformly continuous.
        \item Show that $f(x) = \frac{1}{x}$ is not uniformly continuous on $(0,1)$.
    \end{enumerate}
\end{problem}
\begin{proof}
\end{proof}

%%%%%%%%%%%%%%%%%%%%%%%%%%%%%%%%%%%%%%%%%%%%%%%%%%%%%%%%%%%%%
%-----------------------------------------------------------%
%-----------------------Problem 27---------------------------%
%-----------------------------------------------------------%
%%%%%%%%%%%%%%%%%%%%%%%%%%%%%%%%%%%%%%%%%%%%%%%%%%%%%%%%%%%%%
\newpage
\fancyhead[L]{\scalebox{0.9}{Continuity}}
\fancyhead[R]{\scalebox{0.9}{Appeared on: S18}}
\begin{problem}
    \phantom{a}
    \begin{enumerate}[label = (\arabic*)]
        \item Let $(f_n)_n$ be a sequence of functions defined on $A \subseteq \bfR$ that converges uniformly on $A$ to a function $f$. Prove that if each $f_n$ is continuous at $c \in A$, then $f$ is continuous at $c$.
        \item Give an example to show that the result is false if we only assume that $(f_n)_n$ converges pointwise to $f$ on $A$.
    \end{enumerate}
\end{problem}
\begin{proof}
\end{proof}

%%%%%%%%%%%%%%%%%%%%%%%%%%%%%%%%%%%%%%%%%%%%%%%%%%%%%%%%%%%%%
%-----------------------------------------------------------%
%-----------------------Problem 28---------------------------%
%-----------------------------------------------------------%
%%%%%%%%%%%%%%%%%%%%%%%%%%%%%%%%%%%%%%%%%%%%%%%%%%%%%%%%%%%%%
\newpage
\fancyhead[L]{\scalebox{0.9}{Continuity}}
\fancyhead[R]{\scalebox{0.9}{Appeared on: F16}}
\begin{problem}
    Define $f_n: [0,\infty) \rightarrow \bfR$ by:
        \begin{equation*}
        \begin{split}
            f_n(x) = \frac{\sin(nx)}{1 + nx}.
        \end{split}
        \end{equation*}
    \begin{enumerate}[label = (\arabic*)]
        \item Show that $f_n$ converges pointwise on $[0,\infty)$ and find the pointwise limit $f$.
        \item Show that $(f_n)_n \rightarrow f$ uniformly on $[a,\infty)$ for every $a > 0$.
        \item Show that $f_n$ does not converge uniformly to $f$ on $[0,\infty)$.
    \end{enumerate}
\end{problem}
\begin{proof}
\end{proof}

%%%%%%%%%%%%%%%%%%%%%%%%%%%%%%%%%%%%%%%%%%%%%%%%%%%%%%%%%%%%%
%-----------------------------------------------------------%
%-----------------------Problem 29---------------------------%
%-----------------------------------------------------------%
%%%%%%%%%%%%%%%%%%%%%%%%%%%%%%%%%%%%%%%%%%%%%%%%%%%%%%%%%%%%%
\newpage
\fancyhead[L]{\scalebox{0.9}{Continuity}}
\fancyhead[R]{\scalebox{0.9}{Appeared on: S16}}
\begin{problem}
    Let $f_n:\bfR \rightarrow \bfR$ be a sequence of continuous functions that converges uniformly on $\bfR$ to a function $f$. Let $(x_n)_n$ be a sequence of real numbers that converges to $x_0 \in \bfR$. Prove that $(f_n(x_n))_n \rightarrow f(x_0)$.
\end{problem}
\begin{proof}
\end{proof}

%%%%%%%%%%%%%%%%%%%%%%%%%%%%%%%%%%%%%%%%%%%%%%%%%%%%%%%%%%%%%
%-----------------------------------------------------------%
%-----------------------Problem 30---------------------------%
%-----------------------------------------------------------%
%%%%%%%%%%%%%%%%%%%%%%%%%%%%%%%%%%%%%%%%%%%%%%%%%%%%%%%%%%%%%
\newpage
\fancyhead[L]{\scalebox{0.9}{Continuity}}
\fancyhead[R]{\scalebox{0.9}{Appeared on: F15}}
\begin{problem}
    Let $f:\bfR \rightarrow \bfR$ be a continuous function such that:
        \begin{equation*}
        \begin{split}
            \limit_{x \rightarrow +\infty}f(x) = \limit_{x \rightarrow -\infty}f(x) = +\infty.
        \end{split}
        \end{equation*}
    Prove that $f$ attains an absolute minimum value of $\bfR$. In other words, prove that there exists a real number $c$ such that $f(c) \leq f(x)$ for all $x \in \bfR$.
\end{problem}
\begin{proof}
\end{proof}

%%%%%%%%%%%%%%%%%%%%%%%%%%%%%%%%%%%%%%%%%%%%%%%%%%%%%%%%%%%%%
%-----------------------------------------------------------%
%-----------------------Problem 31---------------------------%
%-----------------------------------------------------------%
%%%%%%%%%%%%%%%%%%%%%%%%%%%%%%%%%%%%%%%%%%%%%%%%%%%%%%%%%%%%%
\newpage
\fancyhead[L]{\scalebox{0.9}{Continuity}}
\fancyhead[R]{\scalebox{0.9}{Appeared on: W25}}
\begin{problem}
    Suppose that $f:[0,\infty)$ is a continuous, increasing, and bounded function. Prove that $f$ is uniformly continuous on $[0,\infty)$.
\end{problem}
\begin{proof}
\end{proof}

%%%%%%%%%%%%%%%%%%%%%%%%%%%%%%%%%%%%%%%%%%%%%%%%%%%%%%%%%%%%%
%-----------------------------------------------------------%
%-----------------------Problem 32---------------------------%
%-----------------------------------------------------------%
%%%%%%%%%%%%%%%%%%%%%%%%%%%%%%%%%%%%%%%%%%%%%%%%%%%%%%%%%%%%%
\newpage
\fancyhead[L]{\scalebox{0.9}{Continuity}}
\fancyhead[R]{\scalebox{0.9}{Appeared on: W24}}
\begin{problem}
    A zero of a continuous function is called \textit{isolated} if there exists an open set containing that zero but no other zeros of $f$.
    \begin{enumerate}[label = (\arabic*)]
        \item Give an example of a continuous function $f:(0,1) \rightarrow \bfR$ with infinitely many isolated zeros.
        \item If $f:[0,1] \rightarrow \bfR$ is continuous and all of its zeros are isolated, show that $f$ has only finitely many zeros on $[0,1]$.
    \end{enumerate}
\end{problem}
\begin{proof}
\end{proof}

%%%%%%%%%%%%%%%%%%%%%%%%%%%%%%%%%%%%%%%%%%%%%%%%%%%%%%%%%%%%%
%-----------------------------------------------------------%
%-----------------------Problem 33---------------------------%
%-----------------------------------------------------------%
%%%%%%%%%%%%%%%%%%%%%%%%%%%%%%%%%%%%%%%%%%%%%%%%%%%%%%%%%%%%%
\newpage
\fancyhead[L]{\scalebox{0.9}{Continuity}}
\fancyhead[R]{\scalebox{0.9}{Appeared on: F24}}
\begin{problem}
    \phantom{a}
    \begin{enumerate}[label = (\arabic*)]
        \item Give a definition for a function $f:[a,b] \rightarrow \bfR$ to be uniformly continuous.
        \item Using your definition (and not a theorem), prove that the function $f(x) = \frac{1}{x}$ is uniformly continuous on $[1,2]$.
    \end{enumerate}
\end{problem}
\begin{proof}
\end{proof}

%%%%%%%%%%%%%%%%%%%%%%%%%%%%%%%%%%%%%%%%%%%%%%%%%%%%%%%%%%%%%
%-----------------------------------------------------------%
%-----------------------Problem 34---------------------------%
%-----------------------------------------------------------%
%%%%%%%%%%%%%%%%%%%%%%%%%%%%%%%%%%%%%%%%%%%%%%%%%%%%%%%%%%%%%
\newpage
\fancyhead[L]{\scalebox{0.9}{Continuity}}
\fancyhead[R]{\scalebox{0.9}{Appeared on: F23}}
\begin{problem}
    Suppose that $f(x)$ is continuous and unbounded on $[a,b)$. Prove that $\limit_{x \rightarrow b^{-}}f(x)$ does not exist.
\end{problem}
\begin{proof}
\end{proof}

%%%%%%%%%%%%%%%%%%%%%%%%%%%%%%%%%%%%%%%%%%%%%%%%%%%%%%%%%%%%%
%-----------------------------------------------------------%
%-----------------------Problem 35---------------------------%
%-----------------------------------------------------------%
%%%%%%%%%%%%%%%%%%%%%%%%%%%%%%%%%%%%%%%%%%%%%%%%%%%%%%%%%%%%%
\newpage
\fancyhead[L]{\scalebox{0.9}{Continuity}}
\fancyhead[R]{\scalebox{0.9}{Appeared on: S23}}
\begin{problem}
    For each $n \in \bfN$, the function:
        \begin{equation*}
        \begin{split}
            f_n(x) = \frac{nx}{e^{nx}}
        \end{split}
        \end{equation*}
    is continuous on $[0,2]$. Find the pointwise limit function $f(x) = \limit_{n \rightarrow \infty}f_n(x)$ and show that $(f_n)_n$ does not converge uniformly to $f$.
\end{problem}
\begin{proof}
\end{proof}

%%%%%%%%%%%%%%%%%%%%%%%%%%%%%%%%%%%%%%%%%%%%%%%%%%%%%%%%%%%%%
%-----------------------------------------------------------%
%-----------------------Problem 36---------------------------%
%-----------------------------------------------------------%
%%%%%%%%%%%%%%%%%%%%%%%%%%%%%%%%%%%%%%%%%%%%%%%%%%%%%%%%%%%%%
\newpage
\fancyhead[L]{\scalebox{0.9}{Continuity}}
\fancyhead[R]{\scalebox{0.9}{Appeared on: S23}}
\begin{problem}
    Let $f$ be continuous on $[0,1]$ with $f(x) > 0$ for all $x \in [0,1]$. Let $S = \sup_{x \in [0,1]}f(x)$. Show that for every $\epsilon > 0$, there is some open interval $I$ on which $f(x) > S - \epsilon$.
\end{problem}
\begin{proof}
\end{proof}

%%%%%%%%%%%%%%%%%%%%%%%%%%%%%%%%%%%%%%%%%%%%%%%%%%%%%%%%%%%%%
%-----------------------------------------------------------%
%-----------------------Problem 37---------------------------%
%-----------------------------------------------------------%
%%%%%%%%%%%%%%%%%%%%%%%%%%%%%%%%%%%%%%%%%%%%%%%%%%%%%%%%%%%%%
\newpage
\fancyhead[L]{\scalebox{0.9}{Continuity}}
\fancyhead[R]{\scalebox{0.9}{Appeared on: W23}}
\begin{problem}
    Show that if $f_n(x)$ is a uniformly continuous function on $[0,1]$ for each $n \in \bfN$ and $(f_n)_n \rightarrow f$ uniformly on $[0,1]$, then $f(x)$ is also uniformly continuous on $[0,1]$.
\end{problem}
\begin{proof}
\end{proof}

%%%%%%%%%%%%%%%%%%%%%%%%%%%%%%%%%%%%%%%%%%%%%%%%%%%%%%%%%%%%%
%-----------------------------------------------------------%
%-----------------------Problem 38---------------------------%
%-----------------------------------------------------------%
%%%%%%%%%%%%%%%%%%%%%%%%%%%%%%%%%%%%%%%%%%%%%%%%%%%%%%%%%%%%%
\newpage
\fancyhead[L]{\scalebox{0.9}{Continuity}}
\fancyhead[R]{\scalebox{0.9}{Appeared on: S24}}
\begin{problem}
    Define a sequence of functions by:
        \begin{equation*}
        \begin{split}
            f_n(x) = \frac{nx^n}{1 + nx^n}
        \end{split}
        \end{equation*}
    for $n \in \bfN$.
    \begin{enumerate}[label = (\arabic*)]
        \item Find the pointwise limit $f(x)$ for each $x \in [0,\infty)$.
        \item Prove that $(f_n)_n$ does not converge uniformly on $[0,\infty)$.
        \item Prove that $(f_n)_n$ converges uniformly on $[1,2]$.
    \end{enumerate}
\end{problem}
\begin{proof}
\end{proof}

%%%%%%%%%%%%%%%%%%%%%%%%%%%%%%%%%%%%%%%%%%%%%%%%%%%%%%%%%%%%%
%-----------------------------------------------------------%
%-----------------------Problem 39---------------------------%
%-----------------------------------------------------------%
%%%%%%%%%%%%%%%%%%%%%%%%%%%%%%%%%%%%%%%%%%%%%%%%%%%%%%%%%%%%%
\newpage
\fancyhead[L]{\scalebox{0.9}{Continuity}}
\fancyhead[R]{\scalebox{0.9}{Appeared on: W22}}
\begin{problem}
    Consider the sequence of functions $f_n:\bfR \rightarrow \bfR$ given by:
        \begin{equation*}
        \begin{split}
            f_n(x) = \frac{nx}{\sqrt{1 + n^2 x^2}}.
        \end{split}
        \end{equation*}
    Find the pointwise limit $f(x) = \limit_{n \rightarrow \infty}f_n(x)$. Does $(f_n)_n$ converge to $f$ uniformly on $\bfR$? Justify your answer.
\end{problem}
\begin{proof}
\end{proof}

%%%%%%%%%%%%%%%%%%%%%%%%%%%%%%%%%%%%%%%%%%%%%%%%%%%%%%%%%%%%%
%-----------------------------------------------------------%
%-----------------------Problem 40---------------------------%
%-----------------------------------------------------------%
%%%%%%%%%%%%%%%%%%%%%%%%%%%%%%%%%%%%%%%%%%%%%%%%%%%%%%%%%%%%%
\newpage
\fancyhead[L]{\scalebox{0.9}{Derivatives and the Mean Value Theorem}}
\fancyhead[R]{\scalebox{0.9}{Appeared on: S21}}
\begin{problem}
    Let $f:\bfR \rightarrow \bfR$ be a differentiable function with a continuous derivative. Suppose there exist four distinct points $w,x,y,z$ in $\bfR$ with $f(w) = f(x)$ and $f(y) = y$ and $f(z) = z$. Prove that there is a point $u$ where $f'(u) = \frac{1}{2}$.
\end{problem}
\begin{proof}
\end{proof}

%%%%%%%%%%%%%%%%%%%%%%%%%%%%%%%%%%%%%%%%%%%%%%%%%%%%%%%%%%%%%
%-----------------------------------------------------------%
%-----------------------Problem 41---------------------------%
%-----------------------------------------------------------%
%%%%%%%%%%%%%%%%%%%%%%%%%%%%%%%%%%%%%%%%%%%%%%%%%%%%%%%%%%%%%
\newpage
\fancyhead[L]{\scalebox{0.9}{Derivatives and the Mean Value Theorem}}
\fancyhead[R]{\scalebox{0.9}{Appeared on: W21}}
\begin{problem}
    Let $f:\bfR \rightarrow \bfR$ be a function. Suppose that $f$ is differentiable, that $f(0) = 1$, and that $|f'(x)| \leq 1$ for all $x \in \bfR$. Prove that $|f(x)| \leq |x| + 1$ for all $x \in \bfR$.
\end{problem}
\begin{proof}
\end{proof}

%%%%%%%%%%%%%%%%%%%%%%%%%%%%%%%%%%%%%%%%%%%%%%%%%%%%%%%%%%%%%
%-----------------------------------------------------------%
%-----------------------Problem 42---------------------------%
%-----------------------------------------------------------%
%%%%%%%%%%%%%%%%%%%%%%%%%%%%%%%%%%%%%%%%%%%%%%%%%%%%%%%%%%%%%
\newpage
\fancyhead[L]{\scalebox{0.9}{Derivatives and the Mean Value Theorem}}
\fancyhead[R]{\scalebox{0.9}{Appeared on: F20}}
\begin{problem}
    Prove that there does not exist a differentiable function $f:\bfR \rightarrow \bfR$ such that $f'(0) = 0$ and $f'(x) \geq 1$ for all $x \neq 0$. [\textit{Hint:} Use the Mean Value Theorem].
\end{problem}
\begin{proof}
\end{proof}

%%%%%%%%%%%%%%%%%%%%%%%%%%%%%%%%%%%%%%%%%%%%%%%%%%%%%%%%%%%%%
%-----------------------------------------------------------%
%-----------------------Problem 43---------------------------%
%-----------------------------------------------------------%
%%%%%%%%%%%%%%%%%%%%%%%%%%%%%%%%%%%%%%%%%%%%%%%%%%%%%%%%%%%%%
\newpage
\fancyhead[L]{\scalebox{0.9}{Derivatives and the Mean Value Theorem}}
\fancyhead[R]{\scalebox{0.9}{Appeared on: S20}}
\begin{problem}
    A function $f:\bfR \rightarrow \bfR$ is \textit{Lipschitz continuous} on a set $A \subseteq \bfR$ if there exists a constant $M \geq 0$ such that $|f(x) - f(y)| \leq M|x-y|$ for all $x,y \in A$.
    \begin{enumerate}[label = (\arabic*)]
        \item Assume that $f$ is a differentiable function on $\bfR$ and that $f'$ is continuous on $[a,b]$. Prove that $f$ is Lipschitz on $[a,b]$.
        \item Prove that a Lipschitz function $f:\bfR \rightarrow \bfR$ is uniformly continuous on $\bfR$.
    \end{enumerate}
\end{problem}
\begin{proof}
\end{proof}

%%%%%%%%%%%%%%%%%%%%%%%%%%%%%%%%%%%%%%%%%%%%%%%%%%%%%%%%%%%%%
%-----------------------------------------------------------%
%-----------------------Problem 44---------------------------%
%-----------------------------------------------------------%
%%%%%%%%%%%%%%%%%%%%%%%%%%%%%%%%%%%%%%%%%%%%%%%%%%%%%%%%%%%%%
\newpage
\fancyhead[L]{\scalebox{0.9}{Derivatives and the Mean Value Theorem}}
\fancyhead[R]{\scalebox{0.9}{Appeared on: S15}}
\begin{problem}
    A function $f:\bfR \rightarrow \bfR$ is \textit{Lipschitz continuous} on a set $A \subseteq \bfR$ if there exists a constant $M \geq 0$ such that $|f(x) - f(y)| \leq M|x-y|$ for all $x,y \in A$.
    \begin{enumerate}[label = (\arabic*)]
        \item Show that $f(x) = \sqrt{x}$ is Lipschitz continuous on $[1,\infty)$ but not $[0,\infty)$.
        \item Prove that a Lipschitz function $f:\bfR \rightarrow \bfR$ is uniformly continuous on $\bfR$.
    \end{enumerate}
\end{problem}
\begin{proof}
\end{proof}

%%%%%%%%%%%%%%%%%%%%%%%%%%%%%%%%%%%%%%%%%%%%%%%%%%%%%%%%%%%%%
%-----------------------------------------------------------%
%-----------------------Problem 45---------------------------%
%-----------------------------------------------------------%
%%%%%%%%%%%%%%%%%%%%%%%%%%%%%%%%%%%%%%%%%%%%%%%%%%%%%%%%%%%%%
\newpage
\fancyhead[L]{\scalebox{0.9}{Derivatives and the Mean Value Theorem}}
\fancyhead[R]{\scalebox{0.9}{Appeared on: S20}}
\begin{problem}
    Show that the function:
        \begin{equation*}
        \begin{split}
            f(x) = 
            \begin{cases}
                x^2, & x \in \bfQ \\
                0, & x \not\in \bfQ
            \end{cases}
        \end{split}
        \end{equation*}
    is differentiable only at $x = 0$.
\end{problem}
\begin{proof}
\end{proof}

%%%%%%%%%%%%%%%%%%%%%%%%%%%%%%%%%%%%%%%%%%%%%%%%%%%%%%%%%%%%%
%-----------------------------------------------------------%
%-----------------------Problem 46---------------------------%
%-----------------------------------------------------------%
%%%%%%%%%%%%%%%%%%%%%%%%%%%%%%%%%%%%%%%%%%%%%%%%%%%%%%%%%%%%%
\newpage
\fancyhead[L]{\scalebox{0.9}{Derivatives and the Mean Value Theorem}}
\fancyhead[R]{\scalebox{0.9}{Appeared on: F19}}
\begin{problem}
    \phantom{a}
    \begin{enumerate}[label = (\arabic*)]
        \item State the Mean Value Theorem.
        \item Use the Mean Value Theorem to prove that $|\tan(x)| \geq |x|$ for all $x \in \left(-\frac{\pi}{2},\frac{\pi}{2}\right)$.
    \end{enumerate}
\end{problem}
\begin{proof}
\end{proof}

%%%%%%%%%%%%%%%%%%%%%%%%%%%%%%%%%%%%%%%%%%%%%%%%%%%%%%%%%%%%%
%-----------------------------------------------------------%
%-----------------------Problem 47---------------------------%
%-----------------------------------------------------------%
%%%%%%%%%%%%%%%%%%%%%%%%%%%%%%%%%%%%%%%%%%%%%%%%%%%%%%%%%%%%%
\newpage
\fancyhead[L]{\scalebox{0.9}{Derivatives and the Mean Value Theorem}}
\fancyhead[R]{\scalebox{0.9}{Appeared on: W19}}
\begin{problem}
    Suppose that $f:\bfR \rightarrow \bfR$ satisfies:
        \begin{equation*}
        \begin{split}
            |f(x) - f(y)| \leq (x-y)^2
        \end{split}
        \end{equation*}
    for all $x,y \in \bfR$. Show that $f$ is a constant function on $\bfR$. (\textit{Hint:} Is $f$ differentiable)
\end{problem}
\begin{proof}
\end{proof}

%%%%%%%%%%%%%%%%%%%%%%%%%%%%%%%%%%%%%%%%%%%%%%%%%%%%%%%%%%%%%
%-----------------------------------------------------------%
%-----------------------Problem 48---------------------------%
%-----------------------------------------------------------%
%%%%%%%%%%%%%%%%%%%%%%%%%%%%%%%%%%%%%%%%%%%%%%%%%%%%%%%%%%%%%
\newpage
\fancyhead[L]{\scalebox{0.9}{Derivatives and the Mean Value Theorem}}
\fancyhead[R]{\scalebox{0.9}{Appeared on: W19}}
\begin{problem}
    \phantom{a}
    \begin{enumerate}[label = (\arabic*)]
        \item Suppose that $f$ is a real valued function on $(0,\infty)$ whose derivative exists and is bounded on $(0,\infty)$. Prove that $f$ is uniformly continuous on $(0,\infty)$.
        \item Give an example of a differentiable real valued function $f$ on $(0,\infty)$ that is uniformly continuous on $(0,\infty)$ yet $f'$ is unbounded on $(0,\infty)$.
    \end{enumerate}
\end{problem}
\begin{proof}
\end{proof}

%%%%%%%%%%%%%%%%%%%%%%%%%%%%%%%%%%%%%%%%%%%%%%%%%%%%%%%%%%%%%
%-----------------------------------------------------------%
%-----------------------Problem 49---------------------------%
%-----------------------------------------------------------%
%%%%%%%%%%%%%%%%%%%%%%%%%%%%%%%%%%%%%%%%%%%%%%%%%%%%%%%%%%%%%
\newpage
\fancyhead[L]{\scalebox{0.9}{Derivatives and the Mean Value Theorem}}
\fancyhead[R]{\scalebox{0.9}{Appeared on: F18}}
\begin{problem}
    Suppose that $f$ is differentiable on $\bfR$ and that $f'(x) \leq 4$ for all $x \in \bfR$. Prove that there is at most one point $x>0$ such $f(x) = x^2$.
\end{problem}
\begin{proof}
\end{proof}

%%%%%%%%%%%%%%%%%%%%%%%%%%%%%%%%%%%%%%%%%%%%%%%%%%%%%%%%%%%%%
%-----------------------------------------------------------%
%-----------------------Problem 50---------------------------%
%-----------------------------------------------------------%
%%%%%%%%%%%%%%%%%%%%%%%%%%%%%%%%%%%%%%%%%%%%%%%%%%%%%%%%%%%%%
\newpage
\fancyhead[L]{\scalebox{0.9}{Derivatives and the Mean Value Theorem}}
\fancyhead[R]{\scalebox{0.9}{Appeared on: S18}}
\begin{problem}
    Suppose that $f:\bfR \rightarrow \bfR$ is differentiable and that $|f'(x)| < 1$ for all $x \in \bfR$. 
    \begin{enumerate}[label = (\arabic*)]
        \item Prove that $f$ has at most one fixed point.
        \item Show that the following function satisfies $|f'(x)| < 1$ for all $x \in \bfR$ but has no fixed points:
        \begin{equation*}
        \begin{split}
            f(x) = \ln(1+e^x).
        \end{split}
        \end{equation*}
    \end{enumerate}
\end{problem}
\begin{proof}
\end{proof}

%%%%%%%%%%%%%%%%%%%%%%%%%%%%%%%%%%%%%%%%%%%%%%%%%%%%%%%%%%%%%
%-----------------------------------------------------------%
%-----------------------Problem 51---------------------------%
%-----------------------------------------------------------%
%%%%%%%%%%%%%%%%%%%%%%%%%%%%%%%%%%%%%%%%%%%%%%%%%%%%%%%%%%%%%
\newpage
\fancyhead[L]{\scalebox{0.9}{Derivatives and the Mean Value Theorem}}
\fancyhead[R]{\scalebox{0.9}{Appeared on: F16}}
\begin{problem}
    \phantom{a}
    \begin{enumerate}[label = (\arabic*)]
        \item Prove that $\ln x \leq x-1$ for all $x > 0$.
        \item Prove that $\ln x \geq x - 1 - \frac{1}{2}(x-1)^2$ for all $x \geq 1$, and that $\ln x \leq x - 1 - \frac{1}{2}(x-1)^2$, for all $0 < x \leq 1$.
    \end{enumerate}
\end{problem}
\begin{proof}
\end{proof}

%%%%%%%%%%%%%%%%%%%%%%%%%%%%%%%%%%%%%%%%%%%%%%%%%%%%%%%%%%%%%
%-----------------------------------------------------------%
%-----------------------Problem 52---------------------------%
%-----------------------------------------------------------%
%%%%%%%%%%%%%%%%%%%%%%%%%%%%%%%%%%%%%%%%%%%%%%%%%%%%%%%%%%%%%
\newpage
\fancyhead[L]{\scalebox{0.9}{Derivatives and the Mean Value Theorem}}
\fancyhead[R]{\scalebox{0.9}{Appeared on: S16}}
\begin{problem}
    Let $f$ be a function that is continuous on $[0,1]$ and differentiable;e on $(0,1)$. Show that if $f(0) = 0$, $|f'(x)| \leq |f(x)|$ for all $x \in (0,1)$, then $f(x) = 0$ for all $x \in [0,1]$.
\end{problem}
\begin{proof}
\end{proof}

%%%%%%%%%%%%%%%%%%%%%%%%%%%%%%%%%%%%%%%%%%%%%%%%%%%%%%%%%%%%%
%-----------------------------------------------------------%
%-----------------------Problem 53---------------------------%
%-----------------------------------------------------------%
%%%%%%%%%%%%%%%%%%%%%%%%%%%%%%%%%%%%%%%%%%%%%%%%%%%%%%%%%%%%%
\newpage
\fancyhead[L]{\scalebox{0.9}{Derivatives and the Mean Value Theorem}}
\fancyhead[R]{\scalebox{0.9}{Appeared on: F15}}
\begin{problem}
    Prove that for all real numbers $x$ and $y$:
        \begin{equation*}
        \begin{split}
            |\cos^2(x) - \cos^2(y)| \leq |x-y|.
        \end{split}
        \end{equation*}
\end{problem}
\begin{proof}
\end{proof}

%%%%%%%%%%%%%%%%%%%%%%%%%%%%%%%%%%%%%%%%%%%%%%%%%%%%%%%%%%%%%
%-----------------------------------------------------------%
%-----------------------Problem 54---------------------------%
%-----------------------------------------------------------%
%%%%%%%%%%%%%%%%%%%%%%%%%%%%%%%%%%%%%%%%%%%%%%%%%%%%%%%%%%%%%
\newpage
\fancyhead[L]{\scalebox{0.9}{Derivatives and the Mean Value Theorem}}
\fancyhead[R]{\scalebox{0.9}{Appeared on: S24}}
\begin{problem}
    Suppose that $f$ is continuous on $[0,1]$. Show that there is some $c \in [0,1]$ with:
        \begin{equation*}
        \begin{split}
            \int_{0}^1 x^2 f(x)dx = \frac{1}{3}f(c).
        \end{split}
        \end{equation*}
\end{problem}
\begin{proof}
\end{proof}

%%%%%%%%%%%%%%%%%%%%%%%%%%%%%%%%%%%%%%%%%%%%%%%%%%%%%%%%%%%%%
%-----------------------------------------------------------%
%-----------------------Problem 55---------------------------%
%-----------------------------------------------------------%
%%%%%%%%%%%%%%%%%%%%%%%%%%%%%%%%%%%%%%%%%%%%%%%%%%%%%%%%%%%%%
\newpage
\fancyhead[L]{\scalebox{0.9}{Derivatives and the Mean Value Theorem}}
\fancyhead[R]{\scalebox{0.9}{Appeared on: F24}}
\begin{problem}
    \phantom{a}
    \begin{enumerate}[label = (\arabic*)]
        \item State the Mean Value Theorem
        \item Show that if $f:\bfR \rightarrow \bfR$ is differentiable, $f(0) = 0$, and for all $x$, $|f'(x)| < |x|^3$, then $|f(x)| \leq x^4$ for all $x$.
    \end{enumerate}
\end{problem}
\begin{proof}
\end{proof}

%%%%%%%%%%%%%%%%%%%%%%%%%%%%%%%%%%%%%%%%%%%%%%%%%%%%%%%%%%%%%
%-----------------------------------------------------------%
%-----------------------Problem 56---------------------------%
%-----------------------------------------------------------%
%%%%%%%%%%%%%%%%%%%%%%%%%%%%%%%%%%%%%%%%%%%%%%%%%%%%%%%%%%%%%
\newpage
\fancyhead[L]{\scalebox{0.9}{Derivatives and the Mean Value Theorem}}
\fancyhead[R]{\scalebox{0.9}{Appeared on: W24}}
\begin{problem}
    Let $f:[a,b] \rightarrow \bfR$ be continuous on $[a,b]$ and differentiable everyone on $(a,b)$ except perhaps at one number $c \in (a,b)$, and let $\limit_{x \rightarrow c}f'(x)$ exist. Show that $f$ is differentiable at $c$ and $f'(c) = \limit_{x \rightarrow c}f'(x)$.
\end{problem}
\begin{proof}
\end{proof}

%%%%%%%%%%%%%%%%%%%%%%%%%%%%%%%%%%%%%%%%%%%%%%%%%%%%%%%%%%%%%
%-----------------------------------------------------------%
%-----------------------Problem 57---------------------------%
%-----------------------------------------------------------%
%%%%%%%%%%%%%%%%%%%%%%%%%%%%%%%%%%%%%%%%%%%%%%%%%%%%%%%%%%%%%
\newpage
\fancyhead[L]{\scalebox{0.9}{Derivatives and the Mean Value Theorem}}
\fancyhead[R]{\scalebox{0.9}{Appeared on: F23}}
\begin{problem}
    Let $f:[a,b] \rightarrow \bfR$ be continuous and twice differentiable on $(a,b)$. Assume that the line segment from $A = (a,f(a))$ and $B = (b,f(b))$ intersects the graph of $f$ in a third point different from $A$ and $B$. Show that $f''(c) = 0$ for some $c \in (a,b)$.
\end{problem}
\begin{proof}
\end{proof}

%%%%%%%%%%%%%%%%%%%%%%%%%%%%%%%%%%%%%%%%%%%%%%%%%%%%%%%%%%%%%
%-----------------------------------------------------------%
%-----------------------Problem 58---------------------------%
%-----------------------------------------------------------%
%%%%%%%%%%%%%%%%%%%%%%%%%%%%%%%%%%%%%%%%%%%%%%%%%%%%%%%%%%%%%
\newpage
\fancyhead[L]{\scalebox{0.9}{Derivatives and the Mean Value Theorem}}
\fancyhead[R]{\scalebox{0.9}{Appeared on: S23}}
\begin{problem}
    Prove that if $f$ is a function which is differentiable on all of $\bfR$ and $f'(x) > 0$ for all $x$, then $f$ is injective.
\end{problem}
\begin{proof}
\end{proof}

%%%%%%%%%%%%%%%%%%%%%%%%%%%%%%%%%%%%%%%%%%%%%%%%%%%%%%%%%%%%%
%-----------------------------------------------------------%
%-----------------------Problem 59---------------------------%
%-----------------------------------------------------------%
%%%%%%%%%%%%%%%%%%%%%%%%%%%%%%%%%%%%%%%%%%%%%%%%%%%%%%%%%%%%%
\newpage
\fancyhead[L]{\scalebox{0.9}{Derivatives and the Mean Value Theorem}}
\fancyhead[R]{\scalebox{0.9}{Appeared on: W23}}
\begin{problem}
    Suppose $f$ and $g$ are continuous on $[a,b]$ and $f'$ and $g'$ are continuous on $(a,b)$ with $f(a) = g(a)$ and $f(b) = g(b)$. Prove there is a number $c \in (a,b)$ such that the line tangent to the graph of $f$ at the point $(c,f(c))$ is parallel to the line tangent to the graph of $g$ at $(c,g(c))$.
\end{problem}
\begin{proof}
\end{proof}

%%%%%%%%%%%%%%%%%%%%%%%%%%%%%%%%%%%%%%%%%%%%%%%%%%%%%%%%%%%%%
%-----------------------------------------------------------%
%-----------------------Problem 60---------------------------%
%-----------------------------------------------------------%
%%%%%%%%%%%%%%%%%%%%%%%%%%%%%%%%%%%%%%%%%%%%%%%%%%%%%%%%%%%%%
\newpage
\fancyhead[L]{\scalebox{0.9}{Derivatives and the Mean Value Theorem}}
\fancyhead[R]{\scalebox{0.9}{Appeared on: S22}}
\begin{problem}
    Find, with proof, the maximum number of real roots of the function $f(x) = x^{16} + ax + b$ where $a$ and $b$ are real numbers.
\end{problem}
\begin{proof}
\end{proof}

%%%%%%%%%%%%%%%%%%%%%%%%%%%%%%%%%%%%%%%%%%%%%%%%%%%%%%%%%%%%%
%-----------------------------------------------------------%
%-----------------------Problem 61---------------------------%
%-----------------------------------------------------------%
%%%%%%%%%%%%%%%%%%%%%%%%%%%%%%%%%%%%%%%%%%%%%%%%%%%%%%%%%%%%%
\newpage
\fancyhead[L]{\scalebox{0.9}{Derivatives and the Mean Value Theorem}}
\fancyhead[R]{\scalebox{0.9}{Appeared on: W22}}
\begin{problem}
    A function $f:\bfR \rightarrow \bfR$ is Lipschitz continuous if there is a constant $M \geq 0$ such that
        \begin{equation*}
        \begin{split}
            |f(x) - f(y)| \leq M|x-y|
        \end{split}
        \end{equation*}
    for all $x,y \in \bfR$.
    \begin{enumerate}[label = (\arabic*)]
        \item Suppose that $f:\bfR \rightarrow \bfR$ is differentiable and $f':\bfR \rightarrow \bfR$ is bounded. Prove that $f$ is Lipschitz continuous.
        \item Give an example, with proof, of a function $f:\bfR \rightarrow \bfR$ that is differentiable but not Lipschitz continuous.
        \item Give an example, with proof, of a function $f:\bfR \rightarrow \bfR$ that is Lipschitz continuous but not differentiable.
    \end{enumerate}
\end{problem}
\begin{proof}
\end{proof}

%%%%%%%%%%%%%%%%%%%%%%%%%%%%%%%%%%%%%%%%%%%%%%%%%%%%%%%%%%%%%
%-----------------------------------------------------------%
%-----------------------Problem 62---------------------------%
%-----------------------------------------------------------%
%%%%%%%%%%%%%%%%%%%%%%%%%%%%%%%%%%%%%%%%%%%%%%%%%%%%%%%%%%%%%
\newpage
\fancyhead[L]{\scalebox{0.9}{Series of Functions}}
\fancyhead[R]{\scalebox{0.9}{Appeared on: S15}}
\begin{problem}
    Let $a > 0$. For each $n \in \bfN$, consider the function $f_n:\bfR \rightarrow \bfR$ given by $f_n(x) = \frac{\sin \left( \sfrac{x}{n} \right)}{\sqrt{1 + n^2}}$
        \begin{enumerate}[label = (\arabic*)]
            \item Show that the series $\sum_{n = 1}^\infty f_n(x)$ converges uniformly on $[-a,a]$.
            \item Show that the series $\sum_{n = 1}^\infty f_n(x)$ is continuously differentiable on $(-a,a)$.
        \end{enumerate}
\end{problem}
\begin{proof}
\end{proof}

%%%%%%%%%%%%%%%%%%%%%%%%%%%%%%%%%%%%%%%%%%%%%%%%%%%%%%%%%%%%%
%-----------------------------------------------------------%
%-----------------------Problem 63---------------------------%
%-----------------------------------------------------------%
%%%%%%%%%%%%%%%%%%%%%%%%%%%%%%%%%%%%%%%%%%%%%%%%%%%%%%%%%%%%%
\newpage
\fancyhead[L]{\scalebox{0.9}{Series of Functions}}
\fancyhead[R]{\scalebox{0.9}{Appeared on: S21}}
\begin{problem}
    Suppose $f$ is continuous on $[0,1]$ and $|f(x)| < 1$ for all $x$ on $[0,1]$. Prove that $F$ is uniformly continuous on $[0,1]$, where
        \begin{equation*}
        \begin{split}
            F(x) = \sum_{k = 1}^\infty(f(x))^k.
        \end{split}
        \end{equation*}
\end{problem}
\begin{proof}
\end{proof}

%%%%%%%%%%%%%%%%%%%%%%%%%%%%%%%%%%%%%%%%%%%%%%%%%%%%%%%%%%%%%
%-----------------------------------------------------------%
%-----------------------Problem 64---------------------------%
%-----------------------------------------------------------%
%%%%%%%%%%%%%%%%%%%%%%%%%%%%%%%%%%%%%%%%%%%%%%%%%%%%%%%%%%%%%
\newpage
\fancyhead[L]{\scalebox{0.9}{Series of Functions}}
\fancyhead[R]{\scalebox{0.9}{Appeared on: W21}}
\begin{problem}
    Consider the function 
    \begin{equation*}
    \begin{split}
        f(x) = \sum_{k = 1}^\infty \frac{x^k}{k^2}.
    \end{split}
    \end{equation*}
    \begin{enumerate}[label = (\arabic*)]
        \item Find the domain of $f(x)$ precisely.
        \item Prove that $f$ is uniformly continuous on this domain.
    \end{enumerate}
\end{problem}
\begin{proof}
\end{proof}

%%%%%%%%%%%%%%%%%%%%%%%%%%%%%%%%%%%%%%%%%%%%%%%%%%%%%%%%%%%%%
%-----------------------------------------------------------%
%-----------------------Problem 65---------------------------%
%-----------------------------------------------------------%
%%%%%%%%%%%%%%%%%%%%%%%%%%%%%%%%%%%%%%%%%%%%%%%%%%%%%%%%%%%%%
\newpage
\fancyhead[L]{\scalebox{0.9}{Series of Functions}}
\fancyhead[R]{\scalebox{0.9}{Appeared on: W20}}
\begin{problem}
    Consider the function
    \begin{equation*}
    \begin{split}
        f(x) = \sum_{n = 1}^\infty \frac{\sin x^n}{n^2 x^n}.
    \end{split}
    \end{equation*}
    \begin{enumerate}[label = (\arabic*)]
        \item Prove that $f$ is continuous on $[1,\infty)$.
        \item Prove that, in fact, $f$ is continuous on $(0,\infty)$.
    \end{enumerate}
\end{problem}
\begin{proof}
\end{proof}

%%%%%%%%%%%%%%%%%%%%%%%%%%%%%%%%%%%%%%%%%%%%%%%%%%%%%%%%%%%%%
%-----------------------------------------------------------%
%-----------------------Problem 66---------------------------%
%-----------------------------------------------------------%
%%%%%%%%%%%%%%%%%%%%%%%%%%%%%%%%%%%%%%%%%%%%%%%%%%%%%%%%%%%%%
\newpage
\fancyhead[L]{\scalebox{0.9}{Series of Functions}}
\fancyhead[R]{\scalebox{0.9}{Appeared on: F19}}
\begin{problem}
    Consider the function $f(x) = \sum_{k = 1}^\infty (1-\cos (\sfrac{x}{k}))$.
        \begin{enumerate}[label = (\arabic*)]
            \item Prove that the series for $f$ converges uniformly on every interval of the form $[-M,M]$ in $\bfR$.
            \item Prove that $f$ is differentiable on $\bfR$.
        \end{enumerate}
    You may use without proof the following inequalities in this problem:
        \begin{equation*}
        \begin{split}
            |\sin t| \leq |t|, \h5 |1-\cos t| \leq \frac{t^2}{2}, \h5 t \in \bfR.
        \end{split}
        \end{equation*}
\end{problem}
\begin{proof}
\end{proof}

%%%%%%%%%%%%%%%%%%%%%%%%%%%%%%%%%%%%%%%%%%%%%%%%%%%%%%%%%%%%%
%-----------------------------------------------------------%
%-----------------------Problem 67---------------------------%
%-----------------------------------------------------------%
%%%%%%%%%%%%%%%%%%%%%%%%%%%%%%%%%%%%%%%%%%%%%%%%%%%%%%%%%%%%%
\newpage
\fancyhead[L]{\scalebox{0.9}{Series of Functions}}
\fancyhead[R]{\scalebox{0.9}{Appeared on: S19}}
\begin{problem}
    Show that the following series converges uniformly on $(r,\infty)$ for any real number $r>1$.
    \begin{equation*}
    \begin{split}
        \sum_{n = 1}^\infty \frac{n \ln (1+nx)}{x^n}.
    \end{split}
    \end{equation*}
\end{problem}
\begin{proof}
\end{proof}

%%%%%%%%%%%%%%%%%%%%%%%%%%%%%%%%%%%%%%%%%%%%%%%%%%%%%%%%%%%%%
%-----------------------------------------------------------%
%-----------------------Problem 68---------------------------%
%-----------------------------------------------------------%
%%%%%%%%%%%%%%%%%%%%%%%%%%%%%%%%%%%%%%%%%%%%%%%%%%%%%%%%%%%%%
\newpage
\fancyhead[L]{\scalebox{0.9}{Series of Functions}}
\fancyhead[R]{\scalebox{0.9}{Appeared on: W19}}
\begin{problem}
    Consider the function
        \begin{equation*}
        \begin{split}
            f(x) = \sum_{n = 1}^\infty \frac{n^2 + x^4}{n^4 x^2}.
        \end{split}
        \end{equation*}
    \begin{enumerate}[label = (\arabic*)]
        \item Prove that the series converges uniformly on $[-R,R]$ for any $R > 0$.
        \item Prove that $f$ is continuous on $\bfR$.
    \end{enumerate}
\end{problem}
\begin{proof}
\end{proof}

%%%%%%%%%%%%%%%%%%%%%%%%%%%%%%%%%%%%%%%%%%%%%%%%%%%%%%%%%%%%%
%-----------------------------------------------------------%
%-----------------------Problem 69---------------------------%
%-----------------------------------------------------------%
%%%%%%%%%%%%%%%%%%%%%%%%%%%%%%%%%%%%%%%%%%%%%%%%%%%%%%%%%%%%%
\newpage
\fancyhead[L]{\scalebox{0.9}{Series of Functions}}
\fancyhead[R]{\scalebox{0.9}{Appeared on: F18}}
\begin{problem}
    Consider the function
        \begin{equation*}
        \begin{split}
            f(x) = \sum_{k = 0}^\infty e^{-kx}\cos kx.
        \end{split}
        \end{equation*}
    \begin{enumerate}[label = (\arabic*)]
        \item Prove that the series converges uniformly on $[a,\infty)$ for any $a > 0$.
        \item Prove that $f$ is a continuous function on $(0,\infty)$.
    \end{enumerate}
\end{problem}
\begin{proof}
\end{proof}

%%%%%%%%%%%%%%%%%%%%%%%%%%%%%%%%%%%%%%%%%%%%%%%%%%%%%%%%%%%%%
%-----------------------------------------------------------%
%-----------------------Problem 70---------------------------%
%-----------------------------------------------------------%
%%%%%%%%%%%%%%%%%%%%%%%%%%%%%%%%%%%%%%%%%%%%%%%%%%%%%%%%%%%%%
\newpage
\fancyhead[L]{\scalebox{0.9}{Series of Functions}}
\fancyhead[R]{\scalebox{0.9}{Appeared on: F17}}
\begin{problem}
    Consider the series
        \begin{equation*}
        \begin{split}
            \sum_{n = 1}^\infty e^{-n{x^2}}\sin(nx).
        \end{split}
        \end{equation*}
    \begin{enumerate}[label = (\arabic*)]
        \item Prove that this series converges uniformly on $[a,\infty)$ for each $a > 0$.
        \item Does this series converge uniformly on $[0,\infty)$? Justify your answer.
    \end{enumerate}
\end{problem}
\begin{proof}
\end{proof}

%%%%%%%%%%%%%%%%%%%%%%%%%%%%%%%%%%%%%%%%%%%%%%%%%%%%%%%%%%%%%
%-----------------------------------------------------------%
%-----------------------Problem 71---------------------------%
%-----------------------------------------------------------%
%%%%%%%%%%%%%%%%%%%%%%%%%%%%%%%%%%%%%%%%%%%%%%%%%%%%%%%%%%%%%
\newpage
\fancyhead[L]{\scalebox{0.9}{Series of Functions}}
\fancyhead[R]{\scalebox{0.9}{Appeared on: S16}}
\begin{problem}
    Let $P = \{2,3,5,7,11,13,...\}$ be the set of prime numbers.
    \begin{enumerate}[label = (\arabic*)]
        \item Find the radius of convergence $R$ of the power series
        \begin{equation*}
        \begin{split}
            f(x) = \sum_{p \in P}x^p = x^2 + x^3 + x^5 + x^7 + ... .
        \end{split}
        \end{equation*}

        \item Show that $0 \leq f(x) \leq \frac{x^2}{1-x}$ for $0 \leq x  < R$.
    \end{enumerate}
\end{problem}
\begin{proof}
\end{proof}

%%%%%%%%%%%%%%%%%%%%%%%%%%%%%%%%%%%%%%%%%%%%%%%%%%%%%%%%%%%%%
%-----------------------------------------------------------%
%-----------------------Problem 72---------------------------%
%-----------------------------------------------------------%
%%%%%%%%%%%%%%%%%%%%%%%%%%%%%%%%%%%%%%%%%%%%%%%%%%%%%%%%%%%%%
\newpage
\fancyhead[L]{\scalebox{0.9}{Series of Functions}}
\fancyhead[R]{\scalebox{0.9}{Appeared on: F15}}
\begin{problem}
    Consider 
        \begin{equation*}
        \begin{split}
            f(x) = \sum_{k = 0}^\infty \frac{1}{2^k}\sin (2^k x).
        \end{split}
        \end{equation*}
    \begin{enumerate}[label = (\arabic*)]
        \item Show that $f$ is continuous on $\bfR$.
        \item Show that $f$ is not differentiable at $x = 0$. (\textit{Hint:} Consider the sequence $\left( \frac{\pi}{2^n} \right)_n$).
    \end{enumerate}
\end{problem}
\begin{proof}
\end{proof}

%%%%%%%%%%%%%%%%%%%%%%%%%%%%%%%%%%%%%%%%%%%%%%%%%%%%%%%%%%%%%
%-----------------------------------------------------------%
%-----------------------Problem 73---------------------------%
%-----------------------------------------------------------%
%%%%%%%%%%%%%%%%%%%%%%%%%%%%%%%%%%%%%%%%%%%%%%%%%%%%%%%%%%%%%
\newpage
\fancyhead[L]{\scalebox{0.9}{Series of Functions}}
\fancyhead[R]{\scalebox{0.9}{Appeared on: W25}}
\begin{problem}
    Suppose that $(a_k)_k$ is a sequence with $|a_k| \leq 1$ for all $k \in \bfN$.
    \begin{enumerate}[label = (\arabic*)]
        \item Prove that the series $\sum_{k = 1}^\infty a_k x^k$ and $\sum_{k = 1}^\infty k a_k x^{k-1}$ converge uniformly and absolutely on any closed interval contained in $(-1,1)$.
        \item Prove that
            \begin{equation*}
            \begin{split}
                \frac{d}{dx}\left( \sum_{k = 1}^\infty a_k x^k \right) = \sum_{k = 1}^\infty k a_k x^{k-1}
            \end{split}
            \end{equation*}
        for all $x \in (-1,1)$.
    \end{enumerate}
\end{problem}
\begin{proof}
\end{proof}

%%%%%%%%%%%%%%%%%%%%%%%%%%%%%%%%%%%%%%%%%%%%%%%%%%%%%%%%%%%%%
%-----------------------------------------------------------%
%-----------------------Problem 74---------------------------%
%-----------------------------------------------------------%
%%%%%%%%%%%%%%%%%%%%%%%%%%%%%%%%%%%%%%%%%%%%%%%%%%%%%%%%%%%%%
\newpage
\fancyhead[L]{\scalebox{0.9}{Series of Functions}}
\fancyhead[R]{\scalebox{0.9}{Appeared on: F24}}
\begin{problem}
    Let $a > 0$ and define $f(x) = \sum_{n = 1}^\infty \frac{1}{n}(ax)^n$.
    \begin{enumerate}[label = (\arabic*)]
        \item Find the interval of convergence.
        \item Let $0 < c < R$ where $R$ is the radius of convergence. Show the convergence is uniform on $[-c,c]$.
    \end{enumerate}
\end{problem}
\begin{proof}
\end{proof}

%%%%%%%%%%%%%%%%%%%%%%%%%%%%%%%%%%%%%%%%%%%%%%%%%%%%%%%%%%%%%
%-----------------------------------------------------------%
%-----------------------Problem 75---------------------------%
%-----------------------------------------------------------%
%%%%%%%%%%%%%%%%%%%%%%%%%%%%%%%%%%%%%%%%%%%%%%%%%%%%%%%%%%%%%
\newpage
\fancyhead[L]{\scalebox{0.9}{Series of Functions}}
\fancyhead[R]{\scalebox{0.9}{Appeared on: S24}}
\begin{problem}
    Let
    \begin{equation*}
    \begin{split}
        f(x) = \sum_{n = 1}^\infty \frac{n^x}{3^n - 7}.
    \end{split}
    \end{equation*}
    Show $f$ is continuous on $[0,\infty)$.
\end{problem}
\begin{proof}
\end{proof}

%%%%%%%%%%%%%%%%%%%%%%%%%%%%%%%%%%%%%%%%%%%%%%%%%%%%%%%%%%%%%
%-----------------------------------------------------------%
%-----------------------Problem 76---------------------------%
%-----------------------------------------------------------%
%%%%%%%%%%%%%%%%%%%%%%%%%%%%%%%%%%%%%%%%%%%%%%%%%%%%%%%%%%%%%
\newpage
\fancyhead[L]{\scalebox{0.9}{Series of Functions}}
\fancyhead[R]{\scalebox{0.9}{Appeared on: W24}}
\begin{problem}
    Show that $f(x) = \sum_{n = 1}^\infty \arctan \left( \frac{x}{n^2} \right)$ is a continuous function on all of $\bfR$.
\end{problem}
\begin{proof}
\end{proof}

%%%%%%%%%%%%%%%%%%%%%%%%%%%%%%%%%%%%%%%%%%%%%%%%%%%%%%%%%%%%%
%-----------------------------------------------------------%
%-----------------------Problem 77---------------------------%
%-----------------------------------------------------------%
%%%%%%%%%%%%%%%%%%%%%%%%%%%%%%%%%%%%%%%%%%%%%%%%%%%%%%%%%%%%%
\newpage
\fancyhead[L]{\scalebox{0.9}{Series of Functions}}
\fancyhead[R]{\scalebox{0.9}{Appeared on: F23}}
\begin{problem}
    Prove that the series
    \begin{equation*}
    \begin{split}
        f(x) = \sum_{n = 1}^\infty \frac{n^2 + x^4}{n^4 + x^2}
    \end{split}
    \end{equation*}
    converges to a continuous function $f:\bfR \rightarrow \bfR$.
\end{problem}
\begin{proof}
\end{proof}

%%%%%%%%%%%%%%%%%%%%%%%%%%%%%%%%%%%%%%%%%%%%%%%%%%%%%%%%%%%%%
%-----------------------------------------------------------%
%-----------------------Problem 78---------------------------%
%-----------------------------------------------------------%
%%%%%%%%%%%%%%%%%%%%%%%%%%%%%%%%%%%%%%%%%%%%%%%%%%%%%%%%%%%%%
\newpage
\fancyhead[L]{\scalebox{0.9}{Series of Functions}}
\fancyhead[R]{\scalebox{0.9}{Appeared on: W23}}
\begin{problem}
    Prove that
        \begin{equation*}
        \begin{split}
            f(x) = \sum_{n = 0}^\infty \left( \frac{x^n}{n!} \right)^2
        \end{split}
        \end{equation*}
    is continuous on $\bfR$.
\end{problem}
\begin{proof}
\end{proof}

%%%%%%%%%%%%%%%%%%%%%%%%%%%%%%%%%%%%%%%%%%%%%%%%%%%%%%%%%%%%%
%-----------------------------------------------------------%
%-----------------------Problem 79---------------------------%
%-----------------------------------------------------------%
%%%%%%%%%%%%%%%%%%%%%%%%%%%%%%%%%%%%%%%%%%%%%%%%%%%%%%%%%%%%%
\newpage
\fancyhead[L]{\scalebox{0.9}{Series of Functions}}
\fancyhead[R]{\scalebox{0.9}{Appeared on: F22}}
\begin{problem}
    Let
    \begin{equation*}
    \begin{split}
        f_n(x) = \frac{x}{(x+\cos \left( \sfrac{x}{n} \right))^n}
    \end{split}
    \end{equation*}
    for each $n \in \bfN$. Prove that $f(x) = \sum_{n = 1}^\infty f_n(x)$ is continuous on $[1,2]$.
\end{problem}
\begin{proof}
\end{proof}

%%%%%%%%%%%%%%%%%%%%%%%%%%%%%%%%%%%%%%%%%%%%%%%%%%%%%%%%%%%%%
%-----------------------------------------------------------%
%-----------------------Problem 80---------------------------%
%-----------------------------------------------------------%
%%%%%%%%%%%%%%%%%%%%%%%%%%%%%%%%%%%%%%%%%%%%%%%%%%%%%%%%%%%%%
\newpage
\fancyhead[L]{\scalebox{0.9}{Series of Functions}}
\fancyhead[R]{\scalebox{0.9}{Appeared on: S22}}
\begin{problem}
    Let $(f_n)_n$ be a sequence of increasing functions on $[a,b]$ with $\sum_{n = 1}^\infty f_n(x)$ absolutely convergent when $x = a$ and when $x = b$. Show that $\sum_{n = 1}^\infty f_n(x)$ converges absolutely for every $x \in [a,b]$ and that also the series converges uniformly on $[a,b]$.
\end{problem}
\begin{proof}
\end{proof}

%%%%%%%%%%%%%%%%%%%%%%%%%%%%%%%%%%%%%%%%%%%%%%%%%%%%%%%%%%%%%
%-----------------------------------------------------------%
%-----------------------Problem 81---------------------------%
%-----------------------------------------------------------%
%%%%%%%%%%%%%%%%%%%%%%%%%%%%%%%%%%%%%%%%%%%%%%%%%%%%%%%%%%%%%
\newpage
\fancyhead[L]{\scalebox{0.9}{Integration}}
\fancyhead[R]{\scalebox{0.9}{Appeared on: S21}}
\begin{problem}
    \phantom{a}
    \begin{enumerate}[label = (\arabic*)]
        \item State the definition for a real valued function $f:[a,b] \rightarrow \bfR$ to be Riemann integrable on the interval $[a,b]$.
        \item Let $f:[a,b] \rightarrow \bfR$ be a bounded function, and assume that the lower integral of $f$ on $[a,b]$ is positive. Show that there exists an interval $[c,d] \subseteq [a,b]$ with $c<d$ with $f(x) > 0$ for $x \in [c,d]$.
    \end{enumerate}
\end{problem}
\begin{proof}
\end{proof}

%%%%%%%%%%%%%%%%%%%%%%%%%%%%%%%%%%%%%%%%%%%%%%%%%%%%%%%%%%%%%
%-----------------------------------------------------------%
%-----------------------Problem 82---------------------------%
%-----------------------------------------------------------%
%%%%%%%%%%%%%%%%%%%%%%%%%%%%%%%%%%%%%%%%%%%%%%%%%%%%%%%%%%%%%
\newpage
\fancyhead[L]{\scalebox{0.9}{Integration}}
\fancyhead[R]{\scalebox{0.9}{Appeared on: W21}}
\begin{problem}
    \phantom{a}
    \begin{enumerate}[label = (\arabic*)]
        \item State the definition for a real valued function $f:[a,b] \rightarrow \bfR$ to be Riemann integrable on the interval $[a,b]$.
        \item Prove that if $f$ is continuous on $[0,1]$, then
            \begin{equation*}
            \begin{split}
                \limit_{n \rightarrow \infty} \int_0^1 f(x^n)dx = f(0).
            \end{split}
            \end{equation*}
    \end{enumerate}
\end{problem}
\begin{proof}
\end{proof}

%%%%%%%%%%%%%%%%%%%%%%%%%%%%%%%%%%%%%%%%%%%%%%%%%%%%%%%%%%%%%
%-----------------------------------------------------------%
%-----------------------Problem 83---------------------------%
%-----------------------------------------------------------%
%%%%%%%%%%%%%%%%%%%%%%%%%%%%%%%%%%%%%%%%%%%%%%%%%%%%%%%%%%%%%
\newpage
\fancyhead[L]{\scalebox{0.9}{Integration}}
\fancyhead[R]{\scalebox{0.9}{Appeared on: F20}}
\begin{problem}
    \phantom{a}
    \begin{enumerate}[label = (\arabic*)]
        \item State the definition for a real valued function $f:[a,b] \rightarrow \bfR$ to be Riemann integrable on the interval $[a,b]$.
        \item Suppose that $f:[0,1] \rightarrow \bfR$ is continuous and monotonically increasing, with $f(0)=0$, $f(\sfrac{1}{2}) = 1$, and $f(1) = 2$. Prove that 
            \begin{equation*}
            \begin{split}
                \int_{0}^1 f(x)dx > \frac{1}{2}.
            \end{split}
            \end{equation*}
    \end{enumerate}
\end{problem}
\begin{proof}
\end{proof}

%%%%%%%%%%%%%%%%%%%%%%%%%%%%%%%%%%%%%%%%%%%%%%%%%%%%%%%%%%%%%
%-----------------------------------------------------------%
%-----------------------Problem 84---------------------------%
%-----------------------------------------------------------%
%%%%%%%%%%%%%%%%%%%%%%%%%%%%%%%%%%%%%%%%%%%%%%%%%%%%%%%%%%%%%
\newpage
\fancyhead[L]{\scalebox{0.9}{Integration}}
\fancyhead[R]{\scalebox{0.9}{Appeared on: W20}}
\begin{problem}
    Let $f:[0,1] \rightarrow \bfR$ be continuous. Prove that 
        \begin{equation*}
        \begin{split}
            \limit_{n \rightarrow \infty} \int_0^1 f(x) x^n dx = 0.
        \end{split}
        \end{equation*}
\end{problem}
\begin{proof}
\end{proof}

%%%%%%%%%%%%%%%%%%%%%%%%%%%%%%%%%%%%%%%%%%%%%%%%%%%%%%%%%%%%%
%-----------------------------------------------------------%
%-----------------------Problem 85---------------------------%
%-----------------------------------------------------------%
%%%%%%%%%%%%%%%%%%%%%%%%%%%%%%%%%%%%%%%%%%%%%%%%%%%%%%%%%%%%%
\newpage
\fancyhead[L]{\scalebox{0.9}{Integration}}
\fancyhead[R]{\scalebox{0.9}{Appeared on: F19}}
\begin{problem}
    \phantom{a}
    \begin{enumerate}[label = (\arabic*)]
        \item State the definition for a real valued function $f:[a,b] \rightarrow \bfR$ to be Riemann integrable on the interval $[a,b]$.
        \item Let $a_n$ be a positive sequence of real numbers converging to $0$ and let $B = \{b_1,b_2,b_3,...\}$ be a countably infinite subset of $[0,1]$. Consider the function $f$ on $[0,1]$ defined by 
            \begin{equation*}
            \begin{split}
                f(x) = 
                \begin{cases}
                    a_n, & x = b_n, \\
                    0, & x \not\in B
                \end{cases}.
            \end{split}
            \end{equation*}
        Use your definition from $(a)$ to prove that $f$ is Riemann integrable on $[0,1]$.
    \end{enumerate}
\end{problem}
\begin{proof}
\end{proof}

%%%%%%%%%%%%%%%%%%%%%%%%%%%%%%%%%%%%%%%%%%%%%%%%%%%%%%%%%%%%%
%-----------------------------------------------------------%
%-----------------------Problem 86---------------------------%
%-----------------------------------------------------------%
%%%%%%%%%%%%%%%%%%%%%%%%%%%%%%%%%%%%%%%%%%%%%%%%%%%%%%%%%%%%%
\newpage
\fancyhead[L]{\scalebox{0.9}{Integration}}
\fancyhead[R]{\scalebox{0.9}{Appeared on: W19}}
\begin{problem}
    \phantom{a}
    \begin{enumerate}[label = (\arabic*)]
        \item State the definition for a real valued function $f:[a,b] \rightarrow \bfR$ to be Riemann integrable on the interval $[a,b]$.
        \item Let $f$ be bounded on $[a,b]$ and assume that there exists a partition $P$ with $L(f,P) = U(f,P)$. Use the definition of Riemann integrability to characterize $f$.
    \end{enumerate}
\end{problem}
\begin{proof}
\end{proof}

%%%%%%%%%%%%%%%%%%%%%%%%%%%%%%%%%%%%%%%%%%%%%%%%%%%%%%%%%%%%%
%-----------------------------------------------------------%
%-----------------------Problem 87---------------------------%
%-----------------------------------------------------------%
%%%%%%%%%%%%%%%%%%%%%%%%%%%%%%%%%%%%%%%%%%%%%%%%%%%%%%%%%%%%%
\newpage
\fancyhead[L]{\scalebox{0.9}{Integration}}
\fancyhead[R]{\scalebox{0.9}{Appeared on: F18}}
\begin{problem}
    \phantom{a}
    \begin{enumerate}[label = (\arabic*)]
        \item State the definition for a real valued function $f:[a,b] \rightarrow \bfR$ to be Riemann integrable on the interval $[a,b]$.
        \item Suppose $f:[a,b] \rightarrow \bfR$ is a bounded function with the property that $f$ is Riemann integrable on $[a,c]$ for all $a < c < b$. Use the definition of Riemann integrability to show that $f$ is Riemann integrable on $[a,b]$.
    \end{enumerate}
\end{problem}
\begin{proof}
\end{proof}

%%%%%%%%%%%%%%%%%%%%%%%%%%%%%%%%%%%%%%%%%%%%%%%%%%%%%%%%%%%%%
%-----------------------------------------------------------%
%-----------------------Problem 88---------------------------%
%-----------------------------------------------------------%
%%%%%%%%%%%%%%%%%%%%%%%%%%%%%%%%%%%%%%%%%%%%%%%%%%%%%%%%%%%%%
\newpage
\fancyhead[L]{\scalebox{0.9}{Integration}}
\fancyhead[R]{\scalebox{0.9}{Appeared on: S18}}
\begin{problem}
    \phantom{a}
    \begin{enumerate}[label = (\arabic*)]
        \item State the definition for a real valued function $f:[a,b] \rightarrow \bfR$ to be Riemann integrable on the interval $[a,b]$.
        \item Use your definition from $(a)$ to prove that if $f:[a,b] \rightarrow \bfR$ is continuous and
            \begin{equation*}
            \begin{split}
                \int_a^b |f(x)| dx = 0,
            \end{split}
            \end{equation*}
        then $f(x) = 0$ for all $x \in [a,b]$.
    \end{enumerate}
\end{problem}
\begin{proof}
\end{proof}

%%%%%%%%%%%%%%%%%%%%%%%%%%%%%%%%%%%%%%%%%%%%%%%%%%%%%%%%%%%%%
%-----------------------------------------------------------%
%-----------------------Problem 89---------------------------%
%-----------------------------------------------------------%
%%%%%%%%%%%%%%%%%%%%%%%%%%%%%%%%%%%%%%%%%%%%%%%%%%%%%%%%%%%%%
\newpage
\fancyhead[L]{\scalebox{0.9}{Integration}}
\fancyhead[R]{\scalebox{0.9}{Appeared on: F17}}
\begin{problem}
    \phantom{a}
    \begin{enumerate}[label = (\arabic*)]
        \item State the definition for a real valued function $f:[a,b] \rightarrow \bfR$ to be Riemann integrable on the interval $[a,b]$.
        \item Use your definition from $(a)$ to prove that 
            \begin{equation*}
            \begin{split}
                f(x) = 
                \begin{cases}
                    1, & x = \frac{1}{n} \h4\text{for some $n \in \bfN$} \\
                    0, & \text{otherwise}
                \end{cases}
            \end{split}
            \end{equation*}
        is integrable on $[0,1]$ and compute the value of the integral $\int_0^1 f(x)dx$.
    \end{enumerate}
\end{problem}
\begin{proof}
\end{proof}

%%%%%%%%%%%%%%%%%%%%%%%%%%%%%%%%%%%%%%%%%%%%%%%%%%%%%%%%%%%%%
%-----------------------------------------------------------%
%-----------------------Problem 90---------------------------%
%-----------------------------------------------------------%
%%%%%%%%%%%%%%%%%%%%%%%%%%%%%%%%%%%%%%%%%%%%%%%%%%%%%%%%%%%%%
\newpage
\fancyhead[L]{\scalebox{0.9}{Integration}}
\fancyhead[R]{\scalebox{0.9}{Appeared on: S17}}
\begin{problem}
    \phantom{a}
    \begin{enumerate}[label = (\arabic*)]
        \item State the definition for a real valued function $f:[a,b] \rightarrow \bfR$ to be Riemann integrable on the interval $[a,b]$.
        \item Use the definition of the Riemann integral to prove that $f(x) = \frac{1}{1+x}$ is Riemann integrable on $[0,b]$, for any $b > 0$.
    \end{enumerate}
\end{problem}
\begin{proof}
\end{proof}

%%%%%%%%%%%%%%%%%%%%%%%%%%%%%%%%%%%%%%%%%%%%%%%%%%%%%%%%%%%%%
%-----------------------------------------------------------%
%-----------------------Problem 91---------------------------%
%-----------------------------------------------------------%
%%%%%%%%%%%%%%%%%%%%%%%%%%%%%%%%%%%%%%%%%%%%%%%%%%%%%%%%%%%%%
\newpage
\fancyhead[L]{\scalebox{0.9}{Integration}}
\fancyhead[R]{\scalebox{0.9}{Appeared on: F16}}
\begin{problem}
    \phantom{a}
    \begin{enumerate}[label = (\arabic*)]
        \item State the definition for a real valued function $f:[a,b] \rightarrow \bfR$ to be Riemann integrable on the interval $[a,b]$.
        \item Let 
            \begin{equation*}
            \begin{split}
                g_n(x) = \begin{cases}
                    n, &0 \leq x \leq \frac{1}{n} \\
                    0, \frac{1}{n} < x \leq 1 
                \end{cases}
            \end{split}
            \end{equation*}
        and let $f$ be any continuous function on $[0,1]$. Use the definition of the Riemann integral to compute 
            \begin{equation*}
            \begin{split}
                \limit_{n \rightarrow \infty}\int_0^1 f(x) g_n(x)dx 
            \end{split}
            \end{equation*}
        in terms of $f$.
    \end{enumerate}
\end{problem}
\begin{proof}
\end{proof}

%%%%%%%%%%%%%%%%%%%%%%%%%%%%%%%%%%%%%%%%%%%%%%%%%%%%%%%%%%%%%
%-----------------------------------------------------------%
%-----------------------Problem 92---------------------------%
%-----------------------------------------------------------%
%%%%%%%%%%%%%%%%%%%%%%%%%%%%%%%%%%%%%%%%%%%%%%%%%%%%%%%%%%%%%
\newpage
\fancyhead[L]{\scalebox{0.9}{Integration}}
\fancyhead[R]{\scalebox{0.9}{Appeared on: F15}}
\begin{problem}
    \phantom{a}
    \begin{enumerate}[label = (\arabic*)]
        \item State the definition for the real valued function $f:[a,b] \rightarrow \bfR$ to be Riemann integrable on the interval $[a,b]$.
        \item Let $f:[a,b] \rightarrow \bfR$ be increasing on the interval $[a,b]$. Use the definition to prove that $f$ is Riemann integrable on $[a,b]$.
    \end{enumerate}
\end{problem}
\begin{proof}
\end{proof}

%%%%%%%%%%%%%%%%%%%%%%%%%%%%%%%%%%%%%%%%%%%%%%%%%%%%%%%%%%%%%
%-----------------------------------------------------------%
%-----------------------Problem 93---------------------------%
%-----------------------------------------------------------%
%%%%%%%%%%%%%%%%%%%%%%%%%%%%%%%%%%%%%%%%%%%%%%%%%%%%%%%%%%%%%
\newpage
\fancyhead[L]{\scalebox{0.9}{Integration}}
\fancyhead[R]{\scalebox{0.9}{Appeared on: W24}}
\begin{problem}
    \phantom{a}
    \begin{enumerate}[label = (\arabic*)]
        \item State a definition for a real valued function $f:[a,b] \rightarrow \bfR$ to be Riemann integrable.
        \item Let $f:[a,b] \rightarrow \bfR$ be given by 
            \begin{equation*}
            \begin{split}
                f(x) = 
                \begin{cases}
                    0,& x \in [a,b] \cap \bfQ \\
                    x,& x \in [a,b] \setminus \bfQ
                \end{cases}.
            \end{split}
            \end{equation*}
        Use your definition to decide with proof if $f$ is Riemann integrable.
    \end{enumerate}
\end{problem}
\begin{proof}
\end{proof}

%%%%%%%%%%%%%%%%%%%%%%%%%%%%%%%%%%%%%%%%%%%%%%%%%%%%%%%%%%%%%
%-----------------------------------------------------------%
%-----------------------Problem 94---------------------------%
%-----------------------------------------------------------%
%%%%%%%%%%%%%%%%%%%%%%%%%%%%%%%%%%%%%%%%%%%%%%%%%%%%%%%%%%%%%
\newpage
\fancyhead[L]{\scalebox{0.9}{Integration}}
\fancyhead[R]{\scalebox{0.9}{Appeared on: F23}}
\begin{problem}
    \phantom{a}
    \begin{enumerate}[label = (\arabic*)]
        \item State a definition for a real valued function $f:[a,b] \rightarrow \bfR$ to be Riemann integrable.
        \item Use this definition to prove that the function $f$ defined on $[0, \frac{\pi}{2}]$ by 
            \begin{equation*}
            \begin{split}
                f(x) =
                \begin{cases}
                    \cos ^2 x,& x \in \bfQ \\
                    0, & \text{otherwise}
                \end{cases}
            \end{split}
            \end{equation*}
        is not Riemann integrable.
    \end{enumerate}
\end{problem}
\begin{proof}
\end{proof}

%%%%%%%%%%%%%%%%%%%%%%%%%%%%%%%%%%%%%%%%%%%%%%%%%%%%%%%%%%%%%
%-----------------------------------------------------------%
%-----------------------Problem 95---------------------------%
%-----------------------------------------------------------%
%%%%%%%%%%%%%%%%%%%%%%%%%%%%%%%%%%%%%%%%%%%%%%%%%%%%%%%%%%%%%
\newpage
\fancyhead[L]{\scalebox{0.9}{Integration}}
\fancyhead[R]{\scalebox{0.9}{Appeared on: W23}}
\begin{problem}
    Let $f:\bfR \rightarrow \bfR$ be continuous, with 
        \begin{equation*}
        \begin{split}
            \int_0^1 f(xt)dt = 0
        \end{split}
        \end{equation*}
    for all $x \in \bfR$. Show that $f(x) = 0$.
\end{problem}
\begin{proof}
\end{proof}

%%%%%%%%%%%%%%%%%%%%%%%%%%%%%%%%%%%%%%%%%%%%%%%%%%%%%%%%%%%%%
%-----------------------------------------------------------%
%-----------------------Problem 96---------------------------%
%-----------------------------------------------------------%
%%%%%%%%%%%%%%%%%%%%%%%%%%%%%%%%%%%%%%%%%%%%%%%%%%%%%%%%%%%%%
\newpage
\fancyhead[L]{\scalebox{0.9}{Integration}}
\fancyhead[R]{\scalebox{0.9}{Appeared on: F22}}
\begin{problem}
    \phantom{a}
    \begin{enumerate}[label = (\arabic*)]
        \item State a definition for a real valued function $f:[a,b] \rightarrow \bfR$ to be Riemann integrable on $[a,b]$.
        \item Let $f:[a,b] \rightarrow \bfR$ be Riemann integrable. Prove that $|f(x)|$ is also Riemann integrable and that 
            \begin{equation*}
            \begin{split}
                \left| \int_a^b f(x) dx \right| \leq \int_a^b |f(x)|dx.
            \end{split}
            \end{equation*}
    \end{enumerate}
\end{problem}
\begin{proof}
\end{proof}

%%%%%%%%%%%%%%%%%%%%%%%%%%%%%%%%%%%%%%%%%%%%%%%%%%%%%%%%%%%%%
%-----------------------------------------------------------%
%-----------------------Problem 97---------------------------%
%-----------------------------------------------------------%
%%%%%%%%%%%%%%%%%%%%%%%%%%%%%%%%%%%%%%%%%%%%%%%%%%%%%%%%%%%%%
\newpage
\fancyhead[L]{\scalebox{0.9}{Integration}}
\fancyhead[R]{\scalebox{0.9}{Appeared on: W22}}
\begin{problem}
    \phantom{a}
    \begin{enumerate}[label = (\arabic*)]
        \item State a definition for a real valued function $f:[a,b] \rightarrow \bfR$ to be Riemann integrable on $[a,b]$.
        \item Let 
            \begin{equation*}
            \begin{split}
                f(x) =
                \begin{cases}
                    1, & 1 \leq x < 2 \\
                    10, & x = 2 \\
                    2, & 2 < x \leq 3.
                \end{cases}
            \end{split}
            \end{equation*}
        Use your definition to prove that $f$ is integrable on $[1,3]$.
    \end{enumerate}
\end{problem}
\begin{proof}
\end{proof}

%%%%%%%%%%%%%%%%%%%%%%%%%%%%%%%%%%%%%%%%%%%%%%%%%%%%%%%%%%%%%
%-----------------------------------------------------------%
%-----------------------Problem 98---------------------------%
%-----------------------------------------------------------%
%%%%%%%%%%%%%%%%%%%%%%%%%%%%%%%%%%%%%%%%%%%%%%%%%%%%%%%%%%%%%
\newpage
\fancyhead[L]{\scalebox{0.9}{Integration}}
\fancyhead[R]{\scalebox{0.9}{Appeared on: S15}}
\begin{problem}
    \phantom{a}
    \begin{enumerate}[label = (\arabic*)]
        \item State a definition for a real valued function $f:[a,b] \rightarrow \bfR$ to be Riemann integrable on $[a,b]$.
        \item Let $f:[a,b] \rightarrow \bfR$ be a continuous function. Use your definition to prove that $f$ is integrable on $[a,b]$.
    \end{enumerate}
\end{problem}
\begin{proof}
\end{proof}

%%%%%%%%%%%%%%%%%%%%%%%%%%%%%%%%%%%%%%%%%%%%%%%%%%%%%%%%%%%%%
%-----------------------------------------------------------%
%-----------------------Problem 99---------------------------%
%-----------------------------------------------------------%
%%%%%%%%%%%%%%%%%%%%%%%%%%%%%%%%%%%%%%%%%%%%%%%%%%%%%%%%%%%%%
\newpage
\fancyhead[L]{\scalebox{0.9}{Integration}}
\fancyhead[R]{\scalebox{0.9}{Appeared on: S14}}
\begin{problem}
    \phantom{a}
    \begin{enumerate}[label = (\arabic*)]
        \item State a definition for a real valued function $f:[a,b] \rightarrow \bfR$ to be Riemann integrable on $[a,b]$.
        \item Let $f,g$ be Riemann integrable functions and suppose that the set $E$ is finite where 
            \begin{equation*}
            \begin{split}
                E = \{x \in (a,b) \mid f(x) \neq g(x) \}.
            \end{split}
            \end{equation*}
        Use your definition of Riemann integrability to show that $\int_a^b f(x)dx = \int_a^b g(x)dx$. (\textit{Hint:} Consider the function $f-g$).
    \end{enumerate}
\end{problem}
\begin{proof}
\end{proof}

%%%%%%%%%%%%%%%%%%%%%%%%%%%%%%%%%%%%%%%%%%%%%%%%%%%%%%%%%%%%%
%-----------------------------------------------------------%
%-----------------------Problem 100---------------------------%
%-----------------------------------------------------------%
%%%%%%%%%%%%%%%%%%%%%%%%%%%%%%%%%%%%%%%%%%%%%%%%%%%%%%%%%%%%%
\newpage
\fancyhead[L]{\scalebox{0.9}{Topology of $\bfR$}}
\fancyhead[R]{\scalebox{0.9}{Appeared on: W20}}
\begin{problem}
    Let $f:\bfR \rightarrow \bfR$ be a continuous, periodic function. Prove that the set $f(\bfR)$ is compact. (Recall that a function $f:\bfR \rightarrow \bfR$ is \textit{periodic} if there exists a nonzero constant $P$ such that $f(x) = f(x+P)$ for all $x \in \bfR$).
\end{problem}
\begin{proof}
\end{proof}

\end{document}