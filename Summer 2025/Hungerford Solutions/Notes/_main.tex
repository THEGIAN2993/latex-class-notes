\documentclass[11pt,twoside,openany]{memoir}
%\usepackage{mlmodern}
%\usepackage{tgpagella} % text only
%\usepackage{mathpazo}  % math & text
\usepackage[T1]{fontenc}
\usepackage[hidelinks]{hyperref}
\usepackage{amsmath}
\usepackage{amsthm}
\usepackage{amssymb}
\usepackage{mathtools}
%\renewcommand*{\mathbf}[1]{\varmathbb{#1}}
%\usepackage{newpxtext}
%\usepackage{eulerpx}
%\usepackage{eucal}
\usepackage{datetime}
    \newdateformat{specialdate}{\THEYEAR\ \monthname\ \THEDAY}
\usepackage[margin=1.5in]{geometry}
\usepackage{fancyhdr}
    \fancyhf{}
    \pagestyle{fancy}
    \cfoot{\scriptsize \thepage}
    \fancyhead[R]{\scalebox{0.7}{\rightmark}}
    \fancyhead[L]{\scalebox{0.7}{\leftmark}}
\usepackage{thmtools}
    \declaretheoremstyle[
        spaceabove=10pt,
        spacebelow=10pt,
        headfont=\normalfont\bfseries,
        notefont=\mdseries, notebraces={(}{)},
        bodyfont=\normalfont,
        postheadspace=0.5em
        %qed=\qedsymbol
        ]{defs}

    \declaretheoremstyle[ 
        spaceabove=10pt, % space above the theorem
        spacebelow=10pt,
        headfont=\normalfont\bfseries,
        bodyfont=\normalfont\itshape,
        postheadspace=0.5em
        ]{thmstyle}
    
    \declaretheorem[
        style=thmstyle,
        numberwithin=section
    ]{theorem}

    \declaretheorem[
        style=thmstyle,
        sibling=theorem,
    ]{proposition}

    \declaretheorem[
        style=thmstyle,
        sibling=theorem,
    ]{lemma}

    \declaretheorem[
        style=thmstyle,
        sibling=theorem,
    ]{corollary}

    \declaretheorem[
        numberwithin=section,
        style=defs,
    ]{example}

    \declaretheorem[
        numberwithin=section,
        style=defs,
    ]{definition}

    \declaretheorem[
        style=defs,
        sibling=theorem,
        numberwithin=section,
    ]{exercise}

    \declaretheorem[
        numbered=unless unique,
        shaded={rulecolor=black,
    rulewidth=1pt, bgcolor={rgb}{1,1,1}}
    ]{axiom}

    \declaretheorem[numberwithin=section,style=defs]{note}
    \declaretheorem[numbered=no,style=defs]{question}
    \declaretheorem[numbered=no,style=defs]{recall}
    \declaretheorem[numbered=no,style=remark]{answer}
    \declaretheorem[numbered=no,style=remark]{solution}
    \declaretheorem[numbered=no,style=defs]{remark}

    %%%%%%%%Named‑theorem environment%%%%%%%%%%%%%%%%%%%%%%%%%%%%%%%
    \usepackage{xparse}   % already loaded once; keep just one \usepackage line
    \newcommand*\namedtheoremtitle{}   % stores the current title
    \declaretheoremstyle[
        spaceabove   = 10pt,
        spacebelow   = 10pt,
        headfont     = \normalfont\bfseries,
        bodyfont     = \normalfont\itshape,
        postheadspace= 0.5em,
        headformat   = {\namedtheoremtitle},   % show only the title text
        headpunct    = {}                     % no automatic punctuation
    ]{namedthmstyle}
    \declaretheorem[
        style   = namedthmstyle,
        numbered= yes,      % create a counter so \label/\ref work
        name    = {}        % we supply the “name” at run‑time
    ]{innernamedtheorem}

    % counter prints the title instead of a number
    \renewcommand\theinnernamedtheorem{\namedtheoremtitle} 

    % Environment: one optional “title” in [...]     (default = empty)
    \NewDocumentEnvironment{namedtheorem}{O{}}{%
        \renewcommand*\namedtheoremtitle{#1}%
        \innernamedtheorem
    }{\endinnernamedtheorem}
    %%%%%%%%%%%%%%%%%%%%%%%%%%%%%%%%%%%%%%%%%%%%%%%%%%%%%%%%%%%%%%%%

\usepackage{enumitem}
\usepackage{titlesec}
    \titleformat{\chapter}[display]
    {\bfseries\Huge\raggedright}
    {Chapter {\thechapter}}
    {1ex minus .1ex}
    {\HUGE}
    \titlespacing{\chapter}
    {3pc}{*3}{40pt}[3pc]

    \titleformat{\section}[block]
    {\normalfont\bfseries\LARGE}
    {\S\ \thesection.}{.5em}{}[]
    \titlespacing{\section}
    {0pt}{3ex plus .1ex minus .2ex}{3ex plus .1ex minus .2ex}
\usepackage[utf8x]{inputenc}
\usepackage{tikz}
\usepackage{tikz-cd}
\usepackage{wasysym}
\linespread{1.00}
%%%%%%%%%%%%%%%%%%%%%%%%%%%%%%%%%%%%%%%%%%%%%%%%%%%%%%%%%%%%%
%%%%%%%%%%%%%%%%%%%%%%%%%%%%%%%%%%%%%%%%%%%%%%%%%%%%%%%%%%%%%
%to make the correct symbol for Sha
%\newcommand\cyr{%
%\renewcommand\rmdefault{wncyr}%
%\renewcommand\sfdefault{wncyss}%
%\renewcommand\encodingdefault{OT2}%
%\normalfont \selectfont} \DeclareTextFontCommand{\textcyr}{\cyr}


\DeclareMathOperator{\ab}{ab}
\newcommand{\absgal}{\G_{\bbQ}}
\DeclareMathOperator{\ad}{ad}
\DeclareMathOperator{\adj}{adj}
\DeclareMathOperator{\alg}{alg}
\DeclareMathOperator{\Alt}{Alt}
\DeclareMathOperator{\Ann}{Ann}
\DeclareMathOperator{\arith}{arith}
\DeclareMathOperator{\Aut}{Aut}
\DeclareMathOperator{\Be}{B}
\DeclareMathOperator{\Bd}{Bd}
\DeclareMathOperator{\card}{card}
\DeclareMathOperator{\Char}{char}
\DeclareMathOperator{\csp}{csp}
\DeclareMathOperator{\codim}{codim}
\DeclareMathOperator{\coker}{coker}
\DeclareMathOperator{\coh}{H}
\DeclareMathOperator{\compl}{compl}
\DeclareMathOperator{\conj}{conj}
\DeclareMathOperator{\cont}{cont}
\DeclareMathOperator{\Cov}{Cov}
\DeclareMathOperator{\crys}{crys}
\DeclareMathOperator{\Crys}{Crys}
\DeclareMathOperator{\cusp}{cusp}
\DeclareMathOperator{\diag}{diag}
\DeclareMathOperator{\diam}{diam}
\DeclareMathOperator{\Dom}{Dom}
\DeclareMathOperator{\disc}{disc}
\DeclareMathOperator{\dist}{dist}
\DeclareMathOperator{\dR}{dR}
\DeclareMathOperator{\Eis}{Eis}
\DeclareMathOperator{\End}{End}
\DeclareMathOperator{\ev}{ev}
\DeclareMathOperator{\eval}{eval}
\DeclareMathOperator{\Eq}{Eq}
\DeclareMathOperator{\Ext}{Ext}
\DeclareMathOperator{\Fil}{Fil}
\DeclareMathOperator{\Fitt}{Fitt}
\DeclareMathOperator{\Frob}{Frob}
\DeclareMathOperator{\G}{G}
\DeclareMathOperator{\Gal}{Gal}
\DeclareMathOperator{\GL}{GL}
\DeclareMathOperator{\Gr}{Gr}
\DeclareMathOperator{\Graph}{Graph}
\DeclareMathOperator{\GSp}{GSp}
\DeclareMathOperator{\GUn}{GU}
\DeclareMathOperator{\Hom}{Hom}
\DeclareMathOperator{\id}{id}
\DeclareMathOperator{\Id}{Id}
\DeclareMathOperator{\Ik}{Ik}
\DeclareMathOperator{\IM}{Im}
\DeclareMathOperator{\Image}{im}
\DeclareMathOperator{\Ind}{Ind}
\DeclareMathOperator{\Inf}{inf}
\DeclareMathOperator{\Isom}{Isom}
\DeclareMathOperator{\J}{J}
\DeclareMathOperator{\Jac}{Jac}
\DeclareMathOperator{\lcm}{lcm}
\DeclareMathOperator{\length}{length}
\DeclareMathOperator*{\limit}{limit}
\DeclareMathOperator{\Log}{Log}
\DeclareMathOperator{\M}{M}
\DeclareMathOperator{\Mat}{Mat}
\DeclareMathOperator{\N}{N}
\DeclareMathOperator{\Nm}{Nm}
\DeclareMathOperator{\NIk}{N-Ik}
\DeclareMathOperator{\NSK}{N-SK}
\DeclareMathOperator{\new}{new}
\DeclareMathOperator{\obj}{obj}
\DeclareMathOperator{\old}{old}
\DeclareMathOperator{\ord}{ord}
\DeclareMathOperator{\Or}{O}
\DeclareMathOperator{\op}{op}
\DeclareMathOperator{\PGL}{PGL}
\DeclareMathOperator{\PGSp}{PGSp}
\DeclareMathOperator{\rank}{rank}
\DeclareMathOperator{\Ran}{Ran}
\DeclareMathOperator{\Rel}{Rel}
\DeclareMathOperator{\Real}{Re}
\DeclareMathOperator{\RES}{res}
\DeclareMathOperator{\Res}{Res}
%\DeclareMathOperator{\Sha}{\textcyr{Sh}}
\DeclareMathOperator{\Sel}{Sel}
\DeclareMathOperator{\semi}{ss}
\DeclareMathOperator{\sgn}{sign}
\DeclareMathOperator{\SK}{SK}
\DeclareMathOperator{\SL}{SL}
\DeclareMathOperator{\SO}{SO}
\DeclareMathOperator{\Sp}{Sp}
\DeclareMathOperator{\Span}{span}
\DeclareMathOperator{\Spec}{Spec}
\DeclareMathOperator{\spin}{spin}
\DeclareMathOperator{\st}{st}
\DeclareMathOperator{\St}{St}
\DeclareMathOperator{\SUn}{SU}
\DeclareMathOperator{\supp}{supp}
\DeclareMathOperator{\Sup}{sup}
\DeclareMathOperator{\Sym}{Sym}
\DeclareMathOperator{\Tam}{Tam}
\DeclareMathOperator{\tors}{tors}
\DeclareMathOperator{\tr}{tr}
\DeclareMathOperator{\Tr}{Tr}
\DeclareMathOperator{\un}{un}
\DeclareMathOperator{\Un}{U}
\DeclareMathOperator{\val}{val}
\DeclareMathOperator{\vol}{vol}

\DeclareMathOperator{\Sets}{S \mkern1.04mu e \mkern1.04mu t \mkern1.04mu s}
    \newcommand{\cSets}{\scalebox{1.02}{\contour{black}{$\Sets$}}}
    
\DeclareMathOperator{\Groups}{G \mkern1.04mu r \mkern1.04mu o \mkern1.04mu u \mkern1.04mu p \mkern1.04mu s}
    \newcommand{\cGroups}{\scalebox{1.02}{\contour{black}{$\Groups$}}}

\DeclareMathOperator{\TTop}{T \mkern1.04mu o \mkern1.04mu p}
    \newcommand{\cTop}{\scalebox{1.02}{\contour{black}{$\TTop$}}}

\DeclareMathOperator{\Htp}{H \mkern1.04mu t \mkern1.04mu p}
    \newcommand{\cHtp}{\scalebox{1.02}{\contour{black}{$\Htp$}}}

\DeclareMathOperator{\Mod}{M \mkern1.04mu o \mkern1.04mu d}
    \newcommand{\cMod}{\scalebox{1.02}{\contour{black}{$\Mod$}}}

\DeclareMathOperator{\Ab}{A \mkern1.04mu b}
    \newcommand{\cAb}{\scalebox{1.02}{\contour{black}{$\Ab$}}}

\DeclareMathOperator{\Rings}{R \mkern1.04mu i \mkern1.04mu n \mkern1.04mu g \mkern1.04mu s}
    \newcommand{\cRings}{\scalebox{1.02}{\contour{black}{$\Rings$}}}

\DeclareMathOperator{\ComRings}{C \mkern1.04mu o \mkern1.04mu m \mkern1.04mu R \mkern1.04mu i \mkern1.04mu n \mkern1.04mu g \mkern1.04mu s}
    \newcommand{\cComRings}{\scalebox{1.05}{\contour{black}{$\ComRings$}}}

\DeclareMathOperator{\hHom}{H \mkern1.04mu o \mkern1.04mu m}
    \newcommand{\cHom}{\scalebox{1.02}{\contour{black}{$\hHom$}}}

\renewcommand{\k}{\kappa}
\newcommand{\Ff}{F_{f}}
%\newcommand{\ts}{\,^{t}\!}


%Mathcal
\newcommand{\cA}{\mathcal{A}}
\newcommand{\cB}{\mathcal{B}}
\newcommand{\cC}{\mathcal{C}}
\newcommand{\cD}{\mathcal{D}}
\newcommand{\cE}{\mathcal{E}}
\newcommand{\cF}{\mathcal{F}}
\newcommand{\cG}{\mathcal{G}}
\newcommand{\cH}{\mathcal{H}}
\newcommand{\cI}{\mathcal{I}}
\newcommand{\cJ}{\mathcal{J}}
\newcommand{\cK}{\mathcal{K}}
\newcommand{\cL}{\mathcal{L}}
\newcommand{\cM}{\mathcal{M}}
\newcommand{\cN}{\mathcal{N}}
\newcommand{\cO}{\mathcal{O}}
\newcommand{\cP}{\mathcal{P}}
\newcommand{\cQ}{\mathcal{Q}}
\newcommand{\cR}{\mathcal{R}}
\newcommand{\cS}{\mathcal{S}}
\newcommand{\cT}{\mathcal{T}}
\newcommand{\cU}{\mathcal{U}}
\newcommand{\cV}{\mathcal{V}}
\newcommand{\cW}{\mathcal{W}}
\newcommand{\cX}{\mathcal{X}}
\newcommand{\cY}{\mathcal{Y}}
\newcommand{\cZ}{\mathcal{Z}}


%mathfrak (missing \fi)
\newcommand{\fa}{\mathfrak{a}}
\newcommand{\fA}{\mathfrak{A}}
\newcommand{\fb}{\mathfrak{b}}
\newcommand{\fB}{\mathfrak{B}}
\newcommand{\fc}{\mathfrak{c}}
\newcommand{\fC}{\mathfrak{C}}
\newcommand{\fd}{\mathfrak{d}}
\newcommand{\fD}{\mathfrak{D}}
\newcommand{\fe}{\mathfrak{e}}
\newcommand{\fE}{\mathfrak{E}}
\newcommand{\ff}{\mathfrak{f}}
\newcommand{\fF}{\mathfrak{F}}
\newcommand{\fg}{\mathfrak{g}}
\newcommand{\fG}{\mathfrak{G}}
\newcommand{\fh}{\mathfrak{h}}
\newcommand{\fH}{\mathfrak{H}}
\newcommand{\fI}{\mathfrak{I}}
\newcommand{\fj}{\mathfrak{j}}
\newcommand{\fJ}{\mathfrak{J}}
\newcommand{\fk}{\mathfrak{k}}
\newcommand{\fK}{\mathfrak{K}}
\newcommand{\fl}{\mathfrak{l}}
\newcommand{\fL}{\mathfrak{L}}
\newcommand{\fm}{\mathfrak{m}}
\newcommand{\fM}{\mathfrak{M}}
\newcommand{\fn}{\mathfrak{n}}
\newcommand{\fN}{\mathfrak{N}}
\newcommand{\fo}{\mathfrak{o}}
\newcommand{\fO}{\mathfrak{O}}
\newcommand{\fp}{\mathfrak{p}}
\newcommand{\fP}{\mathfrak{P}}
\newcommand{\fq}{\mathfrak{q}}
\newcommand{\fQ}{\mathfrak{Q}}
\newcommand{\fr}{\mathfrak{r}}
\newcommand{\fR}{\mathfrak{R}}
\newcommand{\fs}{\mathfrak{s}}
\newcommand{\fS}{\mathfrak{S}}
\newcommand{\ft}{\mathfrak{t}}
\newcommand{\fT}{\mathfrak{T}}
\newcommand{\fu}{\mathfrak{u}}
\newcommand{\fU}{\mathfrak{U}}
\newcommand{\fv}{\mathfrak{v}}
\newcommand{\fV}{\mathfrak{V}}
\newcommand{\fw}{\mathfrak{w}}
\newcommand{\fW}{\mathfrak{W}}
\newcommand{\fx}{\mathfrak{x}}
\newcommand{\fX}{\mathfrak{X}}
\newcommand{\fy}{\mathfrak{y}}
\newcommand{\fY}{\mathfrak{Y}}
\newcommand{\fz}{\mathfrak{z}}
\newcommand{\fZ}{\mathfrak{Z}}


%mathbf
\newcommand{\bfA}{\mathbf{A}}
\newcommand{\bfB}{\mathbf{B}}
\newcommand{\bfC}{\mathbf{C}}
\newcommand{\bfD}{\mathbf{D}}
\newcommand{\bfE}{\mathbf{E}}
\newcommand{\bfF}{\mathbf{F}}
\newcommand{\bfG}{\mathbf{G}}
\newcommand{\bfH}{\mathbf{H}}
\newcommand{\bfI}{\mathbf{I}}
\newcommand{\bfJ}{\mathbf{J}}
\newcommand{\bfK}{\mathbf{K}}
\newcommand{\bfL}{\mathbf{L}}
\newcommand{\bfM}{\mathbf{M}}
\newcommand{\bfN}{\mathbf{N}}
\newcommand{\bfO}{\mathbf{O}}
\newcommand{\bfP}{\mathbf{P}}
\newcommand{\bfQ}{\mathbf{Q}}
\newcommand{\bfR}{\mathbf{R}}
\newcommand{\bfS}{\mathbf{S}}
\newcommand{\bfT}{\mathbf{T}}
\newcommand{\bfU}{\mathbf{U}}
\newcommand{\bfV}{\mathbf{V}}
\newcommand{\bfW}{\mathbf{W}}
\newcommand{\bfX}{\mathbf{X}}
\newcommand{\bfY}{\mathbf{Y}}
\newcommand{\bfZ}{\mathbf{Z}}

\newcommand{\bfa}{\mathbf{a}}
\newcommand{\bfb}{\mathbf{b}}
\newcommand{\bfc}{\mathbf{c}}
\newcommand{\bfd}{\mathbf{d}}
\newcommand{\bfe}{\mathbf{e}}
\newcommand{\bff}{\mathbf{f}}
\newcommand{\bfg}{\mathbf{g}}
\newcommand{\bfh}{\mathbf{h}}
\newcommand{\bfi}{\mathbf{i}}
\newcommand{\bfj}{\mathbf{j}}
\newcommand{\bfk}{\mathbf{k}}
\newcommand{\bfl}{\mathbf{l}}
\newcommand{\bfm}{\mathbf{m}}
\newcommand{\bfn}{\mathbf{n}}
\newcommand{\bfo}{\mathbf{o}}
\newcommand{\bfp}{\mathbf{p}}
\newcommand{\bfq}{\mathbf{q}}
\newcommand{\bfr}{\mathbf{r}}
\newcommand{\bfs}{\mathbf{s}}
\newcommand{\bft}{\mathbf{t}}
\newcommand{\bfu}{\mathbf{u}}
\newcommand{\bfv}{\mathbf{v}}
\newcommand{\bfw}{\mathbf{w}}
\newcommand{\bfx}{\mathbf{x}}
\newcommand{\bfy}{\mathbf{y}}
\newcommand{\bfz}{\mathbf{z}}

%blackboard bold

\newcommand{\bbA}{\mathbb{A}}
\newcommand{\bbB}{\mathbb{B}}
\newcommand{\bbC}{\mathbb{C}}
\newcommand{\bbD}{\mathbb{D}}
\newcommand{\bbE}{\mathbb{E}}
\newcommand{\bbF}{\mathbb{F}}
\newcommand{\bbG}{\mathbb{G}}
\newcommand{\bbH}{\mathbb{H}}
\newcommand{\bbI}{\mathbb{I}}
\newcommand{\bbJ}{\mathbb{J}}
\newcommand{\bbK}{\mathbb{K}}
\newcommand{\bbL}{\mathbb{L}}
\newcommand{\bbM}{\mathbb{M}}
\newcommand{\bbN}{\mathbb{N}}
\newcommand{\bbO}{\mathbb{O}}
\newcommand{\bbP}{\mathbb{P}}
\newcommand{\bbQ}{\mathbb{Q}}
\newcommand{\bbR}{\mathbb{R}}
\newcommand{\bbS}{\mathbb{S}}
\newcommand{\bbT}{\mathbb{T}}
\newcommand{\bbU}{\mathbb{U}}
\newcommand{\bbV}{\mathbb{V}}
\newcommand{\bbW}{\mathbb{W}}
\newcommand{\bbX}{\mathbb{X}}
\newcommand{\bbY}{\mathbb{Y}}
\newcommand{\bbZ}{\mathbb{Z}}
\newcommand{\jota}{\jmath}

\newcommand{\bmat}{\left( \begin{matrix}}
\newcommand{\emat}{\end{matrix} \right)}

\newcommand{\bbmat}{\left[ \begin{matrix}}
\newcommand{\ebmat}{\end{matrix} \right]}

\newcommand{\pmat}{\left( \begin{smallmatrix}}
\newcommand{\epmat}{\end{smallmatrix} \right)}

\newcommand{\lat}{\mathscr{L}}
\newcommand{\mat}[4]{\begin{pmatrix}{#1}&{#2}\\{#3}&{#4}\end{pmatrix}}
\newcommand{\ov}[1]{\overline{#1}}
\newcommand{\res}[1]{\underset{#1}{\RES}\,}
\newcommand{\up}{\upsilon}

\newcommand{\tac}{\textasteriskcentered}

%mahesh macros
\newcommand{\tm}{\textrm}

%Comments
\newcommand{\com}[1]{\vspace{5 mm}\par \noindent
\marginpar{\textsc{Comment}} \framebox{\begin{minipage}[c]{0.95
\textwidth} \tt #1 \end{minipage}}\vspace{5 mm}\par}

\newcommand{\Bmu}{\mbox{$\raisebox{-0.59ex}
  {$l$}\hspace{-0.18em}\mu\hspace{-0.88em}\raisebox{-0.98ex}{\scalebox{2}
  {$\color{white}.$}}\hspace{-0.416em}\raisebox{+0.88ex}
  {$\color{white}.$}\hspace{0.46em}$}{}}  %need graphicx and xcolor. this produces blackboard bold mu 

\newcommand{\hooktwoheadrightarrow}{%
  \hookrightarrow\mathrel{\mspace{-15mu}}\rightarrow
}

\makeatletter
\newcommand{\xhooktwoheadrightarrow}[2][]{%
  \lhook\joinrel
  \ext@arrow 0359\rightarrowfill@ {#1}{#2}%
  \mathrel{\mspace{-15mu}}\rightarrow
}
\makeatother

\renewcommand{\geq}{\geqslant}
\renewcommand{\leq}{\leqslant}
\newcommand{\midd}{\hspace{4pt}\middle|\hspace{4pt}}
    
\newcommand{\bone}{\mathbf{1}}
\newcommand{\sign}{\mathrm{sign}}
\newcommand{\eps}{\varepsilon}
\newcommand{\textui}[1]{\uline{\textit{#1}}}

%\newcommand{\ov}{\overline}
%\newcommand{\un}{\underline}
\newcommand{\fin}{\mathrm{fin}}

\newcommand{\chnum}{\titleformat
{\chapter} % command
[display] % shape
{\centering} % format
{\Huge \color{black} \shadowbox{\thechapter}} % label
{-0.5em} % sep (space between the number and title)
{\LARGE \color{black} \underline} % before-code
}

\newcommand{\chunnum}{\titleformat
{\chapter} % command
[display] % shape
{} % format
{} % label
{0em} % sep
{ \begin{flushright} \begin{tabular}{r}  \Huge \color{black}
} % before-code
[
\end{tabular} \end{flushright} \normalsize
] % after-code
}

\newcommand{\nl}{\newline \mbox{}}

\newcommand{\h}[1]{\hspace{#1pt}}

\newcommand{\littletaller}{\mathchoice{\vphantom{\big|}}{}{}{}}
\newcommand\restr[2]{{% we make the whole thing an ordinary symbol
  \left.\kern-\nulldelimiterspace % automatically resize the bar with \right
  #1 % the function
  \littletaller % pretend it's a little taller at normal size
  \right|_{#2} % this is the delimiter
  }}

\newcommand{\mtext}[1]{\hspace{6pt}\text{#1}\hspace{6pt}}

\newcommand{\lnorm}{\left\lVert}
\newcommand{\rnorm}{\right\rVert}

\newcommand{\ds}{\displaystyle}
\newcommand{\ts}{\textstyle}


\newcommand{\sfrac}[2]{{}^{#1}\mskip -5mu/\mskip -3mu_{#2}}


\makeatletter
\newcommand*{\da@rightarrow}{\mathchar"0\hexnumber@\symAMSa 4B }
\newcommand*{\da@leftarrow}{\mathchar"0\hexnumber@\symAMSa 4C }
\newcommand*{\xdashrightarrow}[2][]{%
  \mathrel{%
    \mathpalette{\da@xarrow{#1}{#2}{}\da@rightarrow{\,}{}}{}%
  }%
}
\newcommand{\xdashleftarrow}[2][]{%
  \mathrel{%
    \mathpalette{\da@xarrow{#1}{#2}\da@leftarrow{}{}{\,}}{}%
  }%
}
\newcommand*{\da@xarrow}[7]{%
  % #1: below
  % #2: above
  % #3: arrow left
  % #4: arrow right
  % #5: space left 
  % #6: space right
  % #7: math style 
  \sbox0{$\ifx#7\scriptstyle\scriptscriptstyle\else\scriptstyle\fi#5#1#6\m@th$}%
  \sbox2{$\ifx#7\scriptstyle\scriptscriptstyle\else\scriptstyle\fi#5#2#6\m@th$}%
  \sbox4{$#7\dabar@\m@th$}%
  \dimen@=\wd0 %
  \ifdim\wd2 >\dimen@
    \dimen@=\wd2 %   
  \fi
  \count@=2 %
  \def\da@bars{\dabar@\dabar@}%
  \@whiledim\count@\wd4<\dimen@\do{%
    \advance\count@\@ne
    \expandafter\def\expandafter\da@bars\expandafter{%
      \da@bars
      \dabar@ 
    }%
  }%  
  \mathrel{#3}%
  \mathrel{%   
    \mathop{\da@bars}\limits
    \ifx\\#1\\%
    \else
      _{\copy0}%
    \fi
    \ifx\\#2\\%
    \else
      ^{\copy2}%
    \fi
  }%   
  \mathrel{#4}%
}
\makeatother

\renewcommand{\div}[2]{%
  \scalebox{0.92}{$#1$}%
  \hspace{1.2pt}%
  \scalebox{1.2}{$\mid$}%
  \hspace{0.75pt}%
  \scalebox{0.92}{$#2$}%
}


\begin{document}
    \chapter*{Preface}
        These are solutions for Hungerford's \textit{Algebra} text. Any theorems cited in the exercises or hints will be included.
    \pagenumbering{roman}
    \tableofcontents
    

    \vfill
    \specialdate
    Last update: \today

    \chapter{Groups}
\pagenumbering{arabic}
%%%%%%%%%%%%%%%%%%%%%%%%%%%%%%%%%%%%%%%%%%%%%%%%%%%%%%%%%%%%%
%%%%%%%%%%%%%%%%%%%%%%%%%%%%%%%%%%%%%%%%%%%%%%%%%%%%%%%%%%%%%
%%%%%%%%%%%%%%%%%%%%%%%%%%%%%%%%%%%%%%%%%%%%%%%%%%%%%%%%%%%%%
%%%%%%%%%%%%%%%%%%%%%%%%%%%%%%%%%%%%%%%%%%%%%%%%%%%%%%%%%%%%%
%%%%%%%%%%%%%%%%%%%%%%%%%%%%%%%%%%%%%%%%%%%%%%%%%%%%%%%%%%%%%
\section{Semigroups, Monoids, and Groups}
    \begin{exercise}
        Give examples other than those in the text of semigroups and monoids that are not groups.
    \end{exercise}
    
    \begin{exercise}
        Let $G$ be a group (written additively), $S$ a nonempty set, and $M(S,G)$ the set of all functions $f:S \rightarrow G$. Define addition in $M(S,G)$ as follows: $(f+g):S \rightarrow G$ is given by $s \mapsto f(s) + g(s) \in G$. Prove that $M(S,G)$ is a group, which is abelian if $G$ is.
    \end{exercise}
        {\color{blue} \begin{proof}
            Clearly $M(S,G)$ is closed under addition defined as above. Assocativity can be seen as follows:
                \begin{equation*}
                \begin{split}
                    [f+(g+h)](s)
                    & = f(s) + (g+h)(s) \\
                    & = f(s) + [g(s) + h(s)] \\
                    & = [f(s) + g(s)] + h(s) \h9\text{\tiny Since $G$ is a group.}\\
                    & = (f+g)(s) + h(s) \\
                    & = [(f+g)+h](s).
                \end{split}
                \end{equation*}
            Define $\mathbf{0}:S \rightarrow G$ by $\mathbf{0}(s) = 0$, the additive identity of $G$. Then:
                \begin{equation*}
                \begin{split}
                    (f+\mathbf{0})(s) 
                    & = f(s) = \mathbf{0}(s) \\
                    & = f(s) + 0 \\
                    & = f(s) \\
                    & = 0 + f(s) \\
                    & = \mathbf{0}(s) + f(s) \\
                    & = (\mathbf{0} + f)(s).
                \end{split}
                \end{equation*}
            Whence $\mathbf{0} \in M(S,G)$ is the identity element. Given $f \in M(S,G)$, define $f^{-1}:S \rightarrow G$ by $s \mapsto -f(s)$. We can see:
                \begin{equation*}
                \begin{split}
                    (f+f^{-1})(s) 
                    & = f(s) + f^{-1}(s) \\
                    & = 0 \\
                    & = (-f(s)) + f(s) \\
                    & = f^{-1}(s) + f(s) \\
                    & = (f^{-1} + f)(s).
                \end{split}
                \end{equation*}
            Thus $M(S,G)$ is a group. If $G$ is abelian, then:
                \begin{equation*}
                \begin{split}
                    (f+g)(s)
                    & = f(s) + g(s) \\
                    & = g(s) + f(s) \h9 \text{\tiny Since $G$ is abelian.} \\
                    & = (g+f)(s).
                \end{split}
                \end{equation*}
            Thus $M(S,G)$ is an abelian group.
        \end{proof}}
    
    \begin{exercise}
        Is it true that a semigroup which has a \textit{left} identity element and in which every element has a \textit{right} inverse is a group?
    \end{exercise}
        {\color{blue} \begin{proof}
            No. Consider $G = \{a,e\}$ with a binary operation defined as follows:
                \begin{equation*}
                \begin{split}
                    ea &= a, \\
                    ae &= e.
                \end{split}
                \end{equation*}
            Note that, for any $a,b,c \in G$:
                \begin{equation*}
                \begin{split}
                    a(bc) &= bc = c, \\
                    (ab)c &= bc = c.
                \end{split}
                \end{equation*}
            Thus $G$ is a semigroup. By construction $G$ admits a left identity element and every element admits a right inverse. Note that $G$ is not a group, since $ae = e \neq a$; i.e., $G$ does not admit a right identity.
        \end{proof}}
    
    \begin{exercise}
        Write out the multiplication table for the group $D_4^\ast$.
    \end{exercise}
    
    \begin{exercise}
        Prove that the symmetric group on $n$ letters, $S_n$, has order $n!$.
    \end{exercise}
        {\color{blue} \begin{proof}
            Starting at $1 \in S$, there are $n$ different elements which $1$ can be mapped to. For $2 \in S$, there are $n-1$ different elements which $2$ can be mapped to (both 1 and 2 cannot be mapped to the same element, otherwise our function is not bijective). This process continues until we reach $n \in S$, which will only have one element which it can be mapped to. Thus there are $n!$ different permutations on the set $S$; i.e., $|S_n| = n!$.
        \end{proof}}
    
    \begin{exercise}
        Write out an addition table for $\bfZ/2\bfZ \oplus \bfZ/2\bfZ$. $\bfZ/2\bfZ \oplus \bfZ/2\bfZ$ is called the \textbf{Klein four group}.
    \end{exercise}
        {\color{blue} \begin{proof}
            \phantom{a}
                \begin{center}
                \begin{tabular}{ c| c | c | c | c |}
                + & (0,0) & (0,1) & (1,0) & (1,1) \\
                \hline
                (0,0) & (0,0) & (0,1) & (1,0) & (1,1) \\ 
                \hline
                (0,1) & (0,1) & (0,0) & (1,1) & (1,0) \\ 
                \hline
                (1,0) & (1,0) & (1,1) & (0,0) & (0,1) \\ 
                \hline
                (1,1) & (1,1) & (1,0) & (0,1) & (0,0) \\ 
                \hline
                \end{tabular}
                \end{center}
        \end{proof}}
    
    \begin{exercise}{\color{red} ***}
        If $p$ is prime, then the nonzero elements of $\bfZ/p\bfZ$ form a group of order $p-1$ under multiplication. [\textit{Hint:} $[a]_{p} \neq [0]_p \Rightarrow \gcd(a,p) = 1$; use \ref{thm:intro-6.5}] Show that this statement is false if $p$ is not prime.
    \end{exercise}
        {\color{blue} \begin{proof}
            Denote the nonzero elements of $\bfZ/p\bfZ$ as $\bfZ/p\bfZ^\ast$. Since there are $p-1$ nonzero elements of $\bfZ/p\bfZ$, then $|\bfZ/p\bfZ^\ast| = p-1$. Moreover, since $\bfZ/p\bfZ$ is a commutative monoid under multiplication by \ref{thm:intro-6.8} and \ref{thm:groups-thm1.5}, it must be that multiplication in $\bfZ/p\bfZ^\ast$ is well-defined.

            We must first show that multiplication is closed in $\bfZ/p\bfZ^\ast$. Let $[a]_p,[b]_p \in \bfZ/p\bfZ^\ast$. Suppose that $[ab]_p = [0]_p$. Then $p \mid ab$. Since $[a]_p \in \bfZ/p\bfZ^\ast$, we know that $[a]_p \neq [0]_p$; i.e., $\gcd(a,p) = 1$. By \ref{thm:intro-6.6}, it must be the case that $p \mid b$. But this contradicts the fact that $[b]_p \in \bfZ/p\bfZ^\ast$. Thus $[ab]_p \neq [0]_p$, giving that $\bfZ/p\bfZ^\ast$ is closed under multiplication.

            Again, since $\bfZ/p\bfZ$ is a commutative monoid under multiplication, it must be that multiplication in $\bfZ/p\bfZ^\ast$ is associative. The identity element is $[1]_p \in \bfZ/p\bfZ^\ast$; observe that for any $[a]_p \in \bfZ/p\bfZ^\ast$:
                \begin{equation*}
                \begin{split}
                    [a]_p [1]_p
                    & = [a\cdot1]_p \\
                    & = [a]_p \\
                    & = [1 \cdot a]_p \\
                    & = [1]_p [a]_p.
                \end{split}
                \end{equation*}
            It remains to show that each element in $\bfZ/p\bfZ^\ast$ has an inverse. Let $[a]_p \in \bfZ/p\bfZ^\ast$ be arbitrary. Then $\gcd(a,p) = 1$. There exists integers $r,s$ so that $ar + ps = 1$. This is equivalent to $ar = ra \equiv 1 \pmod{p}$. We've shown there exists some $r$ such that $[a]_p[r]_p = [ar]_p = [1]_p = [ra]_p = [r]_p[a]_p$. Thus $\bfZ/p\bfZ^\ast$ is a group.

            Suppose that $m$ is a composite integer...
        \end{proof}}
    
    \begin{exercise}
        \phantom{a}
        \begin{enumerate}[label = (\alph*),itemsep=1pt,topsep=3pt]
            \item The relation given by $a \sim b \iff a-b \in \bfZ$ is a congruence relation on the additive group $\bfQ$ [see Groups, Theorem 1.5].
            \item The set $\bfQ/\bfZ$ of equivalence classes is an infinite abelian group.
        \end{enumerate}
    \end{exercise}
        {\color{blue} \begin{proof}
            (a) Our relation $\sim$ is clearly reflexive: $a-a = 0 \in \bfZ$, whence $a \sim a$. If $a \sim b$, then $a-b \in \bfZ$. Since $\bfZ$ is a group under addition, the additive inverse of $a-b$ exists; i.e., $-(a-b) = b-a \in \bfZ$. So $b \sim a$, showing that $\sim$ is symmetric. If $a \sim b$ and $b \sim c$, then $a-b \in \bfZ$ and $b - c \in \bfZ$. Since addition is closed under $\bfZ$, $(a-b) + (b-c) = a - c \in \bfZ$, giving $a \sim c$. Thus $\sim$ is transitive, and altogether it is an equivalence relation. If $a \sim b$ and $c \sim d$, then $a - b \in \bfZ$ and $c - d \in \bfZ$. Again by the closure of $\bfZ$, $(a-b) + (c-d) = (a+c) - (b+d) \in \bfZ$. Thus $\sim$ is a congruence relation, as we've shown $a + c \sim b + d$.

            (b) It is routine to show that all of the group axioms are satisfied by the elements of $\bfQ/\bfZ$. Define $f:\bfN \rightarrow \bfQ/\bfZ$ by $n \mapsto \overline{\frac{1}{n}}$. Observe that:
                \begin{equation*}
                \begin{split}
                    f(n_1) = f(n_2) 
                    & \implies \overline{\frac{1}{n_1}} = \overline{\frac{1}{n_2}} \\
                    & \implies \overline{\frac{1}{n_1}} = \overline{\frac{1}{n_2}} \\
                \end{split}
                \end{equation*} 

            Define $\pi: \bfQ \rightarrow \bfQ/\bfZ$ by $q \mapsto \overline{q}$.
        \end{proof}}
    
    \begin{exercise}
        Let $p$ be a fixed prime. Let $R_p$ be the set of all those rational numbers whose denominator is relatively prime to $p$. Let $R^p$ be the set of rationals whose denominator is a power of $p$ ($p^1$, $i \geq 0$). Prove that both $R_p$ and $R^p$ are abelian groups under ordinary addition of rationals.
    \end{exercise}
    
    \begin{exercise}
        Let $p$ be prime and let $Z(p^\infty)$ be the following subset of the group $\bfQ/\bfZ$ (see pg. 27):
            \begin{equation*}
            \begin{split}
                Z(p^\infty) = \{\overline{\sfrac{a}{b}} \in \bfQ/\bfZ \mid a,b \in \bfZ, \h5 b = p^i \h5 \text{for some}\h5 i \geq 0\}.
            \end{split}
            \end{equation*}
        Show that  $Z(p^\infty)$ is an infinite group under the addition operation of $\bfQ/\bfZ$.
    \end{exercise}
    

    \begin{exercise}
        The following conditions on a group $G$ are equivalent:
            \begin{enumerate}[label = (\roman*),itemsep=1pt,topsep=3pt]
                \item $G$ is abelian;
                \item $(ab)^2 = a^2 b^2$ for all $a,b \in G$;
                \item $(ab)^{-1} = a^{-1}b^{-1}$ for all $a,b \in G$;
                \item $(ab)^n = a^n b^n$ for all $n \in \bfZ$ and all $a,b \in G$;
                \item $(ab)^n = a^n b^n$ for three consecutive integers $n$ and all $a,b \in G$;
            \end{enumerate}
        Show that $(v) \Rightarrow (i)$ is false if "three" is replaced by "two."
    \end{exercise}
    
    \begin{exercise}
        If $G$ is a group, $a,b \in G$ and $bab^{-1} = a^r$ for some $r \in \bfN$, then $b^j ab^{-j} = a^{r^j}$ for all $j \in \bfN$. 
    \end{exercise}
    
    \begin{exercise}
        If $a^2 = e$ for all elements $a$ of a group $G$, then $G$ is abelian.
    \end{exercise}
    
    \begin{exercise}
        If $G$ is a finite group of even order, then $G$ contains an element $a \neq e$ such that $a^2 = e$.
    \end{exercise}
    
    \begin{exercise}
        Let $G$ be a nonempty finite set with an associative binary operation such that for all $a,b,c \in G$ $ab = ac \Rightarrow b = c$ and $ba = ca \Rightarrow b=c$. Then $G$ is a group. Show that this conclusion may be false if $G$ is infinite.
    \end{exercise}
    
    \begin{exercise}
        Let $a_1,a_2,...$ be a sequence of elements in a semigroup $G$. Then there exists a unique function $\psi:\bfN^\ast \rightarrow G$ such that $\psi(1) = a_1$, $\psi(2) = a_1 a_2$, $\psi(3) = (a_1 a_2)a_3$ and for $n \geq 1$, $\psi(n+1) = (\psi(n))a_{n+1}$. Note that $\psi(n)$ is precisely the standard $n$ product $\prod_{i = 1}^n a_i$. [\textit{Hint:} Applying the Recursion Theorem 6.2 of Introduction with $a = a_1$, $S = G$, and $f_n:G \rightarrow G$ given by $x \mapsto x a_{n+2}$ yields a function $\varphi:\bfN \rightarrow G$. Let $\psi = \varphi \theta$, where $\theta:\bfN^\ast \rightarrow \bfN$ is given by $k \mapsto k-1$.] 
    \end{exercise}
%%%%%%%%%%%%%%%%%%%%%%%%%%%%%%%%%%%%%%%%%%%%%%%%%%%%%%%%%%%%%
%%%%%%%%%%%%%%%%%%%%%%%%%%%%%%%%%%%%%%%%%%%%%%%%%%%%%%%%%%%%%
%%%%%%%%%%%%%%%%%%%%%%%%%%%%%%%%%%%%%%%%%%%%%%%%%%%%%%%%%%%%%
%%%%%%%%%%%%%%%%%%%%%%%%%%%%%%%%%%%%%%%%%%%%%%%%%%%%%%%%%%%%%
%%%%%%%%%%%%%%%%%%%%%%%%%%%%%%%%%%%%%%%%%%%%%%%%%%%%%%%%%%%%%
\section{Homomorphisms and Subgroups}
    \begin{exercise}
        If $f:G \rightarrow H$ is a homomorphism of groups, then $f(e_G) = e_H$ and $f(a^{-1}) = f(a)^{-1}$ for all $a \in G$. Show by example that the first conclusion may be false if $G,H$ are monoids that are not groups.
    \end{exercise}
    \appendix
    \chapter*{Appendix}
\addcontentsline{toc}{chapter}{Appendix}

\begin{namedtheorem}[Introduction, Theorem 6.5]\label{thm:intro-6.5}
    If $a_1,a_2,...,a_n$ are integers, not all 0, then $\gcd(a_1,a_2,...,a_n)$ exists. Furthermore there are integers $k_1,k_2,...,k_n$ such that
        \begin{equation*}
        \begin{split}
            \gcd(a_1,a_2,...,a_n) = k_1 a_1 + k_2 a_2 + ... + k_n a_n.
        \end{split}
        \end{equation*}
\end{namedtheorem}

\begin{namedtheorem}[Introduction, Theorem 6.6]\label{thm:intro-6.6}
    If $a$ and $b$ are relatively prime integers (that is, $\gcd(a,b) = 1$) and $a \mid bc$, then $a \mid c$. If $p$ is prime and $p \mid a_1 a_2 ... a_n$, then $p \mid a_i$ for some $i$.
\end{namedtheorem}

\begin{namedtheorem}[Introduction, Theorem 6.8]\label{thm:intro-6.8}
    Let $m>0$ be an integer and $a,b,c,d \in \bfZ$.
    \begin{enumerate}[label = (\roman*),itemsep=1pt,topsep=3pt]
        \item Congruence modulo $m$ is an equivalence relation on the set of integers $\bfZ$, which has precisely $m$ equivalence classes.
        \item If $a \equiv b \pmod{m}$ and $c \equiv d \pmod{m}$, then $a+c \equiv b + d \pmod{m}$ and $ac \equiv bd \pmod{m}$.
        \item If $ab \equiv ac \pmod{m}$ and $a$ and $m$ are relatively prime, then $b \equiv c \pmod{m}$.
    \end{enumerate}
\end{namedtheorem}

\begin{namedtheorem}[Groups, Theorem 1.5]\label{thm:groups-thm1.5}
    Let $\sim$ be an equivalence relation on a monoid $G$ such that $a_1 \sim a_2$ and $b_1 \sim b_2$ imply $a_1b_1 \sim a_2 b_2$ for all $a_i,b_i \in G$. Then the set $G/\hspace{-4pt}\sim$ of all equivalence classes of $G$ under $\sim$ is a monoid under the binary operation defined by $\overline{a} \overline{b} = \overline{ab}$, where $\overline{x}$ denotes the equivalence class of $x \in G$. If $G$ is an (abelian) group, then so is $G/\hspace{-4pt}\sim$.
\end{namedtheorem}
\end{document}